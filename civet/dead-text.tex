\begin{comment}
%- Haven, VC3
%Shielding Applications from an Untrusted Cloud with Haven
Researchers are actively working on developing secure solutions for \sgx{}.
Baumann etal., built a system named 
Haven~\cite{baumann14haven} to defend complete windows applications of cloud users against 
privileged code and physical attacks by leveraging 
the \sgx{} hardware. Haven solves the problem of 
enclave plaintext code by using a separate encrypted 
disk and, like Graphene-\sgx{} library OS~\cite{graphene-sgx}, which uses 
Graphene~\cite{tsai14graphene}
for Linux, Haven uses Drawbridge~\cite{porter11drawbridge} 
LibOS. However, instead of partitioning the 
application into trusted and untrusted parts, Haven 
places trust in the whole application. As a 
result, a vulnerability in any part of the 
application code can cause information leakage. 
While, Haven design causes the TCB to be 
unnecessarily huge, \sysname{} carefully 
categorizes the minimum parts of the application 
as trusted, which runs in the enclave, effectively 
reducing the TCB and attack vectors.

%Intel SDK
On the other hand, Intel provides the SDK to build 
\sgx{} compatible applications in C/C++ by compiling 
the trusted part of the application using the 
\intel{} C/C++ Compiler. Such partitioned 
applications have very small TCB --- the trusted part 
of the application and the hardware --- but 
partitioning is limited to only C/C++ applications. 
Moreover, the partitioning of applications need to be 
done manually by an expert, which is a cumbersome and 
error-prone process. A lot of user applications are 
developed in higher level languages like JAVA. 
\intel{} \sgx{} SDK cannot support such applications. 
\sysname{} helps JAVA developers to automatically 
partition their JAVA applications and provides 
infrastructure to seamlessly interact between the 
trusted and untrusted parts without rewriting the 
whole application. The design of \sysname{} can be 
extended to other high level object oriented 
languages.
\end{comment}
