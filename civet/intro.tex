%\section{Introduction}
%\label{sec:intro}

%% \fixme{Chia-Che Tsai's opening}
%% Modern security hardwares such as TPM or \intel{} \sgx{}
%% provide hardware enforced guarantees of security properties
%% even upon %untrustworthy hosts with
%% compromised operating systems or tampered hardwares.
%% The reasoning of security using these hardwares
%% is often founded as follows:
%% An untamperable hardware package such as CPU will guarantee
%% the integrity of applications loaded
%% exactly identical to the images signed by clients.
%% However, to achieve overall security,
%% the clients are still responsible for ensuring that no vulnerabilities
%% within the signed images
%% will compromise the security properties in runtime.

%\fixmedp{I would start by motivating the goal of application partitioning.
%I think you could add a little more here, but this is a rough sketch.}
An increasing fraction of applications handle sensitive data (e.g., medical applications)
or implement proprietary algorithms (e.g., algorithmic stock trading).
%, such as medical applications, or 
%implement proprietary algorithms, such as algorithmic stock trading.  
These applications are composed of 
libraries and other code from third parties, and may run on 
hardware and hypervisors provided by an untrusted cloud provider.
{\bf Application partitioning} is a technique to 
bound the 
static or dynamic analysis effort to protect sensitive data and algorithms,
while still taking advantage of inexpensive code and cloud hardware.
A partitioned application isolates sensitive data and code, typically using
language or hardware techniques.


%Why \sgx{}
%\fixme{Bhushan's opening}
At the hardware level, 
new CPUs are offering features to protect application-level code and data
from a  compromised or malicious system software stack,
including the OS or hypervisor, or simply isolating portions of the 
application address space from the rest of the application.
Examples include \intel{} \sgx{}~\citep{intelsgx}, 
%\fixmedp{Can we mention/cite others? IBM and ARM?}
IBM SecureBlue++~\citep{secureblue++}, and ARM TrustZone~\citep{trustzone}.
%are offering models of mutual distrust, which facilitate protecting
%an application from a
\sgx{} offers %One feature of \sgx{} is restricted
entry points to an encrypted region of memory, called an \term{enclave}.
Encryption and remote attestation 
prevent the hypervisor or OS from reading or modifying the enclave's contents,
and validate that the enclave was correctly initialized.
This in turn
can exclude 
a cloud provider's hypervisor or OS from the application's trusted computing base (TCB),
requiring only trust in the hardware manufacturer (i.e., Intel).
%Remote attestation can also validate that an enclave was correctly initialized.
%Enclaves also provide remote attestation, where a client
%can ascertain whether the hardware is genuine Intel, and that the application was
%loaded correctly.
\sgx{} also offers protection against malicious devices off the CPU package,
such as a compromised storage device.

Although SGX can restrict enclave entry 
to a few programmer-defined entry points, the programmer must still 
reason about what those entry points should be, protect against 
malicious inputs, and verify that code paths within the enclave that cannot
inadvertently leak sensitive data.
Iago attacks~\citep{checkoway13iago}, where an OS offers malicious system call return values,
 are notoriously difficult to defend against;
although limited countermeasures exist for the OS API (typically with the cooperation of a 
trusted hypervisor)~\citep{kwon2016sego},%\fixmedp{cite the asplos 16 inktag paper},
there are no off-the-shelf defenses against Iago-style attacks from libraries
or compromised application code.
Further, SGX is designed to run a small, native code library inside of a larger
application; support for running managed languages in an enclave is currently limited.  
Writing a security-sensitive application in an unsafe
language like C dramatically increases the risk of exploitable pointer or control flow errors~\citep{nergal2001libc,bletsch2011jump,checkoway2010return}, which ultimately disclose sensitive data or undermine the integrity of the enclave.

%Modern security hardware
%such as  provides hardware protection to applications
%especially the ones deployed in the cloud,
%so that the applications do not have to trust the privileged
%OS layer controlled by the cloud provider. 
%against malicious system stacks, including OS, hypervisors or devices. 
%In principle, these hardware techniques 
%Hardware protection between applications and their hosts is appealing
%in cloud computing, because the cloud client 
%need only trust the 
%the cloud provider's hypervisor can, in principle, 
%be excluded from the application's trusted computing base (TCB).
%These hardware protection is especially essential for cloud computing, due to the multi-tenant environment and potentially compromisable providers.
%\fixme{I want to emphasize it can defend against hardware attack.}
%This type of hardware creates a sanctuary for the most security-sensitive
%components of an application---

%For instance, \sgx{} creates a encrypted memory region 
%called enclave within the memory space of an application,
%which can only be accessed by verified code from the users.
%\fixme{I rather not emphasize partitioning so early, it's not the only way of using enclave.}
%\sgx{} lets the developer partition the application into two parts --- 
%security sensitive and non-security sensitive parts. The security
%sensitive part of the application is run in an hardware isolated
%sandbox called enclave, while the non-security sensitive part runs
%normally outside the enclave.
%% \sgx{} guarantees not only the integrity of loaded application image,
%% but also the confidentiality
%% of application data, %in enclave,
%% avoiding any information leakage to the untrusted exterior world.
%\sgx{} provides a fail-safe way to prevent information leakage from the enclave:
%by guaranteeing the integrity of a measured %and verified
%application binary
%and the confidentiality of application data in memory.
%the OS or hypervisor are simply stopped from peeking any application data, unless the isolated applications release any information. 

Language-level techniques can offer more sophisticated insights into how
to partition the application, as well as stronger protections ~\citep{bittau2008wedge,brumley2004privtrans,khatiwala2006data}.
%\fixmedp{Can we bless some language-level app partitioning tools here?}
Similarly, managed languages such as \java{}, with type-checking,
can prevent memory corruption attacks,
while applications developed in C or C++ are often prone to memory-bug exploitation.  
Because \java{} is memory safe, it is immune to known control flow attacks, such as return-oriented programming,
where control flow is manipulated by unsafe writes to return pointers on the stack or function pointers in objects.

%protection prevents applications from being compromised by
%vulnerabilities that exist inside the applications.
%The vulnerabilities include mistakes made by developers, such as semantic and configuration bugs,
%and unbound application behaviors that can be exploited by attackers,
%such as memory attacks or control flow attacks.
%Based on the different languages used by developers,
%applications can be offered stronger language specific security guarantees.

Partitioned application developers need both hardware and language-level protections,
but these are difficult to combine in practice.
%In general, hardware protection and language-level protection 
%provide separate security guarantees.
%Unfortunately, developers often struggle to combine both types of protection, due to the limitations and lack of support for both approaches.
This paper focuses on \sgx{} and the \java{} language as a running example:
\sgx{} enclaves are designed to run native binaries, primarily developed in C or C++,
so developers must employ considerable effort to port a \java{} application or a \java{} VM to SGX.
To our knowledge, no previous effort has even run a JVM in an enclave, much less partitioned part of a Java application into an enclave.
For \java{} language, there is no API support for \sgx{} enclaves,
making it challenging to leverage isolation features of \sgx{}
in \java{} applications. 
%Developers often have to create an ad hoc integration of hardware and language-level protection,
%or they are forced to exclusively expose the application between
%attacks from the system stacks and vulnerabilities inside the applications.
The current  state-of-the-art is to develop an ad hoc solution for 
a specific use case.
% approaches of combining
%language and hardware protection often rely on ad-hoc solutions that
%are specific to certain use cases or technologies.
For example, VC3 uses \sgx{} to protect mappers and reducers in Hadoop framework~\citep{vc3}.
Although Hadoop is written in Java, the protected code in VC3 cannot be written in Java.

%% dp: this sentence seems off track, unless you want to say something about ``this has to be in C''
%Similarly, S-NFV uses \sgx{} to isolate states of Network Function Virtualization Applications~\citep{shih2016s}.

%\fixmedp{Please check the previous 2 sentences for accuracy; if I'm wrong, try to say something more crisp.
%Also, any other important related work here?}
%to utilize SGX hardware in language runtimes,
%a set of APIs may be exported through the native interface to support certain high-demand use cases, such as Hadoop workloads
%Although these existing solutions nicely support a part of the users,
%and potentially have demonstrated some common wisdom for combining the language and hardware protection,
%the problem of how to generalize the approach
%to cover more use cases or technologies that the authors haven't forseen,
%still remains unsolved.

%% dp: This paragraph is a little too much philosophy for me.  Cutting for now, but can bring some fo this back later in the intro if needed.  I think it would be better for explaining the benfits of Civet.
\begin{comment}
We use SGX vs Java-based protection as an example,
to show how the guarantees ({\em what is secured?})
and features ({\em how is it secured?}) of hardware protection
can be properly modeled at the language level,
so that language protection mechanisms can be extended
to mitigate vulnerabilities in hardware protection. 
For instance,
SGX hardware provides isolation between multiple
levels of security sensitivity in an application process,
by building another privilege level that bypasses the OS or hypervisor.
We model this hardware guarantee at language level by asking the developers
to identify the multi-level security sensitivity in the application.
The developers do not have to worry about
when to enter or leave the enclaves,
what information to flow out of the enclaves,
or even how SGX enclaves fundamentally work.
The language features
such as implicit and explicit information flow tracking can be {\em leveraged}
by the low-level component that interfaces with SGX enclaves
as the criterion for whether certain information
is safe to be released from the enclave.
\end{comment}
%while 
%


%to the privileged
%or un-privileged code outside the \enclave{}.

%One of the primary concerns on the security of enclaves is
%the strong assumption
%that enclaves' security guarantees rely on---
%The security guarantees of hardwares like \sgx{} rely on a strong assumption:
%To provide the confidentiality guarantees,
%the code running in an enclave must not
%the secured environment (\enclave{}) must not
%have %any bugs or
%vulnerabilities that can be exploited for causing any information leakage.
%However, without language support, it is hard to ensure that
%the enclave code is
%free of vulnerabilities, especially if written in C---which provide no memory safety.
%the \sgx{} only supports programs 
%written in C or C++, and it is extremely difficult to find
%all the memory corruption bugs in the enclave code written in C or C++.
%Without memory safety, 
%the control flow of the code %within the enclave
%can be manipulated by exploiting memory corruption.
%This type of vulnerabilities will lead to information leakage,
%defeating the purpose of using \sgx{} enclaves.
%secure hardware such as
%\sgx{} in the first place. 


%% Lack of memory safety using C or C++ in \sgx{}-secured environments \fixme{maybe spell out why it's hard for SGX}
%% enervates any software approaches to reinforce application security, including control flow and information flow integrity, 
%% %These memory corruption bugs can be exploited to change the control flow within the \enclave{} 
%% %as well as leak confidential information outside the \enclave{},
%% defeating the purpose of using %\sgx{}
%% secure hardware in the first place. 
%% \fixme{perhaps justify why not use verification?}

%One solution to guarantee memory safety in applications is to use 
%Using a managed language such as Java
%to implement the security sensitive code
%is a solution to guarantee
%can provide the memory safety
%required by an enclave.
%The solution to avoid the memory corruption bugs is to use a memory safe managed
%language like Java. Thus, partitioning an application written in language such as 
%Java and running the Java based security-sensitive code in an \enclave{}
%can help provide the control flow as well as information flow guarantees that \sgx{} is
%designed for.
%Moreover, Java is widely used to develop many enterprise applications,
%that can improve their security guarantees by leveraging the \sgx{} feature.
%However,
%in absence of Java language support for \sgx{} enclave, 
%it is cumbersome to leverage security guarantees of \sgx{} directly,
%unless developers 
%rewrite the security-sensitive components in C or C++, 
%and integrate %this new code with the original application using a 
%using a foreign function interface such as Java Native Interface (JNI).
%Java Native Interface (JNI).
%to integrate with the rest of the enterprise application.
%Moreover,
%Reimplementing the security-sensitive components
%of the application 
%in C or C++
%Rewriting in C or C++ 
%essentially voids any security benefits brought by managed languages.
%the memory safety guarantees provided by Java. 

%Additionally, having runtime support for \sgx{} in %a managed language like
%Java
%has many benefits.
%Code written in Java %a managed language
%is easier to 
%modularize, partition and reason about the security and information flow.
%Thus, providing Java runtime support for \sgx{} can help provide stronger security guarantees
%for applications running in an \enclave{} as well as make numerous applications
%written in Java secure against shared vulnerabilities of the privileged layer.
%Also, imported library classes can be also secured when used in applications instead of be blindly trusted.
%\fixmebj{Find one more benefit of using Java.} 
%Moreover,
%Java is widely used to develop many enterprise applications,
%which can benefit from the security guarantees of \sgx{}
%without the cost of reimplementation.
%that can improve their security guarantees by leveraging the \sgx{} feature.

%Another notable impediment for effectively leveraging \sgx{} 
%feature in a managed language
%is the trouble developers
%have to endure to partition an existing application into two clear parts and design the interface
%between these two parts. There is a very good chance that an unsuspecting developer may copy
%information which is either security sensitive or derived from security sensitive data to outside 
%the enclave,
%undermining the hardware security guarantees provided by \sgx{}. Thus, we need an 
%easy to use language runtime support for \sgx{}
%that can shield the developer from the complexity of determining whether security sensitive information 
%is leaked outside the enclave.

We present \term{\sysname{}},
a system that combines the benefits of
 \sgx{} enclave isolation with \java{} language security features to protect security-sensitive applications.
\sysname{} avoids the downsides of taking either approach in isolation.
\sysname{} includes three primary parts:
a tool that guides the programmer in partitioning an application, ultimately generating an enclave image that includes
a minimal Java code base;
a runtime framework that loads and verifies classes, and enforces
information flow control across the enclave boundary;
and a \java{} API to seamless access to \sgx{} features such
as attestation and secure provisioning.
%\fixme{Bhushan, read until here.}
%a framework that help Java developers partition their code
%between security sensitive and non-security sensitive parts by adding a few lines of annotations.
%The framework automatically partitions the code into two separate binaries which are run 
%inside and outside the enclave such that the least amount of code as indicated by the
%developer annotations is run inside the enclave, and transparently creates 
%the necessary interface between the two parts. 
%% The interaction between the 2 parts of the application
%% is handled using \emph{reflection library}.
%% We perform dynamic information flow tracking 
%% and control the information flow at the endpoints of the \enclave{}.
%% By default, the framework automatically encrypts any object leaving the \enclave{}, but the
%% developer can override an endorser method to explicitly let some objects pass in the clear --- e.g.,
%% a ciphertext object. Based on information flow tracking, if an endorsed object is affected by some other
%% security sensitive data before leaving the enclave, the developer has to endorse it again.
%We use dynamic information flow tracking to enforce the policy that no information that is security sensitive or
%derived from other security sensitive information is flowing out of enclave in clear, without explicit
%developer directive.
%% Our framework provides ease of partition as well as makes it explicitly clear to the
%% developer when some information is flowing out of enclave in clear. 
%This ease of use and explicitness helps
%developer reason about the security of the code more clearly.
%\sysname{} achieves three properties: {\em security}, {\em practicality} and {\em usability}.
\sysname{} is designed to protect the confidentiality and integrity of 
real-world application code. 
\sysname{} not only requires very little developer effort to adopt, but also provides the developer
with essential insights into how data can flow through the application,
leading to better reasoning about the isolation properties of the partition.
%low developer effort,
%requires minimal porting effort,
%and provides ease of reasoning about the security strength.

%\fixmedp{I would refocus a bit on the contributions.  Make sure I don't overstate}
\sysname{} addresses design challenges at both the hardware and language level.
%one from each of hardware protection and language-level protection.
To extend hardware protection, \sysname{} contributes techniques for dynamically loading \java{} classes in enclaves
with the same code integrity verification as native, static binaries.
\sysname{} uses the Graphene library OS~\citep{tsai14graphene} to facilitate 
running a restricted \jvm{} runtime in an enclave.
%to facilitate and restrict the interface size of a 
%and prove the measurement of loaded classes as part of the hardware-generated attestation.
%Then \sysname{} verifies and attests loaded \java{} classes on the basis of measurement checking and attestation generation from the hardware.
%On the other hand, \sysname{} has to adopt language-level protection to enforce information flow policies against leakage over the border of enclave.
\sysname{} enforces information flow policies at the border of the enclave by incorporating
Phosphor~\citep{bell2014phosphor} instrumentation on all classes in the enclave,
tracing both explicit and implicit flows from any confidential data,
and preventing tainted information from leaving the enclave by any interface, 
except via explicit declassification by the developer.
%either through exported API or low-level interfaces.

%We implemented the prototype for this framework and show three case studies 
%where this framework can improve security of existing Java applications. 
%Firstly, we show how to isolate a security sensitive library like 
%Bouncycastle in the enclave, so that it does not leak the secret key 
%used for cryptography irrespective of any bugs in the other part of the code.
%Secondly, we use this framework to achieve code confidentiality
%for any closed source library implementing proprietary or secret algorithm.
%Finally, we use the enclave to protect security critical data of a Java Web Start (JavaWS) application.

%As a proof-of-the-concept, 
This work presents three use cases for \sysname{}.
First we partition and isolate the 
cryptography library and keys from the rest of an SSH client and  server.
%in which the  used for encrypting the connection
%is partitioned and isolated from the rest of the application, %in the enclave,
%protecting the generated session key while still
%using it for encryption and decryption.
Second, we present a Hadoop workload with a sensitive algorithm;
the %Partitioner%\fixmedp{what is this?},
Mapper, Combiner and Reducer classes are provisioned from a trusted server, protecting confidentiality of the sensitive code.
%always runs in an enclave.
%protecting the confidentiality of the running algorithm.
Third, we present a data manager web application---a building block for medical and other applications that handle sensitive data; 
we protect the secret client data in an enclave on the client's machine.
%\fixmedp{can we explain a bit more about the security property of interest?}
%which can serve as a web application for a website.
For each of the use cases, we show that developers or users
can utilize \sysname{} to partition the applications
with minimal effort
and only an understanding of the isolation properties of SGX,
not the cumbersome, hardware-level details.  %n{high-level understanding  knowledge of %almost no familiarity with how 
%the SGX hardware.

%\sysname{} prevents any bugs either in the untrusted application or the isolated crypto library to cause leakage of the encryption key.
%Another use case is a Hadoop job that is provisioned with a secret algorithm, and automatically encrypts computation results
%to eliminate risks of leaking the secret sauce.  

%The contributions of this work are:
%\begin{compactitem}
%\item An abstraction of hardware isolation, such as \sgx{}, for Java application developers, limiting the trusted computing base to \sysname{} and the hardware.  
%%, with minimal developer  %  to defend against attacks from system stack and hardware (only the CPU chip itself has to be trusted), with minimal porting effort.
%% hardware protection support into \java{} applications, to allow developers to fully utilize \sgx{} hardware for defending against 
%\item Techniques to guide developers in partitioning applications, using information flow analyses to help developers find the ``narrow waist'' in the application.  \sysname{} also uses information flow control to dynamically protect against information leakage %, to sanitize inputs to the enclave, 
%and
%to create a minimal enclave image to reduce the trusted computing base of the protected code.
%%partitioning Java application into enclave images with minimal supporting classes --- to minimize the attack vector in the partitioned applications ---
%%Using \java{}-based language protection and techniques, our framework 
%%and leverage information flow as hints to decide
%%whether data is allowed to be passed outside the enclave.
%%\item We provide a partition tool to generate an enclave image with minimal supporting classes for protecting the security-sensitive execution in an application.
%\item We show several practical use cases for \sysname{} that beneficially combine both hardware and language security,
%such that the sum of the protections is greater than the parts.
%% that \sgx{} enclave security can be hardened using language protection,
%%while \java{} language protection can be further shielded from
%%malicious system stack and hardware using the \sgx{} hardware.
%%\item We prove the opportunities of hardening enclave security using language protection, such as using information flow tracing to prevent leakage from the enclave.
%
%%\item We use memory safe managed language like Java to avoid memory corruption bugs
%%inside the enclave, providing better security guarantees than traditional C or C++.
%%\item We provide a framework for wide variety of Java applications to leverage \sgx{} technology.
%%\item We help developer reason about the security of the application while partitioning and offload
%%a lot of complexity during partitioning from the developer to the framework.
%\end{compactitem}

