\begin{abstract}

  Application partitioning is a technique to move sensitive data or code into
  a protected environment, such as an \sgx{} enclave.
  Partitioning larger existing applications requires subtle reasoning that can be aided with static
  analysis and dynamic, language-level techniques.
  For instance, in-enclave code may include a library that memoizes results in the out-of-enclave heap.
  Using a managed, type-safe language like Java in \sgx{} also brings the benefits of partitioning to a large body of 
  existing code. %, as well as help the developer with language-level analysis tools.
  Unfortunately, running Java partitions in \sgx{} introduces a number of technical
  challenges, many of which arise from resource and system ABI constraints.
  To our knowledge, no prior work has even run a complete Java application in an enclave,
  much less partitioned a Java application to run a piece of application logic in the enclave.
%  That said, no amount of application-level analysis can protect against
%  a compromised OS, without 
  Thus, a number of technical challenges must be addressed to unlock
  the benefit to developers of combining managed languages with hardware such as \sgx{}.


%% \fixmets{Streamlined the argument for combining Java and \sgx{}. Check if this makes sense.}
%% %%There have been significant, but separated %often mutually exclusive,
%% There have been significant, but often mutually exclusive
%% advances in hardware and language-level tools for partitioning applications
%% into multiple security contexts or privilege levels.
%% %Developers who build security-sensitive applications
%% %often face the dilemma of choosing between
%% %managed language features and hardware protections with language restrictions.
%% Developers either 
%% use a hardware solution, such as \intel{} \sgx{},
%% to isolate a piece of consolidated code %small piece of native binary code
%% from an untrusted OS, %but the developer
%% %must reason carefully about potential information flows that could leak data from the enclave, as well as other vulnerabilities in the code.
%% %On the other hand, in
%% or choose to program in a managed type-safe language, such as Java, 
%% %the developer has 
%% which has a wealth of tools and properties for 
%% improving code security and application partitioning.
%% %from type-checking to 
%% %more advanced tools to secure control and information flows.
%% The use of hardware
%% and language-level tools for partitioning are often complementary by concept but incompatible in reality.
%% %mutually exclusive,
%% %due to the limitations in combining the technologies.
%% %This paper observes that these language and hardware security tools 
%% %should complement each other, but are often mutually exclusive.
%% For instance, no language-level analysis %of an application
%% can protect an application against a compromised OS.
%% However, it is fundamentally challenging to
%% factor the protection of \sgx{} into the analysis of a managed type-safe language like \java{}.
%Running a managed language in an enclave is difficult, and language tools
%currently do not factor the hardware capabilities of \sgx{} or similar hardware into their analyses.


%Similarly, the \sgx{} hardware imposes restrictions on loading \java{} classes 
%with \jvm{}s;
%the \java{} language also lacks sensible APIs for using \sgx{} hardware.
%to wrap the semantic of interfaces with the 

%but are not sufficient to defend against a compromised OS or hypervisor.
%but the developer must partition the application 
%and harden the code inside the enclave.
%Managed languages, such as \java{},
%offer a wealth of language-level support for 
%but are not sufficient to defend against a compromised OS or hypervisor.
%The two approaches are often mutually-exclusive:
%we are unaware of any prior work thas has run a \jvm{} inside of an enclave at all,
%For \sgx{} and \java{},
%there is no prior work that has run a \jvm{} inside an enclave,
%much less integrated enclaves into the language's analysis tools.
%\fixmedp{check my claim. Too strong, or justified?}
%\fixmets{Since we are generalizing, we should only claim this is the case for sgx vs. Java. BTW, I prefer showing more certainty when making claims.}


%Application developers can place a library and sensitive data
%in an \intel{} \sgx{} enclave,
%but only if the library is a native binary, generally written in C.
%but must relies on developers to design invulnerable code.

%more advanced tools for detecting information flows.

%These language-level tools can be invaluable in ensuring a correct application partitioning
%and overall application correctness.
%Unfortunately, the most security-sensitive developers must choose between
%hardware and language security tools.


%% Using \intel{} \sgx{} enclaves, applications can
%% protect their most critical components from untrusted system stacks.
%% Language support for \sgx{} enclaves is a missing feature of existing \java{} Virtual Machines.
%% Using a managed language like \java{}
%% provides an opportunity to improve the weakest link of the enclave security---
%% application vulnerabilities that lead to information leakage.
%% \java{} language has effective framework to trace information flows, and provides type safety
%% to prevent memory access vulnerabilities in the enclave.
%% In addition, modularization in \java{} classes allows automatically partitioning
%% an application into enclave components,
%% minimizing the classes included in the enclave's trusted computing base.

This paper presents {\em \sysname{}},
a system that integrates \java{} language tools for partitioning applications
with run-time support for isolated execution in \sgx{} enclaves.
%\fixmedp{Any reason we can't do more than one partition in the enclave?}
%\fixmets{For an application, there is only one partition: the secure-sensitive part. We can create more than one instances in an enclave.}
%and which can execute a portion of the \java{} application in 
%a \java{} utility, runtime framework and API,
%that helps developers design isolated application components,
%and minimizes the attack vectors.
\sysname{} includes both \staticphase{} and \dynamicphase{} components.
The \staticphase{} tool, {\em \statictool{}}, uses code reachability analysis
to determine the minimal required \java{} classes for the isolated execution,
and helps the developer understand the complete partition interface. % between the application partitions.
%\sysname{} helps the developers generate an enclave image
%with minimal required \java{} classes for the protected code, reducing the TCB. %required for isolated execution.
%%\sysname{} includes language-level information flow control at the 
%%enclave border, protecting against unexpected exfiltration of secret
%%data due to application vulnerabilities.
The \dynamicphase{} framework, {\em \dynamicframework{}}, then launches the isolated execution of the enclave image
(a signed JAR archive) in an enclave when the untrusted part calls the interface.
Throughout the execution, \sysname{} enforces end-to-end isolation
on any object instantiated inside the enclave,
and leverages language-level information flow control to determine whether an object contains no sensitive information before the object is released from the enclave.
%to protect against unexpected ex-filtration of secret
%data. % due to application vulnerabilities.
%\fixmebj{How about this?}
%\fixmedp{Is this best-effort?  Only for explicit flows?  The 
%preceding sentence should be crisper}
%identifies information flow at the enclave border,
%preventing enclave secrets
%from being leaked due to any vulnerabilities.
%with minimal classes needed for the isolated computation,
%and narrowest interface for interaction.
%Second, the framework effectively prevents information leakage
%by forbidding any objects tainted by information flow to leave the enclave.
%In \sysname{},
%trusted servers may provision secure objects and classes,
%preserving security policy information. \fixmedp{Don't understand the previous sentence}
%\fixmedp{Do we want to say anything about contributing techniques to get good performance?}
%To secure \java{} applications, 
%With both the \staticphase{} and \dynamicphase{} language support,
\sysname{} also creates language-level abstractions of and extensions \sgx{},
such as attestation and provisioning, simplifying cumbersome aspects of this hardware
for developers.
%extends the security properties (e.g., data confidentiality) and features (e.g., attestation and provisioning)
%of \sgx{}, %hardware protection to cover \java{} classes dynamically loaded by the \jvm{},
%and encapsulates cumbersome aspects of interfacing with the hardware architecture
%keeps the work of bootstrapping and interfacing with 
%from application developers.
%As a proof-of-concept of using advanced language-level tools,
%we use \sysname{} %strengthens hardware protection
%to identify and remove 
%by filtering 
%implicit information flows that would leak secrets from an enclave
 % that leaks secrets
%in our test applications.
%\fixmedp{all of them?  Or just one?}\fixmets{I think this technique applies to all test cases.} % error-prone applications. 
%Finally, the API allows components in enclaves to be seamlessly provisioned with
%secure objects or classes by trusted servers,
%and to preserve their security policy information.
%with security guarantees extended from their sources.
%\sysname{} is evaluated with \jvmname{} on the latest 
%We evaluate \sysname{} % with several case studies 
%on the latest \intel{} Skylake processor.
%Evaluation shows good performance on
%Our overheads are low, 
%adding \fixmedp{xx}\% overhead on an isolated cryptographic library used in SSH client/server,
%and \fixmedp{xx}\% overhead on Hadoop jobs that run a secret, provisioned algorithm.
We demonstrate the usage of \sysname{} with two applications: a SSH server with an isolated cryptographic library to protect its host key,
and a Hadoop framework capable of securely executing jobs using any secret, provisioned algorithm.
%\sysname{} reduces the developer effort required to partition the application code,
%resulting in a smaller 
%and the result of integration is highly generalizable to other managed languages.
%~\fixmebj{Always replace TCB with code footprint size to indicate that we are only talking about size of code - not attack vectors.}
%\fixmets{find a better term for TCB size}
%~\fixmebj{Always replace managed languages with managed type-safe languages.}
%\fixmets{accurately separate managed language and type-safety: managed language causes the challenges, and type-safety improves the security properties; Java happens to be both.
%}
\end{abstract}
