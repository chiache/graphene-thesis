\section{Summary}

This chapter demonstrates the implementation of a \libos{}, or \thelibos{}, with a rich of Linux APIs and abstractions.
Using \thehostabi{},
\thelibos{} faithfully reproduces the behavior of a Linux kernel,
for both single-process and multi-process applications.
In each process of an application,
a \thelibos{} instance serves as an intermediate layer
between the application and the host,
to manage and allocate host abstractions for a wide range of \libos{} abstractions.
This thesis argues for the sufficiency of Linux APIs and abstractions
supported by \thelibos{},
based on the types of applications that are more likely to be ported across host platforms and the abstractions that these applications depend on.


\thelibos{} achieves three primary implementation goals.
First, the \thelibos{} implementation satisfies
several resource management models and requirement, without duplicating or virtualizing the low-level components from the host OS or hypervisor.
Although \thehostabi{} has encapsulated the host resources,
such as pages, CPUs, and I/O devices,
\thelibos{} introduces reasonable emulation and buffering to achieve the resource management model expected by the applications.
Second, \thelibos{} extends a single-process implementation
to multiple \picoprocs{},
with multiple \thelibos{} instances coordinating with each other to present a single OS view.
To maximize the flexibility of placing the \picoprocs{}
on different hosts,
\thelibos{} builds a coordination framework upon RPC messaging instead of shared memory.
Third, \thelibos{} identifies the performance overheads
caused by coordinating over RPC, 
and designs several strategies of optimizing the coordination framework
based on lessons learned during the development of \graphene{}.


This chapter shows that, with reasonable amount of engineering effort,
a \libos{} can be developed upon the previously-defined host ABI, with an extensive coverage of the Linux APIs and abstractions, and plenty of opportunities for performance optimizations.
