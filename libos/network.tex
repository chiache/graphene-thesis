\subsection{Network sockets}
\label{sec:libos:socket}

\thelibos{} supports three most common types of network sockets: TCP stream sockets, UDP datagram sockets, and UNIX-style domain sockets.
A TCP or UDP socket is bounded with a host network interface,
such as an Ethernet card or an loopback interface,
whereas a domain socket is a local IPC (inter-process communication) abstraction
similar to a FIFO (first-in-first-out) or a pipe.
This thesis argues that other types of network sockets
in Linux, such as raw packet sockets, is generally used by administrative programs.
Most networked applications, including server-side and client-side applications, tends to treat a network socket
as a contiguous I/O stream and lacks the incentive to create raw packet sockets or other types of network sockets; based on the intuition,
\thehostabi{} defines a network connection as an I/O stream, which encapsulates the composition and decomposition of network packets
and omits platform-dependent features.

The \graphene{} architecture makes the network stack strictly a component of the host OS.
%and no virtualization of host network interfaces.
The network stack inside an OS contains the implementation of various different network protocol suites or families on top of the network interfaces.
A virtualization solution generally moves or duplicates
the network stack to a guest OS,
and allows the guest OS to implement its own packet processing mechanisms,
on a physical or virtual network interface.
Several Linux network features or APIs assume
the OS owns the network stack, and thus are challenging to implement in \thelibos{}
without any expansion to \thehostabi{}.
For example,
an \syscall{ioctl} opcode, \code{FIONREAD},
returns the number of bytes currently received on a network socket, including the packets queued inside the network stack. 
A use case of \syscall{recvfrom} also allows an application
to ``peek'' into the top of a network queue
and retrieve the first few packets without draining the network queue.
Since \thelibos{} has no direct access to the network stack
inside the host,
it sometimes has to prefetch a network stream
and buffer the incoming data inside the \picoproc{}.
However, in normal cases, \thelibos{} can implement \syscall{sendto} and \syscall{recvfrom} by directly passing the user buffers to \palcall{StreamWrite} and \palcall{StreamRead} and prevents the overhead of memory copy between user buffers and \thelibos{}'s internal buffers. 




%\thelibos{} creates a TCP or UDP socket
%by calling the \hostapi{} \palcall{StreamOpen} with a URI consisting of an interface address and a port number
%(e.g., \code{127.0.0.1:8000} in IPv4 or \code{[::1]:8000} in IPv6).
%If the network connection is successfully established,
%\thelibos{} stores the PAL stream handle for future socket operations, such as \syscall{sendto} and \syscall{recvfrom}.
%A domain socket is not bounded with any network interface;
%\thelibos{} implements an unnamed domain socket (created by \syscall{socketpair})
%by sending data over a host RPC stream.









