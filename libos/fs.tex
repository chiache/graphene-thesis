\subsection{File systems}
\label{sec:libos:fs}


A Linux application depends on a list of indispensable paths
within a POSIX-style, hierarchical file system.
A POSIX-style file system is composed with directories and files, which have a common ancestor as the root directory (``/'').
A POSIX application references each file or directory in the file system
by describing the path %to traverse
from the root directory to the target. % file or directory.
For opening a file during execution,
an application either obtains a canonical or relative path
from the user interface or configuration,
or hard-codes the path in one of the application binaries.




%or reads a hard-coded path from an application binary.




%An application
%often depends on the existence of several paths,
%including paths that are conventional to a POSIX file system (e.g., ``\code{/bin}'', ``\code{/proc}'', and ``\code{/dev}'').





\issue{1.2.e}{describe POSIX file system vs NFS/object stores/other approaches}
