\subsection{File systems}
\label{sec:libos:fs}


A Linux application depends on a list of indispensable resources
within a POSIX-style, hierarchical file system.
A POSIX-style file system is composed with a number of directories and files, including a common ancestor as the root directory (``/'').
A POSIX application searches each file or directory in the file system
by describing the {\em path} %to traverse
from the root directory to the target. % file or directory.
%For opening a file during execution,
An application either obtains a canonical or relative path
from a user interface or configuration,
or hard-codes the path in one of the application binaries.
An application can heavily rely on the existence of specific paths in a file system,
such as \code{/tmp} (a default temporary directory)
and \code{/bin} (a directory for system programs),
or POSIX file sytem features,
such as directory listing and symbolic links.



\thelibos{} creates a consistent file system view containing the dependencies of an application.
A file system in \thelibos{} can be a combination of multiple 
partial file systems,
and the basic file system is a ``chroot (change root)'' file system.
A chroot file system isolates a directory in the host file system,
and maps the directory
to a custom path inside \thelibos{}.
If the application searches for a path under the mapped directory,
\thelibos{} translates the path to a path in the host file system, and supports pass-through operations on the translated path.
The host-level security isolation
should ensure that any access in a chroot file system cannot escape the corresponding directory in the host (hence the name ``chroot''). 
According to the user configuration, \thelibos{} can {\em mount} multiple chroot file systems to different paths in the root file system,
as a way of cherry-picking file resources for an application.



For example, an application can map a host-level directory ``\code{file:/foo}'' as a chroot file system in \thelibos{}, under ``\code{/bin}'' in the guest file system.
If the application search for a path called ``\code{/bin/bash}'', \thelibos{} translates the path to ``\code{file:/foo/bash}'', and redirects access of \code{/bin/bash} to \hostapis{} for accessing \code{file:/foo/bash} in the host OS.
Moreover, the host-level reference monitor enforces policies
to prevent the untrusted application
to escape the mapped directory, even if the application uses ``dot-dot'' to walk back to last level of directory; for example, \thelibos{} cannot redirect a path \code{/bin/../etc/passwd} to \code{file:/etc/passwd}, because \code{file:/etc/passwd} does not belong to the chroot file system mounted at \code{/bin}.
By creating a chroot file system, \thelibos{} creates a sandbox that disguises an unprivileged local directory (i.e., \code{/foo})
as a privileged system directory (i.e., \code{/bin}).





%An application
%often depends on the existence of several paths,
%including paths that are conventional to a POSIX file system (e.g., ``\code{/bin}'', ``\code{/proc}'', and ``\code{/dev}'').





\issue{1.2.e}{describe POSIX file system vs NFS/object stores/other approaches}
