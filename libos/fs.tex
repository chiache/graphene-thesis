\subsection{File systems}
\label{sec:libos:fs}

This section will discuss the implementation of file systems in \thelibos{}, including a pass-through, sandboxed ``chroot'' file system, the virtual file system layer for abstracting common file system operations, and other supported file system types.


\subsubsection{A ``chroot'' file system}


A Linux application depends on a list of indispensable resources
within a POSIX-style, hierarchical file system.
A POSIX-style file system is composed with a number of directories and files, including a common ancestor as the root directory (``/'').
A POSIX application searches each file or directory in the file system
by describing the {\em path} %to traverse
from the root directory to the target. % file or directory.
%For opening a file during execution,
An application either obtains a canonical or relative path
from a user interface or configuration,
or hard-codes the path in one of the application binaries.
An application can heavily rely on the existence of specific paths in a file system,
such as \code{/tmp} (a default temporary directory)
and \code{/bin} (a directory for system programs),
as well as the POSIX file system features,
such as directory listing and symbolic links.



\thelibos{} creates a consistent, guest file system view containing the dependencies of an application.
%A file system in \thelibos{} can be a combination of multiple 
%partial file systems,
A basic file system in \thelibos{} is a ``{\bf chroot (change root)}'' file system.
A chroot file system isolates a directory in the host file system,
and maps the directory
to a custom path inside \thelibos{}.
If the application searches for a path under the mapped directory,
\thelibos{} translates the path to a path in the host file system, and supports pass-through operations on the translated path.
For each application,
\thelibos{} can mount multiple chroot file systems for cherry-picking a few necessary host directories.
The host reference monitor isolates
a \thelibos{} instance within the host directories that are mounted as chroot file systems.
\thelibos{} cannot escape these host directories,
similar to a Linux program calling \syscall{chroot} before starting untrusted execution.


 
%should ensure that any access in a chroot file system cannot escape the corresponding directory in the host (hence the name ``chroot''). 
%According to the user configuration, \thelibos{} can {\em mount} multiple chroot file systems to different paths in the guest file system,
%as a way of cherry-picking host file resources for an application.



For example, \thelibos{} can mount a host directory ``\code{file:/foo}'' as a chroot file system under ``\code{/bin}'' in the guest file system.
If the application search a path called ``\code{/bin/bash}'', \thelibos{} translates the path to ``\code{file:/foo/bash}'', and redirects access of \code{/bin/bash} to \hostapis{} for accessing \code{file:/foo/bash} in the host OS.
Moreover, the host reference monitor enforces policies
to prevent the untrusted application
to escape the mapped directory, even if the application uses ``dot-dot'' to walk back to last level of directory; for example, \thelibos{} cannot redirect a path \code{/bin/../etc/passwd} to \code{file:/etc/passwd}, because \code{file:/etc/passwd} does not belong to the chroot file system mounted at \code{/bin}.
By mounting a chroot file system, \thelibos{} creates a sandbox that disguises an unprivileged local directory (i.e., \code{/foo})
as a privileged system directory (i.e., \code{/bin}) in an application.
  



\subsubsection{The virtual file system layer}







\thelibos{} includes a {\bf virtual file system} layer for abstracting the operations
of different types of file systems,
such as the chroot file system, a pseudo file system (e.g., a ``proc'' or ``devtmpfs'' file system), and a networked file system (NFS).
Similar to the virtual file system in Linux,
the virtual file of \thelibos{} defines a set of file and directory operations that
each file system must implement.
When an application opens a file,
\thelibos{} searches all the mounted file systems for the target path,
and creates a generic file handle containing pointers
to the file and directory operations implemented by the residing file system.
\fixme{not enough explanation}
A virtual file system layer decouples
the implementation of \linuxapis{} and file system operations.


As a feature of the virtual file system, \thelibos{} implements
a local file system directory cache
to reduce the \hostapis{} for retrieving directory information
or attempting to access a nonexistent path.




%An application
%often depends on the existence of several paths,
%including paths that are conventional to a POSIX file system (e.g., ``\code{/bin}'', ``\code{/proc}'', and ``\code{/dev}'')







\subsubsection{Other file systems}


Besides a chroot file system,
%A file system in \thelibos{}
\thelibos{} also supports various types of file systems,
such as a pseudo file system
(e.g., a ``proc'' or ``devtmpfs'' file system).
A pseudo file system is not linked with any physical storage devices,
but a dummy file system
purely designed for retrieving or configuring OS kernel states.
A pseudo file system can be considered as an extension of the \linuxapi{} interface,
for exporting features with more administrative purposes.
A pseudo file system usually re-purposes the operations of a file system, such as \syscall{read}, \syscall{write}, and directory listing,
for administrative-type operations,
such as retrieving a process list and individual process status.
\thelibos{} has selectively implements a set of pseudo file system entries,
based on the importance to the applications targeted by \graphene{}.



\issue{1.2.e}{describe POSIX file system vs NFS/object stores/other approaches}
