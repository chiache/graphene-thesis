\subsection{File systems}
\label{sec:libos:fs}

This section will discuss the implementation of file systems in \thelibos{}, including a pass-through, sandboxed file system, the virtual file system layer for abstracting common file system operations, and other supported file system types.


\subsubsection{A ``chroot'' file system}


A Linux application depends on a list of indispensable resources
within a hierarchical, POSIX file system.
A POSIX file system is composed of a number of directories and files, including a root directory (``/'') as the common ancestor.
A POSIX application searches each file or directory in the file system
by describing the {\em path} %to traverse
from the root directory to the target. % file or directory.
%For opening a file during execution,
An application either obtains a canonical or relative path
from a user interface or configuration,
or hard-codes the path in one of the application binaries.
An application can heavily rely on the existence of specific paths in a file system,
such as \code{/tmp} (a default temporary directory)
and \code{/bin} (a directory for system programs),
as well as the POSIX file system features,
such as directory listing and symbolic links.



\thelibos{} creates a consistent, guest file system view containing the file dependencies of an application.
%A file system in \thelibos{} can be a combination of multiple 
%partial file systems,
A basic file system type in \thelibos{} is a pass-through, sandboxed file system called a ``{\bf chroot (change root)}'' file system.
A chroot file system isolates a directory in the host file system,
and maps such directory
to a custom path inside \thelibos{}.
The mapping creates a restricted view for the application to access the files and directories inside the mounted host directory.
%If the application searches for a path under the mapped directory,
A chroot file simply replaces the prefix of each searched path
with the URI of the mounted host directory,
and redirects the file operations to the host using \thehostabi{}.
For an application, each chroot file system has cherry-picked
file resources in a host directory.
The host reference monitor ensures that
a chroot file system 
is sandboxed within the mounted host directory,
%within the host directories that are mounted as chroot file systems.
so that any \hostapis{} can only access files and directories under the host directory,
similar to a Linux program being \syscall{chroot}'ed
before running any untrusted execution.


 
%should ensure that any access in a chroot file system cannot escape the corresponding directory in the host (hence the name ``chroot''). 
%According to the user configuration, \thelibos{} can {\em mount} multiple chroot file systems to different paths in the guest file system,
%as a way of cherry-picking host file resources for an application.



For example, \thelibos{} can mount a host directory ``\code{file:/foo}'' as a chroot file system under ``\code{/bin}'' in the guest file system.
If the application search a path called ``\code{/bin/bash}'', \thelibos{} will translate the path to ``\code{file:/foo/bash}'', and redirects access of \code{/bin/bash} to \hostapis{} for accessing \code{file:/foo/bash} in the host OS.
Moreover, the host reference monitor enforces policies
to prevent the untrusted application
to escape the mapped directory, even if the application uses ``dot-dot'' to walk back to last level of directory; for example, \thelibos{} cannot redirect a path \code{/bin/../etc/passwd} to \code{file:/etc/passwd}, because \code{file:/etc/passwd} does not belong to the chroot file system mounted at \code{/bin}.
By mounting a chroot file system, \thelibos{} creates a sandbox that disguises an unprivileged local directory (i.e., \code{/foo})
as a privileged system directory (i.e., \code{/bin}) in an application.
  

The implementation of common file operations in a chroot file system is mostly as straightforward as translating to one or few \hostapis{}.
As previously stated, opening a file in a chroot file system
simply requires calling \palcall{StreamOpen} with the file URI translated from the requested path.
If the chroot file system successfully opens the file in the host,
it associates the returned PAL handle with a file descriptor, to translate common file system \linuxapis{} such as \syscall{pread} and \syscall{pwrite}
as \hostapis{} such as \palcall{StreamRead} ad \palcall{StreamWrite}, since \thehostabi{} defines these two \hostapis{} to be positionless.
For the more commonly-used \syscall{read} and \syscall{write},
%Besides any further optimization, 
the chroot file system
simply tracks the current offset of the file descriptor, and atomically retrieves and updates the offset in each \linuxapi{}.
%for implementing \syscall{read} and \syscall{write}
%are merely to store the PAL handle
%and to atomically retrieve and update the current offset inside the opened file.
The batched \syscall{readv} and \syscall{writev}
is translated to multiple \palcall{StreamRead} ad \palcall{StreamWrite} calls
on the same file.
Another two common \linuxapis{},
\syscall{stat} and \syscall{fstat}, which retrieve the metadata of a file or a directory,
need only one more step as
translating the returned host-level stream attributes
(i.e., \code{STREAM\_ATTR})
to the POSIX data structure (i.e., \code{struct stat}).


The definition of \thehostabi{} allows several opportunities of optimizing the latency of file system \linuxapi{}.
Two common techniques being broadly used in \thelibos{}
are buffering and caching.
To improve the latency of reading and writing a file,
\thelibos{} effectively buffers the content of multiple \syscall{read} and \syscall{write} \linuxapis{},
until the application calls \syscall{fsync} or the file offset exceeds the range of buffering.
Buffering file changes potentially delay the timing of writing the data
to physical disks,
but \thelibos{} accelerates the process
by making the buffer a pass-through mapping of the file (using \palcall{StreamMap}).
For an application which
performs lots of small, sequential reads or writes,
or lots of small, random reads or writes with significant spatial locality,
buffering the data
can significantly improve the performance;
evaluation shows that
running \projname{GCC} in \graphene{} to compile 0.7MLoC, on a Linux host, 
is only 1\% slower than running on Linux.
In terms of caching, \thelibos{} contains a file system directory cache
in the virtual file system,
which will aggressively cache any directory information
retrieved from \thehostabi{}.
The file system directory cache of \thelibos{} also benefits other file systems,
and the details will be further discussed in Section~\ref{sec:libos:vfs}.



A chroot file system enforces container-style sandboxing of an application,
but simultaneously
allows sharing part of the file system tree
with other applications and \picoprocs{}.
Since \thelibos{} supports mounting multiple chroot file systems in a \picoproc{},
\graphene{} users can configure a host to selectively export a few host directories containing the file resources in use.
The security isolation of a single chroot file system %enforces the security isolation
is similar to the sandboxing of a Linux container~\cite{lxc}, which restricts all the file operations of an application within a local file system tree unless the container is running on a stackable file system~\cite{aufs}.
\graphene{} allows multiple applications to share a host directory, either read-only or with full access,
and uses a host reference monitor
to enforce AppArmor~\cite{apparmor}-like, white-listed rules
for isolating every file access.
\graphene{} can share most of the system files and binaries,
such as \code{/etc/hosts} and \code{/bin/bash},
without compromising the security isolation of each application.





\subsubsection{Guest-level file systems}
%A file system in \thelibos{}


\issuedone{1.2.e}{describe POSIX file system vs NFS/other approaches}
Other than a pass-through file system,
\thelibos{} can use a different approach as 
implementing the file operations
in a guest-level file system. % which internally implements all file operations.
A guest-level file system does not expose any host files and directories
to applications;
instead, a guest-level file system maintains its own file system states either in memory or in a raw format unknown by the host OS.
A guest-level file system
provides a different option for managing file resources in \thelibos{}. 
Using a guest-level file system,
\thelibos{} potentially has more control over assigning physical resources to each file or directory.


One example of a guest-level file system is a pseudo file system,
including the \code{proc}, \code{sys}, and \code{dev} file systems in Linux and similar OSes.
A pseudo file system %are common features of Linux and similar OSes,
exports an extended system interface for accessing kernel states or raw devices.
%Take a proc file system for example.
An application can use the \code{proc} and \code{sys} file systems %(normally mounted at \code{/proc} and \code{/sys}) 
to obtain information about processes
as well as the whole kernel. 
The \code{proc} and \code{sys} file system have both redefined the common file operations such as \syscall{read}, \syscall{write}, and \syscall{readdir}, for ad-hoc operations of accessing different types of process or kernel states.
%Different from the proc file system, 
On the other hand, the \code{dev} file system %(normally mounted at \code{/dev})
exports both raw, physical devices and dummy, miscellaneous devices to an application.
Examples of miscellaneous devices include \code{/dev/zero}, which outputs an infinite zero stream, and \code{/dev/urandom}, which outputs software-generated, pseudo-random bits.
\thelibos{} has implemented several critical entries of the \code{proc}, \code{sys}, and \code{dev} file systems,
according to the command-line workloads targeted by \graphene{}. 



Anther guest-level file system implemented in \thelibos{}
is a networked file system (NFS), which connects to a NFSv3 server running on either the local host or a remote machine.
A networked file system provides another solution (besides a chroot file system)
for \thelibos{} to share file resources among applications or \picoprocs{}, by relying on a centralized NFS server to multiplex the file resources. 
A benefit of using a networked file system in \thelibos{}
is the natural support of a complete set of POSIX file system features.
A networked file system does not depend on any local file resources, so all the file system features are implemented over a network connection.
Therefore, the implementation of a networked file system is not restricted by
the \hostapis{} defined for file access.
However, 
the overhead of networking \hostapis{}
can have significant impact on
the performance of a networked file system in \thelibos{},
which can be
much slower than an application-level NFS client on Linux (using \code{libnfs}).


%and the client side, which runs in an \thelibos{} instance,
%only depends on a host-level network connection.



Other guest-level file systems can potentially introduce a pre-formatted virtual disk into \thelibos{}.
Several popular file system formats,
including EXT2, FAT, and NTFS, have been supported in either an application-level library (e.g., \code{libext2fs}) or a FUSE (Filesystem in userspace) driver (e.g., \code{NTFS-3G}).
\thelibos{} can potentially modify
these libraries or FUSE drivers as guest-level file system drivers,
to decompose a virtual disk.
The drawback of using a pre-formated virtual disk
is the difficulty of coordinating multiple \picoprocs{} that simultaneously access the same virtual disk.
Since each single write to a file can involves writing to multiple physical blocks (the superblock, inodes, and data blocks),
a guest-level file system driver must consistently coordinate multiple \thelibos{} instances in order to share a virtual disk.
Therefore, without inter-process coordination or a fully-distributed design, the usage of a pre-formated virtual disk in \thelibos{}
is most likely to be restricted to a single-process application.




\subsubsection{A virtual file system}
\label{sec:libos:vfs}


A POSIX file system defines a set of generic file system operations and primitives, making the underlying file system implementations
transparent to most applications.
An administrator can mount a file system at an arbitrary directory, to select among different solutions of managing file resources.
When an application successfully opens a file
under the mount point of a file system,
a generic file descriptor is returned to represent the opened file resources
and is fully independent from the underlying implementations.
By presenting different resources as a generic primitive
as file descriptors,
an application can consistently use identical \linuxapis{},
such as \syscall{read} and \syscall{write}, to invoke file and directory operations
defined by the file system drivers.
In a POSIX file system, an application can mostly be reused
upon different file systems, as long as the required file resources are available in the chosen file systems.



%A POSIX file system defines a set of file and directory operations,
%which are independent from
%the underlying file system implementations.
%The file and directory operations in a POSIX file system
%is based on a few generic primitives,
%such as file descriptors and directory entries.
%Whenever an application accesses a file using a \linuxapi{} such as \syscall{read} or \syscall{write},
%the arguments given to the \linuxapi{}, including a numeric file descriptor, are irrelevant with the type of file system
%where the file is located.
%For example, a user can choose to mount a NTFS partition
%or a NFS share as a root file system;
%yet, the \linuxapis{} for accessing any files on the file system
%will mostly not differ upon different mounts.




%Similar to a Linux kernel, \thelibos{} includes a {\bf virtual file system} layer
%to abstract the file and directory operations of different file system implementations .



%such as the chroot file system, a pseudo file system (e.g., a ``proc'' or ``devtmpfs'' file system), and a networked file system (NFS).
%Similar to the virtual file system in Linux,
%the virtual file of \thelibos{} defines a set of file and directory operations that
%each file system must implement.
%When an application opens a file,
%\thelibos{} searches all the mounted file systems for the target path,
%and creates a generic file handle containing pointers
%to the file and directory operations implemented by the residing file system.
%\fixme{not enough explanation}
%The virtual file system layer has decoupled
%the implementation of \linuxapis{} and file system operations in \thelibos{}.



Similar to a Linux kernel, \thelibos{} includes an abstraction layer for exporting the generic operations and primitives of a POSIX file system,
generally known as {\bf a virtual file system}.
A virtual file system defines a set of file and directory operations as the shared interface of every file system implementation.
When an application opens or queries a target file resource,
the virtual file system
searches the file path among all mounted points,
and then invokes the corresponding file or directory operations implemented by the file system.
In \thelibos{}, each file system, such as a chroot file system
or a pseudo file system,
must provide a data structure to the virtual file system, containing function pointers
referencing to all the file and directory operations  
implemented by the file system.


%As a feature of the virtual file system, \thelibos{} implements
%a local file system directory cache
%to reduce the \hostapis{} for retrieving directory information
%or attempting to access a nonexistent path.




%An application
%often depends on the existence of several paths,
%including paths that are conventional to a POSIX file system (e.g., ``\code{/bin}'', ``\code{/proc}'', and ``\code{/dev}'')


The virtual file system in \thelibos{} enables several file system features
and optimizations
which indistinguishably benefit every file system.
A few atypical file operations, such as the batched \syscall{readv} and \syscall{writev},
can be emulated in the virtual file system
using file operations exported by the underlying file system.
More importantly, 
the virtual file system in \thelibos{} 
includes a local {\bf directory cache},
as an optimization to the latency of searching a path in the guest file system tree,
and retrieving file metadata.
A directory cache
is designed to reduce the frequency
of executing the file operation of walking the file system data structures,
by aggressively caching any directory information
and file metadata
returned from a file system implementation.


The directory cache in \thelibos{} stores each searched path and its parent directories as directory entries
using the spared \picoproc{} memory.
The directory cache in \thelibos{} has a similar architecture
as the Linux file system directory cache.
Each directory entry in the directory cache records the existence or nonexistence of a file system path,
as well as the file attributes
(e.g., file types, sizes, and permissions).
If an application has given a path as a \linuxapi{} argument,
\thelibos{} first looks up
in the directory cache to find any directory entries
that matches with the given path.
If a directory entry is already created for a path, \thelibos{} can bypass the file system operations of querying the paths
in the host file systems or other storage media.


The directory cache in \thelibos{} adopts several techniques for optimizing the latency of \linuxapis{}.
First, the directory cache
can confirm the existence of every parent directory
of a canonical path
found in the whole guest file system.
For example, if \thelibos{} successfully opens a file at \code{/home/foo/bar},
\thelibos{} knows for sure that both \code{/home}
and \code{/home/foo} exist in the guest file system tree,
without doing any extra file system lookups.
%like \palcall{StreamAttributesQuery}. 
By reducing the amount of file system lookups for searching each path
from multiple to exactly one,
the directory cache reduces the overhead of a cache miss,
especially since each directory cache
only caches directory information for one single application.
Second, to look up a path inside the directory cache,
\thelibos{} can use a fast-path algorithm which searches the whole canonical path
instead of each path prefixes~\cite{tsai15dcache}.
For example, in order to search \code{/home/foo/bar} in the directory cache,
a traditional design will have to search
\code{/home}, \code{/home/foo}, and then \code{/home/foo/bar}.
Using the fast-path algorithm,
\thelibos{} can directly search the whole path without walking through each parent directories.
Finally, the directory cache aggressively stores {\em negative} directory entries
for the paths known to be non-existent.
Whenever an application has unlinked or moved a file, \thelibos{} caches the information that the original path no longer exists,
by creating a negative directory entry.





