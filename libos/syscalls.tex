\section{Single-Process API}



\issue{1.1.a}{extend the technical sections}
%\paragraph{Implementing Linux Personality.} 
%\fixmedp{Revisit the logical flow of these paragraphs}
The \graphene{} {\tt libLinux.so} implements a subset 
of the Linux system call API (currently \graphenesyscallnum{} calls)
using only the \pal{} ABI to interact with the host.
We note that Linux exports a very long tail of infrequently used calls.
%applications.
A rough analysis of this tail indicates roughly 100 additional calls that can be implemented
with the existing \pal{} ABI and coordination framework, less than 10 administrative calls that will not make sense to expose to 
an application, such as loading a kernel module or rebooting the system, and roughly 54 that will require 
\pal{} extensions to meaningfully implement, such as controlling scheduling,
NUMA placement, I/O privilege, and shared memory.
In the last category of system calls, the degree to which actual host details should be exported versus emulated is debatable.

%We believe represent the most commonly used system calls.
%When an application requests a call or argument that {\tt libLinux.so} does not implement,
%the picoprocess exits with a distinct error message. 
Each time we have tested \graphene{} with a new application, the number of extra system calls
required has dropped---most recently we only added 4 calls
(namely, epoll\_create, epoll\_wait, semget and semop)
to support the Apache web server.
Thus, we believe \graphene{} implements a representative sample of Linux calls.

%such as {\tt sched\_setparam}, which manipulates scheduler-specific
%parameters or 
%{\tt uselib}, which has been abandoned 
%in {\tt glibc} version 2 in favor of a user-space dynamic linker.
%We do not plan to implement administrative interfaces, such as {\tt reboot}.
%The growth in the set of supported system calls has been driven by 
%the requirements of new applications we use to exercise \graphene{}, and has been 
%slowing considerably over time.

%directly to guests, and thus will not implement them in {\tt libLinux.so}.

% dp: :(
\begin{comment}
Most {\tt libLinux.so} code reimplements
Linux kernel functionality.  We found it expedient to 
read the Linux source in order to understand its behavior and then reimplement 
that behavior on the \pal{} ABI in most cases.
In some cases, such as the file caching code,
%directory entry (dentry) cache, 
we refactored code directly from the Linux kernel.
%% In these cases, we simplified data structures to only include data
%% we needed for a single application, and to hook in with other 
%% {\tt libLinux} subsystems.
%% An interesting direction for future work would identify techniques
%% to automatically import larger swaths of Linux kernel code, facilitating
%% adoption of new features and bugfixes.
\end{comment}

In order to use {\tt libLinux.so}, we modified \libclines{} lines of {\tt glibc} to replace 
system instructions with function calls into {\tt lib\-Linux.so},
and to cooperatively manage thread-local storage with {\tt libLinux.so} (Table~\ref{tab:graphene:loc}).

%All totaled, we only needed to modify \libclines{} lines of glibc source code to redirect all 
%system calls to {\tt libLinux} 

\begin{comment}
\vspace{5pt}
\noindent{\bf ABI Extensions.~}
\graphene{} extends the Drawbridge ABI with 9 additional \pal{} calls.
As discussed above, one creates a new sandbox, and 
5 additional calls were added for IPC.
We also add 3 calls to manage x86 segmentation registers
and exceptions (Bascule~\cite{baumann13bascule} adds
similar extensions).
\end{comment}
