\subsection{Virtual address space}
\label{sec:libos:vma}


\fixme{start with what applications need}
A Linux application expects a contiguous, large virtual address space,
to allocate a number of numerically-addressable memory regions.
A program usually uses a \libc{} allocator, requested by \funcname{malloc} and \funcname{calloc},
or a heap allocator of a managed language runtime,
or reserves space on the current stack,
to allocate fine-grained memory objects.
To support application-level allocation,
an OS is responsible of
maintaining a unique, consistent mapping between virtual memory areas (VMAs) and physical pages,
and managing the virtual address space layout
to prevent collision of VMAs.
The Linux kernel, specifically, provides several ways of memory allocation, such as allocation by \syscall{mmap} and \syscall{brk},
and transparently growing a user stack downward. % exceeding the stack boundary.
A application-level allocator may try several ways of requesting memory resources;
for example, the \glibc{} allocator
uses both \syscall{brk} and \syscall{mmap} to allocate different sizes of memory objects.
Applications depend on different memory allocation mechanisms
of a Linux kernel,
to dynamically allocate space for storing application data.
 



\thelibos{} manages the virtual address space of each \picoproc{}.
To emulate a Linux kernel,
\thelibos{} creates VMAs using two \hostapis{}:
\palcall{VirtMemAlloc} for creating an anonymous memory mapping,
and \palcall{StreamMap} for mapping a file into the virtual address space.
%or maps a file into a new VMA using \palcall{StreamMap}.
Both \hostapis{} creates a page-aligned, fixed range in the
virtual memory space,
with the assumption that the host OS or hypervisor
will assign a physical page to each virtual page being accessed,
and fill the physical page with file content or zeros.
\thelibos{} does not assume
a host to always implement demand paging.
The only assumption that \thelibos{} makes, when \palcall{VirtMemAlloc} or \palcall{StreamMap} returns successfully,
is that 
the the application or 
\thelibos{} is authorized to access any part of the created VMA,
without causing a segmentation fault
or memory protection fault.
It is possible that a host may have statically assign
physical pages to the whole VMA instead of gradually increasing the memory usage.

%each physical page of an allocated VMA
%will be assigned
%before any access to the page, including reading or writing data,
%or executing code.
%In other word, any future, authorized memory access
%in an allocated VMA
%should never cause any segmentation or memory protection faults.



\thelibos{} creates VMAs for two reasons.
First, \thelibos{} allocates memory regions on applications' request.
\thelibos{} also allocates memory for internal usages,
such as maintaining
the bookkeeping of OS states,
and reserving space for buffering and caching.
\thelibos{} contains a {\em slab allocator} (for internal \funcname{malloc}) and several object-caching memory allocators.
For each abstraction, \thelibos{} allocates a handle (e.g., a thread handle)
using internal allocation functions.
Therefore, the memory overhead of \thelibos{} %, regardless of the memory footprint of application itself,
is primarily caused by allocating various types of handles for maintaining or caching OS states,
and is roughly correlated with
the abstractions used by the application. 



%For example, \thelibos{} allocates a thread control block (TCB), or thread handle, for each user thread that an application creates, to store thread-specific attributes
%such as a thread identifier and a given stack address.
%To reduce the memory cost of \graphene{},
%\thelibos{} tries to reduce the VMA allocation for its internal usage
%as long as it does not add significant performance overhead to an application.
%Another usage of a VMA
%is to assign the VMA to a memory area that an application
%assumes to be accessible, such as a heap area created by \syscall{mmap} or a self-growing stack.
%If \thelibos{} allocates a VMA for supporting the memory access in an application,
%the VMA must be at least as large as the size that the application has requested.



\thelibos{} maintains a list of VMAs allocated by either the application
or \thelibos{} itself.
For each VMA, \thelibos{} records
the starting address, size, and the page protection (readable, writable, or executable).
The VMA list traces the free space within the current virtual address space.
When an application allocates a VMA,
\thelibos{} queries the VMA list to search for a sufficient space.
In another case, an application may specify the mapping address,
and the VMA list can determine whether the address has overlapped with an existing VMA, to prevent corrupting the internal states of \thelibos{}.
%Otherwise, \thelibos{} walks the VMA list to find a large enough free space
%to allocate a VMA. 
%In terms of page management,
%the role of \thelibos{} is to keep track of memory addresses that already belong to a VMA.
%In many cases,
%an application or \thelibos{}
%simply needs to allocate a new memory region that has not overlapped
%with existing memory regions.
%If an application gives a memory address
%as an argument to a \linuxapi{},
%\thelibos{} needs verifying the validity of address by checking against existing memory regions. 
%As a result,
%\thelibos{} maintains a list of currently-allocated VMAs,
%and dynamically updates the list whenever allocating, deallocating, or protecting any memory mappings.
%For each VMA, \thelibos{} records
%the starting address, size, protection mode (whether the VMA is readable, writable, or executable), and usage of the VMA.
%By maintaining a list of allocated VMAs,
%%Instead of managing physical pages, 
%\thelibos{} controls the virtual address space layout
%of a guest environment (i.e., a \picoproc{}).
%\thelibos{} keeps track of the free addresses in the current virtual address space,
%to prevent allocating overlapping VMAs.
%To allocate a VMA without a specific address,
%\thelibos{} first walks the VMA list
%to discover an unallocated, large enough address range.
According to the new VMA, \thelibos{} uses
%according to the type of mapping,
\palcall{VirtMemAlloc} or \palcall{StreamMap} 
to create the mapping in the host OS.
The VMA list also contains 
%the states of a
%virtual address space,
the mappings of PAL and the \thelibos{} binary.
%according to information given in the PAL control block.

%with the discovered address,
%to create a memory mapping suitable for the expected usage.
%Therefore, \thelibos{} can control the address
%for allocating a new VMA,
%to service a \syscall{mmap} \linuxapi{},
%or to extend the internal slab allocator of \thelibos{}.
%The bookkeeping of VMAs also includes
%the VMAs preserved by PAL and the VMAs for the \thelibos{} binary mapping,
%to prevent future VMAs
%corrupting the internal states of PAL or \thelibos{}.
%\thelibos{} records these VMAs at the beginning of a \picoproc{}, based on address ranges specified by the PAL control block.


Whenever an application or \glibc{} invokes a \linuxapi{} like \syscall{mmap}, \syscall{mprotect}, or \syscall{munmap}, \thelibos{} updates the VMA list
to reflect the virtual address space layout created by the host.
The basic design of a VMA list is a sorted, double-linked list of unique address ranges.
Because Linux allows arbitrary allocation, protection, and deallocation at page granularity, \thelibos{} often has to shrink or divide a VMA
into smaller regions.
\thelibos{} tries to synchronize
the virtual address space layout with the host OS, by tracing the results of every memory-related \hostapis{}.


%To maintain a VMA list,
%\thelibos{} needs to dynamically modify, shrink, or divide VMAs, to support all corner cases
%of memory allocation.
%Three key \linuxapis{} in Linux---\syscall{mmap}, \syscall{mprotect}, and \syscall{munmap}---permit allocation, protection, and deallocation of pages
%at arbitrary page-aligned addresses.
%If a \syscall{mmap}, \syscall{mprotect}, or \syscall{munmap} \linuxapi{}
%partially frees or modifies a VMA,
%\thelibos{} divides the VMA bookkeeping into two, and then destroys of rewrites one of the VMAs.
%\thelibos{} is responsible of synchronizing the VMA lists of the host OS and \thelibos{}.


%Managing application and internal VMAs in the same virtual address space
%poses a challenge of isolation in \thelibos{}.



Different from a Linux kernel, \thelibos{} does not isolate its internal states from the application data.
\thelibos{} shares a virtual address space with the application,
and allows internal VMAs to interleave with memory mappings created by the application.
In this design, an application does not have to context-switch into another virtual address space to enter \thelibos{}.
A consequence of the design is the possibility that an application will corrupt the states of \thelibos{}, either accidentally or intentionally, by simply writing to arbitrary memory addresses.
The threat model of \graphene{} does not assume
\thelibos{} to defend against applications because both \thelibos{} and applications are untrusted by the host kernel. 

%allows internal VMAs to interleave with the VMAs allocated by the application.
%Despite that \thelibos{} cannot relocates VMAs
%for defragmenting the whole virtual address space,
%\thelibos{} can internally recycle fragmented space for buffering or expanding a slab allocator.
%Interleaving different types of VMAs
%helps filling up the fragmented free space
%with smaller objects or buffers.
%Moreover, there are opportunities to recycle or consolidate the internal heap of \thelibos{},
%once \thelibos{} detects pressure on utilizing the virtual address space.
%To sum up, \thelibos{} can recycle the holes between VMAs allocated by the application, and utilize the space for internal buffering or bookkeeping.



%Interleaving application and internal mappings
%potentially reduces the {\bf external fragmentation} in a virtual address space. % due to arbitrary memory allocation and deallocation.
%External fragmentation happens
%when an application deallocates smaller VMAs and creates holes in the virtual address space where larger memory mappings cannot fit in.
%Having holes in a virtual address space
%is normally acceptable on a Linux kernel;
%A \graphenearch{} Linux kernel sets the virtual address space of a process to be
%as large as 256 terabytes (${2}^{48}$ bytes),
%which is unlikely to wear out due to external fragmentation.
%However, upon a host where \thelibos{} can be given a small virtual address space,
%an application may eventually run out of virtual address space
%despite of the holes created by arbitrary allocation and deallocation.
%For example, a SGX enclave is always restricted within a specific region, which is given as a configuration signed off by the developers.
%To run an application in a SGX enclave, \thelibos{} must fit both application and internal states into a restricted enclave region.
%Therefore, \thelibos{} can recycle the holes between VMAs
%for storing internal OS states, which can be separated and fit into smaller regions.




%if the virtual address space is enormously large and the host OS can swap out physical pages.
%\thelibos{} can simply ignore the address space holes and
%allocate new VMAs at higher or lower addresses. 
%The assumption is problematic on a host where the virtual address space of
%a guest is a limited resource.
%For example, a SGX enclave has a limited virtual address space in the enclave, constrained by the enclave initialization.
%As a result, external fragmentation in the virtual address space
%can potentially cause a significant waste of resources on a specific host like SGX.






%Even if a address space hole
%is much smaller than a normal internal VMA, \thelibos{} can still repurpose the space
%to allocating a small amount of internal object.
%Unlike application VMAs, most internal VMAs can be recycled or relocated if \thelibos{} can trace back pointers to these VMAs.
%Because \thelibos{} interleaves different types of VMAs,
%there is opportunities for \thelibos{} to consolidate 





\paragraph{Implementing \syscall{brk}.}

%Take \syscall{brk} for example.
\syscall{brk} is a Linux \linuxapi{} for 
allocating memory space at the ``program break'', which defines the end of the executable's data segment.
What \syscall{brk} manages is a contiguous ``brk region'', which can be grown or shrunk by an application.
Unlike \syscall{mmap}, \syscall{brk} allocates arbitrary-size memory regions, by simply moving the program break
and returning the address to the application.
% of arbitrary sizes.
The primary use of \syscall{brk} in applications is %as an efficient way of
to allocate small, unaligned memory objects,
as a simple way of implementing \funcname{malloc}-like behaviors.


%Most applications 
%use \syscall{brk} 
%for its speed of allocating 
%small objects by moving the top of heap by a small offset.
%Different from \syscall{mmap},
%\syscall{brk} only allocates a physical page
%when moving the top of heap across page boundaries.
%%to allow gradual allocation of application objects.
%Some applications, such as \gcc{}, can bypass \syscall{brk} by switching to \syscall{mmap}.
%However, to support other applications that depends on \syscall{brk},
%\thelibos{} internally implements the fine-grained heap
%based on VMAs.



\thelibos{} implements \syscall{brk} by dedicating a part of the virtual address space for the brk region.
During the initialization, \thelibos{} reserves an unpopulated memory space
behind the executable's data segment, using \palcall{VirtMemAlloc}.
The size reserved for the brk region is determined
by user configurations.
%preallocates a \code{brk} area for future \syscall{brk} \linuxapis{}.
%The \code{brk} area has a limited, configurable capacity,
%and maintains a \code{brk} pointer to the top of heap currently assigned by the application.
\thelibos{} adjusts the end of the brk region
within the reserved space
whenever the application calls \syscall{brk}, or \funcname{sbrk}, a \libc{} function which internally calls \syscall{brk}.
\thelibos{} reserves the space for \syscall{brk}
to guarantee certain amount of memory resources for all the \syscall{brk} calls,
until the whole \picoproc{} is under memory pressure.



%the \code{brk} pointer
%and returns the latest top of heap.
%A \syscall{brk} call cannot move the \code{brk} pointer beyond the capacity of the \code{brk} area,
%or it will return \code{-ENOMEM} to the application.


\paragraph{Address Space Layout Randomization (ASLR).}

\thelibos{} implements Address Space Layout Randomization (ASLR) as a \libos{} feature.
Linux randomizes the address space layout to defeat or at least delay a remote memory attack, such as
a buffer overflow or a ROP (return-oriented programming) attack.
A remote memory attack
%launched by a user or a remote client
often depends on certain level of knowledge about the virtual address space layout of an application.
For example, in order to launch an effective buffer overflow,
an attacker tries to corrupt an on-stack pointer to make it points to security-sensitive data.
With ASLR, a Linux kernel increases the unpredictability of memory mappings,
so that a remote attacker is harder
to pinpoint a memory target.
%to launch an effective buffer overflow or ROP (return-oriented programming) attack in an application.
To support ASLR,
%Although some host OSes may already support ASLR, \thelibos{} enforces another layer of randomization.
\thelibos{} adds a random factor to the procedure of determining the addresses for allocating new VMAs.
%function that searches for free regions in the virtual address space.
%The randomization will cause \syscall{mmap} to return an unpredictable address, if the application does not specify the address.
\thelibos{} randomizes the results of both \syscall{mmap} and \syscall{brk};
for \syscall{brk}, \thelibos{} creates a random gap (up to 32MB) between the data segment and the brk region.






