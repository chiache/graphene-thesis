\subsection{Virtual address space}
\label{sec:libos:vma}


Instead of directly managing physical pages, \thelibos{} manages the virtual address space of the current process.
\thelibos{} divides the virtual address apace into VMAs (virtual memory areas),
either serving a VMA to a memory-allocating \linuxapi{} such as \syscall{mmap},
or keeping a VMA 
for internal usage of \thelibos{}.


%There are two primary reasons
%that \thelibos{} needs to allocate memory.
%The first reason is to service the requests from the applications and user libraries,
%through system APIs such as \syscall{mmap}.
%The second reason
%is to maintain the internal states
%of the \libos{},
%for either implementing the Linux features and APIs, or managing the allocated host abstractions.



To allocate memory, \thelibos{} creates VMAs (virtual memory areas)
in the host OS.



\thelibos{} is in control of the whole virtual address space of a \picoproc{},
except the initial PAL mappings.








Internally, \thelibos{} maintains a list of VMA records, for managing the virtual address space.









\paragraph{Address Space Layout Randomization (ASLR).}

