\subsection{Virtual address space}
\label{sec:libos:vma}


\thelibos{} allocates memory resources by creating VMAs (virtual memory areas).
\Thehostabi{} defines a VMA
as the basic abstraction for allocating memory pages
in a host;
\thelibos{} either creates an anonymous VMA using a \hostapi{}, \palcall{VirtMemAlloc},
or maps a host file into a VMA using another \hostapi{}, \palcall{StreamMap}.
Both \hostapis{} creates a page-aligned, fixed-size range in the
current virtual memory space,
assuming that each virtual page in the range
will map to an individual physical page, either zeroed or filled with file content.
\thelibos{} does not assume
that the host will always implement demand paging
for every allocated VMAs.
The only assumption that \thelibos{} makes, when using \palcall{VirtMemAlloc} or \palcall{StreamMap},
is that each physical page of an allocated VMA
will be assigned
before any access to the page, including reading or writing data,
or executing code.
In other word, any future, authorized memory access
in an allocated VMA
should never cause any segmentation or memory protection faults.



\thelibos{} allocates VMAs for two usages.
One usage is to allocate pages for internal use of \thelibos{},
including mapping the \thelibos{} binary
(i.e., \thelibos\code{.so})
into the guest environment for execution,
and storing the internal OS states.
For example, \thelibos{} allocates a thread control block (TCB), or thread handle, for each user thread that an application creates, to store thread-specific attributes
such as a thread identifier and a given stack address.
To reduce the memory cost of \graphene{},
\thelibos{} tries to reduce the VMA allocation for its internal usage
as long as it does not add significant performance overhead to an application.
Another usage of a VMA
is to assign the VMA to a memory area that an application
assumes to be accessible, such as a heap area created by \syscall{mmap} or a self-growing stack.
If \thelibos{} allocates a VMA for supporting the memory access in an application,
the VMA must be at least as large as the size that the application has requested.



In terms of page management,
the role of \thelibos{} is to keep track of memory addresses that already belong to a VMA.
In many cases,
an application or \thelibos{}
simply needs to allocate a new memory region that has not overlapped
with existing memory regions.
If an application gives a memory address
as an argument to a \linuxapi{},
\thelibos{} needs verifying the validity of address by checking against existing memory regions. 
As a result,
\thelibos{} maintains a list of currently-allocated VMAs,
and dynamically updates the list whenever allocating, deallocating, or protecting any memory mappings.
For each VMA, \thelibos{} records
the starting address, size, protection mode (whether the VMA is readable, writable, or executable), and usage of the VMA.




By maintaining a list of allocated VMAs,
%Instead of managing physical pages, 
\thelibos{} controls the virtual address space layout
of a guest environment (i.e., a \picoproc{}).
\thelibos{} keeps track of the free addresses in the current virtual address space,
to prevent allocating overlapping VMAs.
If \thelibos{} needs to allocate a VMA without a specific address,
it walks the list of VMAs
to discover an unallocated address range.
Then, \thelibos{} can use \palcall{VirtMemAlloc} or \palcall{StreamMap} with the discovered address,
to create a memory mapping suitable for the expected usage.
Therefore, \thelibos{} can control the address
for allocating a new VMA,
to service a \syscall{mmap} \linuxapi{},
or to extend the internal slab allocator of \thelibos{}.
The bookkeeping of VMAs also includes
the VMAs preserved by PAL and the VMAs for the \thelibos{} binary mapping,
to prevent future VMAs
corrupting the internal states of PAL or \thelibos{}.
\thelibos{} records these VMAs at the beginning of a \picoproc{}, based on address ranges specified by the PAL control block.





\paragraph{Implementing \syscall{brk}.}





\paragraph{Address Space Layout Randomization (ASLR).}

