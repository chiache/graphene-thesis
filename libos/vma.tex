\subsection{Virtual address space}
\label{sec:libos:vma}

\thelibos{} allocates memory resources
for two reasons.
The first reason is to service the requests from the applications and user libraries,
through 
system APIs such as \syscall{mmap}.
The second reason
is to maintain the internal states
of the \libos{},
for either implementing the Linux features and APIs, or managing the allocated host abstractions.



To allocate memory, \thelibos{} creates VMAs (virtual memory areas)
in the host OS.



\thelibos{} is in control of the whole virtual address space of a \picoproc{},
except the initial PAL mappings.








Internally, \thelibos{} maintains a list of VMA records, for managing the virtual address space.





\paragraph{VMA bookkeeping.}




\paragraph{Address Space Layout Randomization (ASLR).}

