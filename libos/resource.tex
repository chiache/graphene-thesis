\section{Resource Management}
\label{sec:libos:resource}


\Thelibos{}
depends on a host OS or hypervisor to manage hardware and privileged OS resources.
\Thehostabi{}
defines the abstractions managed by a host---from an I/O stream, a virtual memory area (VMAs), to a thread. %---for the development of a \libos{}.
%that are available for its guest.
These abstractions encapsulate the ubiquitously-installed hardware resources.
%, such as I/O devices, memory, and CPUs,
%to the host OSes.
Other host abstractions, %defined
%by \thehostabi{},
such as a local RPC stream and a system clock,
represents the low-level, privileged resources of a host OS.
%such as in-kernel queues and time sources.
To run an application with \thelibos{} as a guest,
\graphene{} is able to drop the assumption of exporting or virtualizing %management of
any low-level resources,
and program \thelibos{} with the simplified, easy-to-implement host abstractions of \thehostabi{}.
%to program these resources.

%hardware and privileged OS resources,
%by encapsulating the resources as generic, user-space abstractions.
%of the host ABI.



\issuedone{1.2.a}{discuss the role of libOS in resource management}
The role of \thelibos{} in resource management
is to allocate the host abstractions using \thehostabi{},
as unambiguous requests %to the host
for host-managed resources.
At a high level, the purpose of \thelibos{} is to recreate the Linux abstractions.
\Thelibos{} implements the Linux abstractions
based on managing the host abstractions instead of
the underlying resources.
For example, if a Linux abstraction requires allocating pages
for either usage in an application or
internal bookkeeping,
the implementation in
\thelibos{} will allocate a VMA, which is the memory abstraction of \thehostabi{}, instead of physical pages.
%in a unified, guest virtual address space.
Such a \libos{} design operates on the faith that the host OS will manage and assign pages to VMAs, with reasonable fairness as well as efficiency.
Unless the allocation exceeds user quotas or host limitations,
the \libos{} should be allowed to obtain more host-manged resources,
by increasing the allocation of a host abstraction.



A language runtime, such as a Java virtual machine~\cite{hotspot,j9,alpern2000jalapeno}, or a Python~\cite{python} or Perl~\cite{perl} runtime,
%also allocates OS-managed abstractions,
has a similar role as \thelibos{}.
%The role of \thelibos{} in resource management is close to 
%A language runtime generally relies on system APIs exported by the OSes
%for resource management.
A language runtime commonly uses the existing system interface
to request for resources needed by an application.
For example, a language runtime may use the \syscall{mmap} linuxapi{} to allocate a large heap,
to assign chunks of the heap to an application.
%\linuxapis{} like \syscall{mmap} to allocate a large heap,
%which is chunked into objects
%and assigned to variables in an application.
%Similar strategies have been applied to
%filesystem or threading abstractions in a language runtime. %, which %are likely to
%usually leverage the filesystem or threading APIs of the OSes.
%By managing the OS abstractions,
%exported by the OSes,
%the language runtime creates an independent view
%of system resources for applications.
%Such a design resonates with \thelibos{}. 
Therefore, the development of \thelibos{} and the development of a language runtime share
several challenges,
including bridging the gap of resource allocation models
between the guest and host,
and influencing the host OS to efficiently assign hardware resources to applications.


\thelibos{} reproduces the resource allocation models of Linux
using \thehostabi{}.
To be portable across various hosts,
\thehostabi{} encapsulates the management of host resources.
%A primary task and challenge to managing resources in \thelibos{}
%is to reproduce the idiosyncratic %resource allocation
%features or requirements of Linux, % in Linux,
%using only the host abstractions
%defined by \thehostabi{}.
\graphene{} also simplifies the definition of \thehostabi{},
to include a narrowed set of host abstractions that are necessary
for a guest environment.
A responsibility of \thelibos{}
is to implement different Linux models of allocating a resource.
Take page management for example.
Linux supports several way of memory allocations, including \syscall{mmap} for allocating a fixed-size VMA, growing the stack of a process, and \syscall{brk} for more fine-grained heap allocation.
Since \thelibos{} does not directly manage pages,
it requires different emulation strategies to
implement the allocation models needed by an application. %s expect when using these Linux abstractions.
A strategy repeatedly used in \thelibos{} and language runtimes
is to ``overallocate'' certain host abstractions when an application requests for resources.
The purpose of overallocation is
to keep the flexibility of adjusting the resources afterward.
The caveat of using these kinds of strategies
is that they are based on an assumption that the host
allocates the resources
on demand, instead of populating the resources all at once.
However, such an assumption does not apply to all hosts;
%for an environment like a SGX enclave, overallocating resources such as virtual memory can still have a huge impact on resource footprints.
for example, the current version of SGX requires a static virtual memory layout,
and each VMA is at least fully populated once at enclave creation
for checking the integrity of memory data.
Therefore, overallocating VMAs slows down the creation of an enclave.
\fixme{End with a summary of the paragraph?}




%Since emulation has its cost,
%an important problem to solve in \thelibos{}








\paragraph{Comparison with alternative approaches.}
Virtualization is one of the alternative approaches of guest-level resource management. % for an application.
%managing resources in a guest OS
%is to virtualize the hardware resources,
%such as memory and IO devices,
A virtual machine often runs an unmodified OS kernel,
which has control over the allocation of virtualized or dedicated hardware resources.
To fully virtual hardware resources,
a hypervisor can take one of the two common strategies.
A strategy
is to export a virtual hardware interface
as an emulation of the physical hardware interface,
in a hypervisor
such as QEMU~\cite{qemu} or VMWare ESX~\cite{wldspurger02vmware-esx}.
A virtual hardware interface
allows an unmodified OS kernel to
directly manage virtualized hardware resources like physical hardware resources,
using a set of generic hardware drivers.
%and have full control of hardware resources.
%giving the guest OS the illusion of having full control of all the hardware resources.
Another strategy to leverage the hardware virtualization,
such as IOMMU~\cite{VT-d},
%which allows a hypervisor
to dedicate physical hardware resources
to a virtual machine.
%Virtualization allows a guest OS to directly manage hardware resources, by either emulating a set of virtual hardware,
%or dedicating physical hardware to a guest OS instance.
Both of the virtualization strategies grant a virtual machine with more control over resource management %managing hardware resources
than \thelibos{} in \graphene{}.


%Compared with full virtualization, the \graphene{} approach
%The \thelibos{} design
%is similar to a para-virtualized guest OS,
%which calls out to the host OS or hypervisor for allocating resources.


Exokernel~\cite{engler95exokernel} adopts a \libos{}-like approach
to export application-level system APIs, but grants each application the privilege to directly manage hardware resources. 
%to implement the system APIs and abstractions, but allows the \libos{} to manage hardware resources directly.
The rationale behind Exokernel is to bypass the complicated kernel logics
for abstracting and multiplexing hardware resources,
and to allow opportunities of domain-based optimization for each application.
%Exokernel assigns available hardware resources %, such as physical pages, storage disks, and network devices, 
%to each application,
%with hardware-aided protection.
Exokernel enforces a security binding from machine resources to applications,
so that each application can
manage its own resources using an untrusted \libos{}.
The similarity between the Exokernel and \graphene{} approaches is that
they both delegate the protection and security isolation of hardware resources to the host kernel or hypervisor.


In terms of resource management,
Exokernel and \graphene{} have made different decisions for
the division of labour
between the host and \libos{}. % on managing hardware resources.
Exokernel prioritizes the efficiency of resource management for each application. 
To eliminate overhead of multiplexing resources,
Exokernel exports the low-level hardware primitives, including physical memory, CPU, disks, TLB and page tables. 
Each \libos{} in Exokernel contains drivers to directly interfacing
these hardware primitives,
so that the choice of hardware is not longer
transparent to an application.
\graphene{}, on the other hand, prioritizes compatibility upon plenty of host OS and hardware platforms.
Compared with the primitives exported by Exokernel,
\thehostabi{} of \graphene{} defines abstractions
that are much more high-level and independent from the host OSes, such as files, virtual memory areas, and network sockets.
\graphene{} sacrifices the application-specific opportunities
for optimizing the resource management,
but ensures the compatibility upon any hosts with \thehostabi{}.


%Exokernel allows application-level resource management, by exposing low-level resources, such as physical pages, storage disks, and network devices, at the kernel interface.









\subsection{Virtual address space}
\label{sec:libos:vma}


\fixme{start with what applications need}
A Linux application expects a contiguous, large virtual address space,
to allocate a number of numerically-addressable memory regions.
A program usually uses a \libc{} allocator, requested by \funcname{malloc} and \funcname{calloc},
or a heap allocator of a managed language runtime,
or reserves space on the current stack,
to allocate fine-grained memory objects.
To support application-level allocation,
an OS is responsible of
maintaining a unique, consistent mapping between virtual memory areas (VMAs) and physical pages,
and managing the virtual address space layout
to prevent collision of VMAs.
The Linux kernel, specifically, provides several ways of memory allocation, such as allocation by \syscall{mmap} and \syscall{brk},
and transparently growing a user stack downward. % exceeding the stack boundary.
A application-level allocator may try several ways of requesting memory resources;
for example, the \glibc{} allocator
uses both \syscall{brk} and \syscall{mmap} to allocate different sizes of memory objects.
Applications depend on different memory allocation mechanisms
of a Linux kernel,
to dynamically allocate space for storing application data.
 



\thelibos{} manages the virtual address space of each \picoproc{}.
To emulate a Linux kernel,
\thelibos{} creates VMAs using two \hostapis{}:
\palcall{VirtMemAlloc} for creating an anonymous memory mapping,
and \palcall{StreamMap} for mapping a file into the virtual address space.
%or maps a file into a new VMA using \palcall{StreamMap}.
Both \hostapis{} creates a page-aligned, fixed range in the
virtual memory space,
with the assumption that the host OS or hypervisor
will assign a physical page to each virtual page being accessed,
and fill the physical page with file content or zeros.
\thelibos{} does not assume
a host to always implement demand paging.
The only assumption that \thelibos{} makes, when \palcall{VirtMemAlloc} or \palcall{StreamMap} returns successfully,
is that 
the the application or 
\thelibos{} is authorized to access any part of the created VMA,
without causing a segmentation fault
or memory protection fault.
It is possible that a host may have statically assign
physical pages to the whole VMA instead of gradually increasing the memory usage.

%each physical page of an allocated VMA
%will be assigned
%before any access to the page, including reading or writing data,
%or executing code.
%In other word, any future, authorized memory access
%in an allocated VMA
%should never cause any segmentation or memory protection faults.



\thelibos{} creates VMAs for two reasons.
First, \thelibos{} allocates memory regions on applications' request.
\thelibos{} also allocates memory for internal usages,
such as maintaining
the bookkeeping of OS states,
and reserving space for buffering and caching.
\thelibos{} contains a {\em slab allocator} (for internal \funcname{malloc}) and several object-caching memory allocators.
For each abstraction, \thelibos{} allocates a handle (e.g., a thread handle)
using internal allocation functions.
Therefore, the memory overhead of \thelibos{} %, regardless of the memory footprint of application itself,
is primarily caused by allocating various types of handles for maintaining or caching OS states,
and is roughly correlated with
the abstractions used by the application. 



%For example, \thelibos{} allocates a thread control block (TCB), or thread handle, for each user thread that an application creates, to store thread-specific attributes
%such as a thread identifier and a given stack address.
%To reduce the memory cost of \graphene{},
%\thelibos{} tries to reduce the VMA allocation for its internal usage
%as long as it does not add significant performance overhead to an application.
%Another usage of a VMA
%is to assign the VMA to a memory area that an application
%assumes to be accessible, such as a heap area created by \syscall{mmap} or a self-growing stack.
%If \thelibos{} allocates a VMA for supporting the memory access in an application,
%the VMA must be at least as large as the size that the application has requested.



\thelibos{} maintains a list of VMAs allocated by either the application
or \thelibos{} itself.
For each VMA, \thelibos{} records
the starting address, size, and the page protection (readable, writable, or executable).
The VMA list traces the free space within the current virtual address space.
When an application allocates a VMA,
\thelibos{} queries the VMA list to search for a sufficient space.
In another case, an application may specify the mapping address,
and the VMA list can determine whether the address has overlapped with an existing VMA, to prevent corrupting the internal states of \thelibos{}.
%Otherwise, \thelibos{} walks the VMA list to find a large enough free space
%to allocate a VMA. 
%In terms of page management,
%the role of \thelibos{} is to keep track of memory addresses that already belong to a VMA.
%In many cases,
%an application or \thelibos{}
%simply needs to allocate a new memory region that has not overlapped
%with existing memory regions.
%If an application gives a memory address
%as an argument to a \linuxapi{},
%\thelibos{} needs verifying the validity of address by checking against existing memory regions. 
%As a result,
%\thelibos{} maintains a list of currently-allocated VMAs,
%and dynamically updates the list whenever allocating, deallocating, or protecting any memory mappings.
%For each VMA, \thelibos{} records
%the starting address, size, protection mode (whether the VMA is readable, writable, or executable), and usage of the VMA.
%By maintaining a list of allocated VMAs,
%%Instead of managing physical pages, 
%\thelibos{} controls the virtual address space layout
%of a guest environment (i.e., a \picoproc{}).
%\thelibos{} keeps track of the free addresses in the current virtual address space,
%to prevent allocating overlapping VMAs.
%To allocate a VMA without a specific address,
%\thelibos{} first walks the VMA list
%to discover an unallocated, large enough address range.
According to the new VMA, \thelibos{} uses
%according to the type of mapping,
\palcall{VirtMemAlloc} or \palcall{StreamMap} 
to create the mapping in the host OS.
The VMA list also contains 
%the states of a
%virtual address space,
the mappings of PAL and the \thelibos{} binary.
%according to information given in the PAL control block.

%with the discovered address,
%to create a memory mapping suitable for the expected usage.
%Therefore, \thelibos{} can control the address
%for allocating a new VMA,
%to service a \syscall{mmap} \linuxapi{},
%or to extend the internal slab allocator of \thelibos{}.
%The bookkeeping of VMAs also includes
%the VMAs preserved by PAL and the VMAs for the \thelibos{} binary mapping,
%to prevent future VMAs
%corrupting the internal states of PAL or \thelibos{}.
%\thelibos{} records these VMAs at the beginning of a \picoproc{}, based on address ranges specified by the PAL control block.


Whenever an application or \glibc{} invokes a \linuxapi{} like \syscall{mmap}, \syscall{mprotect}, or \syscall{munmap}, \thelibos{} updates the VMA list
to reflect the virtual address space layout created by the host.
The basic design of a VMA list is a sorted, double-linked list of unique address ranges.
Because Linux allows arbitrary allocation, protection, and deallocation at page granularity, \thelibos{} often has to shrink or divide a VMA
into smaller regions.
\thelibos{} tries to synchronize
the virtual address space layout with the host OS, by tracing each memory allocation.


%To maintain a VMA list,
%\thelibos{} needs to dynamically modify, shrink, or divide VMAs, to support all corner cases
%of memory allocation.
%Three key \linuxapis{} in Linux---\syscall{mmap}, \syscall{mprotect}, and \syscall{munmap}---permit allocation, protection, and deallocation of pages
%at arbitrary page-aligned addresses.
%If a \syscall{mmap}, \syscall{mprotect}, or \syscall{munmap} \linuxapi{}
%partially frees or modifies a VMA,
%\thelibos{} divides the VMA bookkeeping into two, and then destroys of rewrites one of the VMAs.
%\thelibos{} is responsible of synchronizing the VMA lists of the host OS and \thelibos{}.


%Managing application and internal VMAs in the same virtual address space
%poses a challenge of isolation in \thelibos{}.



Different from a Linux kernel, \thelibos{} does not isolate its internal states from the application data.
\thelibos{} shares a virtual address space with the application,
and allows internal VMAs to interleave with memory mappings created by the application.
In this design, an application does not have to context-switch into another virtual address space to enter \thelibos{}.
A consequence of the design is the possibility that an application will corrupt the states of \thelibos{}, either accidentally or intentionally, by simply writing to arbitrary memory addresses.
The threat model of \graphene{} does not assume
\thelibos{} to defend against applications because both \thelibos{} and applications are untrusted by the host kernel. 

%allows internal VMAs to interleave with the VMAs allocated by the application.
%Despite that \thelibos{} cannot relocates VMAs
%for defragmenting the whole virtual address space,
%\thelibos{} can internally recycle fragmented space for buffering or expanding a slab allocator.
%Interleaving different types of VMAs
%helps filling up the fragmented free space
%with smaller objects or buffers.
%Moreover, there are opportunities to recycle or consolidate the internal heap of \thelibos{},
%once \thelibos{} detects pressure on utilizing the virtual address space.
%To sum up, \thelibos{} can recycle the holes between VMAs allocated by the application, and utilize the space for internal buffering or bookkeeping.



%Interleaving application and internal mappings
%potentially reduces the {\bf external fragmentation} in a virtual address space. % due to arbitrary memory allocation and deallocation.
%External fragmentation happens
%when an application deallocates smaller VMAs and creates holes in the virtual address space where larger memory mappings cannot fit in.
%Having holes in a virtual address space
%is normally acceptable on a Linux kernel;
%A \graphenearch{} Linux kernel sets the virtual address space of a process to be
%as large as 256 terabytes (${2}^{48}$ bytes),
%which is unlikely to wear out due to external fragmentation.
%However, upon a host where \thelibos{} can be given a small virtual address space,
%an application may eventually run out of virtual address space
%despite of the holes created by arbitrary allocation and deallocation.
%For example, a SGX enclave is always restricted within a specific region, which is given as a configuration signed off by the developers.
%To run an application in a SGX enclave, \thelibos{} must fit both application and internal states into a restricted enclave region.
%Therefore, \thelibos{} can recycle the holes between VMAs
%for storing internal OS states, which can be separated and fit into smaller regions.




%if the virtual address space is enormously large and the host OS can swap out physical pages.
%\thelibos{} can simply ignore the address space holes and
%allocate new VMAs at higher or lower addresses. 
%The assumption is problematic on a host where the virtual address space of
%a guest is a limited resource.
%For example, a SGX enclave has a limited virtual address space in the enclave, constrained by the enclave initialization.
%As a result, external fragmentation in the virtual address space
%can potentially cause a significant waste of resources on a specific host like SGX.






%Even if a address space hole
%is much smaller than a normal internal VMA, \thelibos{} can still repurpose the space
%to allocating a small amount of internal object.
%Unlike application VMAs, most internal VMAs can be recycled or relocated if \thelibos{} can trace back pointers to these VMAs.
%Because \thelibos{} interleaves different types of VMAs,
%there is opportunities for \thelibos{} to consolidate 





\paragraph{Implementing \syscall{brk}.}

%Take \syscall{brk} for example.
\syscall{brk} is a Linux \linuxapi{} for 
allocating memory space at the ``program break'', which defines the end of the executable's data segment.
What \syscall{brk} manages is a contiguous ``brk region'', which can be grown or shrunk by an application.
Unlike \syscall{mmap}, \syscall{brk} allocates arbitrary-size memory regions, by simply moving the program break
and returning the address to the application.
% of arbitrary sizes.
The primary use of \syscall{brk} in applications is %as an efficient way of
to allocate small, unaligned memory objects,
as a simple way of implementing \funcname{malloc}-like behaviors.


%Most applications 
%use \syscall{brk} 
%for its speed of allocating 
%small objects by moving the top of heap by a small offset.
%Different from \syscall{mmap},
%\syscall{brk} only allocates a physical page
%when moving the top of heap across page boundaries.
%%to allow gradual allocation of application objects.
%Some applications, such as \gcc{}, can bypass \syscall{brk} by switching to \syscall{mmap}.
%However, to support other applications that depends on \syscall{brk},
%\thelibos{} internally implements the fine-grained heap
%based on VMAs.



\thelibos{} implements \syscall{brk} by dedicating a part of the virtual address space for the brk region.
During the initialization, \thelibos{} reserves an unpopulated memory space
behind the executable's data segment, using \palcall{VirtMemAlloc}.
The size reserved for the brk region is determined
by user configurations.
%preallocates a \code{brk} area for future \syscall{brk} \linuxapis{}.
%The \code{brk} area has a limited, configurable capacity,
%and maintains a \code{brk} pointer to the top of heap currently assigned by the application.
\thelibos{} adjusts the end of the brk region
within the reserved space
whenever the application calls \syscall{brk}, or \funcname{sbrk}, a \libc{} function which internally calls \syscall{brk}.
\thelibos{} reserves the space for \syscall{brk}
to guarantee certain amount of memory resources for all the \syscall{brk} calls,
until the whole \picoproc{} is under memory pressure.



%the \code{brk} pointer
%and returns the latest top of heap.
%A \syscall{brk} call cannot move the \code{brk} pointer beyond the capacity of the \code{brk} area,
%or it will return \code{-ENOMEM} to the application.


\paragraph{Address Space Layout Randomization (ASLR).}

\thelibos{} implements Address Space Layout Randomization (ASLR) as a \libos{} feature.
Linux randomizes the address space layout to defeat or at least delay a remote memory attack, such as
a buffer overflow or a ROP (return-oriented programming) attack.
A remote memory attack
%launched by a user or a remote client
often depends on certain level of knowledge about the virtual address space layout of an application.
For example, in order to launch an effective buffer overflow,
an attacker tries to corrupt an on-stack pointer to make it points to security-sensitive data.
With ASLR, a Linux kernel increases the unpredictability of memory mappings,
so that a remote attacker is harder
to pinpoint a memory target.
%to launch an effective buffer overflow or ROP (return-oriented programming) attack in an application.
To support ASLR,
%Although some host OSes may already support ASLR, \thelibos{} enforces another layer of randomization.
\thelibos{} adds a random factor to the procedure of determining the addresses for allocating new VMAs.
%function that searches for free regions in the virtual address space.
%The randomization will cause \syscall{mmap} to return an unpredictable address, if the application does not specify the address.
\thelibos{} randomizes the results of both \syscall{mmap} and \syscall{brk};
for \syscall{brk}, \thelibos{} creates a random gap (up to 32MB) between the data segment and the brk region.







\papersubsection{File systems}
\label{sec:eval:libos:fs}

File system performance in \thelibos{}
is subject to several optimizations including directory caching and buffering.
To reduce the impact of \hostapi{} latency,
a key to the chroot file system implementation
is to lower the average number of expensive \hostapis{}
needed for emulating each system call.
For instance, the directory cache
in \thelibos{} stores path existence and metadata
in spared \picoproc{} memory,
to skip the redundant cost
of querying the host file system
when accessing the same path in the future.
Table~\ref{tab:eval:libos:lmbench-fs} lists the latency of \syscall{open} and \syscall{stat}
for repeatedly opening or querying the same path.
Directory caching
reduces the overheads on \syscall{stat} to 35--41\%
regardless of the hosts.
The latency of \syscall{open}
also benefits from directory caching, but the cost of opening the file in the host OSes
overshadows the optimization,
causing 187--237\% overheads on the Linux host with the \seccomp{} filter and reference monitor,
or 15.2--16.5\x{} in an enclave.

%can be categorized as two types.
%One type of overheads is the costs of directory caching,
%for storing file system hierarchy and metadata
%in \picoproc{} memory to avoid redundant storage lookup.
%Directory caching also
%helps identifying resources among multiple
%chroot'ed file systems
%mounted from host directories
%and pseudo file systems such as \code{/proc} and \code{/dev}.
%\thelibos{}
%adopt a similar design as the directory cache
%in a Linux kernel,
%with an optimization for searching long paths
%in a warm cache~\cite{tsai15dcache}. 


%Figure~\ref{tab:eval:libos:lmbench-fs}
%lists the latency or throughput of system calls
%for accessing an isolated host file system mounted in a \thelibos{} instance,
%or a {\bf chroot} file system.
%Each system call in a chroot file system
%accesses a file or a directory in the host file system,
%and therefore requires
%translation to one or multiple
%host system calls.
%As a result, the system call latency
%is determined by the underlying \hostapi{} latency and the translation cost inside \thelibos{}.
%%Besides, as previously stated, \thelibos{}
%%can optimize system calls such as \syscall{read} and \syscall{write}
%%by buffering read or written data.



%System calls like \syscall{open} and \syscall{stat}
%%access a specific path
%%in the file system.
%%The performance of this type of system calls
%are sensitive to path lengths and depths (i.e., numbers of components).
%As an optimization,
%\thelibos{} implements a file system directory cache
%to store path information and file attributes retrieved from the host OS.
%Because the \lmbench{} tests %for \syscall{stat} and \syscall{open}
%access the same path repeatedly,
%the directory cache
%is guaranteed to optimize every system calls measured.
%As a result,
%\syscall{stat} in both \graphene{} and \graphenesgx{} is only 35--41\% slower than native
%and mostly irrelevant from the host system call latency. 
%\syscall{fstat} also benefits from directory caching
%(35--41\% overheads).
%%Different from \syscall{stat},
%For \syscall{open}, %despite the optimization of directory caching,
%\graphene{} imposes
%extra overheads for opening PAL handles and allocating file descriptors in \thelibos{}.
%To access a path with 2--8 components,
%the overheads on \syscall{open} are 147--197\% for \graphene{} on Linux host, and 187--237\% with \seccomp{} filter and reference monitor.
%For \graphenesgx{}, the overheads are 15.2--16.5\x{}
%without considering the checksum calculation costs.


For \syscall{read} and \syscall{write},
the latency in \graphene{} depends on the buffering strategy in \thelibos{}.
The experiments
are based on a strategy which
buffers reads and writes smaller than 4KB (not including 4KB)
using a 16KB buffer directly mapped from the file.
In Table~\ref{tab:eval:libos:lmbench-fs}, buffered reads and writes (256 bytes and 1KB) on Linux host
are 22--92\% and -45--8\% slower than native, respectively.
The latency of unbuffered reads and writes (4KB and 16KB),
is closer to native
when running on a Linux host,
but suffers significant overheads (10--29\x{}) in an enclave due to copying file contents.



Table~\ref{tab:eval:libos:lmbench-fs} also lists the throughputs of creating and deleting a large amount of files,
measured in operations per second.
Among all the benchmark results, deletion throughputs tend to have much higher overheads than creation throughputs.
Compared to running on the Linux host, 
both file creation and deletion from an enclave suffer significantly higher overheads
(2.7--6\x{}).

\clearpage
\begin{table}[p]
\footnotesize
\centering
\bgroup
\def\arraystretch{1.1}
\setlength{\tabcolsep}{0.4em}
\begin{tabular}{|l|>{\palign{r}}p{4em}r|>{\palign{r}}p{4em}rr|>{\palign{r}}p{4em}rr|>{\palign{r}}p{4em}rr|}
\hline
&\multicolumn{11}{c|}{System call latency (\usec{}), +/- Confidence Interval, \% Overhead} \\
\hline
\multicolumn{1}{|c|}{{\bf Test}} &
\multicolumn{2}{c|}{{\bf Linux \linuxversion{}}} &
\multicolumn{3}{c|}{{\bf \graphene{}}} & \multicolumn{3}{c|}{{\bf \graphene{}+SC+RM}} & \multicolumn{3}{c|}{{\bf \graphenesgx{}}} \\
&
\usec{} & +/- & 
\usec{} & +/- & \%O &
\usec{} & +/- & \%O &
\usec{} & +/- & \%O \\
\hline

\parbox{\widthof{open}}{open} (d=2,len=08)	&	0.947	&	.072	&	2.474	&	.007	&	161	&	2.892	&	.007	&	205	&	16.676	&	.018	&	1,661		 \\\hline
\parbox{\widthof{open}}{open} (d=4,len=16)	&	1.011	&	.074	&	2.895	&	.034	&	186	&	3.241	&	.007	&	221	&	17.456	&	.017	&	1,627		 \\\hline
\parbox{\widthof{open}}{open} (d=6,len=24)	&	1.074	&	.068	&	3.363	&	.070	&	213	&	3.708	&	.007	&	245	&	18.084	&	.018	&	1,584		 \\\hline
\parbox{\widthof{open}}{open} (d=8,len=32)	&	1.131	&	.074	&	4.259	&	.007	&	277	&	4.601	&	.007	&	307	&	19.196	&	.020	&	1,597		 \\\hline
\parbox{\widthof{open}}{stat} (d=2,len=08)	&	0.361	&	.000	&	0.904	&	.000	&	150	&	0.857	&	.000	&	137	&	0.846	&	.000	&	134		 \\\hline
\parbox{\widthof{open}}{stat} (d=4,len=16)	&	0.420	&	.000	&	1.277	&	.001	&	204	&	1.278	&	.006	&	204	&	1.264	&	.001	&	201		 \\\hline
\parbox{\widthof{open}}{stat} (d=6,len=24)	&	0.486	&	.000	&	1.730	&	.000	&	256	&	1.730	&	.001	&	256	&	1.718	&	.000	&	253		 \\\hline
\parbox{\widthof{open}}{stat} (d=8,len=32)	&	0.553	&	.000	&	2.247	&	.000	&	306	&	2.249	&	.001	&	307	&	2.238	&	.001	&	305		 \\\hline
\parbox{\widthof{open}}{fstat} (any length)	&	0.120	&	.000	&	0.193	&	.000	&	61	&	0.193	&	.000	&	61	&	0.187	&	.000	&	56		 \\\hline
																								
\parbox{\widthof{write}}{read} (256bytes)	&	0.207	&	.072	&	0.252	&	.000	&	22	&	0.255	&	.000	&	23	&	0.342	&	.000	&	65		 \\\hline
\parbox{\widthof{write}}{read} (1KB)	&	0.227	&	.072	&	0.435	&	.000	&	92	&	0.434	&	.000	&	91	&	0.805	&	.001	&	255		 \\\hline
\parbox{\widthof{write}}{read} (4KB)	&	0.315	&	.072	&	1.163	&	.001	&	269	&	1.190	&	.000	&	278	&	2.744	&	.006	&	771		 \\\hline
\parbox{\widthof{write}}{read} (16KB)	&	1.022	&	.072	&	4.174	&	.009	&	308	&	4.291	&	.010	&	320	&	10.518	&	.022	&	929		 \\\hline
\parbox{\widthof{write}}{read} (64KB)	&	3.931	&	.072	&	9.504	&	.002	&	142	&	9.790	&	.004	&	149	&	14.619	&	.024	&	272		 \\\hline
\parbox{\widthof{write}}{write} (256bytes)	&	0.515	&	.002	&	0.285	&	.000	&	-45	&	0.287	&	.000	&	-44	&	0.490	&	.000	&	-5		 \\\hline
\parbox{\widthof{write}}{write} (1KB)	&	0.535	&	.001	&	0.575	&	.000	&	7	&	0.580	&	.000	&	8	&	1.420	&	.002	&	165		 \\\hline
\parbox{\widthof{write}}{write} (4KB)	&	0.618	&	.000	&	1.726	&	.003	&	179	&	1.767	&	.000	&	186	&	5.172	&	.006	&	737		 \\\hline
\parbox{\widthof{write}}{write} (16KB)	&	2.034	&	.000	&	6.490	&	.004	&	219	&	6.451	&	.002	&	217	&	19.128	&	.021	&	840		 \\\hline
\parbox{\widthof{write}}{write} (64KB)	&	7.614	&	.001	&	19.250	&	.040	&	153	&	19.545	&	.044	&	157	&	57.574	&	.017	&	656		 \\\hline

\hline
\hline
&\multicolumn{11}{c|}{System call throughput (operations/s), +/- Confidence Interval, \% Overhead} \\
\hline
\multicolumn{1}{|c|}{{\bf Test}} &
\multicolumn{2}{c|}{{\bf Linux \linuxversion{}}} &
\multicolumn{3}{c|}{{\bf \graphene{}}} & \multicolumn{3}{c|}{{\bf \graphene{}+SC+RM}} & \multicolumn{3}{c|}{{\bf \graphenesgx{}}} \\
&
kops/s & +/- & 
kops/s & +/- & \%O &
kops/s & +/- & \%O &
kops/s & +/- & \%O \\
\hline
create (0KB)	&	151,819	&	734	&	122,526	&	343	&	24	&	116,195	&	205	&	31	&	40,471	&	248	&	275		 \\\hline
delete (0KB)	&	247,750	&	1,048	&	133,397	&	424	&	86	&	120,683	&	138	&	105	&	37,706	&	127	&	557		 \\\hline
create (4KB)	&	154,318	&	21	&	83,880	&	201	&	84	&	73,797	&	993	&	109	&	21,989	&	37	&	602		 \\\hline
delete (4KB)	&	250,097	&	461	&	109,782	&	504	&	128	&	101,480	&	480	&	146	&	35,355	&	14	&	607		 \\\hline
create (10KB)	&	102,749	&	90	&	64,693	&	134	&	59	&	62,891	&	72	&	63	&	18,194	&	6	&	465		 \\\hline
delete (10KB)	&	186,029	&	458	&	93,833	&	232	&	98	&	89,493	&	129	&	108	&	33,368	&	94	&	458		 \\\hline
\end{tabular}
\egroup
\caption{File-related system call performance based on \lmbench{}. 
Comparison is among (1) native Linux processes; (2) \graphene{} on Linux host, both without and with \seccomp{} filter ({\bf +SC}) and reference monitor ({\bf +RM}); (3) \graphenesgx{}.
System call latency is in microseconds, and lower is better.
System call throughput is in operations per second, and higher is better. 
Overheads are relative to Linux; negative overheads indicate improvement.} 
\label{tab:eval:libos:lmbench-fs}
\end{table}
\clearpage
\subsection{Network sockets and pipes}
\label{sec:libos:socket}

\subsection{Threading and synchronization}
\label{sec:libos:thread}


A Linux application normally uses POSIX threads, or {\bf pthreads},
for parallelizing computation on a multi-core machine.
%are commonly used in Linux and similar OSes, for developing multi-thread applications.
The pthread library (i.e., \libpthread{}) creates a pthread
by invoking the \syscall{clone} \linuxapi{},
which creates a schedulable task inside the Linux kernel.
The pthread library also maintains a descriptor (\code{pthread\_t}) for signaling or waiting for a pthread from the rest of application.
Finally,
the pthread library contains several scheduling or synchronization primitives,
including mutexes, semaphores, conditional variables,
and barriars.
Few Linux applications may choose an alternative threading library, but no alternatives can avoid creating kernel tasks using \syscall{clone},
to fully utilize the host CPU resources.



\thelibos{} supports creation of pthreads
or other threading primitives
by implementing \syscall{clone} with shared virtual address space. % (no \code{CLONE\_VM} flag).
When creating a new thread, 
\syscall{clone} is usually given a preallocated user stack, a starting function, and an argument to initiate the function.
\thelibos{} implements \syscall{clone} using a \hostapi{}, \palcall{ThreadCreate}, which has similar semantics as \syscall{clone}.


For \thelibos{}, supporting a threading library like pthread
presents two primary challenges.
The first challenge is the implementation
of thread-local storage (TLS), a critical, thread-private region for storing the thread states (e.g., a \code{pthread\_t} structure).
The other challenge is to recreate the OS support
for many application-level scheduling and synchronization primitives.
\thelibos{} achieves the latter by implementing majority of the Linux \syscall{futex} API.



The pthread library allocates a {\bf thread control block (TCB)} for each pthread.
On \graphenearch{} Linux, a pthread %references its own TCB
uses the FS segment register
to reference its own TCB,
followed by the thread-private variables of the application
and user libraries. 
The FS segment register is a privileged context,
and thus an application can only set the address of TCB using the \syscall{arch\_prctl} \linuxapi{}, or pass the address as an argument to \syscall{clone},
unless there is architectural help (an opt-out \graphenearch{} instruction, \assembly{WRFSGSBASE}, allows setting FS/GS registers in the user space).
Therefore, \thelibos{} implements both \syscall{arch\_prctl}
and \syscall{clone} with a \hostapi{}, \palcall{SegmentRegisterSet}, to set the segment register from the host OS or hypervisor.


For each thread, \thelibos{} also maintain an internal TCB,
for storing a pointer to the corresponding thread handle and preserving the register values when \thelibos{} intercepts a \linuxapi{} from \libc{}.
To save the usage of segment registers, \thelibos{} shares the FS register with the pthread library,
by inserting the TCB of \thelibos{} into the pthread TCB.
\thelibos{} copies or recreates its own TCB whenever the pthread library calls \syscall{arch\_prctl} to swap the pthread TCB.



\paragraph{The \syscall{futex} API.}
Most synchronization primitives of the pthread library, including mutexes, semaphores, conditional variables, and barriers,
are based on the \syscall{futex} API.
The \syscall{futex} API contains two primary types of operations:
blocking on a specific memory address to be updated, or waking up threads that are currently blocking. % on a specific memory address.
The futex API allows a thread to forfeit the CPU resources
to a progressing thread,
and can be combined with atomic operation
to simultaneously update a critical variable and notify threads that are waiting for the variable to change to certain condition.
\fixme{not finished yet; maybe add a figure to explain how futex is implemented}










