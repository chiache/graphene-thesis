\section{Resource Management}
\label{sec:libos:resource}


The \libos{}
relies on the host to manage hardware and privileged OS resources.
The host ABI
defines a set of host abstractions,
including I/O streams, virtual memory areas (VMAs), and threads.
%that are available for its guest.
These host abstractions delegate
the management of a few hardware resources, such as I/O devices, memory pages, and CPUs,
to the host OSes.
Other abstractions defined
by the host ABI
include local RPC streams and the system clock,
which depend on low-level, privileged resources in most OSes,
such as in-kernel queues and time sources.
To support the \libos{} as a guest,
the host ABI avoids the requirement of exposing or virtualizing %the management of
these hardware and privileged OS resources,
by encapsulating the resources as the generic, user-space-level abstractions
of the host ABI.



\issuedone{1.2.a}{discuss the role of libOS in resource management}
For resource management, the role of the \libos{}
is to allocate the host abstractions,
as unambiguous requests %to the host
for the host-managed resources.
At a high level, the purpose of the \libos{} is to realize Linux abstractions.
The \libos{} implements Linux abstractions
based on managing the host abstractions instead of
the underlying resources.
For example, if a Linux abstraction requires memory pages
either for applications or
to maintain internal bookkeeping,
the implementation in
the \libos{} will allocate a VMA as the host memory abstraction, instead of directly managing physical pages.
%in a unified, guest virtual address space.
Such a \libos{} design operates on the faith that the host OS will assign physical pages to each VMA, with acceptable fairness and efficiency.
Unless the allocation exceeds user quotas or host limitations,
the \libos{} should be allowed to obtain more host-manged resources,
by increasing the allocation of a host abstraction.


\placeholder[write about the challenges.]{}


\placeholder[list the ``virtual'' resources managed by libOS]{}


\subsection{Virtual address space}
\label{sec:libos:vma}


\thelibos{} allocates memory resources by creating VMAs (virtual memory areas).
\Thehostabi{} defines a VMA
as the basic abstraction for allocating memory pages
in a host;
\thelibos{} either creates an anonymous VMA using a \hostapi{}, \palcall{VirtMemAlloc},
or maps a host file into a VMA using another \hostapi{}, \palcall{StreamMap}.
Both \hostapis{} creates a page-aligned, fixed-size range in the
current virtual memory space,
assuming that each virtual page in the range
will map to an individual physical page, either zeroed or filled with file content.
\thelibos{} does not assume
that the host will always implement demand paging
for every allocated VMAs.
The only assumption that \thelibos{} makes, when using \palcall{VirtMemAlloc} or \palcall{StreamMap},
is that each physical page of an allocated VMA
will be assigned
before any access to the page, including reading or writing data,
or executing code.
In other word, any future, authorized memory access
in an allocated VMA
should never cause any segmentation or memory protection faults.



\thelibos{} allocates VMAs for two usages.
One usage is to allocate pages for internal use of \thelibos{},
including mapping the \thelibos{} binary
(i.e., \thelibos\code{.so})
into the guest environment for execution,
and storing the internal OS states.
For example, \thelibos{} allocates a thread control block (TCB), or thread handle, for each user thread that an application creates, to store thread-specific attributes
such as a thread identifier and a given stack address.
To reduce the memory cost of \graphene{},
\thelibos{} tries to reduce the VMA allocation for its internal usage
as long as it does not add significant performance overhead to an application.
Another usage of a VMA
is to assign the VMA to a memory area that an application
assumes to be accessible, such as a heap area created by \syscall{mmap} or a self-growing stack.
If \thelibos{} allocates a VMA for supporting the memory access in an application,
the VMA must be at least as large as the size that the application has requested.



In terms of page management,
the role of \thelibos{} is to keep track of memory addresses that already belong to a VMA.
In many cases,
an application or \thelibos{}
simply needs to allocate a new memory region that has not overlapped
with existing memory regions.
If an application gives a memory address
as an argument to a \linuxapi{},
\thelibos{} needs verifying the validity of address by checking against existing memory regions. 
As a result,
\thelibos{} maintains a list of currently-allocated VMAs,
and dynamically updates the list whenever allocating, deallocating, or protecting any memory mappings.
For each VMA, \thelibos{} records
the starting address, size, protection mode (whether the VMA is readable, writable, or executable), and usage of the VMA.




By maintaining a list of allocated VMAs,
%Instead of managing physical pages, 
\thelibos{} controls the virtual address space layout
of a guest environment (i.e., a \picoproc{}).
\thelibos{} keeps track of the free addresses in the current virtual address space,
to prevent allocating overlapping VMAs.
If \thelibos{} needs to allocate a VMA without a specific address,
it walks the list of VMAs
to discover an unallocated address range.
Then, \thelibos{} can use \palcall{VirtMemAlloc} or \palcall{StreamMap} with the discovered address,
to create a memory mapping suitable for the expected usage.
Therefore, \thelibos{} can control the address
for allocating a new VMA,
to service a \syscall{mmap} \linuxapi{},
or to extend the internal slab allocator of \thelibos{}.
The bookkeeping of VMAs also includes
the VMAs preserved by PAL and the VMAs for the \thelibos{} binary mapping,
to prevent future VMAs
corrupting the internal states of PAL or \thelibos{}.
\thelibos{} records these VMAs at the beginning of a \picoproc{}, based on address ranges specified by the PAL control block.





\paragraph{Implementing \syscall{brk}.}





\paragraph{Address Space Layout Randomization (ASLR).}


\papersubsection{File systems}
\label{sec:eval:libos:fs}


\begin{table}[t!b!]
\footnotesize
\centering
\bgroup
\def\arraystretch{1.1}
\setlength{\tabcolsep}{0.4em}
\begin{tabular}{|ll|>{\palign{r}}p{3.5em}r|>{\palign{r}}p{3.5em}rr|>{\palign{r}}p{3.5em}rr|>{\palign{r}}p{3.5em}rr|}
\hline
& & \multicolumn{11}{c|}{System call latency (\usec{}), +/- Confidence Interval, \% Overhead} \\
\hline
\multicolumn{2}{|c|}{{\bf Test}} &
\multicolumn{2}{c|}{{\bf Linux \linuxversion{}}} &
\multicolumn{3}{c|}{{\bf \graphene{}}} & \multicolumn{3}{c|}{{\bf \graphene{}+SC+RM}} & \multicolumn{3}{c|}{{\bf \graphenesgx{}}} \\
& &
\usec{} & +/- & 
\usec{} & +/- & \%O &
\usec{} & +/- & \%O &
\usec{} & +/- & \%O \\
\hline

{\tt open}	&	(d=2,len=\hspace{.5em}8)	&	0.947	&	.072	&	2.337	&	.013	&	147	&	2.719	&	.007	&	187	&	16.600	&	.007	&	1,653		 \\\hline
{\tt open}	&	(d=4,len=16)	&	1.011	&	.074	&	2.627	&	.009	&	160	&	2.922	&	.009	&	189	&	17.168	&	.016	&	1,598		 \\\hline
%{\tt open}	&	(d=6,len=24)	&	1.074	&	.068	&	2.719	&	.008	&	153	&	3.131	&	.007	&	192	&	17.527	&	.016	&	1,532		 \\\hline
{\tt open}	&	(d=8,len=32)	&	1.131	&	.074	&	3.360	&	.000	&	197	&	3.812	&	.007	&	237	&	18.415	&	.016	&	1,528		 \\\hline
\hline																										
{\tt stat}	&	(d=2,len=\hspace{.5em}8)	&	0.361	&	.000	&	0.502	&	.000	&	39	&	0.499	&	.000	&	38	&	0.487	&	.000	&	35		 \\\hline
{\tt stat}	&	(d=4,len=16)	&	0.420	&	.000	&	0.585	&	.000	&	39	&	0.584	&	.000	&	39	&	0.571	&	.001	&	36		 \\\hline
%{\tt stat}	&	(d=6,len=24)	&	0.486	&	.000	&	0.685	&	.000	&	41	&	0.685	&	.000	&	41	&	0.671	&	.000	&	38		 \\\hline
{\tt stat}	&	(d=8,len=32)	&	0.553	&	.000	&	0.780	&	.000	&	41	&	0.780	&	.000	&	41	&	0.767	&	.000	&	39		 \\\hline
\hline																										
{\tt fstat} 	&	(any length)	&	0.120	&	.000	&	0.193	&	.000	&	61	&	0.193	&	.000	&	61	&	0.187	&	.000	&	56		 \\\hline
\hline																										
\hline																										
%{\tt read} 	&	(0.25KB)	&	0.207	&	.072	&	0.252	&	.000	&	22	&	0.255	&	.000	&	23	&	0.342	&	.000	&	65		 \\\hline
{\tt read} 	&	(\hspace{.5em}1KB)	&	0.227	&	.072	&	0.435	&	.000	&	92	&	0.434	&	.000	&	91	&	0.805	&	.001	&	255		 \\\hline
{\tt read} 	&	(\hspace{.5em}4KB)	&	0.315	&	.072	&	0.545	&	.001	&	73	&	0.607	&	.000	&	93	&	9.545	&	.006	&	2,930		 \\\hline
{\tt read} 	&	(16KB)	&	1.022	&	.072	&	1.308	&	.000	&	28	&	1.356	&	.000	&	33	&	11.437	&	.022	&	1,019		 \\\hline
%{\tt read} 	&	(64KB)	&	3.931	&	.072	&	4.190	&	.001	&	7	&	4.189	&	.001	&	7	&	17.071	&	.004	&	334		 \\\hline
\hline																										
%{\tt write} 	&	(0.25KB)	&	0.515	&	.002	&	0.285	&	.000	&	-45	&	0.287	&	.000	&	-44	&	0.490	&	.000	&	-5		 \\\hline
{\tt write} 	&	(\hspace{.5em}1KB)	&	0.535	&	.001	&	0.575	&	.000	&	7	&	0.580	&	.000	&	8	&	1.420	&	.002	&	165		 \\\hline
{\tt write} 	&	(\hspace{.5em}4KB)	&	0.618	&	.000	&	0.856	&	.002	&	39	&	0.909	&	.002	&	47	&	9.784	&	.006	&	1,483		 \\\hline
{\tt write} 	&	(16KB)	&	2.034	&	.000	&	2.303	&	.013	&	13	&	2.356	&	.001	&	16	&	19.730	&	.021	&	870		 \\\hline
%{\tt write} 	&	(64KB)	&	7.614	&	.001	&	7.929	&	.020	&	4	&	7.971	&	.002	&	5	&	59.899	&	.017	&	687		 \\\hline

\hline
\hline
& & \multicolumn{11}{c|}{System call throughput (operations/s), +/- Confidence Interval, \% Overhead} \\
\hline
\multicolumn{2}{|c|}{{\bf Test}} &
\multicolumn{2}{c|}{{\bf Linux \linuxversion{}}} &
\multicolumn{3}{c|}{{\bf \graphene{}}} & \multicolumn{3}{c|}{{\bf \graphene{}+SC+RM}} & \multicolumn{3}{c|}{{\bf \graphenesgx{}}} \\
& &
ops/s & +/- & 
ops/s & +/- & \%O &
ops/s & +/- & \%O &
ops/s & +/- & \%O \\
\hline
create	&	(\hspace{.5em}0KB)	&	151,819	&	734	&	122,526	&	343	&	24	&	116,195	&	205	&	31	&	40,471	&	248	&	275		 \\\hline
delete	&	(\hspace{.5em}0KB)	&	247,750	&	1,048	&	133,397	&	424	&	86	&	120,683	&	138	&	105	&	37,706	&	127	&	557		 \\\hline
create	&	(\hspace{.5em}4KB)	&	154,318	&	21	&	83,880	&	201	&	84	&	73,797	&	993	&	109	&	21,989	&	37	&	602		 \\\hline
delete	&	(\hspace{.5em}4KB)	&	250,097	&	461	&	109,782	&	504	&	128	&	101,480	&	480	&	146	&	35,355	&	14	&	607		 \\\hline
create	&	(10KB)	&	102,749	&	90	&	64,693	&	134	&	59	&	62,891	&	72	&	63	&	18,194	&	6	&	465		 \\\hline
delete	&	(10KB)	&	186,029	&	458	&	93,833	&	232	&	98	&	89,493	&	129	&	108	&	33,368	&	94	&	458		 \\\hline
\end{tabular}
\egroup
\caption{File system performance. The host file system is EXT4. Comparison is among (1) native Linux processes; (2) \graphene{} on Linux host, both without and with \seccomp{} filter ({\bf +SC}) and reference monitor ({\bf +RM}); (3) \graphenesgx{}.
System call latency is in microseconds, and lower is better.
System call throughput is in operations per second, and higher is better. 
Overheads are relative to Linux; negative overheads indicate improvement.} 
\label{tab:eval:libos:lmbench-fs}
\end{table}



Figure~\ref{tab:eval:libos:lmbench-fs}
lists the latency or throughput of several system calls
for accessing an isolated host file system mounted into the \thelibos{} instance, or a {\bf chroot} file system.
\subsection{Network connections}
\label{sec:libos:socket}

\subsection{Multi-threading}
\label{sec:libos:thread}







