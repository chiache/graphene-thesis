This chapter uses Linux as a fundamental example
of a \graphene{} host, to illustrate the implementation of \thehostabi{}
and a reference monitor.
The usage of \graphene{} on a Linux host
has primarily two purposes.
One is to create a lightweight, VM-like, guest OS environment for running an application
with a standalone OS view.
The other purpose is
to reduce the host kernel attack surface from an untrusted application,
as the number of vulnerable kernel paths that can be triggered by the application.
This chapter first demonstrates the feasibility of developing a Linux PAL prioritized for minimal Linux \linuxapi{} footprint,
followed by a discussion of the security isolation mechanism.


