This chapter uses Linux as an example
of a \graphene{} host, to illustrate the implementation of \thehostabi{}
and security isolation.
The usage of \graphene{} on a Linux host
has two primary benefits.
One benefit is to create a lightweight, VM-like, guest OS environment for running an application
with a standalone OS view.
The other benefit is
to reduce the host kernel attack surface from an untrusted application,
as the number of vulnerable kernel paths that can be triggered by the application.
This chapter first demonstrates the feasibility of developing a Linux PAL prioritized for minimal Linux system call footprint,
followed by a discussion of the security isolation mechanism.


