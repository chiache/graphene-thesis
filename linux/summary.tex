\section{Summary}



The Linux PAL successfully leverages a limited subset of Linux \linuxapis{},
to implement the whole PAL ABI for running a
full-featured \libos{}.
\Thehostabi{} separates the development of a host OS or hypervisor
from the complexity of emulating a sufficiently-compatible
Linux kernel.
The chapter shows that most calls in \thehostabi{}
can be directly translated to similar \linuxapis{} on a Linux host kernel.
Only a few \hostapis{}, such as process creation and inter-thread synchronization, require additional attention for developing an efficient implementation strategy.



The Linux PAL also enforces robust security isolation
between mutually-untrusting applications,
by placing applications in separate, VM-like sandboxes.
The security isolation on a Linux host is based on \linuxapi{} restriction using a \seccomp{} filter, and a trusted reference monitor. % for checking resource access.
Security isolation at the host interface
restricts an untrusted application to explore vulnerable execution paths
inside a Linux kernel.
A \seccomp{} filter 
enforce a fixed, minimal \linuxapi{} profile, regardless of bloated dependency of an application.
The reference monitor follows
simple, white-listed manifest rules listing 
all the authorized files and network addresses of an application,
using well-known semantics
such as AppArmor~\cite{apparmor} or iptable-like firewall rules~\cite{iptablesman}.
The reference monitor can further enforce dynamic, process-specific isolation by splitting a sandbox
to run a child \picoproc{} under more restricted
resource permissions.
\graphene{} on a Linux host can serve as a sandbox framework
with a reduced attack surface
upon the host kernel.






