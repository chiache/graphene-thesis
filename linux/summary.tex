\section{Summary}


The development of a Linux PAL focuses
on the simplicity of leveraging existing Linux \linuxapis{} to implement
the whole PAL ABI,
and the robustness of enforcing security isolation
from the host kernel.
The development of \thelibos{} proves the sufficiency of \thehostabi{} to emulate Linux abstractions and behaviors in a guest.
\Thehostabi{} isolates each host
and the corresponding PAL
from the complexity of Linux features and APIs required by an applications.
The Linux PAL
shows an example of exporting \thehostabi{} using only \hostsyscallnum{} trimmed \linuxapis{},
and enforcing VM-like security isolation
with a reduced attack surface.


The security isolation on a Linux host is based on \linuxapi{} restriction and a trusted reference monitor.
Since \thelibos{} has significantly reduced the host interface,
a host kernel further restricts the host \linuxapis{} and arguments from an untrusted \picoproc{},
and access to sharable host resources.
A \seccomp{} filter
narrows down the execution path inside a Linux kernel that a \picoproc{} could directly trigger through \linuxapis{},
and effectively eliminates
the risk of exposing any vulnerabilities on the remaining kernel execution path to an untrusted application.
The reference monitor enforces straightforward, white-listing rules
for sharing files and network addresses.
A sandbox created by the reference monitor restricts
the files and network addresses accessed by a set of mutually-trusting \picoproc{}, based on a application-specific manifest.
The reference monitor further enforces process-specific isolation,
by dynamically moving an \picoproc{} to a new sandbox
with a more restricted file and network view and no bridging RPC streams to \picoprocs{} of other sandboxes.





