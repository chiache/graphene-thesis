\section{Implementing \thehostabi{}}
\label{sec:linux:impl}

\issuedone{1.1.a}{extend the technical sections}
The initial design of the Linux PAL is based on an unmodified Linux kernel.
By default, a \graphene{} \picoproc{} should run upon
an off-the-shelf Linux kernel as an unprivileged, normal process, with a PAL loaded for exporting \thehostabi{}.
The Linux PAL demonstrates a minimal effort of implementing \thehostabi{}
on a single host, considering Linux is rich with APIs for programming all sorts of applications.
Only two \graphene{} components on the host requires extension or modification of the Linux kernel: a bulk IPC kernel module and a trusted reference monitor.



On a Linux host,
the majority of \hostapis{} can be implemented with simple wrappers
for similar Linux \linuxapis{},
adding on less than 100 LoC on average for each \hostapi{}.
The most complex \hostapis{} on a Linux host are for exception handling, synchronization, and process creation, and each of these \hostapis{} requires multiple \linuxapis{} and roughly 500--800 LoC in PAL.
For example, process creation (i.e., \palcall{ProcessCreate}) requires 
both the \syscall{vfork} and \syscall{execve} \linuxapis{}
for creating a clean
application instance, and would be more efficiently
implemented inside the Linux kernel.
Finally, the other major \pal{} components are an ELF loader (2 kLoC),
Linux kernel and PAL headers (800 LoC),
and internal support code providing functions like \funcname{malloc} and \funcname{memcpy} (2.3 kLoC).
Table~\ref{tab:libos:loc} lists the lines of code of each components of the \graphene{} architecture on a Linux host.


A PAL developer can use the Linux PAL as a baseline of \thehostabi{} development effort on a full-featured host OS .
About half of the Linux PAL code turns out to be
mostly generic to every host OSes
and thus fully reusable for each PAL.
The generic parts include
the ELF loader, PAL headers, and internal support code, adding up to \roughly{}6,000 LoC.
The rest of the Linux PAL are primarily
wrappers for
Linux \linuxapis{},
and major effort of porting this part would be to replace the Linux \linuxapis{}
with the target host APIs.
If the host OS has exported a UNIX or POSIX-like API,
porting the host-dependent code is mostly straightforward.
For example, as a follow-up exercise, the development of a FreeBSD PAL has recycled most of the Linux PAL code.





\begin{table}[t!b!]
\footnotesize
\centering
\begin{tabular}{|l|rr|}
\hline
{\bf Component} & {\bf Lines} & ({\bf \% Changed})\\
\hline
GNU Library C ({\tt libc}, {\tt ld}, {\tt libdl}, {\tt libpthread}) & \libclines{} & $0.07\%$ \\
\hline
Linux Library OS (\thelibos{}) & 31,112 & \\
Linux PAL & 12,529 & \\
%Extra code for Linux SGX host \pal{} & 9.354 & \\
% updated by Chia-Che on Oct. 10, 2013
\hline
%Storage Server & \fixmedp{XX} & \\
Reference monitor bootstrapper & \reflines{} & \\
Linux kernel reference monitor module ({\tt /dev/graphene}) & \sandboxmodlines{} & \\
Linux kernel IPC module ({\tt /dev/gipc}) & \gipclines{} & \\
\hline
\end{tabular}
\caption[Lines of code written or changed in \graphene{}]
{Lines of code written or changed to develop the whole \graphene{} architecture on a Linux hosts.  The application and other dynamically-loaded libraries are unmodified.}
\label{tab:libos:loc}
\end{table}


%\paragraph{Alternative \pal{} Ports.}
%We prove the platform independence of \graphene{}
%by porting \pal{} to \emph{FreeBSD}, \emph{OSX} and \emph{Windows}.
%With the alternative host \pal{}, unmodified Linux binaries,
%along with {\tt glibc} and {libLinux},
%can be transparently run on the host.
%For FreeBSD,
%only 1.2 kLoC of the host \pal{} code need to be rewritten,
%which are significantly less than FreeBSD Linux compatibility module (10.8 kLoC).
%\pal{} components including ELF loader and internal support code can be shared by any \pal{} ports.

%\fix{We leave host \pal{} ports to non-unix OSes like Windows as future work,
%but previous works~\cite{porter11drawbridge,baumann13bascule} have already shown it feasible.}




%% * most calls are a wrapper, \fixmedp{XX} LoC on average.
%% * Exception handling, sync, and process creation were harder (500-800 LoC each).  Process creation requires a clean instance (vfork+exec), would be simpler to implement in kernel.
%% * Other major components: ELF loader (2kLoC), headers(800 LoC), internal support code (2300 LoC)


%\fixmedp{Chia-Che, update LoC table}


The rest of this section
will discuss a few PAL ABI abstractions that are particularly challenging on a Linux host.
Similar challenges have been presented %have been observed for implementing \thehostabi{}
on other alternative host kernels such as FreeBSD, OS X, and Windows.
%except that other kernels might introduce additional host-specific challenges.


\paragraph{Bootstrapping a \picoproc{}.}
The Linux PAL works as a run-time loader to the \graphene{} \libos{}, or \thelibos{}. % and Linux applications.
First, for the dynamic loading of \thelibos{},
the Linux PAL contains an ELF loader, similar to the functionality of a \libc{} loader (\code{ld.so}),
to map the \thelibos{} binary into a \picoproc{} and resolve the addresses of \hostapis{}.
Second, the Linux PAL constructs a process control block (PCB),
providing information
about the \picoproc{} and the host platform.
For example, a member of the PCB exposes the basic CPU information (e.g., model name and number of cores) to \thelibos{}, 
for implementing the \code{cpuinfo} entry of the \code{proc} file system
as a \libos{} abstraction.
Finally,
the Linux PAL populates the stack with program augments and environment variables from the command line, and starts the execution of \thelibos{}.



\paragraph{RPC streams.}


\paragraph{Exception handling.}


\paragraph{Synchronization.} Perhaps surprisingly, implementing Linux
synchronization appears substantially easier than Windows synchronization, 
as {\tt libLinux} did not require all of the various
synchronization ABIs provided by Drawbridge.
We believe the reason for this is that Linux has consolidated 
all user-level synchronization primitives to use futexes~\cite{franke02futex},
which are essentially kernel-level wait queues.
%In Windows parlance, this is simply an Event associated with a virtual address.
%Thus, our effort implementing synchronization was relatively straightforward.


\paragraph{Process creation.}



\paragraph{Bulk IPC.}


