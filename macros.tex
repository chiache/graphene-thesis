\makeatletter
\newcommand*{\declarecommand}{%
  \@star@or@long\declare@command
}
\newcommand*{\declare@command}[1]{%
  \provide@command{#1}{}%
  % \let#1\@empty % would be more efficient, but without error checking
  \renew@command{#1}%
}

\usepackage{xstring}

\declarecommand{\projname}[1]{{#1}}

\declarecommand{\syscall}[1]{{\tt #1()}}
\declarecommand{\palcall}[1]{{\tt #1()}}
\declarecommand{\code}[1]{{\tt #1}}
\declarecommand{\assembly}[1]{\code{\uppercase{#1}}}
\declarecommand{\funcname}[1]{{\tt #1()}}

% different kinds of seconds
\newcommand{\asec}{S}
\newcommand{\usec}{$\mu$S}
\newcommand{\msec}{mS}
\newcommand{\nsec}{nS}

% tunable symbols
\newcommand{\roughly}[1][]{$\sim$#1}

\newcommand{\palign}[2][]{%
    \IfEqCase{#2}{%
        {l}{\def\@@align{\raggedright}}%
        {r}{\def\@@align{\raggedleft}}%
        {c}{\def\@@align{\centering}}%
    }[\PackageError{palign}{Undefined option to palign: #2}{}%
          \def\@@align{}]%
    \@@align \arraybackslash #1%
}



% slangs
\declarecommand{\loc}{LoC}
\declarecommand{\us}{$\mu$S}
\declarecommand{\x}{$\times$}


% project names
\declarecommand{\graphene}{\projname{Graphene}}
\declarecommand{\graphenesgx}{\projname{Graphene-SGX}}
\declarecommand{\graphenearch}{{x86-64}}
\declarecommand{\civet}{\projname{Civet}}
\declarecommand{\microkernel}{microkernel}
\declarecommand{\haven}{\projname{Haven}}
\declarecommand{\drawbridge}{\projname{Drawbridge}}
\declarecommand{\scone}{\projname{SCONE}}
\declarecommand{\panoply}{\projname{PANOPLY}}
\declarecommand{\sgx}{{SGX}}
\declarecommand{\docker}{{Docker}}


% terminology
\declarecommand{\Libos}{Library OS}
\declarecommand{\Liboses}{Library OSes}
\declarecommand{\libos}{library OS}
\declarecommand{\liboses}{library OSes}
\declarecommand{\Picoproc}{Picoprocess}
\declarecommand{\Picoprocs}{Picoprocesses}
\declarecommand{\picoproc}{picoprocess}
\declarecommand{\picoprocs}{picoprocesses}
\declarecommand{\pal}{PAL}
\declarecommand{\posixapi}{POSIX function}
\declarecommand{\posixapis}{POSIX functions}
\declarecommand{\linuxapi}{system call}
\declarecommand{\Linuxapi}{System call}
\declarecommand{\linuxapis}{system calls}
\declarecommand{\Linuxapis}{System calls}
\declarecommand{\hostapi}{PAL call}
\declarecommand{\hostapis}{PAL calls}
\declarecommand{\thehostabi}{{the PAL ABI}}
\declarecommand{\Thehostabi}{{The PAL ABI}}
\declarecommand{\thelibos}{{\tt libLinux}}
\declarecommand{\Thelibos}{{\tt libLinux}}
\declarecommand{\Glibc}{{Glibc}}
\declarecommand{\glibc}{{Glibc}}
\declarecommand{\libc}{{libc}}
\declarecommand{\libpthread}{{\tt libpthread.so}}
\declarecommand{\libdl}{{\tt libdl.so}}


% stats
\declarecommand{\linuxversion}{4.9}
\declarecommand{\graphenesyscallnum}{145}
\declarecommand{\linuxsyscallnum}{318}
\declarecommand{\palcallnum}{42}
\declarecommand{\hostsyscallnum}{50}
\declarecommand{\enclavecallnum}{28}

\declarecommand{\gipclines}{1,131}
\declarecommand{\sandboxmodlines}{888}
\declarecommand{\reflines}{3,568}
\declarecommand{\libclines}{606}


% miscs
\declarecommand{\sysvipc}{System V IPC}
\declarecommand{\lighttpd}{Lighttpd}
\declarecommand{\gcc}{GCC}
\declarecommand{\lmbench}{LMbench}
\declarecommand{\ab}{ApacheBench}
\declarecommand{\busy}{Bash}
\declarecommand{\sgxsdk}{Intel SGX SDK}

