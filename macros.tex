\makeatletter
\newcommand*{\declarecommand}{%
  \@star@or@long\declare@command
}
\newcommand*{\declare@command}[1]{%
  \provide@command{#1}{}%
  % \let#1\@empty % would be more efficient, but without error checking
  \renew@command{#1}%
}

\usepackage{xstring}

\newcommand{\projname}[1]{{\sc #1}}

\newcommand{\syscall}[1]{{\tt #1()}}
\newcommand{\code}[1]{{\tt #1}}
\newcommand{\funcname}[1]{{\tt #1()}}

% different kinds of seconds
\newcommand{\asec}{S}
\newcommand{\usec}{$\mu$S}
\newcommand{\msec}{mS}
\newcommand{\nsec}{nS}

% tunable symbols
\newcommand{\roughly}{$\sim$}

\newcommand{\palign}[2][]{%
    \IfEqCase{#2}{%
        {l}{\def\@@align{\raggedright}}%
        {r}{\def\@@align{\raggedleft}}%
        {c}{\def\@@align{\centering}}%
    }[\PackageError{palign}{Undefined option to palign: #2}{}%
          \def\@@align{}]%
    \@@align \arraybackslash #1%
}

% slangs

\newcommand{\libos}{libOS}
\newcommand{\liboses}{libOSes}
\newcommand{\Libos}{LibOS}
\newcommand{\Liboses}{LibOSes}

\newcommand{\loc}{LoC}

\declarecommand{\graphene}{\projname{Graphene}}
\declarecommand{\graphenesgx}{\projname{Graphene-SGX}}
\declarecommand{\civet}{\projname{Civet}}
\declarecommand{\intel}{Intel}
\declarecommand{\sgx}{SGX}
\declarecommand{\microkernel}{microkernel}
\declarecommand{\libos}{library OS}
\declarecommand{\liboses}{library OSes}
\declarecommand{\picoproc}{picoprocess}
\declarecommand{\picoprocs}{picoprocesses}
\declarecommand{\haven}{\projname{Haven}}
\declarecommand{\drawbridge}{\projname{Drawbridge}}
\declarecommand{\java}{Java}
\declarecommand{\microbench}{micro-benchmark}
\declarecommand{\sysvipc}{System V IPC}
\declarecommand{\us}{$\mu$S}
\declarecommand{\x}{$\times$}
\declarecommand{\pal}{PAL}
\declarecommand{\graphenesyscalls}{145}
\declarecommand{\linuxsyscalls}{318}
\declarecommand{\palcalls}{35}
\declarecommand{\nativecalls}{50}
\declarecommand{\gipclines}{1,131}
\declarecommand{\sandboxmodlines}{888}
\declarecommand{\reflines}{3,568}
\declarecommand{\libclines}{606}
\declarecommand{\interfacenum}{41}
\declarecommand{\light}{lighttpd}
\declarecommand{\gcc}{gcc}
\declarecommand{\lmbench}{LMbench}
\declarecommand{\ab}{ApacheBench}
\declarecommand{\busy}{Bash}
\declarecommand{\skylake}{Skylake}
