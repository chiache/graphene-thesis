\newpage
\section{Summary}

%Based on this study, we can draw several conclusions about the nature of Linux APIs.
%First, for any OS installation in our data set, the required API size
%is several times larger than the \syscallnum{} system calls in Linux, once one considers
%{\tt ioctl} opcodes and files under {\tt /proc}.  A solid two-thirds of system calls are indispensable.
%We show that a substantial range of system calls and other APIs are rarely or even never used.
%And the paper plots a rough guide for adding system calls to a Linux emulation layer or research prototype.
%
%We expect that this data set will be of use to researchers and developers for further analysis,
%Our methodology and tools can be easily applied to future releases and other distributions.
%%\note{about 1/4 pages}
%Our data set, tools, and other information are available at \projecturl{}.

Traditionally,
the routine procedure for system engineers or researchers
to make implementation decisions
is mostly based on their anecdotal knowledge,
which may be partially credible, but heavily skewed toward their preferred or familiar workloads.
The consequence of the lack of information
can be unfavorable for developers who are building innovative systems with legacy application support.
With the binary, bug-for-bug compatibility,
the developers fail to methodologically evaluate and reasonable about
the completeness of API implementation
in their system prototypes,
until the implementation is completed.
As produced by this study,
a principled approach for determining the priority of API implementation,
to enable more applications or more users
that can plausible use the system,
will guide the developers to make more rewarding decisions.

