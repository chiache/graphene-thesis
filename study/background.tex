\section{Background}
\label{sec:background}

\fixmedp{Refocus on our goals, assumptions, basic approach.  Sec 3 is the detailed metric}

\note{about 1.5 page}
\note{1. What is platform compatibility?}
\note{2. How is compatibility related with system interface?}
\note{3. What to improve interfaces?}
%{\bf Compatibility} is a property often mentioned in different areas such as operating system, micro-architectures or language runtimes.
%With compatibility, other entities are guaranteed to obtain expected services according to the specification they agree to.
%{\bf Platform Compatibility}, specifically mentioned in operating systems, guarantees applications to execute on the defined hardware,
%as long as they follow conventions such as APIs or executable formats.

{\bf Compatibility} ensures that other entities will receive services according to a pre-agreeed specification contract without having to make any changes.
The compatibility property is often mentioned in different areas such as operating system, micro-architectures or language runtimes.
{\bf Platform Compatibility}, specifically mentioned in operating systems, guarantees applications to execute on the defined hardware,
as long as they follow the convention such as APIs or executable formats.


Platform compatibility is essential for decoupling the development of operating systems and applications.
Modern computer systems are complex ecosystems that contain 
%plenty of 
many
applications and subsystems 
%for various
that serve different
purposes.
%When developers change or recreate an OS, preserving platform compatibility will guarantee that developers does not have to pay extra efforts to reimplement existing applications.
Platform compatibility guarantees that the application developers do not have to 
reimplement existing applications when the OS developers change or redesign the OS.
%Sometimes, a OS might not have full compatibility to other OSes, but it only takes {\em recompilation} of applications rather than reimplementation.
Occasionally, applications only need to be recompiled instead of reimplementing to execute an application from one OS to another. 
%These OSes can be consider having lesser platform compatibility, or {\em portability}.
These OSes are considered to have lower platform compatibility, or {\em portability}.
For example, a UNIX operating system like Linux, Solaris or BSD can port most of the applications to one another with the compiler's help.

For a given 
%ideal 
operating system, platform compatibility 
%can be 
is determined by the definition of its {\bf system interfaces}.
Irrespective of the OSes' internal logic, two different OSes are considered as platform compatible, if the applications get the same expected behavior while interacting with each of the OSes.
%The criteria is whether applications can always see the expected behaviors when they interact with the OS.
%As long as the agreed specifications have been perfectly implemented on the interfaces, platform compatibility is justified regardless of the OS's internal logics.
%In reality, there might be ambiguity in the definition of system interfaces, so that platform compatibility has to depend on internal behaviors as well.
In reality however, the platform compatibility may have to depend on internal behavior of OSes, because of possible ambiguity in the definition of system interfaces.
As the OS matures, like Linux, it minimizes the ambiguity of system interfaces
%A mature operating systems like Linux will have minimized the ambiguity,
and always provide a consistent system view to the applications.

%Therefore, platform compatibility can be often confirmed by testing every system interfaces against the specifications.
We can often confirm platform compatibility of an OS by testing its every system interface against the specification.
For instance, to confirm compatibility to Linux, developers can test every system call based on use cases described in the official Linux {\em Manpages}.
However, considering the size and complexity of system interfaces in an OS, it is often impossible to fully verify its platform compatibility.
As long as platform compatibility of an OS is never fully verified,
there is always a possiblity that special efforts need to be taken to port a few applications.
%the cost of porting any applications cannot be eliminated.

Besides API-like system interfaces like system calls,
applications also interact with the OS through other special system interfaces in the form of file systems, databases, or plugin runtimes such as {\em Proc pseudo-filesystem}, {\em FUSE framework} of UNIX, or  Microsoft Windows {\em Registry}.
To interact with these special system interfaces, applications often use a generic set of file-system-like operations,
but applications also need to comprehend the data type or semantics of the special system interface input and output. 
Therefore, these special system interfaces are often more complex than API-like system interfaces.

In general, system interfaces are defined as the boundary between OS kernels and the user space.
However, although located in the user space, some libraries, such as GNU Library C on UNIX, are ubiquitously used by applications, thus can be counted as system interfaces.
The development of these libraries is often closely coupled with the kernel, and their main purpose is to provide additional APIs without overly expanding the kernel code and interfaces.
Some APIs in the libraries are simply wrappers of kernel interfaces,
in order to provide more user-friendly interfaces, or mitigate the difference of kernel interfaces when the OS changes.

\fixmetsai{I need to think about what else to write}

\note{talk about ELF}

