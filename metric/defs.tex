\chapter{Formal Definitions}
\label{chap:defs}

%\fixmedp{The formal definition of importance in section 2.1 is hard to follow, in part because of the naming. The names of variables (Dependentapi, inst) don't make clear the types of the variables (a package, set of packages, set of API, instance, set of instance) etc. It would be helpful to reconsider the terminology to make the variable types more clear.
%In the equations in 2.2, explaining what E means would be helpful, as it is not obvious from the context. Also, should Weighted Compliance be a function of the instance in addition to the API?
%}

\section{\UsageMetric{}}
\label{sec:defs:usagemetric}

%\vspace{0.1in}
%{\noindent
%\fbox{\begin{minipage}{\linewidth}
%\setlength{\parindent}{-0.1in}
%\setlength{\leftskip}{0.1in}
%\setlength{\rightskip}{0.1in}
% Definition: {\bf \UsageMetric{}.} \\
%For a given API, the probability that an installation includes
%at least one application requiring the given API.
%at least one broken application
%(whose API footprint covers the target API),
%if the target API was removed.
%\end{minipage}}}
%\vspace{0.1in}

A system installation ($\mathtt{inst}$)
is a set of packages installed ($\{\mathtt{pkg}_1, \mathtt{pkg}_2, ..., \mathtt{pkg}_k \in \mathtt{Pkg}_\mathtt{all}\}$).
For each package ($\mathtt{pkg}$)  that can be installed by the installer,
we analyze every executable included in the package 
($\mathtt{pkg} = \{\mathtt{exe}_1, \mathtt{exe}_2, ..., \mathtt{exe}_j\}$), and
generate the API footprint of the package as:
\begin{align*}
{\mathtt{Footprint}}_{\mathtt{pkg}} = \{\mathtt{api} \in {\mathtt{API}}_{\mathtt{all}} \mid \text{\parbox{2.3in}{$\exists \mathtt{exe} \in \mathtt{pkg}$,
$\mathtt{exe}$ has usage of $\mathtt{api}$}}\}
\end{align*}

\noindent
The \usagemetric{} is calculated as the probability
that any installation includes
at least one package that uses an API;
i.e.,
the API belongs to the footprint of at least one package.
Using \osdist{}'s package installation statistics, one can calculate the
probability that a specific package is installed as:
\begin{align*}
Pr\{\mathtt{pkg} \in \mathtt{Inst}\} = \frac{\text{\# of installations including $\mathtt{pkg}$}}{\text{total \# of installations}}
\end{align*}

\noindent
Assuming the packages that use an API are  
$\mathtt{Dependents}_\mathtt{api} = \{\mathtt{pkg}|\mathtt{api} \in \mathtt{Footprint}_\mathtt{pkg}\}$.
\Usagemetric{} is the probability that at least one package
from $\mathtt{Dependents}_\mathtt{api}$ is installed on a random installation, which is calculated as follows:
\begin{align*}
&\mathtt{Importance}(\mathtt{api}) = Pr\{\mathtt{Dependent}_\mathtt{api} \bigcap \mathtt{Inst} \neq \emptyset\} \\
&= 1 - Pr\{\forall \mathtt{pkg} \in \mathtt{Dependent}_\mathtt{api}, \mathtt{pkg} \notin \mathtt{Inst}\} \\
&= 1 - \prod_{\mathtt{pkg} \in \mathtt{Dependent}_\mathtt{api}} Pr\{\mathtt{pkg} \notin \mathtt{Inst}\} \\
&= 1 - \prod_{\mathtt{pkg} \in \mathtt{Dependent}_\mathtt{api}} (1 - \frac{\text{\# of installations including $\mathtt{pkg}$}}{\text{total \# of installations}})
\end{align*}

\section{\CompatMetric{}}
\label{sec:defs:compatmetric}

%% An accurate metric for API compatibility must take into account at least two factors:
%% \begin{compactenum}
%% \item For each interface in the API, what are the applications that rely on it and could be affected if the interface is removed? ({\em application footprint})
%% \item For each application, how many installations of the system in the world has included it? ({\em application popularity})
%% \end{compactenum}


%% \fixmedp{Term for each call, vs system overall: API Importance and Effective Coverage?}

%% To accurately evaluate compatibility, we design a metric that is quantifiable, \fixmetsai{one more word here}, and easy to interpret.

\begin{comment}
We defined {\bf platform compatibility} as "{\em the probability of porting any installation of an OS distribution onto the target OS without any effort}".
The definition of an {\em installation} is a combination of application setup on a standalone OS instance.
An Installation can exist on physical machines,
or any machines of generalized sense such as a virtual machines, containers or subsystems.
In our model, installations represent customers of the OS, who are considered equal when providing any services.
\end{comment}

%\vspace{0.1in}
%\fixmetsai{Find a good name for our metric} 
%{\noindent
%\fbox{\begin{minipage}{\linewidth}
%\setlength{\parindent}{-0.1in}
%\setlength{\leftskip}{0.1in}
%\setlength{\rightskip}{0.1in}
%Definition: {\bf \CompatMetric{}.} \\
%For a target system, the fraction of applications supported,
%weighted by the popularity of these applications.
%\end{minipage}}}
%\vspace{0.1in}

\Compatmetric{} is used to evaluate the relative compatibility on a system that
supports a set of APIs ($\mathtt{API}_\mathtt{Supported}$).
For a package on the system, we define it as supported if 
if every API that the package uses is in the supported API set.
In other words, a package is supported if it is a member of the following set:
\begin{align*}
\mathtt{Pkg}_\mathtt{Supported} = \{\mathtt{pkg} | \mathtt{Footprint}_\mathtt{pkg} \subseteq \mathtt{API}_\mathtt{Supported}\}
\end{align*}

\noindent
Using \compatmetric{}, one can estimate the fraction of packages in an installation that end-users can expect a target system to support.
For any installation
that is an arbitrary subset of available packages
($\mathtt{Inst} = \{\mathtt{pkg}_1, \mathtt{pkg}_2, ..., \mathtt{pkg}_k\} \subseteq \mathtt{Pkg}_\mathtt{all}$),
\compatmetric{} is the expected value of
the fraction in any installation ($\mathtt{Inst}$)
that overlaps with the supported packages ($\mathtt{Pkg}_\mathtt{Supported}$):
\begin{align*}
\mathtt{Weighted Completeness}(\mathtt{API}_\mathtt{Supported}) =
E(\frac{|\mathtt{Pkg}_\mathtt{Supported} \bigcap \mathtt{Inst}|}{|\mathtt{Inst}|}) 
\end{align*}
where $E(\mathtt{X})$ is the expected value of $\mathtt{X}$.

Because we do not know which packages are installed together,
except in the presence of explicit dependencies,
%Lacking of the distribution data of each package, we approximate the value of \compatmetric{}
%by ignoring {\em higher order} estimators
%of the expectation, assuming events of installing each package to be independent.
we assume package installation events are independent.
Thus, the approximated value of \compatmetric{} is:
\begin{align*}
\frac{E(|\mathtt{Pkg}_\mathtt{Supported} \bigcap \mathtt{Inst}|)}{E(|\mathtt{Inst}|)}
\sim \dfrac{\sum_{\mathtt{pkg} \in \mathtt{Pkg}_\mathtt{Supported}} (\dfrac{\text{\# of installations including $\mathtt{pkg}$}}{\text{total \# of installations}})}{\sum_{\mathtt{pkg} \in \mathtt{Pkg}_\mathtt{all\hspace{0.21in}}} (\dfrac{\text{\# of installations including $\mathtt{pkg}$}}{\text{total \# of installations}})}
\end{align*}
