\section{Implementation Details}
\label{sec:syspop:framework}

This section provides additional implementation details of our analysis framework. % for mining the API footprint for ELF executables and libraries.

%Analyzing the footprint of an applications requires interpretation of its behavior. As discussed in section~\ref{sec:measure:analysis}, we choose to build our framework by using {\bf static analysis} technique instead of dynamic analysis. The reason is that static analysis can trace potential executions of the applications, regardless of the runtime coverage of the code.

Our analysis is based on disassembling binaries inside each application package, using the standard {\tt objdump} tool.
This approach eliminates the need for source or recompilation, and can handle closed-source binaries.
We implement a simple call-graph analysis to detect system calls reachable from the binary entry point ({\tt e\_entry} in ELF headers). 
%\fixmedp{You do actually parse the elf header for e\_entry, right?}
%In our study, we only count executables during measuring both \usagemetric{} and \compatmetric{}.
We search all binaries, including libraries, for system call instructions ({\tt int 0x80}, {\tt syscall} or {\tt sysenter}) or calling the {\tt syscall} API of \libc{}.
We find that the majority of binaries --- either shared libraries or executables --- do not directly choose system calls, but 
rather use the GNU C library APIs.
Among 66,275 studied binaries, only 7,259 executables and 2,752 shared libraries issue system calls.

% system calls. Therefore, analyzing the libraries that an executable depends on is necessary for retrieving a complete set of API footprint.


%, to predict its runtime behaviors.
%The framework requires no scanning of the source code, or recompiling of the applications with additional debug symbols.
%Consider the size and complexity of Ubuntu repositories,
%static analysis of ELF binaries is certainly easier to automate, and is able to analyze the applications that are close-sourced.

%To trace the footprint of applications, it require {\em Call-Graph Analysis} to rip off potential control flows that can eventually lead to usage of system API.
%Challenges of analyzing call-graph is a known issues for researchers in many area such as security, robustness, etc.

Our call-graph analysis allows us to only select system calls that are actually used by the application, not all the system calls that appear in \libc{}.
Our analysis takes the following steps:
\begin{compactitem}
\item For a target executable or a library, generate a call graph of internal function usage.
\item For each library function that the executable relies on, identify the code in the library that is reachable from each entry point called by the executable.
%trace coverage of code in the binary, and generate the minimal subset of the library's footprint that the function could actually link to.
\item For each library function that calls another library call, recursively trace the call graph and aggregate the results. 
\end{compactitem}
\vspace{10pt}

Precisely determining all possible call-graphs from static analysis is challenging.
Unlike other tools built on 
call-graphs, such as control flow integrity (CFI), our framework can tolerate the error caused by over-approximating the analysis results.
For instance, 
programs sometimes make function call based on a function pointer passed as an argument by the caller of the function. 
Because the calling target is dynamic, it is difficult to determine at the  call site.
Rather, we track sites where the function pointers are assigned to a register, such as using the {\tt lea} instruction with an address
relative to the current program counter.
%Instead of precisely tracking calling target at the {\tt call} instruction, 
%A function pointer is mostly assigned by {\tt lea} instruction to generate a relative address to the current program counter. 
This is an over-approximation because, rather than trace the data flow, we assuming that a function pointer assigned to a local variable will be called.
This analysis could be more precise if it included a data flow component. 
%\fixmedp{As an aside, this level of data flow analysis shouldn't be that hard, right?  Not  that it matters, except for the thesis/journal version.}

We also hard-code for a few common and problematic patterns.
For instance, we generally assume that the registers that pass a system call number to a system call,
or an opcode to a vectored system call, are not the result of arithmetic in the same function.
We spot checked this assumption, but did not do the data flow analysis to detect this case.

%must trace the register values such as {\tt RAX/EAX} for system call number or {\tt RBX/EBX} for parameter to vectored system calls. 

%Based on {\em Case-by-case study}, we assume no arithmetic but direct assignment of integers is possible when the program is issuing a system call. 

%If our analysis cannot precisely determine whether an API is used, we 
%err on the side of assuming an API is used.

%In the case of a run

%allow slightly enlarge the traced footprint of application, to gain simplicity for complex call-graph analysis corner cases. 
%For the specific run-time scenario of the application that can hard to interpret, we do a {\bf Case-by-Case study} to find out strategy of analyze the footprint practicably.

%  that call-graph analysis is complex for generating the accurate control flow of the run time. Fortunately, unlike other studies that relies on 

%In our study, we specifically call it {\bf Over-approximation}. 


%The following is a few example of using {\bf Over-approximation} and {\bf Case-by-Case study} to simplify call-graph analysis:
%\begin{compactenum}
%\item 
%\item 
%\end{compactenum}

Finally, the last mile of the analysis is to recursively aggregate footprint data. We insert all raw data into a {\tt Postgresql} database, and 
use recursive SQL queries to generate the results. 
To scan through all \packagenum{} packages in the repository, collect the data, and generate the results takes roughly three days.

   



