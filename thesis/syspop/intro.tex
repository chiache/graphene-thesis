%\section{Motivation}
%\label{sec:syspop:motivation}

%\note{about 1 page}
%\note{start with a strong topic centence}

When implementing the system APIs and abstractions,
system developers routinely make design choice based on
what they believe to be the important and unimportant features of the system.
%Systems engineers and researchers routinely make design choices based on 
%what they believe to be the common and uncommon behaviors of a system.
%For instance, one recent project optimized the {\tt stat} and {\tt open} system calls
%at the expense of {\tt rename} and {\tt chmod}~\citep{tsai15dcache}.
%In the case of a general-purpose OS, determining
%exactly what the common case is can be challenging.
However, a developer's view of what APIs are important
may be skewed heavily towards that developer's preferred workloads.
%the choice of what features matter skews heavily towards
%workloads particular developers use.
Similarly, developers struggle to evaluate the impact of a 
change that affects backward-compatibility,
primarily because of a lack of metrics.
Deprecating an API is often a lengthy process, wherein
users are repeatedly warned 
and eventually some applications may still be broken.
%For example, a range of security problems arise from the ill-specified behavior of the {\tt signal} system call~\citep{zalewski01signals,cert-signals}.
%Despite 15 years of warnings to move to the more secure {\tt sigaction} call,
%{\tt signal} has not been removed from 32-bit x86 Linux,
%% (it was simply not added to x86\_64),
%because many legacy applications use {\tt signal}.
Eliminating or replacing needless, problematic APIs 
can be good for security, efficiency, and maintainability of OSes,
but in practice this is difficult for OS developers to do without tools to analyze API usage.

Many experimental operating systems add a rough Unix or Linux compatibility 
layer to increase the number of supported applications~\citep{zeldovich+histar, aviram10determinator, xax, appavoo2003providing}.
Such systems generally support a fraction of Linux system calls,
often just enough 
to run a few target workloads.
One metric for compatibility or completeness of a new feature
is the count of supported system APIs~\citep{tsai14graphene, TxOS, baumann13bascule, bergan10dos}.
%Unless all system APIs are used equally,
System call counts do not accurately
estimate the fraction of applications or users that could plausibly use the system.
OS researchers would benefit from
the ability to translate
a set of supported system calls 
to the fraction of applications that
can be directly supported without recompilation.  Similarly, it is useful 
to know which additional APIs would enable the largest range of additional applications to run on the system.
%OS maintainers can use such a metric to focus efforts to deprecate or extend an existing API.
%Finally, users may }
In order to indicate general usefulness, a 
good compatibility metric should factor in 
the fraction of users whose choice of applications can be
completely supported on a system.
%A useful compatibility metric should further 
%identify 
%supported applications
%into
%\rev{rewrite}{An even more critical question is
%whether developers can further deduce the fraction of Linux users %that could realistically
%use the experimental system.}
%As we show, this metric
%can both over- and under-approximate the degree to which a prototype
%can meet the needs of real users.

%Simply counting supported system calls or APIs is a very rough number, and researchers need 
%a better metric to evaluate the degree to which their prototype is applicable to real users.

At the root of these problems is a lack of data sets and analysis of
how system APIs are used in practice.
System APIs are simply not equally important: 
some APIs are used by popular libraries and, thus, by essentially every application.
Other APIs may be used only by applications that are rarely installed.
Evaluating compatibility is fundamentally a measurement problem.
%System call counts do not capture this nuance.
%supporting a system call that is commonly used in popular applications
%is more valuable than
%supporting one that is not used at all.
%Users only care whether their software installations will be broken if they adopt a new system prototype.
%System 
%The gap between the count of system APIs
%implemented by a prototype and the fraction of applications and users
%that could adopt the prototype is
%the primary reason that makes the former an unusable metric.  

%This paper bridges the gap between 
%data that is easy for system builders to measure and the 
%metrics they need, contributing 
%%existing metric and the one we actually need, with
% a methodology and thorough study of API usage in \osarch{} \osdist{}.
%Our study statically analyzes all executable and shared library binaries from all
%\packagenum{} packages in the \osdist{} repository, in order
%to identify the system API ``footprint'' of each binary.
%%\rev{emphasize this}{including both executables and shared libraries}.
%%Our analysis only requires application binaries.
%%all the system calls each application can issue (the system call ``footprint'').
%This paper combines the footprint data with data about how frequently 
%each package is installed, which is measured from the Ubuntu and Debian ``popularity contest'' survey data~\citep{ubuntu-popularity, debian-popularity}.
%By combining these data sources, the paper contributes metrics
%which weigh the usage of each system API by estimated usage in real-world installations.


%We will release the data set and analysis tools with this paper.


%% dp: This is pretty repetitive with the text above.  I propose putting it lower in the interest of brevity
%%% \begin{revs}{List who will care}
%%% The following are some examples in which either OS developers or users will
%%% benefit from the data sets and metrics described by this paper:
%%% \begin{compactitem}
%%% \item Maintainers who have to wait for a decade before confirming that an API is close to retirement.
%%% \item Users who need to evaluate the thoroughness of system compatibility before adopting a new OS prototype.
%%% \item Developers who have to assess the impact of an important design decision about system APIs.
%%% \end{compactitem}
%%% \end{revs}

%This paper contributes a data set and analysis tool
%that can answer several practical questions for systems researchers.
%For instance: in a given prototype, which missing APIs would increase the range of supported applications?
%Or, if a given system API is optimized, what widely-used applications
%would likely benefit?  We expect that the ability to match evaluation workloads to modified or supported system APIs
%will be particularly useful.  
%Similarly, this data and toolset can help 
%OS maintainers evaluate the impact of an API change on applications, and can help users
%evaluate whether a prototype system is suitable for their needs.

%We expect similar benefits for developers, such as automatically generating 
%system call level sandboxes~\citep{seccomp}, finding and reaching out to developers of applications using a deprecated API,
%and optimizing the in-memory layout of standard libraries, such as \libc{}.
%For instance,  out of \syscallnum{} system calls on \osarch{} Linux,
%20 system calls are used by no packages in Ubuntu,
%and an additional 58 are used by applications installed on less than 1\% of surveyed systems.

%\fixmedp{Maybe pontificate more about why compat. as a binary property isn't good enough}

%The contributions of this work are as follows:
%\begin{compactitem}
%\item An approach to measuring platform compatibility, suitable for evaluating the relative completeness of prototype systems.
%Rather than considering compatibility a binary property (``will something break?''),
%%yet for prototypes, which are necessarily in-progress,
%we use a fractional metric 
%(``how many programs will not break?'')
%which is better suited to measuring  the progress of 
%a prototype.
%\item A comprehensive data set of current API usage in \osversion{}.  
%\item Analysis and a range of insights into current API usage patterns. For instance, we identify an efficient path to implementing
%a new Linux compatibility layer, maximizing the additional applications per system call.  We also identify
%that usage of many APIs is similarily distributed: some are widely used, and there is a sharp drop with a very long tail
%of rarely or never-used APIs.  As an example, nearly 40\% of libc APIs are used by less than one percent of applications
%on a typical installation. %\fixmedp{check this}
%%, identifying new opportunities for system designers and important considerations for the evaluation of OS prototypes.
%%% meh - this isn't that novel---the data is key
%%\item An efficient static analysis framework can generates API usage profiles from binaries, with high precision and reasonable efficiency.  For instance, our system can analyze the entire Ubuntu repository on a 48-core machine in 2 days.\fixmedp{right?}
%\end{compactitem}
%
