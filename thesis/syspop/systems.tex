\section{Linux Systems and Emulation Layers}

This section uses \compatmetric{} to evaluate systems or emulation layers with partial Linux compatibility.
We also evaluate several \libc{} variants for their degree of completeness against the APIs exported by \glibc{} 2.21.

\subsection{\CompatMetric{} of Linux Systems}

\begin{table}[t]
\centering
\small
\begin{tabular}{m{1.4in}>{\centering}m{0.5in}>{\raggedright\arraybackslash\footnotesize}m{2.9in}>{\raggedleft\arraybackslash}m{1in}}
\toprule
Systems & \# & Suggested APIs to add & \Compatmetric{} \\
\midrule
\addlinespace
User-Mode-Linux \kernelversion{} & 284 & {\tt name\_to\_handle\_at}, {\tt iopl}, {\tt perf\_event\_open} & 93.1\% \\
\addlinespace
\hline
\addlinespace
L4Linux 4.3 & 286 & {\tt quotactl}, {\tt migrate\_pages}, {\tt kexec\_load} & 99.3\% \\
\addlinespace
\hline
\addlinespace
FreeBSD-emu 10.2 & 225 & {\tt inotify}*, {\tt splice}, {\tt umount2}, {\tt timerfd}* & 62.3\% \\
\addlinespace
\hline
\addlinespace
Graphene  & 143 & {\tt sched\_setscheduler}, {\tt sched\_setparam} & 0.42\% \\
Graphene\textsuperscript{\P} & 145 & {\tt statfs}, {\tt utimes}, {\tt getxattr}, {\tt fallocate}, {\tt eventfd2} & 21.1\% \\
\hline
\end{tabular}
\caption[Evalution of Linux-compatible system or emulation layers]
{\Compatmetric{} of several Linux systems or emulation layers. For each system, we manually identify the number of supported system calls (``\#''), and calculate the \compatmetric{} (``W.Comp.'') . Based on \usagemetric{}, we suggest the most important APIs to add.
(*: system call family.
\P: Graphene after adding two more system calls.) }
\label{tab:linux-compat}
\end{table}

To evaluate the \compatmetric{} of Linux systems or emulation layers,
the prerequisite is to identify the supported APIs of the target systems.
Due to the complexity of Linux APIs and system implementation,
it is hard to automate the process of identification.
However, OS developers are mostly able to maintain such a list based on the internal knowledge. 

We evaluate the \compatmetric{} of four Linux-compatible systems or emulation layers:
User-Mode-Linux~\citep{user-mode-linux}, L4Linux~\citep{hartig97mu}, FreeBSD emulation layer~\citep{freebsd-emu}, and Graphene library OS~\citep{tsai14graphene}.
For each system, we explore techniques
to help identifying the supported system calls,
based on how the system is built.
For example, User-Mode-Linux and L4Linux
are built by modifying the Linux source code,
or adding a new architecture to Linux.
These systems will define architecture-specific system call tables,
and reimplement {\tt sys\_*} functions in the Linux source
that are originally aliases to {\tt sys\_ni\_syscall}
(a function that returns {\tt -ENOSYS}). 
Other systems, like FreeBSD and Graphene,
are built from scratch,
and often maintain their own system call table structures,
where unsupported systems calls
are redirected to dummy callbacks.

%Because the evaluation is simply a proof-of-the-concept,
%we count only system calls,
%omitting other APIs such as vectored system calls and pseudo-files. 

Table~\ref{tab:linux-compat} shows \compatmetric{},
considering only system calls.
The results also identify the most important system calls
that the developers should consider adding. 
User-Mode-Linux and L4Linux both have a \compatmetric{} over 90\%,
with more than 280 system calls implemented.
FreeBSD's  \compatmetric{} is  62.3\% because it is missing some less
important system calls
such as {\tt inotify\_init} and {\tt timerfd\_create}.
Graphene's \compatmetric{} is only 0.42\%.
We observe that the primary culprit is 
scheduling control; by adding two scheduling system calls,
Graphene's \compatmetric{} would be 21.1\%.

\subsection{\CompatMetric{} of \Libc{}}

\begin{table}[t]
\center{
\small
\begin{tabular}{>{\raggedright\arraybackslash}m{1in}>{\raggedleft\arraybackslash}m{0.4in}>{\raggedright\arraybackslash\footnotesize}m{2.4in}>{\raggedleft\arraybackslash}m{0.9in}>{\raggedleft\arraybackslash}m{0.9in}}
\toprule
\Libc{} variants & \# & Unsupported functions (samples) & {\footnotesize \Compatmetric{}} & {\footnotesize \Compatmetric{}} {\scriptsize (normalized)} \\
\midrule
\addlinespace
eglibc 2.19 & 2198 & None & 100\% & 100\% \\
\addlinespace
\hline
\addlinespace
uClibc 0.9.33 & 1867 & {\tt \_\_uflow}, {\tt \_\_overflow}  & 1.1\% & 41.9\% \\
\addlinespace
\hline
\addlinespace
musl 1.1.14 & 1890 & {\tt secure\_getenv}, {\tt random\_r} & 1.1\% & 43.2\% \\
\addlinespace
\hline
\addlinespace
dietlibc 0.33 & 962 & {\tt memalign}, {\tt stpcpy}, {\tt \_\_cxa\_finalize}  & 0\% & 0\% \\
\addlinespace
\hline
\end{tabular}
}
\caption[\Compatmetric{} of \libc{} variants]
{\Compatmetric{} of \libc{} variants. For each variant, we calculate \compatmetric{} based on symbols directly retrieved from the binaries,
and the symbols after reversing variant-specific replacement (e.g.,{\tt printf} becomes {\tt \_\_printf\_chk}).}
\label{tab:libc-compat}
\end{table}

This study also uses \compatmetric{} to evaluate the compatibility of several \libc{} variants --- eglibc~\citep{eglibc}, uClibc~\citep{uclibc}, musl~\citep{musl} and dietlibc~\citep{dietlibc} --- against \glibc{},
listed in Table~\ref{tab:libc-compat}.
We observe that, if simply matching exported API symbols, only eglibc is directly compatible to \glibc{}.
Both uClibc and musl have a low \compatmetric{}, because \glibc{}'s headers replace a number of APIs with safer variants at compile time, using macros.
For example, \glibc{} replaces {\tt printf} with {\tt \_\_printf\_chk}, which performs an additional check for stack overflow.
%Because these new APIs often have the same interface as the old ones,
%we assume other \libc{} variants can easily
%reverse the API replacement during symbol resolution.
After normalizing for this compile-time API replacement, both uClibc and musl are at over 40\% \compatmetric{}.
In contrast, dietlibc is still not compatible with most binaries linked against \glibc{} --- if no other approach is taken to improve its compatibility.
The reason of low \compatmetric{} is that dietlibc does not implement many ubiquitously used \glibc{} APIs such as {\tt memalign} (used by 8887 packages) and {\tt \_\_cxa\_finalize} (used by 7443 packages).
