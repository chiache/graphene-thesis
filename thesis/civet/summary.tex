\section{Summary}
\label{sec:civet:summary}

Partitioning an application into trusted and untrusted components
provides more plausibility for the soundness of the sensitive execution in enclaves,
than isolating the whole application.
Similar to the use cases demonstrated in this chapter,
in many applications,
developers can identify a large fraction of execution to be irrelevant to
the target of isolation.
\term{\sysname{}} allows legacy applications written in high-level,
managed languages such as \java{},
to isolate a piece of the execution in the \sgx{} enclave.
Unlike the non-partition model,
in which the system is only responsible for translating system APIs to untrusted interface,
partitioning a \java{} applications requires generate a clean separation
of trusted and untrusted classes and object,
and a framework to dynamically load and interface objects in enclaves.
\sysname{} essentially enables developers to painlessly partition legacy \java{} applications into enclave,
allowing more flexible use cases of the hardware protection.

At a high level,
Language-based and hardware protection are both valuable
building blocks for secure applications.
\sysname{} shows a constructive synthesis of both technologies
--- combining their strengths
and mitigating downsides of either when used in isolation.
Although this work focuses on Java and Intel SGX as a specific case study,
we believe the technique is more generic
and can apply to
other managed languages and hardware isolation platforms.
%In future work, we plan to incorporate additional analyses
%to further reduce the TCB size within an enclave, and to provide
%stronger integrity protections against adversarial input.
\term{This work is still in progress, and will be completed for the fulfillment of the thesis.}

%% With \systemname{}, we demonstrate how hardware and language-level protection
%% can be integrated,
%% to eliminate vulnerabilities of both approaches.
%% \systemname{} provides guided partitioning for \java{} applications,
%% transparently porting security-sensitive classes
%% into an \intel{} \sgx{} enclave, so that their execution can be isolated
%% from malicious system stacks.
%% The classes in the enclave are harden by \java{}-based language protection,
%% such as type-checking and information flow control,
%% preventing any secret being leaked due to vulnerabilities in the isolated classes.
%% We show several use cases of the framework, for protecting
%% both class implementation and output data,
%% and keeping the development and deployment cost minimal.


