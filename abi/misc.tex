\papersubsection{Miscellaneous}
\label{sec:abi:misc}


Besides managing host resources, 
some miscellaneous features is needed from the host OSes.
Some features, such as exception handling, are specific to the implementation of Linux functionality
in the guest.
%by the library OS
%for implementing functionality specific to Linux.
This section lists these miscellaneous \hostapis{} defined in the host ABI.



\papersubsubsection{Exception handling}



The exception handling in the host ABI
is strictly designed for returning hardware exceptions,
or failures inside the PAL.
The host ABI allows the guest to specify a {\bf handler},
which the execution will be redirected to, when a specific exception is triggered.
%To have a design independent from the host,
%the handler will receive an object,
%for temporarily storing the state of execution,
%and a data structure
%specifying the information regarding the triggered exception,
%such as the register values in the faulting context.
The feature of assigning handlers to specific exceptions
grants a guest the ability of recovering from hardware or host OS failures.




\begin{paldef}
typedef void (*EXCEPTION_HANDLER)
            (void *exception_obj, EXCEPTION_INFO *info);
\end{paldef}



The host ABI defines \palkeyword{EXCEPTION_HANDLE} as the data type of a valid handler function.
A valid handler accepts two arguments.
The first is an exception object, as an opague pointer
which the host OS maintains to store a host-specific state regarding the exception.
The second is a piece of exception information that
is revealed to the exception handler.
The content of the exception information is defined as follows:

\begin{paldef}
typedef struct {
    u8  exception_code;
    u64 fault_addr, registers[REGISTER_NUM];
} EXCEPTION_INFO;
\end{paldef}

The \palkeyword{EXCEPTION_INFO} data structures consists of three fields.
\palkeyword{exception_code} specifies the type of exception.
\palkeyword{fault_addr} specifies the address that triggers a memory fault, or an illegal instruction.
\palkeyword{registers} returns the value of all \graphenearch{} general-purpose registers
when the exception is raised.
The exception code can be one of the following values:
\begin{compactitem}
\item \palkeyword{MEMFAULT}: a protection or segmentation fault.
\item \palkeyword{DIVZERO}: a divide-by-zero fault.
\item \palkeyword{ILLEGAL}: an illegal instruction fault.
\item \palkeyword{TERMINATED}: terminated by the host.
\item \palkeyword{INTERRUPTED}: interrupted by \palcall{ThreadInterrupt} (defined in Section~\ref{sec:abi:thread}).
\item \palkeyword{FAILURE}: a failure in the host ABI.
\end{compactitem}

When an exception is raised,
the current execution is interrupted and redirected to the assigned handler function.
The handler function can try to recover the execution,
based on the information given in the \palkeyword{EXCEPTION_INFO} data structure.
For example, a handler function
can print the interrupted register values to the terminal.
Once a handler function finishes processing the exception, it can return to the original execution,
by calling \palcall{ExceptionReturn} with the exception object given by the host.


\begin{paldef}
void ExceptionReturn     (void *exception_obj);
\end{paldef}

The host ABI defines \palcall{ExceptionReturn} to keep the semantics of exception handling clear and flexible
across host OSes.
A handler function does not assume that it can return to the original execution
using the \code{ret} or \code{iret} instruction.
Instead, a handler function must explicitly call \palcall{ExceptionReturn},
so that the host can destroy the frame that belongs to the handler function,
and return to the interrupted frame.
Also, \palcall{ExceptionReturn} can update the register values pushed to the interrupted frame,
based on the \palkeyword{registers} field in \palkeyword{EXCEPTION_INFO}.



\begin{paldef}
bool ExceptionSetHandler (u8 exception_code,
                          EXCEPTION_HANDLER handler);
\end{paldef}

\palcall{ExceptionSetHandler} assigns a handler function to a specific exception, based on the given \palkeyword{exception_code}.
The assignment of exception handlers
applies to every threads in the same process.
If \palcall{ExceptionSetHandler} is given a null pointer as the handler,
it cancels any handler previously assigned to the exception.
If no handler is ever assigned to a specific exception,
the default behavior of handling the exception is to kill the whole process.
%Once a handler finishes handling an exception,
%it must call \palcall{ExceptionReturn} to return to the execution.







%\papersubsubsection*{Other miscellaneous features}



\papersubsubsection{Querying the system time}


\begin{paldef}
u64 SystemTimeQuery (void);
\end{paldef}


\palcall{SystemTimeQuery} returns the current system time as
the number of microseconds passed since the Epoch, 1970-01-01 00:00:00 Universal Time (UTC).
Querying the system time
requires the host to have a reliable time source.
A common reliable, time source on \graphenearch{}
is a system timer incremented by the hardware alarm interrupts~\cite{love10linux-kernel},
combined with the Time Stamp Counter (TSC), a CPU counter tracing the number of cycles since the system reset.
\palcall{SystemTimeQuery} exports a reliable, simple time source
to the guest, based on the calculation of any arbitrary time sources used in the host.




\papersubsubsection{Reading random bits}



\begin{paldef}
u64 RandomBitsRead (void *buffer, u64 size);
\end{paldef}


\palcall{RandomBitsRead} fills the given buffer with data read from the host random number generator (RNG).
If \palcall{RandomBitsRead} successfully reads up to the number of bytes
specified by \palkeyword{size}, it returns the number of bytes that are actually read.
Based on the host random number generator,
\palcall{RandomBitsRead} may block until there is enough entropy for generating the random data.


The purpose of \palcall{RandomBitsRead} is to leverage the hardware random number generators,
either on-chip or off-chip.
For example, recent Intel and AMD CPUs support the \code{RDRAND} instruction,
which generates random bytes based on an on-chip entropy source.
Other hardware RNGs also exist, mostly based on thermodynamical or photoelectric patterns of the hardware.
\graphene{} only requires each host to export
one trustworthy source of random data,
such as \code{/dev/random}, a pseudo-device in POSIX.




%\papersubsubsection*{Flushing the instruction cache (optional)}
%
%\begin{paldef}
%void ICacheFlush (void *addr, ulong size);
%\end{paldef}
%
%
%On some architectures, such as MIPS, the instruction cache inside the CPU 
%has to be flushed whenever the application code is changed.
%Both \win{} and Linux contain system API for flushing the instruction cache.
%Inherited from \drawbridge{},
%\graphene{} defines a function called \palcall{ICacheFlush} in the host ABI,
%for architectures which requires such a feature.
%On \graphenearch{}, this function is not implemented,
%because the architecture
%does not require the instruction cache to be explicitly flushed.
%Still, Graphene keeps the definition
%for future porting to other architectures.
