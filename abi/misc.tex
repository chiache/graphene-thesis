\subsection{Miscellaneous}
\label{sec:abi:misc}


Besides managing host resources, such as memory or I/O, some miscellaneous features needs to be provided by the host.
Especially, some features, such as exception handling, are required
by the library OS
for implementing functionality specific to Linux.
This section lists these miscellaneous functions in the host ABI.



\subsubsection*{Exception handling}



The exception handling in the host ABI
is strictly designed for returning hardware exceptions,
or failures inside the PAL,
to the library OS.
The library OS specifies a {\bf handler} function which the host ABI
jumps to, when a certain exception is triggered.
To have a design independent from the host,
the handler will receive an object,
for temporarily storing the state of execution,
and a data structure
specifying the information regarding the triggered exception,
such as the register values in the faulting context.





\begin{paldef}
typedef void (*EXCEPTION_HANDLER)
            (void *exception_obj, EXCEPTION_INFO info);
\end{paldef}


The library OS specifies the handler
using \palcall{ExceptionSetHandler}.
The handler is assigned to the whole process,
instead of individual threads.
The function also
cancels out any handler previously assigned to the exception.
If no handler is ever assigned to a specific exception,
the PAL kills the whole process when the exception happens.
Once a handler finishes handling an exception,
it must call \palcall{ExceptionReturn} to return to the execution.


\begin{paldef}
bool ExceptionSetHandler (uint exception,
                          EXCEPTION_HANDLER handler);
void ExceptionReturn     (void *exception_obj);
\end{paldef}




\subsubsection*{Other miscellaneous features}




\begin{paldef}
ulong SystemTimeQuery (void);
\end{paldef}





\begin{paldef}
ulong RandomBitsRead (void *buffer, ulong size);
\end{paldef}




\begin{paldef}
ulong ICacheFlush (void *addr, ulong size);
\end{paldef}
