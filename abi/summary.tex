\section{Summary}
\label{sec:abi:summary}

%The host ABI of \graphene{}
%defines a set of basic OS functions, which must be implemented in the host.
%The host ABI redefines the division of labor
%between host and guest;
%the host ABI decouples hardware resource management and security isolation,
%from API implementation for applications.


\fixmedp{I think related goals are to show that 1) the ABI can be implemented relatively easily on most hosts. 2) that it is sufficient to run a lot of Linux applications.}
The host ABI consists of a sufficient set of simple, UNIX-like OS features.
The goal of defining the host ABI
is to ensure that the host ABI can be implemented relatively easily on most hosts,
and simultaneously, to expose enough host abstractions for developing a library OS that runs a lot of Linux applications.
Functions in the host ABI,
or so-called \hostapis{},
either manage a ubiquitous hardware resource, such as pages or CPU cycles,
or encapsulate an idiosyncratic feature of the host OSes, such as scheduling primitives or exception handling.
For most of the \hostapis{}, system API with similar functionality and semantics
can be found on most host OSes;
Few exceptions that are limited to be implemented on certain host OSes
(e.g., setting FS/GS registers on OSX/Windows) are defined as optional features,
so that the library OS can be prepared with coping strategies.


