\papersection{Summary}
\label{sec:abi:summary}

%The host ABI of \graphene{}
%defines a set of basic OS functions, which must be implemented in the host.
%The host ABI redefines the division of labor
%between host and guest;
%the host ABI decouples hardware resource management and security isolation,
%from API implementation for applications.


\fixmedp{I think related goals are to show that 1) the ABI can be implemented relatively easily on most hosts. 2) that it is sufficient to run a lot of Linux applications.}
\Thehostabi{} consists of a sufficient set of simple, UNIX-like OS features.
The goal of defining \thehostabi{}
is to ensure host abstractions that are easy to port on various host OS but also sufficient for developing a library OS with rich features.
Individual \hostapi{},
either manages a common hardware resource, such as memory pages or CPUs,
or encapsulates a host OS feature, such as scheduling primitives or exception handling.
For most of the \hostapis{}, system API with similar functionality and semantics
can be found on most host OSes;
Few exceptions
(e.g., setting segment registers)
which are challenging to port on certain host OSes (e.g., Windows)
are optional and require workarounds in the \libos{}.


