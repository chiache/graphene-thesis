\papersubsection{Stream I/O}
\label{sec:abi:streams}



\issuedone{1.2.a}{Discuss resource management at host level (I/O)}
%I/O is part of the foundation
%of an OS, to allow an application to interact with
%other machines, users, applications, or system software.
An OS typically supports three types of I/O:
%An application requires interaction with the world during its execution, using I/O devices.
%I/O is a common feature of almost every OSes.
%The typical I/O needed in an application
%can be catagorized into three types:
(1) {\bf storage}, for externalizing data to a permanent store;
(2) {\bf network}, for exchanging data with another machine over the internet;
(3) {\bf RPC} (remote procedure call),
for connecting concurrent applications or processes.
A host OS must contain these I/O abstractions
and manages the related I/O devices such as storage and network devices.
%Therefore, unless an I/O device is virtualized and dedicated to an application or a guest,
%a host OS must take a major role in I/O management;
%for the least, a host OS has to share the resources among multiple applications or guests,
%and contain the drivers to interface with the I/O devices.
It is also important to share these I/O abstractions
among multiple processes of an application.




%externalizing data to outside of the machines (i.e., networking);
% (i.e., storage);
%and  (i.e., remote-procedure call).
%A modern OS may define several abstractions
%for each types of I/O; for example, a file in Linux can be read using several system calls,
%including \syscall{read}, \syscall{pread}, and \syscall{readv};
%RPC in Linux is based on multiple inter-process abstractions,
%including pipes, UNIX sockets, and signals.




\fixmedp{Perhaps you want to start with defining a single byte stream abstraction? And then talk about how different URIs leads to different subclasses?}
The I/O abstractions are simple byte streams.
Byte streams send or receive data over I/O devices or in-kernel queues
as continuous byte sequences.
On a storage device,
a byte stream is logically stored as a sequential file,
but physically divided into blocks.
On a network device, the hardware sends and receives packets for a byte stream, using identification with an IP address and a port number.
An RPC stream can be simply a FIFO (first-in-first-out),
which applications or processes use to exchange messages.
Similar to the API of a UNIX-style OS,
which treats ``everything as a file descriptor''~\cite{ritchie78unix-retro},
\thehostabi{} encapsulates different types of I/O devices through unified APIs such as reading and writing.
Byte streams simplify managing various types of I/O in the library OS.
%The host ABI includes \hostapis{} for single-flavored, stream-type I/O, similar to the API of a UNIX-style OS.
%In general, a UNIX-style OS
%follows the design where ,
%meaning that each I/O abstraction is encapsulated by the file system APIs, such as \syscall{read} and \syscall{write}, to send and receive data on a file, a network socket, or a FIFO (first-in-first-out).
%but categorizes into three types:
%{\bf network connections}, {\bf regular files}, and {\bf RPC streams}.


\Thehostabi{} identifies I/O streams by URIs (unified resource identifier).
%These I/O streams
%are created or connected using a URI (unified resource identifier),
A URI is a unique name to describe an I/O stream,
including a prefix to identify the I/O type and the information for locating an I/O stream on the related I/O device.
The prefix of URI can be one of the following
keywords: ``\palkeyword{file:}'' for regular files; ``\palkeyword{tcp:}'' and ``\palkeyword{udp:}'' for network connections; and ``\code{pipe:}'' for RPC streams.
The rest of the URI represents an identifier of the I/O stream.
For instance, a file URI identifies a file by the path in the host file system.
A network URI identifies the IP address and port number of a network connection.
The URIs standardize identification of I/O resources on host OSes.


%According to the prefix of URI,
%the PAL can create either a synchronous I/O stream (e.g., a file or a connected socket)
%or named I/O server (e.g., a listening TCP socket).
%Modern OSes contain several out-of-band or asynchronous I/O abstractions, to improve the latency or CPU idle time
%when polling I/O streams.
%Although using out-of-band or asynchronous I/O is beneficial for application performance,
%providing these I/O features can be challenging for a host.
%Therefore, in \graphene{}, the host ABI restricts I/O abstractions to only synchronous stream I/O.


\fixmedp{Define these semantics without a reference to POSIX, so that the document is self-contained.}
The \hostapis{} for stream I/O are as follows:
%The host ABI for stream I/O presents a simplified, {\b POSIX file system}.
%A POSIX file system
%encapsulates I/O abstractions,
%including files, sockets, pipes, and even character devices,
%in the file system API.
%The host ABI contains several functions
%which resemble the POSIX API:
\palcall{StreamOpen} creates or opens an I/O stream;
\palcall{StreamRead} and \palcall{StreamWrite}
send and receive data over a stream;
%are similar to \syscall{open}, \syscall{read}, and \syscall{write} in behavior, with a simplified, explicit semantic;
\palcall{StreamMap} maps %a is equivalent to \syscall{mmap} with a \code{MAP\_FILE} flag, mapping
a regular file to the application's memory; %, for reading or writing data.
\palcall{StreamAttrQuery} and \palcall{StreamAttrQuerybyHandle}
retrieves stream attributes;
%as \syscall{stat} and \syscall{fstat} do;
%The POSIX-style functions simplify the porting of the host ABI
%on hosts with a similar specification.
\palcall{ServerWaitForClient} blocks and creates streams for incoming network or RPC connections;
\palcall{StreamSetLength} truncates a file;
\palcall{StreamFlush} clears the I/O buffer inside the host OS.
The following sections will discuss the \hostapis{} in detail.





\papersubsubsection{Opening or creating an I/O stream}




\begin{paldef}
HANDLE StreamOpen (const char *stream_uri,
                   u16 access_flags, u16 share_flags,
                   u16 create_flags, u16 options);
\end{paldef}



\palcall{StreamOpen} opens or creates an I/O stream % for future operations,
based on the URI given by \palkeyword{stream_uri}. % to identifies resources associated with the I/O stream. 
The specification of 
\palcall{StreamOpen} includes interpreting the URI prefixes and syntaxes of \palkeyword{stream_uri},
and allocating the associated resources in the host OS and on I/O devices.
\fixmedp{The term ``Opaque pointer'' is useful here}
If \palcall{StreamOpen} succeeds, it returns a {\bf stream handle}.
The \libos{} stores the stream handle as a reference to the opened I/O stream.
A stream handle is an opaque pointer, and the \libos{} only references the handle as an identifier instead of trying to interpret the handle content.
On the other hand, if \palcall{StreamOpen} fails (e.g., invalid arguments or permission denied), it returns a null pointer with the failure reason delivered with an exception.
% A stream handle is similar to a file descriptor in the POSIX API,
%but defined as a pointer to a data structure containing the stream information.
%The internal structure of a stream handle is up to the I/O stream type and the implementation of each PAL;
%a library OS is supposed to reference a stream handle
%only as an identifier.


\fixmedp{Need a listing of all the values of flags}
Other parameters of \palcall{StreamOpen} specify the options for opening an I/O stream:

\begin{compactitem}

\item
\palkeyword{access_flags} specifies access mode of the I/O stream, to be either \palkeyword{RDONLY} (read-only), \palkeyword{WRONLY} (write-only), \palkeyword{APPEND} (append-only), and \palkeyword{RDWR} (readable-writable).
The first three access modes are only available
for regular files. %; if the opened stream is a network or RPC stream, the access mode is always \palkeyword{RDWR}.
The access modes specify the basic permissions for the \libos{} to access the opened stream. %can request when opening a file.
The access flags are checked by the host OS, based on security policies.
For example, the \libos{} can only append data to a file opened with the \palkeyword{APPEND} mode.

\item
\palkeyword{share_flags} specifies permissions to share a regular file (ignored for other types of  I/O streams)
with other applications. %, either in \graphene{} or in the host OS.
\palkeyword{share_flags} can be a combination of six different values:
\palkeyword{OWNER_R}, \palkeyword{OWNER_W}, and \palkeyword{OWNER_X}
represent the permissions to be read, written, and executed by the creator of the file;
\palkeyword{OTHER_R}, \palkeyword{OTHER_W}, and \palkeyword{OTHER_X}
represent the permissions to be read, written, and executed by everyone else.
%The sharing permissions are also configured in the host file system; access modes in future executions are checked against the sharing permissions given.

\item
\palkeyword{create_flags} specifies the semantics of file creation
when the file is nonexistent on the host file system.
With \palkeyword{TRY_CREATE},
\palcall{StreamOpen} creates the file only when the file is nonexistent.
If \palkeyword{ALWAYS_CREATE},
the \hostapi{} fails if the file already exists.

\item
\palkeyword{options} specifies a set of miscellaneous options to configure the opened I/O stream.
Currently, \palcall{StreamOpen} only accepts one option: \palkeyword{NONBLOCK} specifies that the I/O stream will never block whenever the guest attempts to read or write data.
The nonblocking I/O option is necessary for performing asynchronous I/O in the guest, to overlap the blocking time of multiple streams by polling (using \palcall{ObjectsWaitAny}).

\end{compactitem}


%includes different syntaxes for interpreting the URI and the rest of arguments,
%and different behaviors for creating or opening an I/O stream,
%according to the URI prefixes.
%uses a URI (uniform resource identifier), for specifying the I/O stream.
%An URI identifies both the type of I/O and the distination or location of the I/O stream.
%The type of a I/O stream is determined by URI prefix.
According to the URIs, \palcall{StreamOpen}
can create two types of I/O streams:
A byte stream and
a {\bf server handle} to receive remote connections.
%clients to initiate handshakes
%for establishing a byte-stream connection.
%A server handle can be bound as a network server or a RPC server.
A server handle cannot be directly read or written
but can be given to \palcall{ServerWaitForClient} to block until the next client connection.
The reason that \thehostabi{} must support
I/O servers
%The host ABI includes the abstraction of creating server handles
is that receiving remote connections
requires controls
at the TCP/IP layer and allocating resources
in the network stack,
and thus cannot be emulated in the \libos{}
unless the network stack is virtualized.
%and allocating host resources,
%which cannot be implemented in the guest unless
%the network stack is virtualized.




\palcall{StreamOpen} accepts the following URI prefixes and syntaxes for creating a byte stream or a server handle:


\begin{compactitem}

\item \palkeyword{file:[path]} accesses a regular file on the host file system.
The file is identified by a path,
containing directory names from the file system root (/) or current working directory (CWD).
\fixmedp{Mention CWD is relative to where the app is launched from. It may also be worth noting that this is included for convenience, but there are some security risks to using relative paths.}
The current working directory is the location where the PAL starts and does not change during execution.
%\palcall{StreamOpen} accepts relative paths for the convenience of locating application files packaged and shipped together.
There could be security risks
that the target of a relative path may be ambiguous, especially if the path starts with a ``dot-dot'' (i.e., walking back a directory).
%However, the \libos{} can always check 
%Fortunately, both cases can be checked by the guest, as long as
%the initial directory is specified by the host.
%based on user configurations.
Therefore, a reference monitor should always canonicalize a relative path before checking against security policies.

\item \palkeyword{tcp:[address]:[port]} or \palkeyword{udp:[address]:[port]} creates a TCP or UDP connection to a remote server,
based on the IPv4 or IPv6 address and port number of the remote end.
One a connection is created,
it will exist until it is torn down by both sides.

\item \palkeyword{tcp.srv:[address]:[port]} or \palkeyword{udp.srv:[address]:[port]} create a TCP or UDP server handle which can receive remote client connections.
The address of a TCP or UDP server can be either IPv4 or IPv6, with a port number smaller than 65536.
%If the specified port number is smaller than 1024,
%it may require additional privilege from the host OS.

\item \palkeyword{pipe.srv:[name]} or \palkeyword{pipe:[name]} create a named RPC server or a connection to an RPC server.
The name of an RPC server is an arbitrary, unique string.
An RPC stream is an efficient way for passing messages between applications or processes
running on the same host,
compared to using a network stream locally.
An RPC stream is supposed to have lower latency than a network stream.

%The stream can be either a server which awaits incoming connection from other processes,
%or a client to an existing server.

\end{compactitem}



%Besides the abstraction,
%\palcall{StreamOpen}
%also inherits similar definitions of function arguments,
%including \palkeyword{access_flags}, \palkeyword{share_flags}, \palkeyword{create_flags}, and \palkeyword{options},
%from the POSIX-style \syscall{open}.
%%also inherit similar meanings from the options of \syscall{open}.
%These arguments specify the access type, file permission, creation mode, and other miscellaneous options of the I/O streams:
%for example, the access type can be specified as readable or writable,
%and the creation mode can be either explicit or implicit.
%An important concern is the choice of file permission (specified by \palkeyword{share_flags}), since the host ABI does not expose user credentials
%from the host.
%Setting the file permission in the host
%is mostly a usability feature: 
%an application can run more smoothly if some file permissions are externalized.
%For example, a compilor can create an executable with the execution permission, so that the consecutive building script can run the executable.
%The host ABI also externalizes the file permission
%which specifies specify whether a file can be shared with other applications.


%\palcall{StreamOpen} defines the scope of enforcing and configuring security isolation in the hosts.
%%URIs for \palcall{StreamOpen} represent three types of 
%The host ABI restricts the sharing of host resources
%to type types of simple I/O streams (i.e., file, network, and RPC). 
%Other host resources, such as threads and memory,
%are local to each process, and thus can be isolated by dedicating the host resources.
%%Restricting the shareable host resource to I/O streams simplifies
%%the enforcement of security policies in the hosts.
%Therefore, in the host ABI, \palcall{StreamOpen} is the only \hostapi{}
%which requires permission checks in the hosts.
%Moreover, a user can configure the policies of sharing I/O streams by
%whitelisting the URIs that are permitted for an application.


\palcall{StreamOpen} is easy to port on most host OSes because it limits the types of I/O streams
to three basic and common forms: files, TCP or UDP network sockets, and RPC streams.
The semantics of \palcall{StreamOpen} belong to a subset of POSIX's \syscall{open} semantics, which coincides with \win{}'s \syscall{OpenFile} semantics, with parameter options available in both host OSes.
URIs are also ubiquitously recognized and easily translated
to host-specific identifiers.

This thesis argues that \palcall{StreamOpen} supports sufficient I/O abstractions
for emulating Linux I/O features that most server or command-line applications need.
Low-level abstractions such as storage blocks or raw sockets are mostly only important to administration-type applications such as \code{fsck} and \code{ipconfig}.
Other Linux I/O features, such as asynchronous I/O and close-on-\syscall{execve},
are emulatable in the \libos{}.




\papersubsubsection{Reading or writing an I/O stream}



\begin{paldef}
u64 StreamRead  (HANDLE stream_handle, u64 offset, u64 size,
                 void *buffer);
u64 StreamWrite (HANDLE stream_handle, u64 offset, u64 size,
                 const void *buffer);
\end{paldef}                   
              
\palcall{StreamRead} and \palcall{StreamWrite} synchronously
read and write data over an opened I/O stream.
Both \hostapis{} receive four arguments: a \palkeyword{stream_handle} for referencing the target I/O stream;
\palkeyword{offset} from the beginning of a regular file
(ignored if the stream is a network or RPC stream);
\palkeyword{size} for specifying how many bytes are expected to be read or written;
and finally, a \palkeyword{buffer} for storing the read or written data.
At success, the \hostapis{} return the number of bytes
actually being read or written.


     




%The host ABI features include synchronously reading and writing data over an I/O stream.
%The behavior of \palcall{StreamRead} and \palcall{StreamWrite} is slightly different between a file stream and other type of stream:
%when reading or writing a file stream, \palcall{StreamRead} and \palcall{StreamWrite} accesses the file at a specific offset from the beginning of the file;
%otherwise, when accessing a network or RPC stream,
%the argument \palkeyword{offset} is ignored, and thus the ABI works similar to \syscall{read} and \syscall{write}.



\palcall{StreamRead} and \palcall{StreamWrite} avoid the semantics of sequential file access
to skip migrating stream handles.
The \hostapis{} only read or write at absolute offsets
from the beginning of an opened file,
and do not rely on states stored in the host OSes as the file cursors.
Stateless file access 
allows migrating a \libos{} to another process or host 
%reopen the I/O stream on another host
without migrating the host OS states.
All host OS states associated with an I/O stream is only meaningful to the host
and can always be recreated by the \libos{}.



%The design is to keep the stream handle stateless inside the PAL,
%for cleanly migrating a library OS.
%Since the library OS is responsible of maintaining the offset of a file descriptor,
%a library OS instance can be easily detach from a PAL,
%by simply ``invalidating'' a stream handle;
%the library OS should be able to always reopen the stream after migration,
%or after a failure of accessing a stream handle,
%to recover the state of an I/O stream.



The \hostapis{} do not support asynchronous I/O, or peeking into network or RPC buffers.
%Another challenge in the library OS, regarding I/O streams,
%is to support asynchronously I/O, or peeping data received over a network or RPC stream.
%This design decision is made to keep the host ABI simple.
Asynchronous I/O and buffer peeking are
essential OS features to many applications
%that an application may depend on.
%Although the features are not included in the host ABI,
and the \libos{} can emulate
these features
using \palcall{StreamRead}, \palcall{StreamWrite}
and other \hostapis{}
(e.g., \palcall{ObjectsWaitAny}).
%to prevent blocking on an I/O stream.
The \libos{} can maintain in-memory buffers to store data prematurely received from an I/O stream,
to service asynchronous I/O and buffer peeking.
Chapter~\ref{chap:libos} further discusses these features in detail.

%To implement the full Linux I/O features,
%the library OS must emulate asynchronous I/O,
%using the synchronous
%\palcall{StreamRead} and \palcall{StreamWrite}.
%The emulation requires opening the I/O stream in a non-blocking mode,
%and polling the stream handle before reading or writing data.
%The library OS can also buffer the data being read or written over a stream,
%as long as the buffered state is coordinated over every processes
%which share the same stream.



The fact that
all three types of I/O streams
supported by \palcall{StreamOpen} are simply byte streams justifies the portability of the \hostapis{}.
\palcall{StreamRead} and \palcall{StreamWrite} cover all access types on the I/O streams,
using synchronous semantics available in most host OSes.
\palcall{StreamRead} and \palcall{StreamWrite}
also cover both random and sequential access
to a regular file
and transferring over a TCP or UDP socket. 
Other I/O operations (e.g., asynchronous I/O) can mostly be emulated in the \libos{} using the two \hostapis{}.

\paragraph{Alternatives.}
An alternative strategy is to define asynchronous I/O in the host ABI instead of synchronous I/O.
Although asynchronous I/O may not be universally portable,
%An asynchronous read or write
%does not return a result immediately; instead, it creates an event handle
%which can be polled arbitrarily.
a \libos{} can easily emulate synchronous I/O with asynchronous I/O using a thread to poll I/O events.
Asynchronous I/O
potentially has more predictable semantics,
because the \libos{} can explicitly tell which \hostapis{} will be blocking.
This strategy
is later taken by Bascule~\cite{baumann13bascule}.
\graphene{} chooses synchronous I/O in \thehostabi{}
to prioritize portability,
but will explore alternative designs in the future.
%because synchronous I/O is a more common feature in host OSes.


 


%Defining  \hostapis{} for asynchronous read and write
%may potentially sacrifice portability, \fixme{cites some host that doesn't have async IO}
%since asynchronous I/O is less common seen in OSes.
%However, 
%can potentially be a more flexible option for emulating other I/O features in the guest.
%For example, the guest can emulate a synchronous read by polling a stream followed by asynchronously reading it.



%An asynchronous I/O ABI can be potentially
%more flexible for implementing the library OS features,
%as it can emulate both synchronous and asynchronous behaviors
%without buffering.
%The only reason that we choose synchronous over asynchronous I/O
%in the host ABI is to reduce the porting effort,
%especially for a host which lacks
%asynchronous I/O features.



\papersubsubsection{Mapping a file to memory}
                   
\begin{paldef}            
u64 StreamMap (HANDLE stream_handle, u64 expect_addr,
               u16 protect_flags, u64 offset, u64 size);
\end{paldef}


\palcall{StreamMap} maps a file stream to an address in memory, for reading and writing data, or executing code stored in a binary file.
\palcall{StreamMap} creates a memory region
as either a copy of the file,
or a pass-through mapping which shares file updates with other processes.
When calling \palcall{StreamMap},
the \libos{} can specify an address in memory to map the file, or a null address (i.e., zero) to map at any address decided by the host OS.
\palkeyword{expect_addr}, \palkeyword{offset}, and \palkeyword{size} must be aligned
to allocation granularity decided by the host OS (more discussion in Section~\ref{sec:abi:memory}).
\palkeyword{protect_flags} specifies the protection mode
of the memory mapping, as a combination of \palkeyword{READ} (readable), \palkeyword{WRITE} (writable), \palkeyword{EXEC} (executable), and \palkeyword{WRITE_COPY} (writable local copy).
At success, \palcall{StreamMap} returns the mapped address; otherwise, the \hostapi{} returns a null pointer.




\Thehostabi{} defines \palcall{StreamMap} for two reasons. First, memory-mapped I/O is suitable for file access in certain applications, and cannot be fully emulated
using \palcall{StreamRead} and \palcall{StreamWrite}.
An application may choose memory-mapped I/O for
efficiency, %, especially when the application contains
especially for smaller, frequent file reads and writes.
Second, memory-mapped I/O is asynchronous by its nature.
An OS is supposed to lazily flush the data written to a file-backed memory mapping.
The feature is difficult to emulate
%due to lack of ability to efficiently determine which pages are recently updated,
% and thus should be synchronized with the storage.
without an efficient way to mark recently-updated pages (e.g., using page table dirty bits).
% can only be accessed in host OSes).
%without access to dirty bits in the page table.




%However, the challenge to implementing this behavior
%is to externalize the latest file state
%written by the application,
%on a host which disallows file-backed memory.
%For example, in a \sgx{} enclave, all memory will have to be private memory,
%to be individually encrypted by the CPU.
%For a host which doesn't support pass-through file mappings,
%\palcall{StreamMap}
%can only guarantee writing out
%the latest file state to the disk when the memory is unmapped, using \palcall{VirtMemFree}.


\fixmedp{there should be an explicit semantics about when the mapping is visible back to the host, like on a sync, with the option to flush earlier}
Although \palcall{StreamMap} allows multiple processes to map the same file into memory, it is hard to guarantee coherent file sharing on every host OSes.
To support memory-mapped I/O,
a complete implementation of \thehostabi{}
must include coherent file sharing. 
For some host target where coherent file sharing is impossible,
such as an \sgx{} enclave,
\thehostabi{} implementation must be considered incomplete.
Potentially the \libos{} can implement a remote memory access protocol, with significant overheads to intercept memory access and trace unflushed contents.  
For most monolithic host OSes,
\palcall{StreamMap} with coherent file sharing
is easy to implement using APIs like \syscall{mmap} (POSIX) and \syscall{CreateFileMapping} (\win{}).
 
%Because memory-mapped I/O is asynchronous,
%the data written in the memory are only externalized when the memory mapping is unmapped or explicitly flushed.
%The design is to drop the assumption of memory sharing, especially for an isolated environment like \sgx{} or architectures without coherent memory.
%A host OS can optionally flush written data or maintain coherent sharing across multiple processes if possible.


\palcall{StreamMap} is sufficient for most use cases of memory-mapped I/O. The \libos{} can further manage file-backed memory using \hostapis{} for page management (e.g., \palcall{VirtMemProtect} and \palcall{VirtMemFree}). A few hardware features, such as huge pages and data execution protection (DEP), will require extensions to \thehostabi{} though.
Fortunately, these hardware features are mostly considered optional in applications
and rarely dictate usability.












\papersubsubsection{Listening on a server}


\begin{paldef}
HANDLE ServerWaitforClient (HANDLE server_handle);
\end{paldef} 


\palcall{ServerWaitforClient} waits on a network or RPC server handle, %created with a URI that starts with \palkeyword{tcp.srv:}, \palkeyword{udp.srv:}, or \palkeyword{pipe.srv:},
to receive an incoming client connection.
%Besides the I/O streams which can be directly read or written,
%the host ABI also supports creation
%of I/O stream server, which can be
%connected from another process (to a RPC stream server), or another host (to a network server), as a client of the server.
A network or RPC server handle cannot be accessed by \palcall{StreamWrite} or \palcall{StreamRead};
instead, the host OS listens on the server handle,
and negotiates the handshakes for incoming connections.
Once a connection is fully established,
the host OS returns a client stream handle, which can be read or written as a byte stream.
Before any connection arrives, \palcall{ServerWaitforClient} blocks indefinitely.
If a connection arrives before the guest calls \palcall{ServerWaitforClient},
the host can optionally buffer the connection in a limited backlog; the maximal size of server backlogs is up to the user configurations. The host will drop incoming connections when the backlog is full.
Other than adjusting backlog sizes, \palcall{ServerWaitForClient} covers most of the server-specific behaviors
and is easy to port on most host OSes.



%I/O stream server has to block until a client
%connects the server, using \palcall{StreamWaitForClient}.
%\palcall{StreamWaitForClient} will return a stream handle, which represents a connection with the client.
%Other implementation
%is up to the host: for example,
%the host may decide a maximal number of incoming connections to buffer.









\papersubsubsection{File and stream attributes}

%The host ABI features also include retrieving the metadata of an I/O stream.
%The retrieval of metadata is not limited to files,
%but also network sockets and RPC streams. %, to query the information regarding the streams.
%An example of metadata is the address of a network stream;
%when an unbound network stream is created,
%the host will randomly bind the stream to a local, temporary port, for identifying the connection at the IP (internet protocol) level;
%the POSIX API
%reveals the local port number
%to the applications,
%using \syscall{getsockname}.
%Other stream metadata required by Linux or POSIX functionality
%includes the total bytes written over an I/O stream, and the permission to sharing an I/O stream with other applications.
%The host ABI includes two functions for querying stream metadata:

\begin{paldef}
bool StreamAttrQuerybyHandle (HANDLE stream_handle,
                              STREAM_ATTRS *attrs);
bool StreamAttrQuery (const char *stream_uri, STREAM_ATTRS *attrs);

\end{paldef}

\palcall{StreamAttrQuerybyHandle} and \palcall{StreamAttrQuery} query the attributes of an I/O stream, and fill in a data structure as \palkeyword{STREAM_ATTRS}.
The only difference between the two \hostapis{} is that \palcall{StreamAttrQuerybyHandle} queries an opened stream handle whereas \palcall{StreamAttrQuery} queries a URI without opening the I/O stream in advance.
\palcall{StreamAttrQuery} is convenient for querying stream attributes when the \libos{} does not plan to access the data of an I/O stream.
%Because \palcall{StreamOpen} involves more operations in the host OS,
%\palcall{StreamAttrQuery} can quickly retrieve the attributes without actually opening the stream.
Both \hostapis{} return true or false for whether the stream attributes are retrieved successfully.



%Both functions fill out a data structure given by the library OS,
%with metadata regarding an I/O stream
%identified by either a stream handle or a URI.
%Keeping both functions can be convenient for the library OS to query a file without creating a stream handle;
%however, we can always consolidate the host ABI
%with only \palcall{StreamAttrQuerybyHandle},
%because \palcall{StreamAttrQuery} can be replaced by \palcall{StreamAttrQuerybyHandle}
%after \palcall{StreamOpen}.




\begin{paldef}
typedef struct {
    u16 stream_type, access_flags, share_flags, options;
    u64 stream_size;
    u64 recvbuf, recvtimeout;
    u64 sendbuf, sendtimeout;
    u64 lingertimeout;
    u16 network_options;
} STREAM_ATTRS;
\end{paldef}


The \palkeyword{STREAM_ATTRS} data structure consists of multiple fields specifying the attributes assigned to an I/O stream since creation.
\palkeyword{stream_type} specifies the type of I/O stream that the handle references to.
\palkeyword{access_flags}, \palkeyword{share_flags}, and \palkeyword{options} are the same attributes assigned to an I/O stream when the stream is created by \palcall{StreamOpen}.
\palkeyword{stream_size} has different meanings for files and network/RPC streams:
if the handle is a file, \palkeyword{stream_size} specifies the total size of the file;
if the handle is a network or RPC stream, \palkeyword{stream_size} specifies the size of pending data currently received and buffered in the host.


The remaining attributes are specific to network or RPC streams.
\palkeyword{recvbuf} and \palkeyword{sendbuf} specify the limitation of buffering the pending bytes, either inbound or outbound.
\palkeyword{recvtimeout} and \palkeyword{sendtimeout} specify the receiving or sending timeout (in microseconds)
before the other end abruptly disconnects the stream.
\palkeyword{lingertimeout} specify the timeout for closing or shutting down a connection
to wait for the pending outbound data.
\palkeyword{network_options} is a combination of flags that specify the options for configuring a network stream.
Currently, \palkeyword{network_options} accepts the following generic options:
\palkeyword{KEEPALIVE} (enabling keep-alive messages), %\palkeyword{CORK} (don't send out partial data),
\palkeyword{TCP_NODELAY} (no delay in sending small data),
and \palkeyword{TCP_QUICKACK} (no delay in sending ACK responses).


\begin{paldef}
bool StreamAttrSetbyHandle (HANDLE stream_handle,
                            const STREAM_ATTRS *attrs);
\end{paldef}


%\palcall{StreamAttrSetByHandle} is a \hostapi{} newly 
Introduced by \graphene{},
\palcall{StreamAttrSetByHandle} can configure
the attributes of a file or an I/O stream
in the host OS.
%, and externalizes the change to the host OS.
\palcall{StreamAttrSetByHandle} accepts an updated \palkeyword{STREAM_ATTRS} data structure,
which contains new attributes to assign
to the I/O stream.
%\palkeyword{stream_type} cannot be changed, as well as any attributes that violate the limitation imposed by the host.


It is a dilemma 
to decide which stream attributes
to define in \palkeyword{STREAM_ATTRS}.
especially for a network socket.
A network socket in a monolithic OS often provides several options to fine-tune the behavior of network stacks and drivers.
% (i.e., flags included in \palkeyword{network_option}).
%A network stream attribute can be derived from an optional feature inside the host network stack,
%or a configuration at the NIC level.
Exposing these options on \thehostabi{} allows the \libos{} to emulate network socket APIs more completely.
However, extending \thehostabi{} with network socket options also compromises portability
on host OSes that do not provide equivalence of these features.
Eventually, the \libos{} should not expect every attributes defined in \palkeyword{STREAM_ATTRS} to be configurable on every host OSes.





%To complete the library OS implementation, \graphene{} introduces a new function,
%,
%for changing the metadata of an I/O stream.
%The main reason for changing metadata
%is to configure a network or RPC stream with several device-specific options,
%such as the number of lingering connections,
%or enabling the TCP keepalive feature.


%\papersubsubsection*{Truncating a file or flushing a stream}




%To externalize the change to an I/O stream, the library OS must ensure a file is truncated to the right length (\palcall{StreamSetLength}), or a network or RPC Stream has flushed the host buffer (\palcall{StreamFlush}).
%Both functions are private to a process; if multiple processes try
%to set the file length or flush a stream at the same time, one of the function calls may be ignored by the host.


\begin{paldef}
bool StreamSetLength (HANDLE stream_handle, u64 length);
\end{paldef}


Finally, \palcall{StreamSetLength} expands or truncates a file stream to a specific length.
In general, the data blocks on storage media are allocated dynamically
to a file when the file length grows.
If \palcall{StreamWrite} writes data beyond the end of a file, it automatically expands the file, by allocating new data blocks on the storage media.
However, a file-backed memory mapping created by \palcall{StreamMap}
lacks an explicit timing to expand the file
when writing to the memory mapped beyond the end of the file.
\palcall{StreamSetLength} can explicitly request the host to expand a file to an appropriate length
so that sequential memory write will never raise memory faults.
\palcall{StreamSetLength} can also shrink a file to the actual data size
if the file has overallocated resources earlier.


Potentially \palcall{StreamSetLength} can be absorbed by \palcall{StreamAttrSetByHandle}. Currently, the \hostapi{} is preserved as a legacy from previous versions of \thehostabi{} and as an optimization for file operations which need to frequently update file sizes (\palcall{StreamAttrSetByHandle} is much slower than \palcall{StreamSetLength}).






%\papersubsection*{\graphene{}-specific stream I/O features}

%\begin{paldef}
%void StreamDelete (HANDLE stream_handle, uint direction);
%\end{paldef}


\paragraph{Listing a directory.}
\fixmedp{Always start with what you did, and then discuss alternatives.\\ I think the heart of issue isn't clear. It seems that you are adopting a POSIX model of treating a directory like a file.}
\graphene{} extends the stream I/O feature in the host ABI to retrieve directory information.
A file system usually organizes files in directories,
and allows applications to retrieve a list of files in a given directory.
Instead of adding new \hostapis{} for directory operations,
the host ABI uses existing \hostapis{}, namely \palcall{StreamOpen} and \palcall{StreamRead},
for listing a directory.
When \palcall{StreamOpen} opens a file URI that points to a directory,
such as ``\code{file:/usr/bin}'',
it returns a stream handle
which allows consecutive \palcall{StreamRead} calls to read the file list
as a byte stream.
The stream handle returns a series of file names as null-terminated strings.
The stream handle cannot be written or mapped into memory.



%A POSIX file system contain a hierarchy of directories
%containing files and subdirectories.
%The file I/O in POSIX requires listing the entries in a directory;
%a POSIX function, \syscall{readdir}, returns a list of file and subdirectory names
%in a directory.
%In \graphene{}, we face a decision of whether to include directory I/O
%in the host ABI.
%An option is to maintain a local file in each directory
%to store a list of file and subdirectory names;
%however, this solution will requires maintaining the list whenever a new file or subdirectory is created.
%Therfore, we extend the host ABI,
%to allow opening a directory as a stream handle.
%The library OS can read a list of file and subdirectory names from a directory stream handle,
%generated by the hosts.




\paragraph{Character devices.}
The host ABI also supports reading or writing data over a character device, such as a terminal.
A terminal can be connected as a stream handle,
using a special URI called \palkeyword{dev:tty}.
Other character devices include the debug stream of a process (the URI is \palkeyword{dev:debug}),
equivalent to writing to \code{stderr} in POSIX.




