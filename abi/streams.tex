\section{Stream I/O}
\label{sec:abi:streams}


An application requires interaction with the world during its execution.
I/O is a common feature of almost every OSes.
The typical I/O needed in an application
can be catagorized into three types:
externalizing data to outside of the machines (i.e., networking);
externalizing data to a permanent store (i.e., storage);
and communicating with other applications running on the same host (i.e., remote-procedure call).
A modern OS may define several abstractions
for each types of I/O; for example, a file in Linux can be read using several system calls,
including \syscall{read}, \syscall{pread}, and \syscall{readv};
RPC in Linux is based on multiple inter-process abstractions,
including pipes, UNIX sockets, and signals.




The host ABI
encapsulates I/O as three different types of byte streams
to send data in and out of applications:
network sockets, regular files, and RPC streams (similar to pipes).
These I/O streams
are created or connected using a URI (unified resource identifier),
a unique name which describes the destinations or resource locations of the I/O streams.
According to the prefix of URI,
the PAL can create either a synchronous I/O stream (e.g., a file or a connected socket)
or named I/O server (e.g., a listening TCP socket).


Modern OSes contain several out-of-band or asynchronous I/O abstractions, to improve the latency or CPU idle time
when polling I/O streams.
Although using out-of-band or asynchronous I/O is beneficial for application performance,
providing these I/O features can be challenging for a host.
Therefore, in \graphene{}, the host ABI restricts I/O abstractions to only synchronous stream I/O.



The host ABI for stream I/O presents an equivalent of a {\bf POSIX file system}.
POSIX
encapsulates most I/O abstractions,
including files, sockets, pipes, and even character devices,
in the file system API.
The host ABI contains several functions
which resemble the POSIX API:
\syscall{StreamOpen}, \syscall{StreamRead}, and \syscall{StreamWrite}
share the specfication with \syscall{open}, \syscall{read}, and \syscall{write};
\syscall{StreamAttrQuery} and \syscall{StreamAttrQuerybyHandle} return file and I/O metadata, similar to \syscall{stat} and \syscall{fstat};
both \syscall{StreamMap} and \syscall{mmap} maps a regular file into memory, for reading or writing data.
The POSIX-style host ABI
simplifies the porting above
similar system interfaces.






\subsection*{Opening or creating an I/O stream}




\begin{paldef}
HANDLE StreamOpen (const char *stream_uri,
                   uint access_flags, uint share_flags,
                   uint create_flags, uint options);
\end{paldef}



Unlike \syscall{open}, \syscall{StreamOpen} uses a URI (uniform resource identifier), for specifying the I/O stream.
An URI identifies both the type of I/O and the distination or location of the I/O stream.
I/O types can be determined by the prefix of a URI;
the host ABI supports the following prefixes:


\begin{compactitem}
\item \palkeyword{file:[path]}: A regular file mapped from the host file system. The file is located by a file path on the host. The file path can be either an absolute path from the root the host file system, or a relative path from the current working directory on the host.
\item \palkeyword{tcp:[address]:[port]} or \palkeyword{udp:[address]:[port]}: A TCP or UDP connection to a remote socket, identified by a IPv4 or IPv6 address and a listening port. The TCP connection exists until it is teared down by the both parties.
\item \palkeyword{tcp.srv:[address]:[port]} or \palkeyword{udp.srv:[address]:[port]}: A TCP or UDP server waiting for remote connections, bound on a IPv4 or IPv6 address and an idle port.
\item \palkeyword{pipe.srv:[name]} or \palkeyword{pipe:[name]}: a named RPC stream server, which can be connected from other processes running on the same host, 
or a connected RPC stream,
for cross-process message passing.
\end{compactitem}



Besides the abstraction,
other function arguments of \syscall{StreamOpen}, including \code{access\_flags}, \code{share\_flags}, \code{create\_flags}, and \code{options},
also inherit similar meanings from the options of \syscall{open}.
These arguments specify the access type, file permission, creation mode, and other miscellaneous options of an I/O stream.
For example, the access type can be specified as readable or writable,
and the creation mode can be either explicit or implicit.
An interesting option is the choice of file permission (\code{share\_flags}), since the host ABI does not expose user credentials
from the host.
Setting the file permission in the host
is mostly a usability feature: 
an application can run more smoothly by if some file permissions are externalized.
For example, a compilor can create an executable with the execution (X) permission, so that the consecutive building script can run the executable.
The host ABI also externalizes the file permission
which specifies specify whether a file can be shared with other applications.


\syscall{StreamOpen} returns a {\bf stream handle}. A stream handle is similar to a file descriptor in the POSIX API,
but mostly defined as a pointer to a stream-specific data structure.
The definition of stream handles is up to the implementation of PAL,
and the library OS simply references the stream handles as identifiers for opened I/O streams.




\subsection*{Reading or writing an I/O stream}



\begin{paldef}
ulong StreamRead  (HANDLE stream_handle, ulong offset,
                   ulong size, void *buffer);
ulong StreamWrite (HANDLE stream_handle, ulong offset,
                   ulong size, const void *buffer);
void *StreamMap   (HANDLE stream_handle,
                   void *expect_addr, uint protection,
                   ulong offset, ulong size);
\end{paldef}



\subsection*{Querying metadata}

\begin{paldef}
bool StreamAttrQuerybyHandle (HANDLE stream_handle,
                              STREAM_ATTRS attrs);
bool StreamAttrQuery (const char *stream_uri,
                      STREAM_ATTRS attrs);

\end{paldef}



\subsection*{Truncating a file or flushing a stream}


\begin{paldef}
bool StreamSetLength (HANDLE stream_handle, ulong size);
void treamFlush (HANDLE stream_handle);
\end{paldef}


\subsection*{Listening on a server}


\begin{paldef}
HANDLE StreamWaitforClient (HANDLE stream_handle);
\end{paldef} 


\subsection*{\graphene{}-specific stream I/O features}

\begin{paldef}
void StreamDelete (HANDLE stream_handle, uint direction);
bool StreamAttrSetbyHandle (HANDLE stream_handle,
                            STREAM_ATTRS attrs);
\end{paldef}