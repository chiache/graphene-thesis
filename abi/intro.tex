This chapter specifies \thehostabi{} as a component
of the \graphene{} architecture.
\fixmedp{A sentence on motivation might be good here.  Same with implication.  Like, what is the high level goal?  What properties would a good abi have (or how to judge goodness)?}
\polish{\Thehostabi{} acts as a boundary between the host and the \libos{},
and defines several UNIX-like features. % for \libos{} development.
The definition of \thehostabi{} is primarily based on two criteria:
{\em simplicity}, to bound the development effort per host, and {\em sufficiency}, to encapsulate enough host abstractions for \libos{} development.}
This chapter also explains
the rationale behind the definitions.
%according to the experience of porting the host ABI to several host examples, such as Linux, \win{}, and \sgx{}.







%The host ABI of \graphene{} represents a specification of which OS and hardware abstractions should be implemented in the hosts, 
%compared to what should be in the library OS.
%, in order to implement the library OS.
%and how to develop
%a PAL (platform adaption layer).
%which translates the host ABI
%to host OS features.
%This chapter shows how the host ABI is initially defined and extended,
%based on design principles of \graphene{}. %, including simplicity and sufficiency.
%and security isolation.
%determine the definition of the host ABI,
%as a guideline to adding host features.
%with new abstractions.





%and 
%the design principles behind the definition, for both the simplicity of porting, and the sufficiency of implementing Linux functionality.













%ion solution (e.g., Xen) can deduplicate OS components, such as  schedulers and page fault handlers, between a host and a guest,
%by modifying the guest OS to make hypercalls into the hypervisor.



%The strategies for defining the host ABI in \graphene{}
%can be summarized as follows:
%First, \graphene{} inherits most host OS functions from \drawbridge{}~\cite{porter11drawbridge},
%with minor extensions. % for implementing Linux system calls.
%\graphene{} and \drawbridge{} share a similar motivation: reusing unmodified applications across host platforms.
%%\drawbridge{} 
%%tries to reuse \win{} code
%%in most of its library OS implementation.
%%As a result, most functions defined in the \drawbridge{} host ABI have certain level of similarity with 
%%the \win{} API.
%%such as the scheduling primitives.
%%\drawbridge{} makes the decision
%%primarily for maximally reusing the \win{} OS code in the library OS.
%Although inheriting the host ABI from \drawbridge{} 
%is mostly an engineering decision
%in \graphene{},
%this strategy covers the typical UNIX-style OS features,
%with a definition largely distinct from Linux system calls in style.
%
%
%
%Second,
%\graphene{} adds several host functions for
%extending the library OS features which are unique to Linux.
%For example, 
%use of x86 segment registers (FS/GS) is necessary for Linux applications hard-coded with
%direct TLS (thread-local storage) references.
%The segment registers can only be written in ring 0;
%although a user-ring instruction, \code{WRFSGSBASE}, can also write to the registers,
%the feature is not opt-in on x86.
%Therefore, adding a 











%layer, which defines the entire host features
%available for implementing the library OS.
%The traditional virtual machines reuse unmodified OS kernels on a virtual hardware interface.
%Unlike a virtual machine,
%the host ABI partitions OS components into the library OS, and retains only a set of basic OS functions to interact with the hosts.
%The functions defined in the host ABI
%are either API for requesting hardware resources, or convenient features, such as scheduling primitives, which commonly exist in many OSes.
%%The functions in the host ABIs are similar to
%%the UNIX-style API, or a small portion of POSIX,
%The host ABI is selected as a pinch point for OS virtualization,
%to recycle the virtualized Linux system call implementation, and to harvest the existing host OS functionality for porting the host ABI. 
%%as a leverage point for harvesting the OS functionality implemented on the hosts.

%The \graphene{} host ABI defines a set of {\em functions}, similar to the API of UNIX or POSIX.
%The functions are directly called by the library OS, along with the arguments given either in the registers or on the stack.
%A host-specific \graphene{} loader is responsible for resolving the linking, from the library OS to the host ABI.


%\papersection{Formal Definition}
%
%
%This section formally defines the calling convention, data types, and functions
%in the host ABI.
%The host ABI
%specifies the scope of PAL development,
%and the host features required by the library OS.
%%The definition specifies the research problem
%%in the development of \graphene{},
%%which is the translation between the Linux system interface, the host ABI, and various host interfaces.
%



\papersection{PAL calling convention}



%The key component to implement the features of the host ABI
%is a PAL (platform adaption layer).
%The host ABI
%%is not only a description of the interface to the host and library OS,
%is essentially a specification of how to develop a PAL.
%The specification of a PAL
%includes two parts: the first is the functions used in the programming of the library OS;
%the second is the calling convention
%that a PAL enforces for the library OS to interact with the host ABI.
%This section will discuss the PAL calling convention.



%The host ABI avoids making assumption about the host,
%including the host system call convention.
%Every call from the library OS to enter the PAL is a normal function call.
%To resolve function call addresses,
%the PAL has to work as a dynamic loader, similar to \code{ld.so}.
%Dynamic linking allows shipping the library OS as a separate binary from the host-specific PAL binaries.
%and thus can be loaded on any hosts.
%We choose to dynamically link the library OS for keeping the initialization
%adaptable to different hosts;
%For example,
%A host such as \sgx{} can impose additional restriction on the application code to be loaded,
%including to validate a signature of application code at beginning.
%which requires all trusted code to execute in one context.
 



%to the calling convention used by the native, host system interface.
%For example, a PAL running on a Linux host
%is equivalent to a simplified version of standard C library, which contains wrappers to
%translate functions to system calls.
%Besides, the Linux PAL also works as
%a dynamic loader,
%similar to a simplified \code{ld.so},
%to bootstrap the dynamic linking between the host ABI
%and the library OS.


\fixmedp{Make this sentence more direct: "The library OS is loaded by the host OS as user libraries."}
\Thehostabi{} partially adopts the \graphenearch{} Linux convention. %, to simplify compilation and linking.
A PAL contains a simple run-time loader that can load the library OS as an ELF (executable and linkable format) binary~\cite{elf-format}, similar to \code{ld.so} for loading a user library.
%The host ABI inherit the Linux calling convention,
%for the convenience of development.
%The primary architecture targeted by \graphene{} is \graphenearch{};
%The primary target of architecture in \graphene{} is x86-64;
%in the x86-64 Linux call convention,
%The host ABI follows the Linux calling convention,
%which, on the x86 architecture, passes 
%the Linux calling convention on \graphenearch{}
%specifies the arguments of a function to be passed in general-purpose registers---specifically in \code{rdi}, \code{rsi}, \code{rdx}, \code{rcx}, \code{r8}, and \code{r9}~\cite{system-v-abi}.
Inheriting the Linux convention
simplifies the dynamic linking between the application, library OS and PAL,
and enables compiler-level optimizations for linking,
such as function name hashing.
%The host ABI also assumes the library OS to be compiled in ELF (extensible linking format).
%The decision is to match the calling conventions
%of application, library OS, and PAL,
%to allow a commodity C compiler to improve the linking procedure.
%For example, the GNU C Compiler (GCC) can optimize function name comparison
%using hashing,
%which can be simply adopted by the PAL
%if it shares a linking scheme with the library OS.
Another benefit is simplifying debugging with GDB, since GDB only recognizes one calling convention at a time.



%If the library OS and the hosts share the same calling convention, the PAL can directly call
%the host ABI, without any redirection.
%For other hosts,
% that do not share the calling convention with the library OS,
%such as \osx{} and \win{}, the PALs redirect
%the function arguments to the correct registers or offsets on the stack frames assigned by the calling convention.
%Dynamic linking is beneficial for
%building a pluggable, host-independent library OS binary, and leveraging the compiler optimizations across the host ABI.





%By definition, the host ABI is limited to the OS abstractions to the local process.

%Most functions in the host ABI
%only have local effect to a picoprocess.
%Using the host ABI, the library OS can  llocate a memory region inside the current picoprocess, or create a handle that is only accessible by the current picoprocess.
%Only functions related with stream I/O
%can share OS resources with other picoprocesses, such as writing to the same file, sharing a network sockets, or connecting through a local RPC (remote-procedure call) stream.



\paragraph{Host differences.}
%For a host with a different calling convention from Linux,
A PAL is responsible for translating the calling convention between the host ABI and a system interface on the host.
For example,
\win{} or \osx{} applications follow different calling conventions and binary formats
from Linux applications.
\fixmedp{Make this point clearer}
On each host, a PAL acts as a simple ELF loader.
On \win{} and \osx{}, the PALs are developed as application binaries recognized by the host OSes,
but contain ELF loading code for linking 
%but rather a host-specific binary that includes an ELF loader.
Linux application binaries and the \libos{}.
%therefore, a \win{} PAL needs
%to swap the register values when the library OS calls the host ABI.
%\win{} and \osx{} also accepts a different type of binary format,
%and requires the PAL to implement an ad-hoc linking scheme for the library OS as an ELF binary.
The PALs also resolve calling convention inconsistency
such as placement of function arguments
in registers or on stacks.




\paragraph{Error codes.}
To be explicit, a \hostapi{} only returns two types of results:
a non-zero number or pointer if the call succeeds, or zero if it fails.
Unlike the Linux convention, the host ABI does not return negative values as error codes (e.g., \code{-EINVAL}).
For applications,
interpreting negative return values from system APIs
causes confusing corner cases
that applications may easily miss.
For instance, error codes of failed \syscall{read} system calls may be misinterpreted as
negative input sizes,  
causing bugs in applications.
Instead, a PAL delivers the failure of a \hostapi{}
as an exception, to be captured by the library OS's assigned exception handler.
The design avoids confusing semantics
as interpreting negative return values.



\paragraph{Dynamic linking vs static linking.}
\graphene{} dynamic links an application, \libos{}, and the PAL inside of a user process.
Dynamic linking ensures complete reuse of an unmodified application,
as well as an unmodified library OS.
\graphene{} allows the binaries of application, library OS, and PAL to be deployed individually to the users.
The dynamic linking of applications is a prerequisite
to running most Linux applications without modification or recompilation.



There are cases where static linking is preferred on a host (e.g., \sgx{}), and recompilation is acceptable to the users.
Compiling an application, library OS, and PAL
into a single binary is useful for reducing the memory footprint in unikernels~\cite{unikernels},
by compiling out unnecessary code and data segments. 
It is possible for \graphene{} to statically link an application with the library OS and a PAL,
but this technique is out of the scope of this thesis and left for future work.


\papersection{The PAL ABI}

\fixmedp{make this chapter self-contained; should be like a manual of the host ABI; avoid taking about libOS and focus on PAL}
This section defines the \thehostabi{},
%functions in the host ABI, according on the type of OS features.
as a developer's guide to porting. %implementing \hostapis{} for a new host.
This section describes 
the usage of each \hostapi{},
followed by justifications of sufficiency and simplicity of porting on a new host OS.


