This chapter describes the formal definition of the host ABI.
The host ABI contains a specification of which OS and hardware abstractions should be implemented in the hosts, %, in order to implement the library OS.
and how to develop
a PAL (platform adaption layer).
%which translates the host ABI
%to host OS features.
This chapter shows how the host ABI is initially defined and extended,
based on design principles of \graphene{}. %, including simplicity and sufficiency.
%and security isolation.
%determine the definition of the host ABI,
%as a guideline to adding host features.
%with new abstractions.


%and 
%the design principles behind the definition, for both the simplicity of porting, and the sufficiency of implementing Linux functionality.













%ion solution (e.g., Xen) can deduplicate OS components, such as  schedulers and page fault handlers, between a host and a guest,
%by modifying the guest OS to make hypercalls into the hypervisor.



%The strategies for defining the host ABI in \graphene{}
%can be summarized as follows:
%First, \graphene{} inherits most host OS functions from \drawbridge{}~\cite{porter11drawbridge},
%with minor extensions. % for implementing Linux system calls.
%\graphene{} and \drawbridge{} share a similar motivation: reusing unmodified applications across host platforms.
%%\drawbridge{} 
%%tries to reuse Windows code
%%in most of its library OS implementation.
%%As a result, most functions defined in the \drawbridge{} host ABI have certain level of similarity with 
%%the Windows API.
%%such as the scheduling primitives.
%%\drawbridge{} makes the decision
%%primarily for maximally reusing the Windows OS code in the library OS.
%Although inheriting the host ABI from \drawbridge{} 
%is mostly an engineering decision
%in \graphene{},
%this strategy covers the typical UNIX-style OS features,
%with a definition largely distinct from Linux system calls in style.
%
%
%
%Second,
%\graphene{} adds several host functions for
%extending the library OS features which are unique to Linux.
%For example, 
%use of x86 segment registers (FS/GS) is necessary for Linux applications hard-coded with
%direct TLS (thread-local storage) references.
%The segment registers can only be written in ring 0;
%although a user-ring instruction, \code{WRFSGSBASE}, can also write to the registers,
%the feature is not opt-in on x86.
%Therefore, adding a 










%layer, which defines the entire host features
%available for implementing the library OS.
%The traditional virtual machines reuse unmodified OS kernels on a virtual hardware interface.
%Unlike a virtual machine,
%the host ABI partitions OS components into the library OS, and retains only a set of basic OS functions to interact with the hosts.
%The functions defined in the host ABI
%are either API for requesting hardware resources, or convenient features, such as scheduling primitives, which commonly exist in many OSes.
%%The functions in the host ABIs are similar to
%%the UNIX-style API, or a small portion of POSIX,
%The host ABI is selected as a pinch point for OS virtualization,
%to recycle the virtualized Linux system call implementation, and to harvest the existing host OS functionality for porting the host ABI. 
%%as a leverage point for harvesting the OS functionality implemented on the hosts.

%The \graphene{} host ABI defines a set of {\em functions}, similar to the API of UNIX or POSIX.
%The functions are directly called by the library OS, along with the arguments given either in the registers or on the stack.
%A host-specific \graphene{} loader is responsible for resolving the linking, from the library OS to the host ABI.


%\section{Formal Definition}
%
%
%This section formally defines the calling convention, data types, and functions
%in the host ABI.
%The host ABI
%specifies the scope of PAL development,
%and the host features required by the library OS.
%%The definition specifies the research problem
%%in the development of \graphene{},
%%which is the translation between the Linux system interface, the host ABI, and various host interfaces.
%



\section{Subroutine Calling Convention}



The host ABI assumes linking the library OS as
a user library.
The \graphene{} features include dynamically links an application, with the library OS and PAL,
as one executable running inside a process.
The host ABI does not
rely on the system call convention;
every call that enters a PAL is strictly a function call.
We choose to dynamically link the library OS for keeping the initialization
adaptable to different hosts;
For example,
a host like SGX imposes restriction on the application context,
including to validate a signature of application code at beginning.
%which requires all trusted code to execute in one context.
 


In \graphene{}, the role of PAL is similar to a system library,
which translates a user-level calling convention,
to the calling convention used by the native, host system interface.
For example, a PAL running on a Linux host
is equivalent to a simplified version of standard C library, which contains wrappers to
translate functions to system calls.
Besides, the Linux PAL also works as
a dynamic loader,
similar to a simplified \code{ld.so},
to bootstrap the dynamic linking between the host ABI
and the library OS.













%with dynamic linking initialized at the creation of each process (or picoprocess).
%The platform adaption layers (PALs)
%are responsible for
%resolving the calling addresses inside the library OS, and passing the function arguments and return values in and out of the host ABI.



The host ABI inherit the Linux calling convention,
for the convenience of development.
The primary target of architecture in \graphene{} is x86-64;
in the x86-64 Linux call convention,
%The host ABI follows the Linux calling convention,
%which, on the x86 architecture, passes 
function arguments are passed in general-purpose registers, i.e., \code{rdi}, \code{rsi}, \code{rdx}, \code{rcx}, \code{r8}, and \code{r9}, in that order.
The host ABI also assumes the library OS to be compiled in ELF (extensible linking format).
The decision is to match the calling conventions
of application, library OS, and PAL,
to allow a commodity C compiler to improve the linking procedure.
For example, the GNU C Compiler (GCC) can optimize function name comparison
using hashing,
which can be simply adopted by the PAL
if it shares a linking scheme with the library OS.
Another benefit is to simplify the debugging with GDB, because GDB only recognizes one calling convention at a time.



%If the library OS and the hosts share the same calling convention, the PAL can directly call
%the host ABI, without any redirection.
%For other hosts,
% that do not share the calling convention with the library OS,
%such as OSX and Windows, the PALs redirect
%the function arguments to the correct registers or offsets on the stack frames assigned by the calling convention.
%Dynamic linking is beneficial for
%building a pluggable, host-independent library OS binary, and leveraging the compiler optimizations across the host ABI.





%By definition, the host ABI is limited to the OS abstractions to the local process.

%Most functions in the host ABI
%only have local effect to a picoprocess.
%Using the host ABI, the library OS can  llocate a memory region inside the current picoprocess, or create a handle that is only accessible by the current picoprocess.
%Only functions related with stream I/O
%can share OS resources with other picoprocesses, such as writing to the same file, sharing a network sockets, or connecting through a local RPC (remote-procedure call) stream.



\paragraph{Translating calling conventions.}
For a host with a different calling convention from Linux,
the PAL is responsible for translating the calling convention between the host ABI and the host system interface.
For example,
in Windows, function arguments are passed in a different order among general-purpose registers,
and thus require the Windows-specific PAL
to swap the register values when the library OS calls the host ABI.
Windows or OSX also accepts a different type of binary format,
and requires the PAL to implement an ad-hoc linking scheme for the library OS as an ELF binary.




\paragraph{Returning error codes.}
For explicity, a function in the host ABI only returns two types of results:
a non-zero number or pointer if the function succeeds, or zero if it fails.
Unlike the Linux call convention, the host ABI does not return negative values as error codes (e.g., \code{-EINVAL}).
Instead, the host ABI delivers the failure of a function call
as an exception, with the library OS capturing the failure by an assigned exception handler.
The design is to avoid confusing
the interpretation of return values from the host ABI.



\paragraph{Dynamic linking vs static linking.}
\graphene{} dynamic links the application, library OS, and PAL
in a process.
Dynamic linking ensures the complete reuse of an unmodified application,
as well as an unmodified library OS implementation.
\graphene{} allows the binaries of application, library OS, and PAL to be deployed individually to the users,
and be swapped with different implementations.
The dynamic linking of applications is a prerequisite
to running most of the Linux applications without modification or recompilation.



However, there are cases where static linking is preferred on a host (e.g., SGX), and recompilation is acceptable to the users.
Compiling the applications, library OS, and potential a PAL
into a single binary is similar to the technique of unikernels~\cite{unikernels},
which has the benefit of compiling out unnecessary code and execution paths
from the binary. 
Theoretically, it is possible for \graphene{} to statically link an application with the library OS and a PAL,
but this technique is out of the scope of this thesis.







