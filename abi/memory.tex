\subsection{Page management}
\label{sec:abi:memory}



%Linux applications 
%are generally developed and compiled under the assumption that the memory is managed
%by the OS with page-level granularity.
%%generally requires system API
%%for allocating, protecting, and deallocating memory at the same granularity.
%%A x86 application may require memory allocation to be strictly at the granularity of four-kilobyte pages,
%The primary Linux memory API,
%including \syscall{mmap}, \syscall{mprotect}, and \syscall{munmap},
%allows an application
%to arbitrarily create, modify, and destroy VMAs (virtual memory areas),
%wholly or partially,
%as long as the requested areas align to
%4KB.
%%Developers can avoid hard-coding the granularity in an application,
%%by retrieving the system setting using \syscall{getpagesize}.
%Specifically, the Linux dynamic loader (i.e., \code{ld.so}) %, an ubiqutous user-space component in Linux,
%uses \syscall{mmap} to map a binary file to a large memory area,
%and then divides the mapping into 4KB-aligned code and data segments.
%To implement the Linux memory API,
%the memory management scheme of the host ABI
%must be at least as fine-grained as 4KB pages.

\issuedone{1.2.a}{Discuss resource management at host level (pages)}
In the hos ABI, the abstraction for page management
is a {\bf virtaul memory area (VMA)}, a page-aligned, nonoverlaping region
in the guest's virtual address space.
A virtual memory area
specifies the memory region that requires the host OS to allocate the page resources,
either statically or dynamically.
There are two types of VMAs: one is a file-backed VMA, which is created by \palcall{StreamMap} and backs the memory pages with file data blocks.
The other type is an anoumous VMA, which is purely in DRAM and not backed by any files.
Either types of VMAs is part of the virtual address space,
and a VMA should never overlap with others created in the same virtual address space.
The host OS can choose to populate all the pages for a VMA immediately at creation of the VMA,
or delay the allocation until the first memory access (i.e., demand paging).






%The host ABI for memory management is as follows:
%\palcall{VirtMemAlloc} creates an anonymous, non-file memory mapping in a process, similar as \syscall{mmap};
%\palcall{VirtMemProtect} changes the read (R), write (W), or execution (X) permission in an address range;
%\palcall{VirtMemFree} frees the an address range.




\begin{paldef}
u64  VirtMemAlloc   (u64 expect_addr, u64 size, u16 protect_flags);
\end{paldef}


\palcall{VirtMemAlloc} creates an anonymous VMA in the guest memory. When \palcall{VirtMemAlloc} is given an expected address, the host OS must try to create the VMA at the exact address.
Otherwise, if no address is given, \palcall{VirtMemAlloc} can create the VMA at wherever the host OS sees fit, and does not overlap with existing VMAs.
Both \palkeyword{expect_addr} and \palkeyword{size}
must be page-aligned, and never exceed the permitted range in the guest's virtual address space.
\palkeyword{protect_flags} specifies the page protection in the created VMA, and can be given a combination of the following values: \palkeyword{READ}, \palkeyword{WRITE}, and \palkeyword{EXEC} (similar as \palcall{StreamMap} but without \palkeyword{WRITE_COPY}).
If \palcall{VirtMemAlloc} succeeds, it returns the starting address
of the created VMA, which the guest is permitted to access up to the given size.





\begin{paldef}
bool VirtMemFree    (u64 addr, u64 size);
bool VirtMemProtect (u64 addr, u64 size, u16 protect_flags);
\end{paldef}


\palcall{VirtMemFree} and \palcall{VirtMemProtect} modify one or more VMAs, 
by either freeing the pages
or adjusting the page protection in an address range.
Both \hostapis{} specify the starting address and size of the address range to modify;
the given address range must be page-aligned,
but can be any part of the guest virtual address apace,
and overlap with any VMAs, either file-backed or anonymous.
If the given address range overlaps with a VMA, the overlapped part is divided into a new VMA, and be destroyed or protected accordingly.


 



%The host ABI delegates physical memory management to the host. The division of labor between the library OS and the host is as follows:
%the library OS manages only the virtual memory layout of a process;
%the host is the one who decides the allocation of physical memory resources, among VMAs of all processes.
%The delegation avoids the competition between the host and guest
%on managing the physical memory, which is a common issue for virtualization.
%For example, a VM may suffer a problem called ``double-paging'' when both the host and guest is trying to swap unused pages out of memory; a VM needs to use techniques like ballooning~\cite{wldspurger02vmware-esx} to subtly coordinate with the host to make paging decisions,
%or use paravirtualization~\cite{vmware_vmi}.



\fixme{Moved from the beginning of this section}
The challenge to defining the host ABI for page management
is to accommodate different allocation models and granularities of the hosts.
A POSIX-style OS often assumes dynamic allocation with page granularity (normally with four-kilobyte pages);
the assumption is deeply ingrained in the design of page fault handler sand page table management
inside an OS like Linux or BSD;
the page management component in an OS
is usually low-level and closely interacting with the hardware interface,
to serve the needs of both the OS and applications.
Such an OS design makes it difficult to move page management
into the guest, unless using hardware virtualization such as
\fixmedp{Needs a cite, and probaly a more specific mention of what feature of VT you have in mind}
VT~\cite{VT}, which virtualizes page fault handlers and page table management to the guest.



