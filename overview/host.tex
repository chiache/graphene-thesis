\section{Host ABI compatibility}
\label{sec:overview:host}

A host ABI (application binary interface) defines the convention of application binaries, such as the binary format and the linking procedure. A host ABI also includes the system interfaces used to program the applications, each of which is called an application programming interface (API).
The host ABI describes the minimal functionality that a host OS has to implement, in order to support an application and other libraries, including a library OS.


The host ABI defines the acceptable binary format as a simplified version of the ELF (Excutable and Linkable Format).
Each host of \graphene{} is supposed implement a minimal dynamic loader,
which can load the \graphene{} library OS binary in ELF.
The library OS then provides the full-functioned dynamic loader,
which will loads and links the rest of application binaries, just like the native Linux loader (i.e., \code{ld.so}).




\subsection{The host abstractions}
\label{sec:overview:host:abstractions}


The host ABI shares several common abstractions with most OSes. The interfaces, as part of the host ABI, which access these host abstractions, are ultimately simplified to reduce the porting effort on each host.
Unlike the system interfaces in the OS, the host ABI does not prioritize backward compatibility. Therefore, the host ABI includes only the minimum interfaces that the library OS needs to interact with the host. The host ABI does not have to include any of  the legacy system interfaces from a production OS, let alone preserving different flavors of system interfaces for backward compatibility.


To bootstrap the host ABI, we partially adopt the high-level semantics of a host ABI defined by a prior work called \drawbridge{}~\cite{porter11drawbridge}.
\drawbridge{} is a library OS for reusing single-process Windows applications.
We use the definition of the \drawbridge{} host ABI, as a hint to create a list of host abstractions necessary for the library OS, including as streams, memory, threads, and processes. We also added other abstractions that are essential to the implementing of the Linux functionality, such as exception handing and the control of segment registers (i.e., \code{FS}/\code{GS} registers).
The host ABI eventually contains \palcalls{} host functions (Table~\ref{tab:overview:host-abi}), which are proved to be sufficient to implementing a large chunk of the Linux system interfaces.


\begin{table*}[t]
\centering
\begin{tabular}{|p{.2\textwidth}|p{.3\textwidth}|p{.5\textwidth}|}
\hline
{\bf Abstraction inherited from \drawbridge{}} & {\bf Function Names} & {\bf Description} \\
\hline
Streams & 
\begin{minipage}{.2\textwidth}
{\tt DkStreamOpen} \\
{\tt DkStreamRead} \\
{\tt DkStreamWrite} \\
{\tt DkStreamAttributesQuery} \\
{\tt DkStreamAttributesQuerybyHandle}
\end{minipage}
& 
Batch allocations of PIDs, children generally created using local state at parent.  \\
\hline
Signaling & PID to \picoproc{} map & Local signals call handler; remote signal delivery by RPC.  Cache mapping of PID to \picoproc{} ID. \\
\hline
\raggedright
Exit notification & 
\raggedright
Remote process status  & Exiting processes issue an RPC, or one synthesized if child becomes unavailable.  The {\tt wait} system call blocks until notification received by IPC helper. \\
\hline
{\tt /proc/[pid]} & Process metadata & Read over RPC.  \\
\hline
Message Queues & 
\raggedright
Key mapping, queue contents & Mappings managed by a leader, contents stored in various \picoprocs{}.  When possible, send messages asynchronously, and migrate queues to the consumer.\\
\hline
Semaphores & 
\raggedright
Key mapping, count. & Mappings managed by leader, migrate ownership to \picoproc{} most frequently acquiring the semaphore. \\
\hline
\raggedright
File System & 
\raggedright
File truncate sizes, FIFO files, domain sockets, symbolic links, file locks & No coordination, create special files in the host to represent symbolic links and file locks. \\
\hline
\raggedright
Shared File Descriptors & 
\raggedright
Seek pointers & Mappings managed by parent, migrate ownership to \picoproc{} most frequently accessing the file descriptors. \\
\hline
\end{tabular}
\caption{Overview of the \graphene{} host abstractions.}
\label{tab:overview:host-abi}
\end{table*}



\subsection{Porting the implementation of the host ABI}
\label{sec:overview:host:port}


\subsection{Remote procedure calls}
\label{sec:overview:host:rpc}


\subsection{Security isolation on the host abstractions}
\label{sec:overview:host:security}




\begin{comment}
The Platform Adaptation Layer (\pal{}) exports a generic host kernel interface to the guest, summarized in Table~\ref{tab:abi}.
Our Linux host \pal{} consists of a user library and a small Linux kernel module.
Our \pal{} library is implemented using  \nativecalls{} native Linux system calls.
%The specific division of labor between the host kernel and user library
%could be different in another implementation.
%We currently implement only a Linux host \pal{}.  
If a different host kernel implements the same \pal{} ABI,
{\tt libLinux.so} and all binaries
should execute without modification.


%How we connect the shim to the \pal{}?  Expedient to hack the loader to add it to the link tables.  Describe.  
%-Dynamically linked, included in loader (ld.so can't mmap the \pal{}; conceptually should be mapped by the OS)
%-imported by libc
%-- Does a new binary type require kernel mods?  If so, make claim about unmod host
%-- conceptual leakage by tightly coupling pal and loader; pal should be provided by the host OS
% validated that no other native system calls
%% were being issued by application directly to the host kernel with {\tt strace} 
%% and by disassembling application binaries.
%% In future work, we will replace the native Linux system call table 
%% with a restricted system call (or hyper call) table to host.
%potentially migrating much of the \pal{} implementation into the host kernel.

We selected the Drawbridge ABI~\cite{porter11drawbridge} as a starting point for the \sysname{} ABI,
because it strikes a good balance between host platform independence 
and avoiding duplicate functionality.
In porting Linux to the Drawbridge ABI, which was designed for a Windows library OS,
we strove to only add additional interfaces when otherwise unavoidable for
functionality or performance.
In some cases, revisiting the Drawbridge design choices 
could simplify {\tt libLinux},
but we wanted to understand how well the Drawbridge design generalizes to other guests, 
without undermining its simplicity.
\end{comment}

%We note that we reimplemented the Drawbridge ABI according to the published description and 
%expect that both systems two could be interoperable
%with a modest amount of reconciliation effort where the published specification is ambiguous;
%because Drawbridge is not publicly available, we cannot test interoperability of the library OSes.


\begin{comment}

\noindent \sysname{} requires  the following additional \pal{} calls:
\begin{compactenum}
%% dp: I think these are the key ones
%\item \fixmedp{What is missing here}
% What is: DkThreadSelf - is this necessary?  
% DkThreadPrivate
\item {\tt DkSetSegmentRegister} --- Set the optional segmentation
  register values ({\tt fs} and {\tt gs} on {\tt x86\_64}).  This functionality 
  is required
  for thread-local storage, which is initialized in user-space
  by the Linux loader.
  The Windows kernel initializes these registers before process execution begins, 
  obviating the need to initialize them from user-space.
  We further note that different OSes use these registers differently
  (e.g., 32-bit Windows uses {\tt gs} for TLS, whereas Linux and 64-bit Windows use {\tt fs} for TLS),
  therefore management of these registers should be in the domain of the guest.
\item {\tt DkSetExceptionHandler} --- Register a callback for a hardware exception, 
  such as a divide by zero error or segmentation fault.  This is roughly analogous 
  to a software emulated interrupt descriptor table (IDT).  
  %In future work,
  %we will adopt techniques proposed previously to send processor exceptions directly to the guest~\cite{belay12dune}.
\item {\tt DkExceptionReturn} --- Returns from an exception back to the host kernel.
   %% In our current design, the host kernel is responsible for returning control flow 
   %% back to the thread that was running when an exception occurs.
   %% Alternatively, the \sysname{} ABI could require exception delivery to include
   %% a register trapframe sufficient to resume execution directly in user space.
% Async upcall; return necessary?
%\fixmetsai{I am not sure how to put this, but the design of Return is to leave \pal{} independent from its host semantic
%%%    With the extra call, guests no long need to worry about the stack structure that
%%%    \pal{} created at exception handling. For example, \pal{} might put extra objects on stack, or use an
%%% alternative stack. These design will make the guest crash while trying to return.
\item {\tt DkCreatePhysicalMemoryChannel}, {\tt Dk\-Physical\-Memory\-Send}, and {\tt Dk\-Physical\-Memory\-Recv} --- These calls mark a set of pages in the sender's address space as copy-on-write (COW) 
  and then map the pages as COW in the receiver's address space.
  Although these calls are not strictly required, as
  data can be sent over a stream, we found the ability to exchange 
  large quantities of data efficiently to be an important 
  performance optimization for {\tt fork}  (\S\ref{sec:fork}).

\item {\tt DkHandleSend}, {\tt DkHandleRecv} --- Send a stream 
  handle out-of-band to another guest in the same sandbox.
  This is used to establish connections both for namespace
  coordination (\S\ref{sec:namespaces}) and 
  handle inheritance (\S\ref{sec:fork}).
  

%  This mechanism required extending the Linux kernel with a \gipclines{} line module,
%  discussed below.
%  This module does not require any kernel changes or recompilation.
  
\end{compactenum}
\fixmedp{isolate pal call?}

%%% Extra \pal{} calls we needed.  Describe issues. \fixmedp{Can I get a list?}
%%% -mutex (don)
%%% -cv (don)
%%% -IPC physical memory calls send/recv
%%% -set\_thread\_private - sets segmentation registers for TLS
%%% -synchronous event - sets hardware exception handlers + OS callbacks (seg fault, div zero, external kill, async io, interrupts, etc.) roughly analogous to idt
%%% -system initialize - get control block address for checkpointing/etc. - avoids assumptions about address space layout

\paragraph{Fast bulk data movement.} 
We extend the Linux host kernel with a
fast bulk IPC mechanism.  The bulk IPC mechanism does a 
copy-on-write (COW) data transfer between two guests.
A sender designates virtual pages to send to 
the receiver;
these pages are then placed in a first-in-first-out queue.
The receiver then requests these pages be mapped into its address space,
optionally providing a target virtual address (similar to {\tt mmap}).
The sent pages need not be virtually contiguous, 
allowing the guest to efficiently gather and send scattered pages.
%This is useful for implementing guest-managed page caches.

\sysname{} uses bulk IPC in conjunction with control
messages on a byte stream; 
the byte stream communicates the expected meaning and ordering 
of the corresponding bulk data transfer.
In the common case, these physical pages are quickly shared between
two processes, minimizing data copying
for user-level implementations of system calls like {\tt fork}.
%The CAS server commonly
%unmaps sent pages, which will reduce the page reference 
%count to one and eliminate the need to copy the page on a 
%on a write fault in the receiver.

Our bulk IPC module is \gipclines{} lines of code, 
runs on multiple versions of Linux (2.6 and 3 series kernels), and
does not require
Linux kernel changes or recompilation.
\end{comment}

%% dp: So sad, but something has to go
\begin{comment}
\paragraph{Efficient creation of clean processes.}
We implement a process creation ABI in the host 
that only creates 
clean child picoprocesses that do not inherit memory mappings
or open file handles from the parent, similar to booting another VM.
This design choice strengthens security isolation.

Linux does not provide a process creation mechanism other than {\tt fork()} (or variants, such as {\tt clone()}).
Thus, the \pal{} is must implement this abstraction on Linux processes.
A simple implementation of {\tt Dk\-Pro\-cess\-Create()} would {\tt fork} 
a new process,
clean the open file handles, and then {\tt exec} the new binary.
This simple implementation of process creation
wastes work implementing an ABI that will always {\tt exec()} a clean process.
%%% The {\tt fork-exec} approach bring unwanted overhead to {\tt fork} and {\tt execve}
%%% system call. Compare to the native system calls, reloading the whole \sysname{}
%%% system causes duplicated works such as parsing manifests and linking libraries.

The \pal{} optimizes process creation by forking 
a newly launched picoprocess after initializing all of the operating system state,
but before starting any application-specific work, such as calling the ELF loader.
Dune leverages a similar optimization to efficiently create
isolated threads in the Wedge case study~\cite{belay12dune}.
The checkpoint process is not visible to the application.
This checkpoint listens for a message from the \pal{} to 
fork itself in order to create a clean child.

Because new processes are created by forking the checkpoint process,
as a guest creates children, they are actually siblings
from the perspective of the host kernel.
The guest implements the abstraction of a Linux process tree inside the library OS.

This checkpointing strategy works well even if the application 
requests that the child load a different application
binary, because the checkpoint is taken before {\tt ld.so} links in the application.
In a simple microbenchmark,
this checkpointing strategy improved picoprocess creation time
 a factor of two.

%This approach is \fixmedp{XX\%} faster than simply {\tt exec}-ing 
%the \pal{} in the child to clean up parent state.

%%% This strategy of checkpointing the process manager allows \sysname{} to quickly 
%%% create clean child processes in user-space on an unmodified Linux kernel.  
%%% In a simple microbenchmark,
%%% this pre-forking strategy improved {\tt DkProcessCreate} time
%%% by a factor of two.


\end{comment}
