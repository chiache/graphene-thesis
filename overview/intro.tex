This chapter gives an overview of \graphene{}.
%\fixme{define compatibility upfront; a lightning-rod definition}
%The compatibility of an OS is the ability to reuse the application code developed on the system interfaces.
%By introducing an additional compatibility layer as the library OS,
%the existing applications can be reused on a set of simpler, more adaptable interfaces.
The design of \graphene{} is divided into two parts:
a host ABI (application binary interface) for porting to new OSes and hardware,
and a library OS which is compatible against Linux applications.
This chapter first provides an overview of the host ABI, and the design principles behind the definition.
%A host ABI is defined to be ported to each host, with design principles to be both simple and sufficient for running the library OS.
%a narrowed set of %system interfaces, a.k.a. the 
%a set of host \fixmedp{spell out ABI at the first appearance} ABIs,
%\fixmedp{this definition is confusing}
%including OS functions which export OS states to the hosts (\S\ref{sec:overview:host}).
%on a host (i.e., an OS paired with a physical machine) in the OS design.
The rest of chapter
discusses the design of the library OS, including the architecture,
approaches to implementing the Linux API,
and trade-offs between compatibility, performance, and security isolation.


