%\fixme{define compatibility upfront; a lightning-rod definition}
%The compatibility of an OS is the ability to reuse the application code developed on the system interfaces.
%By introducing an additional compatibility layer as the library OS,
%the existing applications can be reused on a set of simpler, more adaptable interfaces.
This chapter gives an overview of \graphene{}.
% as a compatibility layer to reuse unmodified Linux applications. For OS developers, \graphene{} reduces the development effort for achieving compatibility,
%on an OS either developed from scratch or ported for new hardware.
%on an OS ported for new hardware or a new OS prototype.
%, with compatibility against a wide range of Linux applications.
%, \fixmedp{the flow here is wrong; do applications -> hosts} which facilitates the implementation %of compatibility on host platforms \fixmedp{incompatible with makes more sense} incompatible to unmodified Linux applications.
The design of \graphene{} is divided into two parts:
a host ABI (application binary interface) for porting to new OS or hardware,
and a library OS for restoring Linux compatibility.
This chapter first provides an overview of the host ABI, and the design principles for defining the host ABI.
%A host ABI is defined to be ported to each host, with design principles to be both simple and sufficient for running the library OS.
%a narrowed set of %system interfaces, a.k.a. the 
%a set of host \fixmedp{spell out ABI at the first appearance} ABIs,
%\fixmedp{this definition is confusing}
%including OS functions which export OS states to the hosts (\S\ref{sec:overview:host}).
%on a host (i.e., an OS paired with a physical machine) in the OS design.
Then, the rest of chapter
discusses the design points of the library OS, including the architecture,
approaches to implementing Linux functionality and API,
and trade-offs between compatibility, performance, and security isolation.


%and the trade-offs between growing Linux functionality, performance overhead, memory footprint, and security isolation.
%will reproduce a rich of Linux functionality, including multi-process abstractions, with acceptable trade-offs of performance and security isolation.
%\graphene{} is
%a library OS built upon the narrowed host ABIs (\S\ref{sec:overview:libos}), to reuse unmodified Linux applications
%\fixmedp{maybe break into different sentence, or reorder it}
%on various hosts with either alternative kernels (e.g., Windows) or new hardware platforms (e.g., Intel SGX).
%(e.g., Intel SGX). 
%, to improve the compatibility of the host.
%More details of the host ABI and the library OS will be discussed in the following chapters.
%The discussion will focus on the design principles followed by the whole \graphene{} library OS,
%including 