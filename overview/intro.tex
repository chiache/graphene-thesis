%\fixme{define compatibility upfront; a lightning-rod definition}
%The compatibility of an OS is the ability to reuse the application code developed on the system interfaces.
%By introducing an additional compatibility layer as the library OS,
%the existing applications can be reused on a set of simpler, more adaptable interfaces.
This chapter gives a overview of the \graphene{} library OS, as a compatibility layer to reuse unmodified Linux applications. For OS developers, \graphene{} targets on reducing the development effort for achieving reasonable compatibility,
on each OS built from scratch or ported for new hardware.
%on an OS ported for new hardware or a new OS prototype.
%, with compatibility against a wide range of Linux applications.
%, \fixmedp{the flow here is wrong; do applications -> hosts} which facilitates the implementation %of compatibility on host platforms \fixmedp{incompatible with makes more sense} incompatible to unmodified Linux applications.
The design of \graphene{} is divided into two parts: a host ABI for porting, and a library OS for reproducing Linux system interfaces.
Section~\ref{sec:overview:host} will describe the principles in the definition of the host ABI. 
%A host ABI is defined to be ported to each host, with design principles to be both simple and sufficient for running the library OS.
%a narrowed set of %system interfaces, a.k.a. the 
%a set of host \fixmedp{spell out ABI at the first appearance} ABIs,
%\fixmedp{this definition is confusing}
%including OS functions which export OS states to the hosts (\S\ref{sec:overview:host}).
%on a host (i.e., an OS paired with a physical machine) in the OS design.
Section~\ref{sec:overview:libos} will discuss the architecture of the library OS,
and the trade-offs between building a rich of Linux functionality, performance, memory footprint, and security isolation among applications.
%will reproduce a rich of Linux functionality, including multi-process abstractions, with acceptable trade-offs of performance and security isolation.
%\graphene{} is
%a library OS built upon the narrowed host ABIs (\S\ref{sec:overview:libos}), to reuse unmodified Linux applications
%\fixmedp{maybe break into different sentence, or reorder it}
%on various hosts with either alternative kernels (e.g., Windows) or new hardware platforms (e.g., Intel SGX).
%(e.g., Intel SGX). 
%, to improve the compatibility of the host.
%More details of the host ABI and the library OS will be discussed in the following chapters.
%The discussion will focus on the design principles followed by the whole \graphene{} library OS,
%including 