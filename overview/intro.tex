\issuedone{1.1.a}{Reorganization: Graphene overview}
This chapter gives the overview of \graphene{}.
%\fixme{define compatibility upfront; a lightning-rod definition}
%The compatibility of an OS is the ability to reuse the application code developed on the system interfaces.
%By introducing an additional compatibility layer as the library OS,
%the existing applications can be reused on a set of simpler, more adaptable interfaces.
The development of \graphene{} is divided into two parts.
The first part is a host interface which encapsulates the abstractions needed from a host OS.
On each host target, a PAL (platform adaption layer) instantiates the host ABI,
including \palcallnum{} \hostapis{}.
The second part is a \libos{}, or \thelibos{}, which emulates a substantial subset of Linux system calls.
%and the \graphene{} library OS. % to be compatible against Linux applications.
This chapter first introduces the host ABI definitions and principles,
%A host ABI is defined to be ported to each host, with design principles to be both simple and sufficient for running the library OS.
%a narrowed set of %system interfaces, a.k.a. the 
%a set of host \fixmedp{spell out ABI at the first appearance} ABIs,
%\fixmedp{this definition is confusing}
%including OS functions which export OS states to the hosts (\S\ref{sec:overview:host}).
%on a host (i.e., an OS paired with a physical machine) in the OS design.
followed by discussion of the \thelibos{} architecture,
%ble approaches to
emulation strategies, % to emulate Linux functionality,
and trade-offs.
 % between compatibility, performance, and security isolation.


