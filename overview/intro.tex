%\fixme{define compatibility upfront; a lightning-rod definition}
%The compatibility of an OS is the ability to reuse the application code developed on the system interfaces.
%By introducing an additional compatibility layer as the library OS,
%the existing applications can be reused on a set of simpler, more adaptable interfaces.
This chapter gives a overview of the \graphene{} library OS.
The overview is focused on the design for improving OS compatibility against applications. For OS developers, \graphene{} targets on reducing the development effort 
to port an OS to new hardware, or to build an OS prototype.
%, \fixmedp{the flow here is wrong; do applications -> hosts} which facilitates the implementation %of compatibility on host platforms \fixmedp{incompatible with makes more sense} incompatible to unmodified Linux applications.
Section~\ref{sec:overview:host} will describe
the host ABI to implement on each host, and design principles for keeping the host ABI simple but sufficient to run a library OS with rich Linux functionality.
%a narrowed set of %system interfaces, a.k.a. the 
%a set of host \fixmedp{spell out ABI at the first appearance} ABIs,
%\fixmedp{this definition is confusing}
%including OS functions which export OS states to the hosts (\S\ref{sec:overview:host}).
%on a host (i.e., an OS paired with a physical machine) in the OS design.
Section~\ref{sec:overview:libos} will describe the library OS design, and how it is built upon the simple host ABI, with reasonable performance overhead and security isolation model.
%\graphene{} is
%a library OS built upon the narrowed host ABIs (\S\ref{sec:overview:libos}), to reuse unmodified Linux applications
%\fixmedp{maybe break into different sentence, or reorder it}
%on various hosts with either alternative kernels (e.g., Windows) or new hardware platforms (e.g., Intel SGX).
%(e.g., Intel SGX). 
%, to improve the compatibility of the host.
%More details of the host ABI and the library OS will be discussed in the following chapters.
%The discussion will focus on the design principles followed by the whole \graphene{} library OS,
%including 