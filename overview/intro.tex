%\fixme{define compatibility upfront; a lightning-rod definition}
%The compatibility of an OS is the ability to reuse the application code developed on the system interfaces.
%By introducing an additional compatibility layer as the library OS,
%the existing applications can be reused on a set of simpler, more adaptable interfaces.
This chapter gives an overview of the \fixmedp{which OS?} OS design, \fixmedp{the flow here is wrong; do applications -> hosts} which facilitates the implementation of compatibility on host platforms \fixmedp{incompatible with makes more sense} incompatible to unmodified Linux applications.
The chapter will first describe %a narrowed set of %system interfaces, a.k.a. the 
a set of host \fixmedp{spell out ABI at the first appearance} ABIs,
\fixmedp{this definition is confusing}
including OS functions which export OS states to the hosts (\S\ref{sec:overview:host}).
%on a host (i.e., an OS paired with a physical machine) in the OS design.
The discussion then presents an overview of \graphene{},
%\graphene{} is
a library OS built upon the narrowed host ABIs (\S\ref{sec:overview:libos}), to reuse unmodified Linux applications
\fixmedp{maybe break into different sentence, or reorder it}
on various hosts with either alternative kernels (e.g., Windows) or new hardware platforms (e.g., Intel SGX).
%(e.g., Intel SGX). 
%, to improve the compatibility of the host.