The compatibility of an OS relies on reusing the source code developed on the system interfaces.
By introducing an additional layer as the {\em library OS},
the existing applications can be reused on this layer, which can be then reused on simpler, more adaptable interfaces.
This chapter will first describe a set of system interfaces (known as the host ABI), which are defined for adaptability on various OSes and hardware platforms. The discussion is then followed by the overview of \graphene{}, a library OS built on top of the host ABI, in order to reuse unmodified Linux applications.