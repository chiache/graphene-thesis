%\fixme{define compatibility upfront; a lightning-rod definition}
%The compatibility of an OS is the ability to reuse the application code developed on the system interfaces.
%By introducing an additional compatibility layer as the library OS,
%the existing applications can be reused on a set of simpler, more adaptable interfaces.
This chapter gives a overview of the \graphene{} library OS, for compatibility against unmodified Linux applications. For OS developers, \graphene{} targets on reducing the development effort on fulfilling application compatibility,  
for porting an OS to new hardware, or building an OS prototype.
%, with compatibility against a wide range of Linux applications.
%, \fixmedp{the flow here is wrong; do applications -> hosts} which facilitates the implementation %of compatibility on host platforms \fixmedp{incompatible with makes more sense} incompatible to unmodified Linux applications.
A host ABI is defined to be ported to each host, with design principles to be both simple and sufficient for running the library OS.
%a narrowed set of %system interfaces, a.k.a. the 
%a set of host \fixmedp{spell out ABI at the first appearance} ABIs,
%\fixmedp{this definition is confusing}
%including OS functions which export OS states to the hosts (\S\ref{sec:overview:host}).
%on a host (i.e., an OS paired with a physical machine) in the OS design.
Then, the library OS built upon the host ABI
will reproduce a rich of Linux functionality, including multi-process abstractions, with acceptable performance and security isolation.
%\graphene{} is
%a library OS built upon the narrowed host ABIs (\S\ref{sec:overview:libos}), to reuse unmodified Linux applications
%\fixmedp{maybe break into different sentence, or reorder it}
%on various hosts with either alternative kernels (e.g., Windows) or new hardware platforms (e.g., Intel SGX).
%(e.g., Intel SGX). 
%, to improve the compatibility of the host.
%More details of the host ABI and the library OS will be discussed in the following chapters.
%The discussion will focus on the design principles followed by the whole \graphene{} library OS,
%including 