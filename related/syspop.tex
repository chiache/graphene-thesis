\section{Measurement and Study of System APIs}

%\fixmedp{I am not sure about this paragraph.  I probably need to discuss in person} \fixmebj{any better?}

Concurrent with our work, Atlidakis et al.~\citep{atlidakis16posix} conducted a similar 
study of POSIX.
A particular focus of the POSIX study is measuring fragmentation across different POSIX-compliant OSes
(Android, OS X, and Ubuntu), as well as identifying points where higher-level frameworks
are driving this fragmentation, such as the lack of a ubiquitous abstraction for graphics.
Both studies identify long tails of unused or lightly-used functionality in OS APIs.
The POSIX study factors in dynamic tracing, which can yield performance insights;
our study uses installation metrics, which can yield insights about the impact of incompatibilities end-users.
Our paper contributes complimentary insights, such as a metric and incremental path for 
completeness of an emulation layer, as well as analysis of the importance of less commonly-analyzed 
APIs, such as pseudo-files under {\tt /proc}.

%% \fixmetsai{My crack on discussing this work.}
%% Atlidakis, et. al use a related technique of static analysis,
%% to trace POSIX API linkage in multiple OSes such as Android, OS X and Ubuntu.
%% The study also uses dynamic analysis, as to intercept applications' runtime API usage using a library called {\tt libtrack},
%% to trace dynamic invocation of POSIX abstractions in 45 popular Android applications.
%% While our work is focused on
%% studying more universal, modern Linux API importance to end users on a large scale,
%% their study targets on elaborating the dependency on POSIX APIs
%% in applications,
%% and more importantly,
%% the performance implications of POSIX APIS to application in the runtime.

A number of previous studies have investigated how other portions of the Operating System
interact, often at large scale.
Kadav and Swift~\citep{understand-drivers} studied the effective API the Linux kernel exports 
to device drivers, as well as device driver interaction with Linux---complementary to our study
of how applications interact with the kernel or core libraries.
Palix et al.\ study faults in all subsystems of the Linux kernel and identify the 
most fault-prone subsystems~\citep{linux-faults}.
They find architecture-specific subsystems have highest fault rate, followed by file systems.
Harter et al.~\citep{file-not-file} studied
 the interaction of a set of Mac OS X applications with the file system APIs---identifying a number
of surprising I/O patterns.
%Other systems have studied specialized set of software that 
%interact with specific components of Linux kernel.
%%For instance, Harter et.al. study only I/O behavior of iBench applications on Mac to conclude that each of these applications form a mini-filesystem amongst themselves and their interactions with the
%filesystem kernel component are different than other standard applications~\citep{file-not-file}.
%Another study of device drivers classifies drivers into 5 different classes after analyzing the interacion of the drivers with the kernel, devices, and buses~\citep{understand-drivers}.
%SymDrive identifies busgs in device drivers in the kernel by studying I/O changes after the driver patches~\citep{symdrive}. \fixmedp{Is this really related?  I was thinking of a different swift paper}
Our approach is complementary to these studies, with a focus on overall API 
usage across an entire Linux distribution.

%% Our study however is more general as 
%% we focus on all the applications in the Debian/Ubuntu repositories
%% and the applications' use of different kernel interfaces including the 
%% standard system calls, vectored system calls and other interfaces such as
%% /proc, /sys and /dev.


%\note{about 1/2 page}

A number of previous studies have drawn inferences about user and developer behavior
using Debian and Ubuntu package metadata and popularity contest statistics.
Debian packages have been analyzed to study the evolution 
of the software itself~\citep{macro-study, life-death, mining-over-time}, 
to measure the popularity of application programming 
languages~\citep{upgrade-libre}, to analyze dependencies between the packages~\citep{package-dependency}, 
to identify trends in package sizes~\citep{pigs-to-stripes}, 
the number of developers involved in developing and maintaining a package~\citep{toy-story},
%\fixmedp{right?}\fixmebj{yeah}, 
and estimating the cost of development~\citep{measuring-woody}.
Jain et al.\ used popularity contest survey data to 
prioritize the implementation effort for new system security policies~\citep{jain14setuid}. % \fixmebj{nice one :)}.
This study is unique in using this information to infer the relative importance of 
of system APIs to end users, based on frequency of application installation.

%%% Although our study is unique in its focus on footprint of interaction
%%% between the applications and the kernel interfaces, a body of similar application studies
%%% exist that analyze different questions than the ones discussed in this paper. 

%%% On the other hand, we analyze the Debian and Ubuntu package repositories to 
%%% study the kernel interface footprints of the applications
%%% and the minimal set of interfaces necessary to support those applications.

%\fixmedp{How do these studies quantify compatibility?}
A number of previous projects develop techniques or tools to identify software incompatibilities, with 
the goal of avoiding subtle errors during integration of software components.
The Linux Standard Base (LSB)~\citep{linux-standard-base} 
predicts whether an application can run on a given distribution based on the 
symbols imported by the application from system libraries.
%exported symbols and libraries provided by that distribution.
%examines
%applications to get the binary symbols and libraries used by those applications.
%LSB goal is
%Several previous studies have examined compatibility of different software library implementations~\citep{linux-standard-base}
Other researchers have studied application compatibility 
across different versions of same library,
creating rules for library developers
to maintain the compatibility across versions~\citep{shared-library}.
Previous projects have also developed tools to verify backward compatibility of libraries, based on checking for any changes in library variable 
type definitions and function signatures~\citep{backward-compatibility}.
%that can cause backward binary compatibility problems
Another variation of compatibility looks at integrating independently-developed components of a larger software project; 
solutions examine various attributes of the components' source code, such as recursive functions and strong coupling of different classes~\citep{component-compatibility}.
In these studies, compatibility is a binary property, reflecting a focus on correctness. Moreover, these studies are focused on the interface between the application and the libraries or distribution ecosystem.
In contrast, this paper proposes a metric for relative completeness of a prototype system. % with a focus on the compatibility between the applications or libraries and the kernel ABIs.


%%% Research has been done on the compatibility of components during software development~\citep{component-compatibility}.
%%% There are verifiers for compatibility
%%% between software components such as binaries and libraries based on
%%% broad spectrum of compatibility problems~\citep{backward-compatibility}.
%%% Other studies focus on compatibility of application binaries with the different 
%%% versions of libraries~\citep{shared-library} and with libraries provided in different 
%%% distribution environments~\citep{linux-standard-base}.
%%% However, we study the kernel-userspace ABI compatibility between the application binaries
%%% and systems that support linux ABI.

%\fixmedp{I'm not sure what the theme of this paragraph is.  Everything is listed here, but it is pretty terse.
%I like to frame each paragraph as ``what is known'', ``how it has been used'', and ``what is new here'' (and maybe why important)}
%\fixmebj{This paragraph is about all the related work that does any sort of system call profiling for applications and what do they do with it.}

Identifying the system call footprint of an application is useful for a number of reasons;
our work contributes data from studying trends in system API usage in a large set of application software.
The system call footprint of an application can be extracted by static or dynamic analysis.
The trade-off is that dynamic analysis is easier to implement quickly, but the results are input-dependent.
Binary static analysis, as this paper uses, 
can be thwarted by obfuscated binaries,
which can confuse the disassembler~\citep{zhang13cfi}.
%is also subject to the limitations of the disassembler to deal with
Static binary analysis has been used to automatically generate application-specific sandboxing 
policies~\citep{policy-extraction}.
Dynamic analysis has been used to compare system call sequences of two applications
as an indicator of potential intellectual property theft~\citep{software-theft},
to identify opportunities to batch system calls~\citep{multi-call},
to model power consumption on mobile devices~\citep{power-modelling},
and to repackage applications to run on different systems~\citep{cde}.
These projects answer very different questions than ours, but could, in principle, benefit
from the resulting data set.

%Many previous works have used static as well as dynamic analysis of applications to determine the system calls that they use.
%While dynamic analysis is the easiest to get a system call trace of the applications, it is very difficult to guarantee that all branches in the code
%are exercised. On the other hand while static analysis can cover the complete binary, we share the same assumption as other static analysis solutions that the application is not obfuscated.
%Researches have used static analysis to identify system call footprints of applications to automatically generate application-specific sandboxing policies
%% Others have also used dynamic analysis to determine and compare system call sequences within applications to detect software threat assuming that common system call sequences may imply
%% common or stolen code~\citep{software-theft}.
%% \fixmedp{I don't get this one---a little more detail please?}\fixmebj{how about now?}
%% Another work reduces the number of kernel boundary crossings by replacing group of system calls, that are identified by dynamically profiling applications, with a single system call~\citep{multi-call}.
%% %and optimize the performance by clustering multiple related system calls~\citep{multi-call}\fixmedp{or this one?}.
%% Runtime system call tracing of applications has also been used to model 
%% fine-grained power consumption for Smartphones~\citep{power-modelling}.
%% CDE~\citep{cde} use system call information of x86 Linux 
%% programs to package them differently to run on other x86 Linux machines.
%% All these research works can benefit from our study and can be applied to large number of applications in the Ubuntu repository.
%% Moreover, our study answers a different question of what fraction of these packages
%% can be compatible to a Linux ABI based system.

 


