\section{Library OSes}


\begin{comment}
Recent library OSes, including Graphene,
search for a better
division of labor between the host kernel and guests.
Paravirtualized VMs attempt to move away from modeling specific hardware designs in software
toward a more virtualization-friendly hardware model~\citep{barham03xen,whitaker02denali, eiraku09outsourcing}.
Library OSes can be viewed as extreme paravirtualization---attempting
to find the most ideal interface between guest and host. %abstractions of hardware without also baking-in host semantics.
\end{comment}

\paragraph{Previous library OSes.}
This work extends previous library OSes~\citep{porter11drawbridge,xax,unikernels,baumann13bascule,osv},
which focused on single-process applications,
to support coordination abstractions required 
for multi-process applications, such as shell scripts.


Bascule~\citep{baumann13bascule} implements a Linux library OS on a variant of the Drawbridge ABI,
but does not include support for multi-process abstractions such as signals or copy-on-write fork.
The Bascule Linux library OS also implements fewer Linux system calls than Graphene, missing
features such as signals.
Bascule demonstrates a complementary facility to Graphene's multi-process support: composable library OS extensions, 
such as speculation and record/replay.
OSv is a recent open-source, % project to design and implement a 
single-process 
library OS to support a managed language runtime, such as Java, on a bare-metal hypervisor~\citep{osv}.


A number of recent projects have provided a minimal, isolated environment
for web applications to download and execute native code~\citep{nacl,xax,howell13refactoring,gazelle,atlantis}.
The term ``picoprocess'' is adopted from some of these designs, and they share 
the goal of pushing system complexity out of the kernel and into the application.
%%% Despite these common design themes, the motivations
%%% for these projects varies widely, including
%%% robustly enforcing browser policy\citep{gazelle};
%%% ensuring that a web page is compatible with the JavaScript, CSS, and other interpreters in the browser~\citep{atlantis};
%%% and simply the ability to push code from the data center onto the client where it makes 
%%% sense~\citep{nacl, embassies, howell13refactoring}.
Unlike a library OS, these systems
generally sacrifice the ability to execute unmodified application code, 
eliminate common UNIX multi-process functionality (e.g., fork), or both.


The term library OS also refers to an older generation of research
focused on tailoring hardware management heuristics 
to individual application needs~\citep{kaashoek97exokernel,anderson92libos,cheriton94cache,leslie96nemesis,libra},
whereas newer library OSes, including Graphene, focus
on providing application compatibility
across different hosts without dragging along an entire legacy OS.
A common thread among all libOSes is moving functionality from the kernel
into applications and reducing the TCB size or attack surface.
Kaashoek et al.~\citep{kaashoek97exokernel} identify multi-processing as a problem for an Exokernel libOS,
and implemented some shared OS abstractions.
The Exokernel's sharing designs rely on shared memory rather than byte streams,
and would not work on recent libOSes,
nor will they facilitate dynamically sandboxing two processes.

%% In moving services out of the kernel and into user libraries our design resembles a microkernel~\citep{liedtke95sosp, Baron:1985:MOE}.
%% %liedtke93sosp,liedtke95sosp, Baron:1985:MOE, Jones:1986:MMK, Rashid:1987:RAM}.
%% An important distinction is that
%% microkernels generally centralize system services in a single trusted daemon,
%% whereas Graphene distributes management of coordination APIs among
%% cooperating guests.

\begin{comment}
Dune~\citep{belay12dune} %merges the concept of a process and VM,
leverages virtualization hardware 
to allow an application
to safely manage privileged CPU features
such as page tables and interrupts.
%Dune leverages virtualization hardware  to safely 
%isolate privileged hardware access within a process.
Dune's goals are complimentary to ours; we expect that
certain aspects of the PAL implementation would be simplified on Dune.
\end{comment}

User Mode Linux~\citep{user-mode-linux} (UML) executes a Linux kernel inside a process
by replacing  architecture-specific code with 
code 
that uses Linux host system calls. % to emulate this functionality.
%Because UML runs a complete Linux kernel and multiple processes,
%it is effectively
UML is best described as  an alternative approach to paravirtualization~\citep{barham03xen},
and, unlike a library OS, does not deduplicate functionality.

\paragraph{Distributed coordination APIs.}
Distributed operating systems, such as LOCUS~\citep{locus83sosp,fleisch86locus}, Amoeba~\citep{mullender90amoeba,cheriton89naming} and  
Athena~\citep{champine90athena} required a consistent namespace for process identifiers and other IPC abstractions
across physical machines.
%A key requirement for naming in a distributed OS is network transparency---that the name should be independent of the physical placement of an object in the system.
Like microkernels, these systems generally centralize all management in a single, system-wide service.
%For instance, Amoeba has a single session server that allocates process identifiers.
%This system service may be replicated or partitioned for improved performance on a large network or to
%divide management responsibility across administrative domains~\citep{cheriton89naming}.
%Each Graphene guest also needs to coordinate naming of Unix abstractions, such as processes or System V IPC keys.
%Although Graphene draws some inspiration from these distributed OSes, 
%Graphene's goals and usage model are different, and 
Rote adoption of a central name service does not meet our goals
of security isolation and host independence.

Several aspects of the Graphene host kernel ABI are similar to the
Plan 9 design~\citep{pike90plan9}, including the unioned view of the host file system
and the inter-picoprocess byte stream.
Plan 9 demonstrates how to implement this host kernel ABI,
whereas Graphene uses a similar ABI 
to encapsulate multi-process coordination 
in the libOS.

%% or coordinating coordinate a distributed file system
%% across multiple workstations,
%% but rather to demonstrate how it can be leveraged to 
%% implement a more efficient guest library OS. instance, Plan 9 uses the 9P protocol to coordinate a distributed file system
%% across multiple workstations. In contrast, Graphene uses streams




Barrelfish~\citep{baumann09barrelfish} argues that multi-core scaling is best served 
by replicating shared kernel abstractions at every core, and using message passing 
to coordinate updates at each replica, as opposed to using cache coherence to update a shared data structure.  
Barrelfish is a new OS; in contrast,
Cerberus~\citep{song11eurosys} applies similar ideas to coordinate abstractions
across multiple Linux VMs running on Xen.
%Cerberus coordinates a unified view of the process tree, file system, network card,
%and the address space when threads run on separate cores---forwarding 
%some system calls to other kernels as remote procedure calls.
In order for a library OS to provide multi-process abstractions, 
Graphene must solve some similar problems, but innovates by
replicating additional classes of coordination abstractions, such as System V IPC,
and facilitates dynamic sandboxing.
%and optimizes performance through 
%techniques such as lazy discovery.
The focus of this paper is not on multi-core scalability, but on security isolation and compatibility with legacy, multi-process applications.
That said, we expect that systems like Barrelfish~\citep{baumann09barrelfish} 
could leverage our implementation techniques to efficiently 
construct higher-level OS abstractions, such as System V IPC and signals.

%% We note that several recent projects argue that multi-core scalability 
%% is enhanced by eschewing cache coherence and shared data structures, 
%% in favor of a similar 
%% design which
%% replicates state across cores
%% {\em within an OS kernel}~\citep{baumann09barrelfish} or {\em across legacy guest OSes}~\citep{song11eurosys}.
%% The focus of this paper is not on multi-core scalability, but on security isolation and compatibility with legacy, multi-process applications.
%% That said, we expect that systems like Barrelfish~\citep{baumann09barrelfish} 
%% could leverage our implementation techniques to efficiently 
%% construct higher-level OS abstractions, such as System V IPC and signals.

L3 introduced a ``clans and chiefs'' model of IPC redirection, in which IPC to
a non-sibling process was validated by a the parent (``chief'') before a message could leave the clan~\citep{liedtke92clans}.
%This model was primarily used to enforce access control on system abstractions,
%and abandoned for capabilities on kernel objects in L4
Although this model was abandoned as cumbersome for general-purpose access control~\citep{elphinstone13microkernels},
the Graphene sandbox design observes
that a stricter variation is a natural fit
for security isolation among multi-process applications.

Cerberus focuses on replicating lower-level state, such as process address spaces
which Graphene leaves in the host kernel.
As a result, the performance characteristics are different.
Although this comparison is rough, 
we replicated their test of ping-ponging 1000 
{\tt SIGUSR1} signals and compare the ratio to their reported data, 
albeit with different hardware and our baseline kernel is newer 
(3.2 vs 2.6.18).  
When signals are sent inside of a single guest on Graphene, they are {\em faster}
by 79\%, whereas performance drops by a 5.5--18$\times$ on Cerberus.
When passing signals across coordinating guests both approaches are competitive:
Graphene's cross-process signal delivery is 4.6$\times$ slower than native, whereas Cerberus ranges from 
3.3--11.3$\times$ slower, depending on the hardware.


%% For instance, Graphene guests may require a mixture of isolated and shared namespaces.
%% Moreover, Graphene naming must support dynamic detachment of one guest from a confederation of other guests.
%% Our work contributes  an implementation of distributed naming inside the library OS, facilitating security isolation, 
%% and application-transparent detachment.

\paragraph{Migration and security isolation.}
Researchers have added checkpoint and migration support to Linux~\citep{linux-cr}
by serializing kernel data structures to a file
and reloading them later.  
This introduces several challenges, including
security concerns of loading data structures into the OS kernel from a potentially 
untrusted source.
%, as well as significant engineering and maintenance effort.
In contrast, Graphene checkpoint/restore 
%on a simple host ABI
requires little more than a guest memory dump.
%The host state can be easily restricted by a sandbox.

%% supports migration of one or more process across
%% hosts with the same kernel.
%% To maintain consistency of namespace and resource throughout migration,
%% Zap provides a thin virtualization layer upon which
%% application can have the same virtualized and conflict-free view of system.
%% To simplify process dependency that complicates migration, Zap migrates
%% processes in the unit of Pods, a container with virtualized namespace,
%% memory checkpoints and private file system. 

%Docker - http://www.docker.io/ - really looks like it just automates creation of a minimal hard drive

OS-based virtualization, such as 
Linux VServer~\citep{vserver},  containers~\citep{bhattiprolu08containers},
and Solaris Zones~\citep{price04zones},
implement security isolation by maintaining multiple copies of kernel data structures,
such as the process tree,
in the host kernel's address space.
In order to facilitate sandboxing, 
Linux has added support for launching single processes
with isolated views of namespaces, including process IDs and network interfaces~\citep{lwn-namespaces}.
FreeBSD jails apply a similar approach to augment an isolated {\tt chroot} environment
with other isolated namespaces, including the network and hostname~\citep{jails}.
Similarly, Zap~\citep{osman02zap} migrates groups of process, called a Pod,
which includes a thin layer virtualizing system resource names.
In these approaches, all guests must use the same OS API, and the host kernel
still exposes hundreds of system calls to all guests.
Library OSes move these data structures into the guest, enabling
a range of personalities to run on a single guest and limiting the attack surface
of the host.


Shuttle~\citep{shan12shuttle} permits selective violations of strict isolation
to communicate with host services 
under OS-based virtualization.
For example, collaborating applications may communicate using the Windows Common Object Model (COM);
Shuttle develops a model to permit access to the host COM service.
%Windows guest applications 
%may rely on the host's Service Control Manager
%for certain functionality, and this functionality cannot be replicated inside of each guest.
%Similarly, some collaborating applications may 
Rather than attempting to secure host services,
Graphene moves these services out of the host
and into collaborating guests.


%From an API perspective, this is similar to Graphene's isolation.  
%This approach requires all applications to run on the same host kernel,
%and 
%However, these approaches are implemented in the same address space 
%as the host kernel and still expose a wide attack surface area.  
%The library OS
%approach moves much of this attack surface area out of the kernel and into the guest.
