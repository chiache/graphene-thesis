\begin{abstract}

\intel{} \sgx{} hardware enables
applications to protect themselves from potentially-malicious
OSes or hypervisors.  
In cloud computing and other systems, many users and applications
could benefit from \sgx{}. % protections.
Unfortunately, current applications will not work out-of-the-box on \sgx{}.
Although previous work has shown that %, in principle,
a library OS can execute unmodified 
applications on \sgx{}, a belief has developed that 
a library OS will be ruinous for performance and TCB size,
making 
application code modification
an implicit prerequisite to adopting \sgx{}.

%A protected context is called an {\em enclave} include encrypting 


%isolates a program in an encrypted context, 
%called an {\em enclave},
%protecting it against malicious system components (e.g., rootkits),
%or hardware-level attacks (e.g., cold-boot attacks).
%\fixmedp{can we say something more precise?}
%\fixme{is this okay?}
% attacks from a compromised operating system (OS),
% hypervisor, or other sytem software.
%%to resist attacks from 
%%provide a tamper-resistant, isolated execution
%%environment for application code---a promising building block for secure systems.
%%has rapidly gained the attention of developer and research communities.
%Despite the fast-growing interest in \sgx{},
%%Despite a surge of research and development interest in \sgx{},
%adopting the hardware in existing applications has been far from efficient,
%due to the porting effort for overcoming the limitations of legacy frameworks.
%%% Current approaches to legacy application suuport on \sgx{}
%%% requires either modifying the binaries (SCONE, Panoply),
%%% or packaging all binaries into an encrypted archive to be provisioned from a remote server (Haven).
%%% \fixmedp{is this right?  What about Haven?}\fixme{Try to be clear}
%%% These requirements make adopting \sgx{} unstraightforward to unprofessional users.
%%% Moreover, COTS (commercial off-the-shelf) applications cannot be directly
%%% isolated by \sgx{} using the current approaches,
%%% due to difficulty of attesting off-the-shelf binaries
%%% and supporting the system ABI.
%%% For close-source applications or applications that requires modification
%%% to be ported to \sgx{},
%%% users need a framework to accelerate the porting process without the intervention of application developers.
%%% \fixme{is this better?}

%which make \sgx{} harder to adopt, if not impossible
%for closed-source applications without developer support.
%\fixmedp{I would rephrase these as problems, rather than benefits, but I don't understand these points yet:
%(2) reducing (re)deployment cost at upgrades.
%(3) isolating an application at emergency, with the measurement preserved for auditing.
%}

%As existing OS support for \sgx{} requires custom-making applications,
%Not only does Graphene simplify the effort to port an application
%to \sgx{}, but it also introduces benefits including:
%The COTS support not only minimizes the porting effort of an application to merely configuration,
%but introduces benefits such as:
%Using either the \sdk{} or a shielding system
%(e.g., \haven{}~\cite{baumann14haven}),
%the procedure of porting an application to \sgx{} mostly requires centralized effort---one trusted user or entity
%has to be responsible of customizing, packaging and signing the application code to run with \sgx{}, as well as maintaining the correspondent validation and provisioning services.
%%porting of existing applications has been far from efficient,
%%due to the bottleneck on developers to
%%customize, package and authenticate the application code
%%to comply with the legacy frameworks. % (either \sdk{} or \sgx{} \libos{}es).
%%Besides relying on applications tailored to \sgx{},
%Many users %who own an \sgx{}-enabled infrastructure will
%can benefit from a framework that
%seamlessly transits COTS (commercial off-the-shelf) applications to \sgx{} without bottlenecking on porting procedure. 
%%and keeps developers from the critical path of the porting process.
%Some immediate reasons for targeting COTS applications on \sgx{} include
%(1) securing close-source applications only available in stores.
%to facilitate porting executables that heavily rely on dynamic linking (e.g., Apache, OpenJDK),
%and (3) to create throwaway containers for applications abruptly escalated to processing sensitive data, and preserve the attestation information for auditing.
%For reusing legacy applications,
%\sysname{} has the least restriction and human intervention
%compared to its alternatives.


%programming this hardware is cumbersome because of many missing OS abstractions.
%This is fundamental: in order to secure applications from an untrusted OS,
%some risky OS interaction models must be constrained, breaking backward-compatibility.
%%developers often find programming for \sgx{} cumbersome,
%%due to the lack of legacy OS abstractions and APIs in the current infrastructure.
%%To address the problem,
%Developers need a platform with sufficient OS compatibility to get existing code running
%in an enclave quickly (but without defeating the purpose of using \sgx{}), so they can focus on
%their effort on further hardening their applications with other \sgx{} features, such as remote attestation or finer-grained partitioning.
%%so most of an application and its supporting libraries can be reused inside an \sgx{} enclave.
%%The cutting-edge Software Guard Extension (\sgx{}) is at the point of being universally launched
%%in upcoming \intel{} processors.
%%However, the software development for applying this technology, at wherever opportunities lie,
%%falls behind the introduction of the hardware.
%%The main obstacle is
%%the concentration and expertise required
%%for developing software specialized for \sgx{}.
%%As an answer to the issue,


%This paper focuses on Linux COTS applications, to benefit the majority of public clouds
%that use Linux as the backbone OS.
%For securing Linux COTS applications with \sgx{}, two critical challenges present in the legacy frameworks.
%First, Linux executables are commonly linked with shared, local libraries,
%while \sgx{} requires the enclave code to be static and authenticated ahead-of-time.
%%First, for an executable composed of dynamically-linked, regular binaries,
%%the framework must launch the execution with \sgx{} and allow authentication by trusted provisioners.
%Second, the legacy frameworks are not sufficiently compatible with the features
%that many Linux applications depend on.
%A key feature that has been missing is the support of multi-process applications---i.e., ones that \fork{} the isolated execution into another enclave---while retaining the integrity guarantee.

This paper demonstrates that 
these concerns are exaggerated, and 
that a fully-featured
library OS can rapidly deploy unmodified applications on \sgx{}
with overheads
comparable to applications modified to use ``shim'' layers.
%This paper presents \sysname{}, a framework for running unmodified, Linux COTS applications in enclaves.
%This paper describes \sysname{}, a framework for transiting Linux COTS applications to \sgx{},
%and decentralizing the effort of packaging application code, authenticating benign applications, and building the trust in provisioning services.
We present a port of Graphene to \sgx{},
as well as a number of improvements to make the security benefits
of \sgx{} more usable, such as integrity support for dynamically-loaded libraries,
and secure multi-process support.
\sysname{} supports a wide range of unmodified applications, including
Apache, GCC, and the R interpreter.
%, and has been used by several other research groups to 
%build concurrent submissions exploring \sgx{}.
%of \sgx{} more usable, such as integrity support for dynamically loaded binaries,
%and secure multi-process support.
%remote attestation, secure inter-process communication, and enclave-level fork.
%Users need only configure and sign the enclave parameters %and sign the configuration.
%to supports a wide range of unmodified applications, including
%both server and desktop workloads.
%Apache, MySQL, the OpenJDK Java Virtual Machine, the Python and R interpreters,
%and Memcached, and has been used by several other research groups to 
%build concurrent submissions exploring \sgx{}.
%\sysname{} is secure, has acceptable performance costs,
%and is can run a range of substantial, unmodified Linux applications in \sgx.
%without any development effort for code modification or recompilation.
%reducing the development effort
%of running Linux applications in an \sgx{} enclave.
%we present \sysname{}, an open-source \libos{}
%for accelerating the software advancement of putting \sgx{} to effective use.
%a fully-ported 
%which is capable of running Linux executables
%inside an \sgx{} enclave.
%\sysname{} includes a 
%\libos{} that encapsulates 
%most OS APIs, and maps these onto a narrower enclave interface.
%\sysname{} designs a narrow enclave interfaces
%with defenses 
%we implement the defense 
%against malicious inputs potentially returned by the untrusted OS
%(i.e., {\em Iago attacks}).
The performance overheads of \sysname{} range from matching a Linux
process to less than $2\times$ in most single-process cases;
these overheads are largely attributable to current \sgx{} hardware 
or missed opportunities to optimize Graphene internals, 
and are not necessarily fundamental to leaving the application unmodified.
%Using the \libos{} reduces the complexity of protecting a broad, state-leaking system interface
%like Linux system calls.
%\sysname{} includes support for multi-process applications,
%where different processes run in separate enclaves;
%\sysname{} also protects multi-process abstractions from the untrusted OS.
% shields the security-sensitive OS states, including ones shared by multi-process applications which span across enclaves.
%For compatibility, \sysname{} implements a major portion of the core Linux system calls,
%supporting xx.x\% \fixme{update this number} of official applications installed on each Ubuntu machine.
\sysname{} is open-source and has been used concurrently by other groups  
for \sgx{} research.


%extends the \graphene{} \libos{}---\graphene{} sustains a narrow interface
%to the untrusted OS, retaining the isolation benefits of an \sgx{} \libos{}.
%Despite this narrow interface, \graphene{} supports and secures many challenging
%Linux features across \sgx{} enclaves, including fork(), signals, namespaces, and other IPC.
%%Below the \libos{}, a narrow interface is exported
%%for integration with rest of the application,
%%and fully configurable for self-validating the inputted resources.
%%The goal of \sysname{} is a framework
%%for loading natively-developed Linux executables into isolated environment,
%%regardless of the deployment and integration models of applications.
%%First, we remove the prerequisite of a trusted server from the dynamic loading process,
%%and use truly application-dependent measurements to establish trustworthy execution.
%%This approach facilitates many options, %deployment and integration options, 
%%Using the multi-process support of \graphene{},
%%our platform contributes several integration options over prior \sgx{} \libos{}es,
%%including process-to-enclave integration, and multi-enclave applications.
%For each application, we make the signatures of \sgx{} enclaves reflect the loaded binaries,
%based on analyzing application dependencies across processes---allowing third-party provisioning servers to attest the execution.
%%The platform has a framework for validating input from the untrusted OS,
%%making several contributions over prior \sgx{} \libos{}es,
%%including asynchronous deployment, enclave-process integration,
%%and multi-enclave environment.
%%\fixmedp{Can you crispen the contributions?}
%%This paper also contributes thorough measurements of applications running with the platform,
%%as well as detailed analysis of the performance overheads of \sgx{} and 
%%tuning strategies to mitigate the pitfalls.
%The \sysname{} framework is contributed back to the open-source \graphene{} project to help 
%accelerate \sgx{} implementation and research.
%%have been 


%Second, \sysname{} exports several \sgx{} features,
%such as attestation and sealing keys,
%as library OS abstractions ready for employment.
%In addition, by using off-the-shelf \intel{} processors for development and evaluation,
%to minimize the gap from the real-world adoption.
%During the development,
%we identify several pivotal factors to the \sgx{} performance,
%which can be fine-tuned in either the \libos{} or applications.
%and provide insights for developers to tune their applications.
%For instance, memory usage impacts the start-up time and paging overhead,
%so we adjust the design toward small memory space and working set.
%In summary, we expect \sysname{} to become the foundation of software development
%for pervasive application protection derived from the rising \sgx{} technology.
%This work is anonymously released as part of the \graphene{} project for early adoption.




%\intel{} \sgx{} enclaves isolate applications
%from untrusted system stacks, as well as attacks to hardware or firmware.
%Previous works have shown how to
%use \sgx{} to isolate a complete, single-process,
%legacy application,
%or a small piece of it---in a single enclave.
%The open problem this work addresses is how to span an application,
%as multi-process and with multiple security principles,
%across several enclaves.
%Due to the lack of tools for developers to partition
%a large application,
%the default approach to using enclaves leads to a large trusted computing base (TCB),
%thereby increasing the risk of exploitable vulnerabilities.
%A multi-enclave model splits the TCB, but introduces
%challenges such as
%extending integrity guarantees to cloned enclaves;
%%minimizing per-enclave provisioning costs;
%and enforcing security policies on shared in-enclave abstractions.

%% \sgx{} hardware can isolate security-critical components of
%% applications in a protected memory space called an {\em enclave}.
%% In an enclave, the CPU guarantees confidentiality and integrity,
%% and offers remote attestation.


%Hardware-isolated execution brought by \sgx{}
%allows applications to guard their most critical security components
%in a sanctuarized memory space ({\it enclave}),
%with confidentiality and integrity guaranteed and attested by processors.
%% Recent works like \haven{}~\cite{baumann14haven} use \libos{}es to
%% facilitate migration of native \win{} applications to \sgx{},
%% providing ease of use and
%% a sensible security model to build up the trust for applications
%% running in the cloud.
%% However, users of \haven{} must tolerate the limitations that
%% only single-process \win{} applications are supported,
%% and unpacking of application binaries must alway be provisioned
%% from remote servers.
%% Moreover, lack of methods to partition the applications
%% causes large trusted computing base in enclaves,
%% leading to risks of exposing security vulnerabilities to attackers.

%This paper presents {\bf \sysname{}}, %, a shielding system derived from a
%a library OS that creates and coordinates the multi-enclave environment,
%to isolate legacy Linux multi-process applications.
%%supports multi-enclave execution.
%%to migrate Linux applications to \sgx{} enclaves.
%%For multi-process applications, % that are natively partitioned into processes,
%%\sysname{} can place each process in a separate enclave.
%\sysname{} contributes a
%decentralized trust model, where remote hosts
%can attest and provision each binary of an application individually,
%preventing compromised enclaves from
%jeopardizing the security of the whole application.
%\sysname{} also extends hardware integrity measurements to
%dynamically-loaded binaries---a prerequisite for 
%inter-process attestation.
%%We use an off-line model to guarantee the integrity of 
%%by involving software-verified measurements in hardware-generated attestations.
%%We present a
%We evaluate \sysname{} on
%\intel{} \skylake{} processors,
%and contribute baseline measurements of \sgx{} hardware costs.
%Our results show reasonable application performance overheads,
%such as xxx\% throughput cost for the Apache web server and
%xxx\% latency cost on OpenJDK Java virtual machine.

\end{abstract}
