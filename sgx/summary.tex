\section{Summary}

This chapter describes \graphenesgx{},
a port of the \graphene{} \libos{} on the security-centered, Intel SGX platforms.
SGX facilitates the protection of applications
against the whole untrusted system stack ranging from off-chip hardware to the OS,
but imposes several restrictions to running unmodified applications.
\graphenesgx{} removes the restrictions
by servicing Linux \linuxapis{} inside an in-enclave \libos{} instance and defining an enclave interface with explicit checks for responses from the untrusted OS.
Compared with other thin, shielding layers~\cite{osdi16scone,shinde17panoply}, \graphenesgx{} shields a large range of Linux \linuxapis{},
%for the typical applications targeted by the cloud,
without extending the host ABI that an enclave needs to check.



\graphenesgx{} shows the feasibility
of protecting an unmodified application and \libos{} on an untrusted OS, using three shielding techniques.
First, \graphenesgx{}
shields the dynamic loading process
by generating a unique cryptographic measurement that verifies all the binaries of one application.
Second, \graphenesgx{} further defines a simple
enclave interface
below \thehostabi{} to shield an enclave from a series of subtle semantic attacks
launched by the untrusted OS, known as Iago attacks~\cite{checkoway13iago}.
For most of the enclave calls that reach out to the untrusted OS,
\graphenesgx{} either restricts the possible responses
from the OS to one predictable answer,
or ensures that deliberate failures of the OS are benign to the application.
Finally, \graphenesgx{} spans an multi-process application
into multiple mutually-trusting, cooperative enclaves.
To shield the inter-enclave collaboration from the untrusted OS,
\graphenesgx{} establishes mutual trust between enclaves
using local attestation of Intel CPUs,
and negotiates TLS-secured RPC streams across enclaves.


