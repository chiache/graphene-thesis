\section{Introduction}

The complexity of modern operating systems has become
a major source of security vulnerabilities,
pressuring developers to seek solutions to exclude the host OSes
from their trust models.
The untrustworthiness of OSes get severe when applications are run in the cloud,
because the applications may cohabit with compromised tenants,
or be hosted by malicious providers.

To address the concern,
isolated execution hardwares such as \sgx{} are introduced
for the sake of confidentiality and integrity in
security-sensitive applications,
upon untrusted hosts.
\sgx{} supports new instructions to create a
critical space called {\it enclave} in the application's memory.
\sgx{} guarantees that only signed code loaded in the enclave can access the critical data,
and is able to attest the integrity for remote clients.

%Protecting applications without the requirement of trusting the hosts
%has become valuable in modern computer systems for many reasons.
%The complexity of the existing monolithic operating systems has become a major source of
%vulnerabilities to exploit.
%Once the host operating systems are compromised,
%any security mechanisms enforced in the applications can be easily circumvented.
%If the applications run in the cloud,
%the hosts have to be controlled by the cloud providers,
%which is capable of compromising both host operating systems and hardwares.
%The attacks from the cloud is possible even if
%the operating systems' vulnerabilities are never exploited.

Unfortunately, leveraging \sgx{} for security
often requires rewriting the applications to a large extent.
Because the OSes cannot be trusted,
enclaves are not allowed to directly access any system interface
for any OS features
such as files, networks or scheduling.
Only a limited subset of system library API can be exported in enclaves;
for example, the software development kit for \sgx{}
only supports spin-locking as scheduling tool,
because other primitives require cooperation from the OS.

To alleviate the pain of migrating applications to \sgx{},
recent works like \haven{}~\citep{baumann14haven} provides a mechanism
of loading native applications into enclaves,
alongside a library OS (\libos{) }to facilitate rich OS features.
These systems are called ``shielding systems''~\citep{xu15controlledchannel},
which means these systems can secure execution environments
without trusting the host.

\haven{} effectively migrates native Windows applications to enclaves,
providing a shortcut for developers
to easily boost the security of their products.
However, there are several limitations in \haven{}, including
excessive trusted computing base (TCB),
a restrictive model of adapting applications,
and missing support for Linux or multi-process applications.

We present \graphenesgx{}, a \libos{}-based shielding system
targeting Linux multi-process applications,
as improvement to the \haven{} model of migrating native application to \sgx{}.
\graphenesgx{} is derived of \graphene{} \libos{}~\citep{tsai14graphene},
which supports 139 out of 300 Linux system calls.
\graphene{} runs numbers of popular applications
such as Apache web server, OpenJDK virtual machine, or shell scripts,
all of which relies on multi-process abstractions
such as {\tt fork}, {\tt execve}, signals or System V IPC.

By supporting Linux multi-process applications,
\graphenesgx{} simply applies to a broader range of cloud-based applications
than \haven{}.
Survey shows that Linux yields a market share of up to 75 percent in
enterprise cloud providers~\citep{linuxcloud2014}.
This data suggests majority of the existing cloud users can
more easily adopt \graphenesgx{} than \haven{}.

\graphenesgx{} uses a different model from \haven{}
to bootstrap the trust of loaded applications.
\haven{} builds up the trust by packing all binaries
in an encrypted virtual disk,
with secret key provisioned consensually from a remote server
trusted by the clients.
Contrastingly, \graphenesgx{} relies on a model directly derived from
the trust model of \sgx{},
by including applications' checksums
into the enclave measurement to be attested by hardware.
\graphenesgx{} can migration applications for enclaves off-line,
and share binaries (e.g., shared libraries) to save bandwidths for deployment.

Security-wise speaking,
\graphenesgx{} provides a way to reduce the TCB of enclaves,
with a smaller code footprint
and applications natively partitioned as processes.
\haven{} in general yields a large TCB
including a \libos{} as large as 209MB,
and application binaries around 10s \textasciitilde 100s of MB.
The image of \graphenesgx{} is merely 10MB,
and Linux applications generally follow the principle of dividing
code into smaller, more testable binaries.
\graphenesgx{} isolates multiple binaries (or processes) of an application
in separate enclaves, and authenticates inter-process coordination
by binary-based policies.
Even if an enclave is compromised, other enclaves of the same application
can still stay secure as long as
critical information never flows to the victim.

The contributions of this paper are listed as follows:
\begin{compactitem}
\item \graphenesgx{}, a Linux-compatible \libos{}-based shielding system,
which supports execution of native, unmodified Linux applications
in \sgx{} enclaves.
\item A secure implementation of multi-process support
for shielded applications,
using standalone enclaves to solate processes.
\item A decentralized trust model for multi-process applications to shield,
where each binaries individually attested and provisioned.
\end{compactitem}

\graphenesgx{} provides cloud users
ease of shielding Linux applications
and flexibility of implementing any desired security model
suitable for each binary.

