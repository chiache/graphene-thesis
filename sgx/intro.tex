Intel \sgx{}~\cite{intelsgx} shows
a compelling example
where an unmodified application fails to run inside a beneficial, new host environment.
\sgx{} provides an opportunity of running trusted applications with a strong threat model where
OSes, hypervisors, and peripheral devices can be malicious.
\sgx{} presents challenges to running an unmodified application, including shielding the application
from malicious host system calls.
\graphene{} significantly reduces the complexity
of resolving both compatibility and security issues for running unmodified applications on \sgx{}.


This chapter summarizes the development of an \sgx{} framework using \graphene{} to protect unmodified Linux applications from the untrusted host OS.
This chapter starts with the overview of \sgx{}-specific challenges
for porting an application,
followed by a comparison of approaches to shielding an application from an untrusted host~\cite{osdi16scone,shinde17panoply,baumann14haven}.
This chapter then describes the design of \graphenesgx{},
an \sgx{} port of \graphene{};
\graphenesgx{} fits dynamically-linked, unmodified applications into the paradigm of \sgx{}-ready applications,
and customizes an interface to an untrusted OS
to simplify security checks against malicious host system calls.
%This chapter shows
%that by checking a narrowed host interface,
%\graphenesgx{} shields Linux system calls inside a \thelibos{} instance.
