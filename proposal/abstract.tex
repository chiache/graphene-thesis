Unknown security vulnerabilities and performance bottlenecks exist
in modern applications and operating systems,
%The pitfalls hidden in the legacy code affect users
and even developers can hardly prevent them.
A known strategy to restrain
the outcome of hidden system pitfalls %potential vulnerabilities or bottlenecks
is to partition a system that is complex and hard to predict,
into small pieces with reasonably simple logics.
%Partitioning a complex system into
%isolated, reasonably simple pieces
%is a method
%to restrict the aftermath of potential vulnerabilities or
%performance bottlenecks.
In a partitioned system, developers
are more likely to reason about
the soundness of the design and implementation.
%in the partitions
%that are more valuable to users.
%instead of demanding arbitrary trust from the users.
For instance, library OSes
~\citep{porter11drawbridge, baumann13bascule, unikernels}
split the functionality of kernels into in-process OS instances,
or so-called \term{\picoprocs{}},
to isolate mutually distrusting applications.
The security isolation enforced by library OSes
is based on the fact that
every \picoproc{} have only a narrow attack surface to the kernel,
while execution in the library OSes is
bounded by the address space.
In term of performance benefits,
%The performance benefits
%from partitioning the host resource and congesting logics
%are presented in the 90's,
%solutions like
Exokernel~\citep{engler95exokernel}
and Nemesis~\citep{leslie96nemesis}
identify and evict the OS logics causing congestion in the kernel,
such as multiplexing resources or allocating buffers,
into applications or library OSes.
For each applications,
the unneeded or redundant system logics can be dismissed from the critical path.
%can be removed from the critical paths.



%Furthermore, more recent works such as HiStar~\citep{zeldovich+histar}
%or isolated execution
%~\citep{flicker, trustzone, intelsgx, criswell2014virtualghost}
%aim for minimizing the partition that needs to be trusted
%for highly sensitive applications.


%Modern system stacks and applications have grown in term of both complexity and requirement.
%The complexity of system stacks --- from operating systems to the hardware --- have made them prone to bugs and vulnerabilities,
%leading to risk of exploitation.
%In a multi-tenant environment like the cloud, benign and malicious users share the same trusted computing base in the system stack,
%in which the malicious users can actively attack the system.
%Moreover, even if the operating systems are not attacked by the users,
%the hardware can be attacked or counterfeited physically, if the providers or maintainers of the hosts are malicious.
%Overall, an application can face risk of exploitation,
%by trusting the system stacks to maintain their integrity, whereas the application itself can also be vulnerable due to its complexity.

%Among these partitioned solutions,
%the majority require
%development effort to rewrite the legacy application code
%in order to execute in the partitioned systems.
%A primary reason %of the price
%is that these systems
%have redefined or reduced the specifications,
%to gain a clean, narrow interface betweens the partitions.
%Many existing partitioned systems only demonstrate POSIX compatibility
%~\citep{engler95exokernel, leslie96nemesis, zeldovich+histar},
%partially due to research prototyping.
%A complex application
%will have to be reprogrammed to be adopted into these systems.
%Unikernel~\citep{unikernels} requires applications to be implemented
%in a functional language for the sake of compilation-time
%partitioning and optimization.
%Only few works~\citep{porter11drawbridge, baumann13bascule, tsai14graphene}
%are designed for isolating or optimizing
%legacy applications adopted from mainstream operating systems such as Windows or Linux.


%To reduce the risk of exploitation in the system stacks,
%system researchers have engaged in effort to build secure operating systems, such as Exokernel~\citep{engler95exokernel} or HiStar~\citep{zeldovich+histar}.
%These secure operating systems attempt to minimize the trusted computing base of the privileged components run in the kernel (as the concept of micro-kernels) and scooping the operating systems into the userspace (as the concept of library OSes).
%However, existing micro-kernels or library OSes are mostly designed as prototypes, which are incapable of supporting sophisticated applications in the modern days.
%The center of the problem is the {\em compatibility} requirement of applications:
%many applications depend on the modern system APIs and features of a monolithic operating system (e.g., Linux), and cannot be easily redesigned for other platforms.
%For existing library OSs,
%one of the system features that are especially missing or incomplete for Linux applications is the multi-process support,
%which are often used in server-type workloads or shell scripts.
%On the other hand, virtualization can provide both compatibility and security isolation for native applications built for monolithic operating systems, but often requires too much resources such as memory, to run each guests in multi-tenant environments.

A key challenge in a partitioned or optimized system is
to maximally reuse legacy code, %from prior systems and applications,
in order to harvest the development efforts made in previous systems.
Essentially, millions of applications built and deployed
on mainstream OSes,
such as Windows and Linux,
will require porting efforts if users need to adapt them to another platform. 
%maximizing the legacy support in a system
%will lower the barricade for users to adopt it.
In recent library OSes~\citep{porter11drawbridge, baumann13bascule, tsai14graphene, baumann14haven},
legacy application support is achieved
%are designed for isolating or optimizing
%legacy applications adopted from mainstream operating systems such as Windows or Linux.
by emulating
%The key challenge for reusing legacy application code
%is to emulate 
the specifications or personalities of the applications' origins,
on top of the system's building blocks
that enforces security isolation, platform independence, migration, or other features.
%--- even when the requirements oppose the system assumptions.
%Essentially, a system supporting legacy applications
%must reproduce the abstractions and assumptions that applications depend on,
%besides translating
%language slangs and system interfaces.
%Based on the principle,
A personality that is especially hard to emulate in \picoprocs{}
is multi-process abstractions,
which will require multiple \libos{} instances to work collaboratively. 
We design a library OS called \term{\graphene{}}~\citep{tsai14graphene},
which support various Linux multi-process abstractions used in legacy applications,
yet provided comparable security isolation
as single-process \picoprocs{}.
%For instance,
%\picoprocs{} isolate the OS states for each applications
%within the process boundaries~\citep{porter11drawbridge, baumann13bascule, baumann14haven}.
%As the Linux personality, a multi-process application
%should be split into several \picoprocs{},
%yet the \picoprocs{} must cooperate to provide a unified system view.
%For Linux applications that are designed as multi-process,
%such as servers or shell scripts,
%the OS states must be maintained across \picoprocs{}
%so the OS views can be arbitrarily unified or partitioned
%Generally speaking,
%Take Linux applications for example,
%it is a well-known design principle to break down the execution
%into smaller binaries that run in multiple processes,
%and the decision shall be honored
%when partitioning the underlying operating systems.
The idiosyncratic multi-process abstractions of Linux, such as forking, signals, \sysvipc{}, etc,
are coordinated across \picoprocs{},
over simple, pipe-like RPC streams on the host.
Therefore, \graphene{} can simplified the complexity of isolating shareable abstractions
by simply forbidding RPC streams across distrusting applications.
\graphene{} can expand the support for Linux multi-process applications
to various platforms, including \term{Intel SGX enclaves}.
In Intel SGX enclaves, \graphene{} can isolate Linux multi-process applications
in a sealed environment,
fully immune to 
attacks from host OSes and hardware peripherals.

%As the purpose is to keep the design
%clean and reasonable,
%a vital principle is to dissociate the logics of preserving personalities,
%from security mechanisms and performance optimizations.
%Take \graphene{} for instance,
%the implementation of Linux's idiosyncratic IPC abstractions
%are contained in the library OS,
%using simple RPC streams
%that can be straightforwardly isolated on the host.
%Motivated from partitioned systems,
%an opportunity of significantly reducing the lookup latency
%in file system directory cache
%can be demonstrated in both Linux~\citep{tsai15dcache} and \graphene{}.
%The optimization is founded on
%decoupling the hit lookup latency and permission checks,
%as well as miscellaneous features in the Linux file system hierarchy.


The principles for developing legacy support are vastly distinct
for different specifications,
and entangled with the implementation of
security mechanisms and performance optimizations.
Many systems provide legacy support
in a complex, case-by-case way,
potentially causing more security glitches and performance suboptimality.
An insight
for solving this problem %that affects both partitioned and non-partitioned system
is to avoid
emulating specifications and checking security policies
on the critical path of common cases.
We demonstrate a case of optimizing a performance-centric, heavily engineered components,
the file system directory cache in Linux,
while preserving the legacy file-system features and security models
~\citep{tsai15dcache}.
The original Linux kernel interleaves path component searching
in the directory cache
with permission checks against security modules,
and retrieving file attributions.
The lookup logic can be optimized
by sparing the hit latency from permission and attribute checks,
and searching the directory cache in constant time.
The optimized Linux
can improve the latency of git version control system on Linux by up to 25\%.
On the other hand,
the process of building up legacy application support
can be more principled,
using a fractal measurement to evaluate the completeness of emulation,
instead of the traditional binary measurement.
%
% a representative measurement of the completeness of emulating specifications.
%To address the concern,
We conduct a study on Linux system API usage
~\citep{tsai16apistudy}
to reveal the impact of API implementation on users
and suggest an optimal path for maximizing the completeness in a system prototype.
%The study reveals that all system APIs are not equally important for emulating,
%and by prioritizing API emulation
%developers can plan a optimal path to support a wide range of applications.


%We present \term{\graphene{} library OS}, to enforce security isolation for native Linux applications,
%with reasonable compatibility, performance and resource cost.
%\graphene{} library OS supports and secures many multi-process abstractions that are native to Linux, such as forking, signaling, and system V IPC (message queues and semaphores), etc.
%For multi-process applications, \graphene{} run each process in separate pico-processes loaded with a library OS,
%yet each pico-processes can coordinate to provide a unified view of the operating system.
%Within each pico-process, \graphene{} implements a large portion of Linux system calls, using a much narrower, platform-independent host ABI.
%To enforce security isolation on multi-process abstractions,
%\graphene{} coordinate all the namespaces and operations
%over a simple pipe-like RPC streams, with can be easily sandboxed from the hosts.
%By restricting coordination over RPC streams,
%\graphene{} effectively enforce security isolation on various of multi-process abstractions whose mechanisms can be complicated and idiosyncratic.

As in-progress and proposed works,
we explore possibility of reusing legacy code when partitioning an application,
to purse fine-grained isolation.
The improvement of granularity is critical for protecting highly sensitive components,
to reduce the execution paths
potentially exploitable.
%If the subject of partitioning is a legacy system or application
%in which code reuse is desired,
%the challenge falls on
%clean compartmentalization of code
%and seamless access of objects across partitions.
%Code compartmentalization
%~\citep{addistant, jorchestra, jif-split, swift}
%is a technique adopted from
%programming languages and code analysis.
%However, 
From the OS perspective,
there are system design efforts that
can fundamentally improve the result of legacy code reuse,
based on other techniques from programming languages and code analysis.
These efforts include
lifting the limitations imposed on more sophisticated languages
and creating clean, smooth transition of system states
which the partitioning is based on.
We demonstrates an example
in which applications can be painlessly partitioned
into Intel SGX enclaves~\citep{intelsgx},
by leveraging advantages from the \java{} language,
on a system designed to transparently load and interface legacy \java{} classes in enclaves.


%Regardlessly, creating an adaptable design
%in any partitioned system,
%in order to resurrect as much legacy system or application code as possible,
%is a nontrivial effort.
%The process of building a compatible system will benefit from a more principled procedure,
%in which the completeness of emulation can be measured.
%Given an API specification,
%the compliance of a partial implementation can be evaluated
%based on a study of API usage in applications and application popularity
%~\citep{tsai16apistudy}.
%The measurement provides a methodology
%for system developers to reason about the
%preservation of specifications and personalities.






%\graphene{} \libos{} can also secure native applications from potentially compromised system stacks, if it is combined with isolated execution.
%Isolated execution, either enforced by hardware (e.g., \intel{} \sgx{}~\citep{intelsgx}) or hypervisors (e.g., Flicker~\citep{flicker}, VirtualGhost~\citep{criswell2014virtualghost}, etc), creates a privileged context for applications,
%and prevents the system stacks above the hardware or the hypervisors
%to compromise the confidentiality and integrity of the applications.  
%We focus on combining \graphene{} with \sgx{}, as {\em \graphene{}-SGX},
%to provide stronger isolated execution than hypervisor-based solutions,
%for defending against both OS-level and hardware-level attacks.
%\gsgx{} improves the usability of \sgx{}, by minimizing the effort of porting native applications to the \sgx{} platform,
%but still maintains compatible security guarantees as the applications that built for the platform.
%Compare to other library OSes that are combined with \sgx{} (e.g., \haven{}~\citep{baumann14haven}),
%\gsgx{} provides more fine-grained protection for applications binaries and processes,
%breaking down the isolated execution of
%an application that are already partitioned into processes,
%to yield smaller trusted computing base.

%Despite the effort of using \libos{} and isolated execution
%to reduce the risk of exploitation in the system stacks,
%the application itself can still be vulnerable
%--- due to the its own complexity.
%We observe that with the help of managed languages,
%applications can be partitioned by automation,
%and the trusted components needed to be loaded into the isolated execution
%can be effectively reduced.
%We present {\em \civet{}}, a framework that combines \sgx{} and Language protection in \java{},
%to benefit from both types of protection.
%Using Civet,
%developers of Java applications can enforce the partitioning and security guarantees they desired.
%Moreover, the components loaded into \sgx{} can be harden by language-based protection,
%such as filtering the output
%that are tainted by the information flow.



%In addition, from the experience of building more practical and secure operating systems,
%we observe that the compatibility requirement on modern operating systems have made significant impact
%on the development of the operating systems,
%even effecting performance and security of the systems.
%For example, we discover that the performance of looking up a longer, cached path in the Linux file system directory cache
%is suboptimal due to the requirement of checking directory permissions during the lookup.
%We can further optimize the latency of lookup operations
%by decoupling the lookup with permission checking,
%without breaking the compatibility of any file system operations.
%Another observation we made is that
%the compatibility requirement of operating systems can be balanced with other system properties,
%using a more practical measurement.
%We propose measurements that correlate both the usage of APIs in applications
%and the popularity of applications among user,
%to replace the traditionally used bug-for-bug compatibility.
%With the formal observation, we optimize the file system directory cache in both Linux and \graphene{};
%with the latter, we measure and improve the partial completeness of the Linux application support in \graphene{}. 







%As future works, we will focus on improving \graphene{}, \gsgx{} and \civet{},
%to build more generalized models of security isolation, and minimize the weakness we have observed in these solutions.