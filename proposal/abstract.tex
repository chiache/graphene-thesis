Modern system stacks and applications have grown in term of both complexity and requirement.
The complexity of system stacks --- from operating systems to the hardware --- have made them prone to bugs and vulnerabilities,
leading to risk of exploitation.
In a multi-tenant system like cloud, benign and malicious users share the same trusted computing base in the system stack,
in which the malicious users can actively attack the system.
Moreover, system stacks can be attacked from the hardware, if the providers of the hosts are malicious.
Overall, applications are facing risks of being compromised increasingly,
by trusting the system stacks to maintain their integrity.

System researchers have been designing secure operating systems, such as Exokernel~\citep{engler95exokernel} or HiStar~\citep{zeldovich+histar}, to minimize the risk of exploitation.
However, these secure operating systems are designed as prototypes that are unable to support sophisticated applications.
The center of the problem is the compatibility requirement:
many applications depend on the modern system APIs and features of a monolithic operating system (e.g., Linux), and cannot be easily redesigned.
For previous library operating systems,
one of the system features that are especially missing or incomplete for Linux applications is the multi-process support
--- which are often used in server-type workloads or shell scripts.
On the other hand, virtualization can isolate native applications from each other in a multi-tenant system,
but often requires too much resources such as memory, to run each guests.

To enforce security isolation for native Linux applications in multi-tenant systems, we present {\em Graphene Library OS},
a library OS that supports unmodified Linux multi-process applications.
