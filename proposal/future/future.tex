\chapter{Future Works and Proposals}
\label{chap:future}

In our previous works, we explore several models of
reusing or resurrecting the development efforts in legacy code
that occur in diverse scenarios.
The cases studied include
partitioned systems such as library OSes,
conventional components in legacy OSes such as file system directory cache in Linux,
and partitioning a legacy \java{} application for isolated execution.
For future works, and the fulfillment of this thesis,
we will explore improving out solutions in more generalized models,
and reducing the weaknesses we observe in these solutions.
%that occur in diverse scenarios.
%For instance, \graphene{} and \gsgx{} both utilize a highly compatible library OS,
%but isolate applications with drastically different assumptions.
%In addition, \gsgx{} and \civet{} relies on two distinct strategies to maximize the usability
%--- \gsgx{} secures an application as it is, whereas \civet{} benefits from language techniques.
%As future works, we will focus on improving \graphene{}, \graphenesgx{} and \civet{},
%to build more generalized models, and minimize the weakness we have observed in these solutions.



\section{Generalizing Platform Independence}

In library OSes, legacy applications are reused
by the \picoprocs{} implementing abstractions and personalities of the development platforms.
Library OSes essentially translate the abstractions or APIs
to the interfaces and conventions, often reduced or redefined, on the host platforms.
In particular, a few library OSes~\citep{porter11drawbridge, baumann13bascule, baumann14haven, tsai14graphene}
relies on a host ABI defined to be \term{platform independent}
--- the semantics and assumptions of the ABI are supposed to be
reasonably feasible for implementation
on most host platforms.
For instance, most platforms provide the abstractions and features
of a hierarchical file system,
with operations for opening,
accessing and querying the files,
using similar interfaces.


Platform independence in library OSes are demonstrated by the complexity of legacy applications being supported,
on top of various host platforms.
The rationale behind the argument of such a proof-of-the-concept is that
a complex application will exercise a wide range of system features and conventions,
that mostly overlap with the common footprints of other applications.
For instance, Drawbridge~\cite{porter11drawbridge} is tested with
popular Windows applications such as Internet Explorer and Microsoft Powerpoint.
To demonstrate the platform independence of the host ABI,
\graphene{} is ported to many platform such as FreeBSD, Windows, OSX and \intel{} \sgx{} enclaves,
and most applications can be supported on all these platforms
transparently.

The narrowness of host interfaces used by library OSes can cause limitations to the ability of implementing OS personalities.
The primary reason is that the definition of host interfaces
is mostly biased by the specifications that developers choose to implement,
and the applications preferred for testing.
For instance, the \drawbridge{} host ABI, which is adopted by many other library OSes,
is defined for a library OS of Windows personality.
In \graphene{}, we implement Linux personality over the \drawbridge{} host ABI,
and observe that many required host abstractions are missing,
due to the difference between Windows and Host applications.
For instance, Linux applications heavily rely on exception handling, which is missing in \drawbridge{} but added in both Bascule and \graphene{}.

In this thesis we focus on implement Linux personality:
\graphene{} supports \syscalls{} Linux system calls,
based on what developers considered to be the most commonly used.
However, according to our study of Linux system API usage in a large, representative application sample,
for any Linux installations,
there are only 0.4\% of the installed applications whose system API footprint
are completely supported by \graphene{}.
Moreover, the study suggests that completeness of \graphene{} can be largely improved by add two scheduling APIs.


\begin{table}[t]
\footnotesize
\centering
\begin{tabular}{|p{1.2in}|p{2.4in}|p{2.4in}|}
\hline
{\bf Host ABI Function} & {\bf Functions in the \libos{}} & {\bf Description} \\
\hline
Scheduler policies & {\tt sched\_setscheduler}, {\tt sched\_setaffinity} & Specifying scheduler policies such as priority level or CPU affinity.\\
\hline
Huge pages & {\tt MAP\_HUGETLB} flags for {\tt mmap} & Specifying virtual memory to be backed by huge pages. \\
\hline
Memory sharing & {\tt shmget}, {\tt shmat}, {\tt shmdt} & Sharing memory using RPC. \\
\hline
Raw sockets & {\tt socket} with {\tt SOCK\_RAW}  & Sending RAW packets (e.g., DHCP clients) \\
\hline
NUMA & {\tt set\_mempolicy}, {\tt migrate\_page} & Accessing host NUMA features. \\
\hline
\end{tabular}
\caption[List of host ABI functions to be added in \graphene{} as future works]
{List of host ABI functions that are missing in \graphene{} host ABI.
We anticipate providing these functions in the future,
by either providing a platform-independent interface,
or allowing configuration in manifests.}
\label{tab:future:abi}
\end{table}


The Linux system calls that are not implemented in \graphene{},
besides the one we anticipate to be supported in the near future,
are primarily for two purposes:
specifying policies for multiplexing host resources;
and accessing host hardware features.
An example for the former type of unsupported system calls is scheduling APIs, such as {\tt sched\_setscheduler}.
The latter type of unsupported system calls includes
a wide range of {\tt ioctl} opcodes,
and NUMA-related system calls like {\tt migrate\_pages}.
These abstractions cannot be implemented in the library OS
without extension of the host ABI.
The advance the completeness of \graphene{},
we will design host ABI to by
either providing an platform-independent interface
for \picoprocs{},
or allowing users to specify the resource multiplexing policy in the manifests.
Table~\ref{tab:future:abi} lists the top-priority host abstractions
that should be provided to \picoprocs{}.


\begin{table}[t]
\footnotesize
\centering
\begin{tabular}{|p{1.2in}|>{\raggedright\arraybackslash}p{0.8in}|>{\raggedright\arraybackslash}p{1.6in}|>{\raggedright\arraybackslash}p{2.2in}|}
\hline
{\bf Feature} & {\bf Platform} & {\bf Limitations} & {\bf Strategy} \\
\hline
Memory mappings & Windows & No fine-grained deallocation or protection & Redesign ABI functions for platform independence \\
\hline
Thread-local storage & Windows, OSX & FS or GS register is not available & Overwrite TLS allocation in libc, or binary translation \\
\hline
Position-dependent binaries & Windows, \intel{} \sgx{} & Memory addresses are occupied or restricted & Reserve virtual memory before \picoprocs{} start \\
\hline
Special instructions & \intel{} \sgx{} & {\tt CPUID} and {\tt RDTSC} are forbidden & Exception handling or binary translation \\
\hline
\end{tabular}
\caption[List of platform limitations affecting host ABI porting]
{List of limitations on host features that affects porting the host ABI to the target platforms,
and the coping strategies.}
\label{tab:future:abi-limit}
\end{table}

Besides the feasibility of implementing OS personalities,
we also demonstrate the neutrality of the host ABI,
to prove that \graphene{} is portable to diverse host platforms.
However, the assumptions made in the definition of the host ABI
can be violated in a host platform with limited execution environment.
For instance, although not defined explicitly in the host ABI of \graphene{},
the \picoprocs{} requires the host platforms
to reserve address space for loading legacy binaries that are not compiled as position independent.
The requirement can be a challenge on a platform
where part of the \picoprocs{}' virtual memory is occupied by the host,
or processes are only allowed to access a limited range of virtual memory.
Table~\ref{tab:future:abi-limit}
list the limitations from the platforms that affects the implementation of host ABI.
We propose strategies to remove the limitations for the reinforcement of platform independence.


The proof-of-the-concept for the platform independence of \graphene{}
requires porting the host ABI to more platform variants,
especially ones that does not rely on a monolithic kernels.
One example is to port \graphene{} to a \term{\microkernel{}}
such as L4 and SeL4~\citep{l4family, klein09sel4}.
Porting \picoprocs{} to \microkernel{} will allows running legacy applications
in a host with restricted resource,
at a smaller cost than porting a monolithic kernel ported to \microkernel{}
(e.g., L4Linux~\citep{hartig97mu}).
Another direction is to port \graphene{} to \term{Barrelfish}~\citep{baumann09barrelfish},
a host platform designed for multi-core heterogeneous architectures,
using asynchronous message passing across heterogeneous cores pinned to applications.
The distributed POSIX implementation of \graphene{}
can be an advantage for adapting to Barrelfish.
%In addition, the OSX and Windows ports of \graphene{} are still in-progress.
Overall, when generalizing the host ABI implementation to more platforms,
we observe that a large portion of \graphene{} \pal{}
can be reused for another platform.
In particular, a wide range of platforms adopt the POSIX specifications.
By translating the \graphene{} host ABI to standard POSIX APIs,
the platform independence of \graphene{} can be guaranteed on
every POSIX-compliant hosts.



\section{Partitioning Legacy Applications}



%Although \graphene{}, \gsgx{} and \civet{} can fundamentally reduce the TCB required for an application,
%we observe missed opportunities in these solutions to further minimize the risk.
%In \graphene{}, the \libos{} of untrused applications are evicted from the TCB,
%but the host kernel must still be trusted. The system call restriction enforced by Seccomp filters is helpful, but it is hard to reason that the vulnerabilities are eliminated in the kernel.
%For a system that requires stronger enforcement, the \graphene{} PAL can be ported to a bare-metal or a \microkernel{} such as L4~\citep{l4family}.
%
%In \gsgx{} and \civet{}, we observe more opportunities of reducing TCB with engineering efforts.
%For instance, \gsgx{}, the size of \libos{} and supporting libraries
%can be as large as tens to hundreds of megabytes.
%The supporting classes partitioned by the \civet{} design-time tool
%may contain more than thousands of \java{} classes.
%All these code and data are not always necessary, and can be shredded more carefully or in finer granularity.


In our ongoing work,
we explore a system design (\term{\civet{}}) that automatically partitions a legacy \java{} application,
to isolate a minimal, security-sensitive component
in \intel{} \sgx{} enclaves.
We have a prototype framework to show the proof-of-the-concept.
For future works,
the implementation can be extended to handle more use cases as well as corner cases.

In particular, a primary challenge in \civet{} is
the effectiveness of the partitioning
--- in other word, the minimality of the split, security-sensitive components.
Another aspect of the partitioning is the effort needed from the developers,
to identify the boundary of the partition.
We observe that, tracing from
a few developer-identified interfaces to reach all of the supporting classes
will find at least xxx classes, xxx methods,
and xxx lines of code \fixme{Need to evaluate the actual number}
that needs to be included in a partition,
even if the partitioned component
is simply a ``Hello World''.
The overloaded TCB (trusted computing base) includes
methods used by the 
\java{} VM class loader and the \civet{} framework;
overridden super-class methods;
and methods that are part of a supporting classes
but never used.
We anticipate that the isolated components
can be further minimized,
by partitioning applications at the granularity of methods,
and shredding any methods that will never be called.

Nevertheless, \civet{} requires developers to provide two hints to guide the partitioning:
the classes that are used as interfaces,
and the classes that are dynamically loaded by the class loader.
However, identifying these classes are not always straightforward,
as we found in our use cases.
For example, some applications do not have the obvious interfaces
to interact with the sensitive components,
or, in the worst case, are not well modularized (e.g., all methods are in one main class).
On the other hand,
sometimes the \java{} VM loads classes based on some inputs from the users or developers,
such as loading cryptography APIs based on hard-coded or inputed strings
which describe the algorithms and options.
Even if developers who conduct the partitioning know the algorithms that will be used,
it is hard to identify the actual classes that will be loaded
unless the developers have knowledge about the internals of cryptography APIs.


An alternative approach will be to determine the minimal partition ``bottom-up'',
starting from the secrets or methods that need to be protected.
Instead of using a \emph{Shredder} for removing unused classes,
the partitioning can instead rely on a \emph{Collector} that incorporate the classes
that access any protected classes,
until it eventually determines a clean,
minimal partition
that is self-converging and has the narrowest interface.
This approach will require static analysis of data flow to conduct proper partitioning.

Besides in the partitioned application itself,
we also find opportunities in the \civet{} framework, the OpenJDK \java{} VM,
and the underlying \graphene{} \libos{},
to further reduce the TCB.
The unused library functions, \java{} VM features, JNI code, and system calls in \graphene{}
can all technically be removed with manageable efforts.


\section{Non-Binary Security Isolation}

%In \graphene{} or many others like VM or container-based solutions,
%applications are protected with a trust model of all-or-nothing.
%In other word, each process or component of an application can only be fully trusted or not trusted at all.
%The key reason of such a restriction is that the applied security isolation cannot reflect the complexity of privilege model in the application.
%In reality, multiple security principles can co-exist in an application.
%For instance, a web server that serves requests from clients identified as different clearance
%will need to maintain the correspondent confidentiality levels.
%The web server may perform operations that are more security sensitive (e.g., retrieve a secret key)
%or more vulnerable (e.g., execute privilege-escalating scripts).
%For components in the same process, we have seen examples, such as the heart-bleed bug~\citep{heartbleed} in OpenSSL, in which sensitive components are intruded by more functional components.


A partitioned system that sandboxes mutually untrusting applications
must restrict any interaction or coordination between the sandboxes.
Take \graphene{} for instance;
the related \picoprocs{} (created from the same application)
in a sandbox will trust each other
and be permitted to coordinate freely over the host RPC streams.
On the contrary, two unrelated \picoprocs{} from separate sandboxes will be completely detached,
with the \graphene{} reference monitor enforcing a strict,
all-or-nothing access control on RPC streams.
The rationale behind this design is to enforce simple but strong isolation from the host,
instead of relying on permission checks that are fragile and subtle
in a complex operating system.

In general, to enforce security isolation between sandboxes,
\graphene{} must assume
unrelated applications will never share any resources or states,
unless completely mediated by the host.
In other word, if two mutually untrusting applications
will ever share a resource or state,
\graphene{} must place the \picoprocs{} of both applications in a sandbox
and allow any coordination among them.
No partial isolation can be enforced in \graphene{}.
The all-or-nothing security isolation will
cause violation to the \emph{least privilege principle}
if the involved parties
require more fine-grained access control.
In fact, partial isolation is required by many applications,
when one process untrusts another because of the security vulnerabilities,
still have to interact in a limited, controlled way.
For instance,
a web server (e.g.,Apache) may prefer to be completely detach from a PHP engine (e.g.,FastCGI-PHP),
due to the much larger attack surface in the PHP engine.
However, A \picoproc{} that runs the web server
still needs to maintains minimal coordination with the \picoproc{} running the PHP engine,
such as connection through pipes,
or sharing the process ID namespace for signaling.
In a sense, \graphene{} does provide dynamic sandboxing
for \picoprocs{} that have dropped privileges,
but the sandboxed \picoproc{} will be completely detached from the others,
and appear to them as being destroyed.

A plausible model to allow partial security isolation among \picoprocs{}
is to enforce \term{namespace} access control.
The mechanism is similar to the namespace in UNIX system:
when a process clones a child using a flags for namespace isolation
(e.g., {\tt CLONE\_NEWNS}, {\tt CLONE\_NEWPID}),
the kernel data structures related with the namespaces will be strictly separated.
In \graphene{}, similar policies can be enforced
by setting up firewall rules on the host RPC streams.
Assuming a \picoproc{} strictly uses different RPC streams to coordinate different namespaces or abstractions,
host policies can be assigned to allow another untrusted
\picoproc{} accessing specific RPC streams.
A more fail-safe model will be assign firewall rules to filter the contents of RPC streams,
to prevent \picoprocs{} to make mistakes on separating RPC streams.
The \picoprocs{} can submit \term{BPF-style} (Berkeley Packet Filter) firewall rules to the reference monitors, to filter the fields in RPC messages.
Either approaches are platform-independent, and the former one requires no changes to the current host ABI.


Another model of non-binary security isolation
is to enforce \term{multi-level security} (MLS) among \picoprocs{},
so applications can be protected
with the ``no read up, no write down'' rule.
This model is similar to security isolation model of HiStar~\citep{zeldovich+histar}.
The \liboses{} will enforce DAC-type (Discretionary Access Control) policies,
by labeling \picoprocs{} based on the level of secrecy or integrity.
Among the \picoprocs{}, \term{information flow} can be implied to
determine whether a \picoproc{} is allowed to send or receive RPC messages
on the host streams connected with other \picoprocs{}.
The directional access control on host RPC streams is
a reasonable extension to the host ABI.






\section{Seamless Transition of Partitioned Systems}




%\section{Future Directions}

%
%\paragraph{Security Isolation for Multi-Principle Applications.}

%
%\paragraph{Seamless Transition of Security Isolation.}
%Each existing solution of security isolation can protect applications
%under specific security principles and assumptions.
%For instance, \graphene{} or other \picoproc{}-based solutions isolate mutually untrusting applications on a trusted host,
%whereas enclaves protect more sensitive applications on an untrusted host.
%No existing solutions can support all security principles and assumptions.
%Moreover, many solutions provide a container-like environment in which the operating system views are completely isolated.
%These limitations cause different solutions to be mutually exclusive,
%and users are held responsible for making the decisions of choosing the solution
%--- simply put, to explicitly run {\tt pal} or {\tt pal-sgx} to load applications in \graphene{} or \gsgx{}. 
%The penalties of the security solution such as performance overhead or incompatibility
%will make users to be reluctant
%to choose one solution to run all related applications,
%if given the choice.
%
%%Security isolation for applications mostly requires users to run applications in a container-like environment, consciously and explicitly.
%%Simply put, a user must always launch an application in \graphene{} or \gsgx{} by executing their loaders, {\tt pal} or {\tt pal-sgx}.
%%In practice, however, users often have insufficient knowledge of the security requirement of applications
%%to decide whether to enforce stronger security isolation.
%%The common result of the problem is that security isolation solutions becomes mutually exclusive for an operating system to choose.
%
%While operating systems have sufficient information to determine the security principles of an application,
%the existing solutions of security isolation
%are not designed to seamlessly transit into one another.
%We can use \graphene{}, \gsgx{} and Linux as an example of transition between solutions.
%%We observe that \graphene{} and \gsgx{} provide compatible Linux personality,
%%so that applications can run seamlessly in these environments.
%The Linux personality of \graphene{} and \gsgx{}
%make it feasible to dynamically migrate applications from Linux to a \picoproc{}
%or an enclave according to the security principles.
%%Operating system can determine the appropriate security isolation for applications, based on the security principles.
%%For instance, operating systems can determine how to isolate an application based on its origin
%A security-sensitive application
%that is signed by the developers to always run in an enclave,
%can demand a regular process to be migrated into another enclave
%if the the process is requesting any interaction
%--- a requirement that can be verified by the enclave, without trusting the Linux host.
%%while a suspicious, downloaded application will run in a \picoproc{}.
%%If a regular application interacts with the sensitive one, the latter's \libos{} can demand the former to be migrated into an enclave.
%%In contrast, if a application becomes tainted by a suspicious application,
%%operating system can migrate the application to a \picoproc{}.
%Similar technique can be applied to drop the privilege of a regular application to a \picoproc{}
%if tainted by a low-security applications already isolated in another \picoproc{}.
%
%
%\paragraph{Minimizing TCB.}



\section{Proposed Schedule for Fulfilling Thesis Requirement}