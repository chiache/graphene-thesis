%\section{Evaluation}
%\label{sec:eval}
%\note{about 1.5page}

\begin{table}[t!b!]
\footnotesize
\centering
\begin{tabular}{p{2.45in} r}
\toprule
\textbf{Evaluation Criteria} & \textbf{Size}\\
\midrule
\addlinespace
Source Lines of Code (Python) & 3,105 \\
\addlinespace
Source Lines of Code (SQL) & 2,423 \\
\addlinespace
Total Rows in Database & 428,634,030 \\
%\addlinespace
%Missed/Unknown system call instances & 1,643\\
%\addlinespace
%Missed/Unknown vectored call instances & 2,212\\
\end{tabular}%
\caption{Implementation of the API usage analysis framework.}
\label{tab:syspop:eval}%
\end{table}%

Our implementation is summarized in Table~\ref{tab:syspop:eval}.
We wrote 3,105 lines of code in Python and 2,423 lines of code in SQL (Postgresql).
The database contains 48 tables with over 428 Million entries.
%\rev{Update false-negative}
%{Due to the limitations of our approach,
%we cannot identify the system call numbers at 4.2\% of system call locations,
%and the opcodes at 3.7\% of vectored system call locations,
%among all binaries studied.}
%\fixmedp{I still don't understand what this means.  How did you know you missed some?}
%4.2\% of locations in binary 
%1,643 instances of missed/unknown system call instances
%and 2,212 missed/unknown vectored call instances.
% \fixmedp{What does this mean, exactly?  What does it mean to miss an instance?}
