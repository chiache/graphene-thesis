\section{Summary}
\label{sec:graphene:summary}

%\sysname{} extends library OS designs 
%to include multi-process APIs required by common applications, such as a shell or 
%web server.
%\sysname{} demonstrates efficient, selective
%coordination of shared state across multiple library OS 
%instances---maintaining host independence.
%%simplifying security sandboxing of otherwise unwieldy OS features.
%Applications on \sysname{} enjoy both 
%strong security isolation with acceptable performance and low memory overheads.
%% from unrelated programs 
%%and seamless shared namespaces 
%%among a group of coordinating guests.
%%% Although this paper focuses on distributed coordination
%%% to facilitate the efficiency benefits,
%%% expect our experiences with distributed coordination 
%%% may also be particularly relevant to highly scalable OS designs, 
%%% which avoid the bottlenecks of shared OS data structures~\citep{baumann09barrelfish, song11eurosys}.
%%Graphene's overheads are acceptable and the memory 
%%footprint is substantially lower than a VM.



%% , which could benefit from the reduced memory footprint
%% in a cloud 

%% by introducing a novel design for  coordination APIs. 
%% to a new OS (Linux),
%% new classes of applications,
%% and introduces a
%% %an alternative design point for storage virtualization.
%% Our results further demonstrate the feasibility of the library OS model.
%% % generally,
%% Applications on Graphene enjoy both 
%% strong security isolation from unrelated programs 
%% and seamless shared namespaces 
%% among a group of coordinating guests.
%% Although we explore this concept in a library OS,
%% we expect the namespace coordination framework 
%% could also be adapted to limit the attack surface area between
%% processes in a traditional OS.
%% We expect these experiences with distributed coordination 
%% may also be particularly relevant to highly scalable OS designs, 
%% which avoid the bottlenecks of shared OS data structures~\citep{baumann09barrelfish}.
%and specifically of content-addressable storage as the primary virtual storage abstraction.
%%% This work opens up a number of interesting questions for future work, 
%%% including studying opportunities for low-level storage optimization within the CAS server,
%%% making CAS the root file system,
%%% eliminating storage management in the host kernel, and 
%%% investigating the impact of frequent migration among devices.

Enabling legacy applications in a restricted environment,
such as \picoprocs{} or enclaves,
requires extra effort for mitigating the limitations of platforms,
in order to support typical OS personalities.
\term{\graphene{}}, as described in this chapter, extends the existing \libos{} designs
from isolating single-process or unshared abstractions
to include multi-process APIs required by many UNIX applications,
such as servers or shell scripts.
The challenge that \graphene{} primarily overcomes
is the requirement for coordinating shared states across multiple \picoprocs{},
to provides a collaborative, unified OS view.
Essentially, \graphene{} implements all shared, multi-process abstractions
and OS states
based on coordination over host-provided, pipe-like RPC streams.
The RPC-based, distributed OS implementation enables multi-process support in \graphene{}, with minimal extension to the host interface,
and a sweet-spot for enforcing inter-application security isolation,
by simply sandboxing the RPC streams.
Such a model largely reduces the complexity of
enforcing security isolation
on idiosyncratic multi-process abstractions
and shared states.
Because the corporative nature of \picoprocs{} in \graphene{},
an application can even dynamically impose sandboxing on one of its processes,
to reflect per-process, variable security policies.


In principle, we attempt to use \graphene{} to justify the platform independence
of the \libos{} design,
without sacrificing its qualitative benefits,
such as isolating mutually-untrusting applications
and a narrow attack surface to kernels.
\graphene{} implements a considerable number of common Linux system calls,
to support popular, modern applications
such as Apache web server, GNU Make, OpenJDK \java{} VM and the Python runtime.
\graphene{} translates the high-level system APIs used by applications
to a host ABI
inherited and extended from a previous Windows-compatible \libos{}~\citep{porter11drawbridge}.
In addition, we port the \pal{} (Platform Adaption Layer) of \graphene{}
to various platforms,
including FreeBSD, OSX, Windows, and even a more restricted environment, the \intel{} \sgx{} enclaves.
In particular, \graphene{} being ported to \intel{} \sgx{}
(\term{\graphenesgx{}})
can isolate applications --- either single-process or multi-process
--- on a host where neither the operating system nor the hardware (except the CPU package)
is trusted by the applications. 
Overall, \graphene{} shows that,
by simply porting the reasonably sized host ABI
to a new platform,
a whole large spectrum of legacy applications tested on the previous platforms
can be activated all together.
