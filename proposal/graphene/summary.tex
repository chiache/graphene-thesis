\section{Summary}
\label{sec:graphene:summary}

%\sysname{} extends library OS designs 
%to include multi-process APIs required by common applications, such as a shell or 
%web server.
%\sysname{} demonstrates efficient, selective
%coordination of shared state across multiple library OS 
%instances---maintaining host independence.
%%simplifying security sandboxing of otherwise unwieldy OS features.
%Applications on \sysname{} enjoy both 
%strong security isolation with acceptable performance and low memory overheads.
%% from unrelated programs 
%%and seamless shared namespaces 
%%among a group of coordinating guests.
%%% Although this paper focuses on distributed coordination
%%% to facilitate the efficiency benefits,
%%% expect our experiences with distributed coordination 
%%% may also be particularly relevant to highly scalable OS designs, 
%%% which avoid the bottlenecks of shared OS data structures~\citep{baumann09barrelfish, song11eurosys}.
%%Graphene's overheads are acceptable and the memory 
%%footprint is substantially lower than a VM.




%% , which could benefit from the reduced memory footprint
%% in a cloud 

%% by introducing a novel design for  coordination APIs. 
%% to a new OS (Linux),
%% new classes of applications,
%% and introduces a
%% %an alternative design point for storage virtualization.
%% Our results further demonstrate the feasibility of the library OS model.
%% % generally,
%% Applications on Graphene enjoy both 
%% strong security isolation from unrelated programs 
%% and seamless shared namespaces 
%% among a group of coordinating guests.
%% Although we explore this concept in a library OS,
%% we expect the namespace coordination framework 
%% could also be adapted to limit the attack surface area between
%% processes in a traditional OS.
%% We expect these experiences with distributed coordination 
%% may also be particularly relevant to highly scalable OS designs, 
%% which avoid the bottlenecks of shared OS data structures~\citep{baumann09barrelfish}.
%and specifically of content-addressable storage as the primary virtual storage abstraction.
%%% This work opens up a number of interesting questions for future work, 
%%% including studying opportunities for low-level storage optimization within the CAS server,
%%% making CAS the root file system,
%%% eliminating storage management in the host kernel, and 
%%% investigating the impact of frequent migration among devices.

