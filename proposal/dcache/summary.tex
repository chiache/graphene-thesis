\newpage
\section{Summary}
\label{sec:dcache:summary}

The \term{file system directory cache} is a sophisticated subsystem
of the Linux virtual file system (VFS),
which is designed for caching the file paths and metadata
retrieved from the storage.
%in the main memory.
We observe missed opportunities in the design of the directory cache,
primarily in removing the redundant operations
that contribute to the lookup latency at cache hits.
These operations include checking prefix permissions against security models,
detecting mount points,
and resolving symbolic links, etc.
A fast path for looking up already cached information
will fully exploit the locality of path access in the file system,
to cache the results of permission checks and other redundant operations.
The optimized design not only
minimizes the hit latency in favor of frequently accessed paths,
but also improves the hit rate,
by capturing other missed information (e.g., nonexistent paths) that
shall be cached. % in the memory.
%For instance, for the paths confirmed to be nonexistent in the storage,
%especially when the path is recently deleted,
%or one of its prefix is a nonexistent directory,
%a \emph{negative entry} can be stored in the directory cache.
Essentially, we optimize the hit latency by penalizing the infrequent operations
of updating path attributes and permissions,
as an acceptable trade-off in most scenarios.

The optimization on the directory cache
shows an example for introducing performance improvement
into a legacy system,
while preserving the existing
system abstractions and security mechanisms.
We start by investigating the opportunity of
rewriting the component-by-component lookup algorithm
in the directory cache,
and identify the culprit of the suboptimality
as the interleaving of looking up data structures and processing file attributes.
The principle behind this optimization
is to decouple the implementation of
an efficient, frequently occuring operation
from addressing security concerns and fulfilling system features
--- a principle we also use
in implementing an efficient \libos{}.




%This paper presents a directory cache design
%that efficiently 
%maps file paths to in-memory data structures
%in an OS kernel.
%Our design decomposes the directory cache into separate caches
%for permission checks and path indices, enabling single-step path lookup,
%as well as facilitating new optimizations based
%on signatures and caching symbolic link resolution.
%For applications that frequently 
%interact with the file system directory tree, these optimizations 
%can improve performance by up to \updatedbspeedup\%.
%% is an important task for 
%%modern OS kernels. \fixmedp{more justification?}
%%This paper demonstrates how this space used to cache the directory hierarcy can be put to even more effective use,
%%improving application performance by up to 30\%.
%Our optimizations
%maintain compatibility with a range of applications and kernel extensions,
%making them suitable for practical deployment.

%\tsai{14 pages, not including references}


