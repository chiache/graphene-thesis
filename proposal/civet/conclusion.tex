\section{Conclusion}
\label{sec:conclusion}

Language-level analyses and hardware memory protection are both valuable
building blocks for secure applications.
\systemname{} shows a constructive synthesis of both technologies---combining their strengths
and mitigating downsides of either when used in isolation.
Although this paper focused on Java and Intel SGX as a specific case study,
we believe the techniques in this paper are generic enough to apply to
other managed languages and hardware isolation platforms.
In future work, we plan to incorporate additional analyses
to further reduce the TCB size within an enclave, and to provide
stronger integrity protections against adversarial input.


%% With \systemname{}, we demonstrate how hardware and language-level protection
%% can be integrated,
%% to eliminate vulnerabilities of both approaches.
%% \systemname{} provides guided partitioning for \java{} applications,
%% transparently porting security-sensitive classes
%% into an \intel{} \sgx{} enclave, so that their execution can be isolated
%% from malicious system stacks.
%% The classes in the enclave are harden by \java{}-based language protection,
%% such as type-checking and information flow control,
%% preventing any secret being leaked due to vulnerabilities in the isolated classes.
%% We show several use cases of the framework, for protecting
%% both class implementation and output data,
%% and keeping the development and deployment cost minimal.
