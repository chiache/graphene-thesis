\section{Implementation Details}
\label{sec:civet:impl}

In this section, we discuss in detail about how the framework of \sysname{}
is implemented. 


\paragraph{Graphene library OS for \sgx{}.}
Similar to Drawbridge in Haven, Graphene \libos{} can map a larger number of system interfaces
to a much narrower host interface.
The untrusted interface of Graphene libOS after porting to \sgx{}
is mostly identical to the original host interface,
therefore Graphene libOS has a finite  bound on interface for the enclave.

Graphene libOS uses checksums of files to verify the integrity of supporting binaries and configuration.
The checksums of files are collected by a compile-time Signer tool of Graphene libOS.
The Signer tool ensures the integrity of the file checksums
by including them as part of the enclave's measurement.
Even if the same Graphene libOS is used
to run the applications,
different binaries in the enclaves yield different measurements. 
Such a design decouples the problem of distributing the library OS
and guaranteeing code integrity for each application, as the enclave integrity is based on the integrity of the application --- not just the \libos{}.
 
\sysname{} makes minor modifications to Graphene \libos{} to allow
the enclave to run
in the same process as the the untrusted \java{} VM.
The trusted and untrusted VM share a ring buffer (as shown in ~\ref{fig:civet:runtime} for communication during control transfer,
but the ring buffer itself is not trusted. The \sysname{} framework encrypts tainted security-sensitive data before passing the data on the ring buffer.
 
\paragraph{Control transfer at enclave entry.}

\sysname{} transparently transfers the control from the untrusted classes to trusted classes --- triggering enclave entry in the process ---
by intercepting the trusted classes's method or constructor invocation by untrusted classes.
\sysname{} intercepts in 2 cases:
(a) For an entry class that has finite entry points,
\sysname{} creates a bogus class that redirects all the constructors and static methods to the enclave.
(b) For other trusted classes, the \sysname{} adds the redirection calls only when a reference is returned by the enclave.
%(b) For other trusted classes, the interception is installed when a reference of 
%object is returned to the untrusted classes.
On future access of the object in the enclave, a proxy object is created to trigger the interception, using CGLib.

When the control transfers between the trusted and untrusted classes,
the arguments and return values of the methods
are stored in the ring buffer.
\sysname{} relies on serialization and deserialization to safely transfer object in and out of the enclave.
When \sysname{} deserializes an object, the \java{} VM perform type-checking on the object to sanity check the members.

After \sysname{} transfers the arguments, the \java{} VM thread that
invoke the intercepted  method does not directly enter the enclave.
On the other hand, several enclave threads that are pre-created during the enclave creation that are block-waiting on the ring buffer.
One of the enclave threads consumes the queued job, invokes the method,
and returns to block-waiting for more new requests.
No variable on the stack has to be persistent for the enclave threads,
and only instances of the trusted classes are persisted across enclave entry and exit.

Moreover, upon enclave entry and exit, the objects that can be safely transferred in or out of the enclave must be serialized / deserialized.
However, not all classes implement the interface {\tt Serializable},
especially when the classes contain internal states that cannot be simply interpreted.
But, \sysname{} assumes that the types of all arguments and return objects must be {\tt Serializable}. 

\paragraph{Framework limitations.}

As we leverage dependency tracking for automated partition, there are few corner cases that the Shredder cannot gracefully handle.
Because CGLib creates subclasses of objects when intercepting them,
it requires the intercepted object to be never finalized.
If a trusted class is also a final class, developers have to manually modify the class definition. Even though final attribute prevents further extension of the class, we argue that the application developer who is building the enclave can safely remove the final attribute as the \sysname{} signs the enclave jar.

 We also do not let trusted classes make method or constructor invocation of the untrusted classes, as the untrusted classes may be able to influence and interfere with the execution of trusted classes. Moreover, we only consider the provisioned code and data as security-sensitive. Even though this limits the usage scenarios, we argue that for any right usage of \sgx{} hardware, these limitations are not disruptive.
