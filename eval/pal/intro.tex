The section evaluates the performance of \thehostabi{}
on each host.
As a baseline, the latency or throughput
of a \hostapi{} is always bound by the performance of the underlying host system interface.
For instance, the Linux PAL implements \palcall{StreamRead}
using the \syscall{read} system call,
and thus introduces the same performance
patterns to the \hostapi{}.
However, the performance of \thehostabi{} and the underlying host system interface
is unlikely to be equivalent,
due to the cost of translating between
the two system interfaces
when they have different semantics. 
Moreover, a host like SGX imposes compatibility challenges
to applications,
which the PAL must address.
Part of the overheads on the SGX PAL
subject to enclave exits,
since enclave code cannot invoke system calls directly.
The SGX PAL
also suffer memory access overheads
when bringing memory into the EPC (enclave page cache)
or decrypting memory
on a last-level cache miss.
%The latency on the SGX PAL is then mostly dominated by the overheads
%of switching the context between the enclave
%and the untrusted, external PAL using SGX instructions,
%and bringing memory into the EPC (enclave page cache).
% or decrypting memory on a last-level cache miss.


%In theory, the definition of \thehostabi{}
%is meant to minimize the translation cost to most host system interfaces,
%by adopting similar, UNIX-like semantics.
%%Since most \hostapis{} are just wrappers
%%of the underlying system calls,
%%the translation cost is mostly reduced to simply replacing the generic identifiers and control flags
%%to the semantics that are recognized by the host OS.
%A major cost
%in translation is to parse a URI (unified resource identifier)
%for determining the file path or network address
%%and compose a equivalent file system path or network address
%to request host OS services.
%%The second factor is the translation cost
%%of the \hostapis{}.
%%For portability, \thehostabi{} is defined with generic semantics, without host-specific notions
%%such as process identifiers, file descriptors,
%%file system paths,
%%and system call flags.
%%The PAL must translate the arguments of \hostapis{}, including PAL handles, URIs (Uniform Resource Identifiers), and generic flags,
%%to the arguments interpretable by the host kernel.
%Other translation costs,
%such as converting optional flags,
%are relatively marginal, compared with the overall latency of a \hostapi{}.
%%because the translation is mostly straightforward,
%%and only requires simply logics
%%and little memory copy.
%%Without extra security checks,
%%a \hostapi{} is more likely to suffer high overheads if the implementation
%%requires additional system calls for complementary operations
%%to the base abstraction,
%%or constantly retrying a system calls at failures.



More significant overheads
on a few \hostapis{} contribute to security checks or enforcements,
to protect applications
inside host-specific threat models.
%for ensuring the safety of running an application
%within the threat model
%of a host.
%Finally, the third factor that impacts the PAL call efficiency
%is the cost of security checks,
%either inside the host kernel or the guest.
%The cost of security checks varies between hosts, and is correlated with the presumed security models.
For example, the threat model on a Linux host
focuses on the attacks between mutually-untrusting applications via system interfaces;
therefore,
the security checks on the Linux PAL
restrict the sharing of host resources and block system calls that are not required by the Linux PAL.
%using both the reference monitor and \seccomp{} filter.
In another threat model, with the SGX enclave, 
security checks for each \hostapi{}
protects the application and \libos{} against malicious inputs from an untrusted OS,
%and thus focus on validating the results of system calls,
using either cryptographic techniques or semantic checks.
%Cryptographic techniques are used to: (1) validate the file against the secure hash, at \palcall{StreamOpen}, (2) check the file chunks against a Merkle tree of hash values, at \palcall{StreamRead}, and (3) establish a TLS connection over inter-enclave RPC, at \palcall{ProcessCreate}.
On the SGX PAL, the latency of a \hostapi{} may be dominated
by security checks,
especially the ones based on cryptographic operations.
%on large chunks of data exported to the untrusted OS.


The evaluation in this section is based on micro-benchmark programs similar to \lmbench{} 2.5~\cite{McVoy:lmbench}.
For each \hostapis{}, the evaluation also shows the breakdowns
of its latency or throughput,
by benchmarking the PALs both with and without the security mechanisms, such as the \seccomp{} filter and reference monitor on the Linux PAL,
as well as testing under different
implementation strategies.
The evaluation focuses on \hostapis{} that are especially sensitive for the performance of the \graphene{} \libos{}.


% the efficiency of \hostapis{} on both Linux and SGX hosts, and shows the impact of each performance factor.
%The evaluation is based on micro-benchmark programs similar to \lmbench{} 2.5~\cite{McVoy:lmbench},
%and is compared against
%similar system calls on Linux.


















