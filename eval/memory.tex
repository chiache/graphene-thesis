%\fixmedp{Error bars on mem usage graphs missing}

\subsection{Memory footprint}
\label{sec:eval:perf:memory}

We begin by measuring the minimal memory footprint of a simple ``hello world'' program 
on Linux (352 KB) and \graphene{} (1.4 MB).
Thus, one would expect roughly 1 MB of extra memory usage for any single-\picoproc{} application.
Because of copy-on-write sharing, however, the incremental cost of adding additional ``hello world'' children
is only about 790 KB per process.
%, although this will vary based on usage.

%Figure~\ref{fig:graphene:memusage} lists memory overheads of 
%a diverse set of {\em unmodified}
%applications, including 
%a {\tt make -j 4} of Gra\-phene's {\tt libLinux}
%using the {\tt \gcc{}} compiler (v4.4.5), 
%%\busy, a software which combines tiny versions of most common UNIX utilities into a single
%%small executable;
%the \light{} web server (v1.4.30) with 4 threads,
%the Apache web server (v2.4.3) with 4 processes,
%and the \busy{} shell (v4.1) executing the shell script test ({\tt multi.sh}) from the Unixbench suite (v5.1.3)~\cite{unixbench}.
%We measure memory utilization based on the maximum kernel-reported resident set size
%of each process or VM. For most applications, memory usage was fairly constant across inputs,
%so we only display representative examples.
%
%\begin{figure}[t!]
%\centering
%\includeplot[0.5]{apps-memusage}
%\centerline{\includegraphics[width=\linewidth]{figures/memusage.pdf}}
%\caption{Memory usage of applications on Linux, \graphene{}, and KVM, in MB.  Lower s better.
%\label{fig:graphene:memusage}}
%\end{figure}


We found that the memory footprints of compilation were a function of the 
size of the source base, even on Linux; we select compile of {\tt libLinux} as a representative example.
\graphene{} adds less than 15\% overhead in all cases.
%every compilation we measured on \graphene{} had less than 10\% memory overhead.

Unixbench on \graphene{} uses substantially more memory at a given time than native Linux---more than double.
In these samples, however, \graphene{} also had 3--4\x{} as many processes
running; this is because Unixbench simply spawns all of the tasks in the background, rather than
executing them sequentially.   Because \graphene{} processes execute more slowly (attributable to a slower {\tt fork}---\S\ref{sec:eval:graphene}),
a given sample will include more \picoprocs{}, pushing total memory usage higher.
Thus, we expect that further tuning {\tt fork} performance will lower sampled memory usage.


Across all workloads, \graphene{}'s memory footprint is 3--20\x{}  
smaller than KVM.  
For all tests, we used a minimal KVM disk image, 
%formatted as a 30GB\fixmewkj{this can be changed to be smaller, but I don't know if disk size would impact memory} raw disk image with ext4. The root image was 
generated using debootstrap 1.0.39ubuntu0.3 and supplemented only by packages required to obtain, compile, and run the experiments.
In order to make memory footprint measurements as fair as possible to KVM, 
we used both the virtio balloon driver and kernel same page merging (KSM)~\cite{ksm}.
We also reduced the RAM allocated to the VM to the smallest size without harming performance---128MB.  We note that memory measured includes memory used by QEMU for VM device emulation, 
adding a few dozen MB.
%hence the total is 
%a few dozen MB higher than the RAM allocated to the VM.

If the smallest usable Linux VM consumes about 150 MB of RAM, our measurements indicate that 
one could run anywhere from 12--188 libOSes within the same footprint.