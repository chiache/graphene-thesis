\papersubsection{Single-process system calls}
\label{eval:perf:syscalls}


%In order to understand the overheads of individual system calls,
Table~\ref{tab:eval:lmbench-syscalls} lists 
a representative sample of 
tests from \lmbench{} 2.5 benchmark suite~\cite{McVoy:lmbench}
(extended with additional experiments).
Each row reports a mean and 95\% confidence interval,
assuming the benchmark results are normally distributed;
to improve the precision,
the number of iterations in each test is increased to at least a thousand times, which effectively lower the variance
in most tests.
Although assuming a normal distribution may not be realistic for most benchmark results,
the error is likely to be marginal with the very low variance
observed in the tests.
%we use the default number of iterations for each test case.
%We have added code to \lmbench{} to also calculate 95\% confidence intervals 
%within a run~\footnote{The lmbench authors deliberately exclude variation statistics
%because most methods assume a known distribution, generally a normal distribution---an 
%assumption which is often not the case for a computer microbenchmark~\cite{staelin05lmbench}.
%Though confidence intervals should be taken with a grain of salt, 
%we include them because they clearly indicate that these experiments have very low variance. In 
%a few cases of minor performance improvement, one can assess the impact of noise.}.
Besides, to measure the marginal cost of the \seccomp{} filter and reference monitor on a Linux host,
the experiments include the cases both with
and without the \seccomp{} filter and reference monitor.


The evaluation results categorize
the system calls emulated inside \thelibos{}
as two types.
The first type of system calls is completely serviced inside \thelibos{};
the evaluation results show that these system calls are even faster than native, because \thelibos{} does not switch context into the kernal space
to service the system calls.
For instance,
a null system call (i.e., \syscall{getppid}) and installing a signal handler with \syscall{sigaction}
are up to three time as fast as the native performance.
The second type of system call requires translation to a native host system calls.
%For instance, 
%the self-signaling test (sig overhead)
%just calls the signal handler as a function,
%which is almost twice as fast
%as the Linux kernel implementation.

\begin{table}[htp!]
\footnotesize
\centering
\bgroup
\def\arraystretch{1.1}
\setlength{\tabcolsep}{.5em}
\begin{tabular}{|ll|>{\palign{r}}p{3em}r|>{\palign{r}}p{3em}rr|>{\palign{r}}p{3em}rr|>{\palign{r}}p{3em}rr|}
\hline
& & \multicolumn{11}{c|}{System call latency (\usec{}), +/- Confidence Interval, \% Overhead} \\
\hline
\multicolumn{2}{|c|}{{\bf Test}} &
\multicolumn{2}{c|}{{\bf Linux \linuxversion{}}} &
\multicolumn{3}{c|}{{\bf \graphene{}}} & \multicolumn{3}{c|}{{\bf \graphene{}+SC+RM}} & \multicolumn{3}{c|}{{\bf \graphenesgx{}}} \\
& &
\usec{} & +/- & 
\usec{} & +/- & \%O &
\usec{} & +/- & \%O &
\usec{} & +/- & \%O \\
\hline																					
\multicolumn{2}{|l|}{{\tt getppid}}			&	0.045	&	.000	&	0.015	&	.000	&	-67	&	0.015	&	.000	&	-67	&	0.015	&	.000	&	-67		 \\\hline
\multicolumn{2}{|l|}{{\tt getppid} (direct)}			&	0.045	&	.000	&	\multicolumn{3}{c|}{Not supported}					&	1.155	&	.000	&	2,467	&	5.800	&	.001	&	12,789		 \\\hline
\hline																										
{\tt open}	&	{\tt /dev/zero}	&	0.997	&	.072	&	1.247	&	.000	&	25	&	1.256	&	.000	&	26	&	1.207	&	.004	&	21		 \\\hline
{\tt stat}	&	{\tt /dev/zero}	&	0.362	&	.000	&	0.466	&	.000	&	29	&	0.467	&	.000	&	29	&	0.451	&	.000	&	25		 \\\hline
{\tt fstat}	&	{\tt /dev/zero}	&	0.117	&	.000	&	0.111	&	.000	&	-5	&	0.111	&	.000	&	-5	&	0.107	&	.000	&	-9		 \\\hline
{\tt read}	&	{\tt /dev/zero}	&	0.116	&	.000	&	0.121	&	.000	&	4	&	0.121	&	.000	&	4	&	0.115	&	.000	&	-1		 \\\hline
{\tt write}	&	{\tt /dev/zero}	&	0.077	&	.000	&	0.116	&	.000	&	51	&	0.116	&	.000	&	51	&	0.112	&	.000	&	45		 \\\hline
\hline																										
install	&	sigaction	&	0.146	&	.000	&	0.113	&	.000	&	-23	&	0.113	&	.000	&	-23	&	0.110	&	.000	&	-25		 \\\hline
send	&	{\tt SIGUSR1}	&	0.895	&	.000	&	0.189	&	.000	&	-79	&	0.187	&	.000	&	-79	&	0.178	&	.000	&	-80		 \\\hline
catch	&	{\tt SIGSEGV}	&	0.379	&	.000	&	1.526	&	.000	&	303	&	1.575	&	.000	&	316	&	6.117	&	.000	&	1,514		 \\\hline

\end{tabular}
\egroup
\caption{System call benchmark results based on \lmbench{} 2.5. Comparison is among (1) native Linux processes, (2) \graphene{} \picoprocs{} on Linux host, both without and with JIT-optimized SECCOMP filter ({\bf +SC}) and reference monitor ({\bf +RM}), and (3) \graphene{} in \sgx{} enclaves.
System call latency is in microseconds, and lower is better.
System call bandwidth and throughput are in megabytes per second and operations per second, respectively, and higher is better. 
%The file system is measured in thousands operations per second, and higher is better.
Overheads are relative to Linux \linuxversion{}; negative overheads indicate improved performance.} 
\label{tab:eval:lmbench-syscalls}
\end{table}


