\papersubsection{Single-process system calls}
\label{eval:perf:syscalls}


\begin{table}[t!b!]
\footnotesize
\centering
\bgroup
\def\arraystretch{1.1}
\setlength{\tabcolsep}{.5em}
\begin{tabular}{|ll|>{\palign{r}}p{3em}r|>{\palign{r}}p{3em}rr|>{\palign{r}}p{3em}rr|>{\palign{r}}p{3em}rr|}
\hline
& & \multicolumn{11}{c|}{System call latency (\usec{}), +/- Confidence Interval, \% Overhead} \\
\hline
\multicolumn{2}{|c|}{{\bf Test}} &
\multicolumn{2}{c|}{{\bf Linux \linuxversion{}}} &
\multicolumn{3}{c|}{{\bf \graphene{}}} & \multicolumn{3}{c|}{{\bf \graphene{}+SC+RM}} & \multicolumn{3}{c|}{{\bf \graphenesgx{}}} \\
& &
\usec{} & +/- & 
\usec{} & +/- & \%O &
\usec{} & +/- & \%O &
\usec{} & +/- & \%O \\
\hline																					
\multicolumn{2}{|l|}{{\tt getppid}}			&	0.045	&	.000	&	0.015	&	.000	&	-67	&	0.015	&	.000	&	-67	&	0.015	&	.000	&	-67		 \\\hline
\multicolumn{2}{|l|}{{\tt getppid} (direct)}			&	0.045	&	.000	&	\multicolumn{3}{c|}{Not supported}					&	1.155	&	.000	&	2,467	&	5.800	&	.001	&	12,789		 \\\hline
\hline																										
{\tt open}	&	{\tt /dev/zero}	&	0.997	&	.072	&	1.247	&	.000	&	25	&	1.256	&	.000	&	26	&	1.207	&	.004	&	21		 \\\hline
{\tt stat}	&	{\tt /dev/zero}	&	0.362	&	.000	&	0.466	&	.000	&	29	&	0.467	&	.000	&	29	&	0.451	&	.000	&	25		 \\\hline
{\tt fstat}	&	{\tt /dev/zero}	&	0.117	&	.000	&	0.111	&	.000	&	-5	&	0.111	&	.000	&	-5	&	0.107	&	.000	&	-9		 \\\hline
{\tt read}	&	{\tt /dev/zero}	&	0.116	&	.000	&	0.121	&	.000	&	4	&	0.121	&	.000	&	4	&	0.115	&	.000	&	-1		 \\\hline
{\tt write}	&	{\tt /dev/zero}	&	0.077	&	.000	&	0.116	&	.000	&	51	&	0.116	&	.000	&	51	&	0.112	&	.000	&	45		 \\\hline
\hline																										
install	&	sigaction	&	0.146	&	.000	&	0.113	&	.000	&	-23	&	0.113	&	.000	&	-23	&	0.110	&	.000	&	-25		 \\\hline
send	&	{\tt SIGUSR1}	&	0.895	&	.000	&	0.189	&	.000	&	-79	&	0.187	&	.000	&	-79	&	0.178	&	.000	&	-80		 \\\hline
catch	&	{\tt SIGSEGV}	&	0.379	&	.000	&	1.526	&	.000	&	303	&	1.575	&	.000	&	316	&	6.117	&	.000	&	1,514		 \\\hline

\end{tabular}
\egroup
\caption{Single-process system call performance based on \lmbench{}. Comparison is among (1) native Linux processes, (2) \graphene{} \picoprocs{} on Linux host, both without and with SECCOMP filter ({\bf +SC}) and reference monitor ({\bf +RM}), and (3) \graphene{} in \sgx{} enclaves.
System call latency is in microseconds, and lower is better.
Overheads are relative to Linux; negative overheads indicate improvement.} 
\label{tab:eval:lmbench-syscalls}
\end{table}


%In order to understand the overheads of individual system calls,








\papersubsection{File systems}
\label{eval:perf:syscalls}


\begin{table}[t!b!]
\footnotesize
\centering
\bgroup
\def\arraystretch{1.1}
\setlength{\tabcolsep}{0.4em}
\begin{tabular}{|l|>{\palign{r}}p{4em}r|>{\palign{r}}p{4em}rr|>{\palign{r}}p{4em}rr|>{\palign{r}}p{4em}rr|}
\hline
&\multicolumn{11}{c|}{System call latency (\usec{}), +/- Confidence Interval, \% Overhead} \\
\hline
\multicolumn{1}{|c|}{{\bf Test}} &
\multicolumn{2}{c|}{{\bf Linux \linuxversion{}}} &
\multicolumn{3}{c|}{{\bf \graphene{}}} & \multicolumn{3}{c|}{{\bf \graphene{}+SC+RM}} & \multicolumn{3}{c|}{{\bf \graphenesgx{}}} \\
&
\usec{} & +/- & 
\usec{} & +/- & \%O &
\usec{} & +/- & \%O &
\usec{} & +/- & \%O \\
\hline

\parbox{\widthof{open}}{open} (d=2,len=08)	&	0.947	&	.072	&	2.474	&	.007	&	161	&	2.892	&	.007	&	205	&	16.676	&	.018	&	1,661		 \\\hline
\parbox{\widthof{open}}{open} (d=4,len=16)	&	1.011	&	.074	&	2.895	&	.034	&	186	&	3.241	&	.007	&	221	&	17.456	&	.017	&	1,627		 \\\hline
\parbox{\widthof{open}}{open} (d=6,len=24)	&	1.074	&	.068	&	3.363	&	.070	&	213	&	3.708	&	.007	&	245	&	18.084	&	.018	&	1,584		 \\\hline
\parbox{\widthof{open}}{open} (d=8,len=32)	&	1.131	&	.074	&	4.259	&	.007	&	277	&	4.601	&	.007	&	307	&	19.196	&	.020	&	1,597		 \\\hline
\parbox{\widthof{open}}{stat} (d=2,len=08)	&	0.361	&	.000	&	0.904	&	.000	&	150	&	0.857	&	.000	&	137	&	0.846	&	.000	&	134		 \\\hline
\parbox{\widthof{open}}{stat} (d=4,len=16)	&	0.420	&	.000	&	1.277	&	.001	&	204	&	1.278	&	.006	&	204	&	1.264	&	.001	&	201		 \\\hline
\parbox{\widthof{open}}{stat} (d=6,len=24)	&	0.486	&	.000	&	1.730	&	.000	&	256	&	1.730	&	.001	&	256	&	1.718	&	.000	&	253		 \\\hline
\parbox{\widthof{open}}{stat} (d=8,len=32)	&	0.553	&	.000	&	2.247	&	.000	&	306	&	2.249	&	.001	&	307	&	2.238	&	.001	&	305		 \\\hline
\parbox{\widthof{open}}{fstat} (any length)	&	0.120	&	.000	&	0.193	&	.000	&	61	&	0.193	&	.000	&	61	&	0.187	&	.000	&	56		 \\\hline
																								
\parbox{\widthof{write}}{read} (256bytes)	&	0.207	&	.072	&	0.252	&	.000	&	22	&	0.255	&	.000	&	23	&	0.342	&	.000	&	65		 \\\hline
\parbox{\widthof{write}}{read} (1KB)	&	0.227	&	.072	&	0.435	&	.000	&	92	&	0.434	&	.000	&	91	&	0.805	&	.001	&	255		 \\\hline
\parbox{\widthof{write}}{read} (4KB)	&	0.315	&	.072	&	1.163	&	.001	&	269	&	1.190	&	.000	&	278	&	2.744	&	.006	&	771		 \\\hline
\parbox{\widthof{write}}{read} (16KB)	&	1.022	&	.072	&	4.174	&	.009	&	308	&	4.291	&	.010	&	320	&	10.518	&	.022	&	929		 \\\hline
\parbox{\widthof{write}}{read} (64KB)	&	3.931	&	.072	&	9.504	&	.002	&	142	&	9.790	&	.004	&	149	&	14.619	&	.024	&	272		 \\\hline
\parbox{\widthof{write}}{write} (256bytes)	&	0.515	&	.002	&	0.285	&	.000	&	-45	&	0.287	&	.000	&	-44	&	0.490	&	.000	&	-5		 \\\hline
\parbox{\widthof{write}}{write} (1KB)	&	0.535	&	.001	&	0.575	&	.000	&	7	&	0.580	&	.000	&	8	&	1.420	&	.002	&	165		 \\\hline
\parbox{\widthof{write}}{write} (4KB)	&	0.618	&	.000	&	1.726	&	.003	&	179	&	1.767	&	.000	&	186	&	5.172	&	.006	&	737		 \\\hline
\parbox{\widthof{write}}{write} (16KB)	&	2.034	&	.000	&	6.490	&	.004	&	219	&	6.451	&	.002	&	217	&	19.128	&	.021	&	840		 \\\hline
\parbox{\widthof{write}}{write} (64KB)	&	7.614	&	.001	&	19.250	&	.040	&	153	&	19.545	&	.044	&	157	&	57.574	&	.017	&	656		 \\\hline

\hline
\hline
&\multicolumn{11}{c|}{System call throughput (operations/s), +/- Confidence Interval, \% Overhead} \\
\hline
\multicolumn{1}{|c|}{{\bf Test}} &
\multicolumn{2}{c|}{{\bf Linux \linuxversion{}}} &
\multicolumn{3}{c|}{{\bf \graphene{}}} & \multicolumn{3}{c|}{{\bf \graphene{}+SC+RM}} & \multicolumn{3}{c|}{{\bf \graphenesgx{}}} \\
&
kops/s & +/- & 
kops/s & +/- & \%O &
kops/s & +/- & \%O &
kops/s & +/- & \%O \\
\hline
create (0KB)	&	151,819	&	734	&	122,526	&	343	&	24	&	116,195	&	205	&	31	&	40,471	&	248	&	275		 \\\hline
delete (0KB)	&	247,750	&	1,048	&	133,397	&	424	&	86	&	120,683	&	138	&	105	&	37,706	&	127	&	557		 \\\hline
create (4KB)	&	154,318	&	21	&	83,880	&	201	&	84	&	73,797	&	993	&	109	&	21,989	&	37	&	602		 \\\hline
delete (4KB)	&	250,097	&	461	&	109,782	&	504	&	128	&	101,480	&	480	&	146	&	35,355	&	14	&	607		 \\\hline
create (10KB)	&	102,749	&	90	&	64,693	&	134	&	59	&	62,891	&	72	&	63	&	18,194	&	6	&	465		 \\\hline
delete (10KB)	&	186,029	&	458	&	93,833	&	232	&	98	&	89,493	&	129	&	108	&	33,368	&	94	&	458		 \\\hline
\end{tabular}
\egroup
\caption{System call benchmark results based on \lmbench{} 2.5. Comparison is among (1) native Linux processes, (2) \graphene{} \picoprocs{} on Linux host, both without and with JIT-optimized SECCOMP filter ({\bf +SC}) and reference monitor ({\bf +RM}), and (3) \graphene{} in \sgx{} enclaves.
System call latency is in microseconds, and lower is better.
System call bandwidth and throughput are in megabytes per second and operations per second, respectively, and higher is better. 
%The file system is measured in thousands operations per second, and higher is better.
Overheads are relative to Linux \linuxversion{}; negative overheads indicate improved performance.} 
\label{tab:eval:lmbench-fs}
\end{table}


%In order to understand the overheads of individual system calls,
Table~\ref{tab:eval:lmbench-syscalls} lists 
a representative sample of 
tests from \lmbench{} 2.5 benchmark suite~\cite{McVoy:lmbench}
(extended with additional experiments).
Each row reports a mean and 95\% confidence interval,
assuming the benchmark results are normally distributed;
to improve the precision,
the number of iterations in each test is increased to at least a thousand times, which effectively lower the variance
in most tests.
Although assuming a normal distribution may not be realistic for most benchmark results,
the error is likely to be marginal with the very low variance
observed in the tests.
%we use the default number of iterations for each test case.
%We have added code to \lmbench{} to also calculate 95\% confidence intervals 
%within a run~\footnote{The lmbench authors deliberately exclude variation statistics
%because most methods assume a known distribution, generally a normal distribution---an 
%assumption which is often not the case for a computer microbenchmark~\cite{staelin05lmbench}.
%Though confidence intervals should be taken with a grain of salt, 
%we include them because they clearly indicate that these experiments have very low variance. In 
%a few cases of minor performance improvement, one can assess the impact of noise.}.
Besides, to measure the marginal cost of the \seccomp{} filter and reference monitor on a Linux host,
the experiments include the cases both with
and without the \seccomp{} filter and reference monitor.


The evaluation results categorize
the system calls emulated inside \thelibos{}
as two types.
The first type of system calls is completely serviced inside \thelibos{};
the evaluation results show that these system calls are even faster than native, because \thelibos{} does not switch context into the kernal space
to service the system calls.
For instance,
a null system call (i.e., \syscall{getppid}) and installing a signal handler with \syscall{sigaction}
are up to three time as fast as the native performance.
The second type of system call requires translation to a native host system calls.
%For instance, 
%the self-signaling test (sig overhead)
%just calls the signal handler as a function,
%which is almost twice as fast
%as the Linux kernel implementation.



\papersubsection{Network sockets and pipes}
\label{eval:perf:streams}


\begin{table}[t!b!]
\footnotesize
\centering
\bgroup
\def\arraystretch{1.1}
\setlength{\tabcolsep}{.5em}
\begin{tabular}{|ll|>{\palign{r}}p{4em}r|>{\palign{r}}p{4em}rr|>{\palign{r}}p{4em}rr|>{\palign{r}}p{4em}rr|}
\hline
& & \multicolumn{11}{c|}{System call latency (\usec{}), +/- Confidence Interval, \% Overhead} \\
\hline
\multicolumn{2}{|c|}{{\bf Test}} &
\multicolumn{2}{c|}{{\bf Linux \linuxversion{}}} &
\multicolumn{3}{c|}{{\bf \graphene{}}} & \multicolumn{3}{c|}{{\bf \graphene{}+SC+RM}} & \multicolumn{3}{c|}{{\bf \graphenesgx{}}} \\
& &
\usec{} & +/- & 
\usec{} & +/- & \%O &
\usec{} & +/- & \%O &
\usec{} & +/- & \%O \\
\hline

pipe	&		&	2.412	&	.290	&	4.440	&	.464	&	84	&	4.491	&	.138	&	86	&	13.364	&	.000	&	454		 \\\hline
UNIX	&	socket	&	4.386	&	.293	&	5.587	&	.048	&	27	&	5.824	&	.044	&	33	&	14.431	&	.000	&	229		 \\\hline
UDP	&	socket	&	6.300	&	.708	&	9.451	&	.235	&	50	&	9.938	&	.219	&	58	&	17.538	&	.703	&	178		 \\\hline
TCP	&	socket	&	7.217	&	.505	&	9.422	&	.203	&	31	&	10.075	&	.209	&	40	&	17.925	&	.001	&	148		 \\\hline
	
\hline			
\hline
& & \multicolumn{11}{c|}{System call bandwidth (MB/s), +/- Confidence Interval, \% Overhead} \\
\hline
\multicolumn{2}{|c|}{{\bf Test}} &
\multicolumn{2}{c|}{{\bf Linux \linuxversion{}}} &
\multicolumn{3}{c|}{{\bf \graphene{}}} & \multicolumn{3}{c|}{{\bf \graphene{}+SC+RM}} & \multicolumn{3}{c|}{{\bf \graphenesgx{}}} \\
& &
MB/s & +/- & 
MB/s & +/- & \%O &
MB/s & +/- & \%O &
MB/s & +/- & \%O \\
\hline
							
pipe	&		&	4,791	&	295	&	12,262	&	204	&	-61	&	12,151	&	130	&	-61	&	177	&	0	&	2,613		 \\\hline
UNIX	&	socket	&	12,643	&	419	&	12,179	&	85	&	4	&	12,173	&	306	&	4	&	177	&	0	&	7,059		 \\\hline
TCP	&	socket	&	7,465	&	38	&	7,011	&	52	&	6	&	6,932	&	64	&	8	&	4,242	&	2	&	76		 \\\hline

\end{tabular}
\egroup

\caption{System call benchmark results based on \lmbench{} 2.5. Comparison is among (1) native Linux processes, (2) \graphene{} \picoprocs{} on Linux host, both without and with JIT-optimized SECCOMP filter ({\bf +SC}) and reference monitor ({\bf +RM}), and (3) \graphene{} in \sgx{} enclaves.
System call latency is in microseconds, and lower is better.
System call bandwidth and throughput are in megabytes per second and operations per second, respectively, and higher is better. 
%The file system is measured in thousands operations per second, and higher is better.
Overheads are relative to Linux \linuxversion{}; negative overheads indicate improved performance.} 
\label{tab:eval:lmbench-streams}
\end{table}


%In order to understand the overheads of individual system calls,
Table~\ref{tab:eval:lmbench-syscalls} lists 
a representative sample of 
tests from \lmbench{} 2.5 benchmark suite~\cite{McVoy:lmbench}
(extended with additional experiments).
Each row reports a mean and 95\% confidence interval,
assuming the benchmark results are normally distributed;
to improve the precision,
the number of iterations in each test is increased to at least a thousand times, which effectively lower the variance
in most tests.
Although assuming a normal distribution may not be realistic for most benchmark results,
the error is likely to be marginal with the very low variance
observed in the tests.
%we use the default number of iterations for each test case.
%We have added code to \lmbench{} to also calculate 95\% confidence intervals 
%within a run~\footnote{The lmbench authors deliberately exclude variation statistics
%because most methods assume a known distribution, generally a normal distribution---an 
%assumption which is often not the case for a computer microbenchmark~\cite{staelin05lmbench}.
%Though confidence intervals should be taken with a grain of salt, 
%we include them because they clearly indicate that these experiments have very low variance. In 
%a few cases of minor performance improvement, one can assess the impact of noise.}.
Besides, to measure the marginal cost of the \seccomp{} filter and reference monitor on a Linux host,
the experiments include the cases both with
and without the \seccomp{} filter and reference monitor.


The evaluation results categorize
the system calls emulated inside \thelibos{}
as two types.
The first type of system calls is completely serviced inside \thelibos{};
the evaluation results show that these system calls are even faster than native, because \thelibos{} does not switch context into the kernal space
to service the system calls.
For instance,
a null system call (i.e., \syscall{getppid}) and installing a signal handler with \syscall{sigaction}
are up to three time as fast as the native performance.
The second type of system call requires translation to a native host system calls.
%For instance, 
%the self-signaling test (sig overhead)
%just calls the signal handler as a function,
%which is almost twice as fast
%as the Linux kernel implementation.