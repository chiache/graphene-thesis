\papersubsection{Network sockets and pipes}
\label{sec:eval:libos:streams}


\begin{table}[t!b!]
\footnotesize
\centering
\bgroup
\def\arraystretch{1.1}
\setlength{\tabcolsep}{.5em}
\begin{tabular}{|ll|>{\palign{r}}p{4em}r|>{\palign{r}}p{4em}rr|>{\palign{r}}p{4em}rr|>{\palign{r}}p{4em}rr|}
\hline
& & \multicolumn{11}{c|}{System call latency (\usec{}), +/- Confidence Interval, \% Overhead} \\
\hline
\multicolumn{2}{|c|}{{\bf Test}} &
\multicolumn{2}{c|}{{\bf Linux \linuxversion{}}} &
\multicolumn{3}{c|}{{\bf \graphene{}}} & \multicolumn{3}{c|}{{\bf \graphene{}+SC+RM}} & \multicolumn{3}{c|}{{\bf \graphenesgx{}}} \\
& &
\usec{} & +/- & 
\usec{} & +/- & \%O &
\usec{} & +/- & \%O &
\usec{} & +/- & \%O \\
\hline

pipe	&		&	2.412	&	.290	&	4.440	&	.464	&	84	&	4.491	&	.138	&	86	&	13.364	&	.000	&	454		 \\\hline
UNIX	&	socket	&	4.386	&	.293	&	5.587	&	.048	&	27	&	5.824	&	.044	&	33	&	14.431	&	.000	&	229		 \\\hline
UDP	&	socket	&	6.300	&	.708	&	9.451	&	.235	&	50	&	9.938	&	.219	&	58	&	17.538	&	.703	&	178		 \\\hline
TCP	&	socket	&	7.217	&	.505	&	9.422	&	.203	&	31	&	10.075	&	.209	&	40	&	17.925	&	.001	&	148		 \\\hline
	
\hline			
\hline
& & \multicolumn{11}{c|}{System call bandwidth (MB/s), +/- Confidence Interval, \% Overhead} \\
\hline
\multicolumn{2}{|c|}{{\bf Test}} &
\multicolumn{2}{c|}{{\bf Linux \linuxversion{}}} &
\multicolumn{3}{c|}{{\bf \graphene{}}} & \multicolumn{3}{c|}{{\bf \graphene{}+SC+RM}} & \multicolumn{3}{c|}{{\bf \graphenesgx{}}} \\
& &
MB/s & +/- & 
MB/s & +/- & \%O &
MB/s & +/- & \%O &
MB/s & +/- & \%O \\
\hline
							
pipe	&		&	4,791	&	295	&	12,262	&	204	&	-61	&	12,151	&	130	&	-61	&	177	&	0	&	2,613		 \\\hline
UNIX	&	socket	&	12,643	&	419	&	12,179	&	85	&	4	&	12,173	&	306	&	4	&	177	&	0	&	7,059		 \\\hline
TCP	&	socket	&	7,465	&	38	&	7,011	&	52	&	6	&	6,932	&	64	&	8	&	4,242	&	2	&	76		 \\\hline

\end{tabular}
\egroup

\caption{Network socket and pipe performance based on \lmbench{}. Comparison is among (1) native Linux processes; (2) \graphene{} on Linux host, both without and with \seccomp{} filter ({\bf +SC}) and reference monitor ({\bf +RM}); (3) \graphenesgx{}.
System call latency is in microseconds, and lower is better.
System call bandwidth is in megabytes per second, and higher is better. 
%The file system is measured in thousands operations per second, and higher is better.
Overheads are relative to Linux; negative overheads indicate improvement.} 
\label{tab:eval:libos:lmbench-streams}
\end{table}


%Figure~\ref{tab:eval:libos:lmbench-streams}
%lists the latency and bandwidth of pipes, UNIX domain sockets, UDP sockets, and TCP sockets.
%Based on the latency of transmitting single-byte messages, 
%\graphene{} causes at most 30--85\% overheads
%on all types of I/O streams.
%pipes and UNIX domain sockets have similar latency and throughput since \thelibos{} uses the same host abstractions (i.e., RPC streams) to emulate both.
\Thehostabi{} performance
largely determine
the performance of network socket and pipe emulation
in \thelibos{}.
In general, \thelibos{} does not implement a network or pipe stack,
neither does \thelibos{}
buffer data read or written to a network socket or a pipe.
Table~\ref{tab:eval:libos:lmbench-streams} shows similar benchmark results as
unbuffered file reads and writes discussed in the previous section;
the overheads on a network socket or a pipe are up to 33--86\% on the Linux host,
or 148--454\% from an enclave,
when measuring the latency of sending single-byte messages
over the abstractions.
More overheads
are observed on pipes than UNIX domain sockets because of both PAL implementations
evaluated in this section have chosen
UNIX domain sockets
to implement the RPC streams underlying \thelibos{}.
 
