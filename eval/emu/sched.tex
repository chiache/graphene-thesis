\papersubsection{Threading}
\label{sec:eval:libos:threading}


\begin{table}[t!b!]
\footnotesize
\centering
\bgroup
\def\arraystretch{1.1}
\setlength{\tabcolsep}{.3em}
\begin{tabular}{|ll|>{\palign{r}}p{3em}r|>{\palign{r}}p{3em}rr|>{\palign{r}}p{3em}rr|>{\palign{r}}p{3em}rr|}
\hline
& & \multicolumn{11}{c|}{System call latency (\usec{}), +/- Confidence Interval, \%/\x{} Overhead} \\
\hline
\multicolumn{2}{|c|}{{\bf Test}} &
\multicolumn{2}{c|}{{\bf Linux \linuxversion{}}} &
\multicolumn{3}{c|}{{\bf \graphene{}}} & \multicolumn{3}{c|}{{\bf \graphene{}+SC+RM}} & \multicolumn{3}{c|}{{\bf \graphenesgx{}}} \\
& &
\usec{} & +/- & 
\usec{} & +/- & O &
\usec{} & +/- & O &
\usec{} & +/- & O \\
\hline

\multicolumn{2}{|l|}{{\tt clone}+{\tt exit}}			&	10.486	&	.316	&	19.218	&	.456	&	83	\% &	19.325	&	.463	&	84	\% &	120.297	&	18.987	&	10	$\times$	 \\\hline
\multicolumn{2}{|l|}{{\tt pthread\_create}+{\tt join}}			&	8.456	&	.138	&	23.622	&	1.386	&	179	\% &	23.884	&	1.841	&	182	\% &	168.981	&	26.777	&	19	$\times$	 \\\hline
\hline																																			
select files 	&	(\hspace{.5em}64 fds)	&	0.742	&	.005	&	144	&	12	&	193	\x{} &	149	&	11	&	200	\x{} &	461	&	0	&	620	\x{}	 \\\hline
select files 	&	(128 fds)	&	1.276	&	.001	&	427	&	0	&	333	\x{} &	440	&	0	&	344	\x{} &	1,119	&	0	&	876	\x{}	 \\\hline
select files 	&	(256 fds)	&	2.287	&	.001	&	1,486	&	5	&	649	\x{} &	1,539	&	0	&	672	\x{} &	2,977	&	1	&	1,301	\x{}	 \\\hline
%select files 	&	(512 fds)	&	4.381	&	.002	&	5,706	&	11	&	1,301	\x{} &	5,799	&	0	&	1,323	\x{} &	8,799	&	31	&	2,007	\x{}	 \\\hline
select TCP	&	(\hspace{.5em}64 fds)	&	1.975	&	.001	&	45	&	0	&	22	\x{} &	51	&	0	&	25	\x{} &	371	&	17	&	187	\x{}	 \\\hline
select TCP	&	(128 fds)	&	4.609	&	.001	&	104	&	0	&	22	\x{} &	117	&	0	&	24	\x{} &	855	&	32	&	185	\x{}	 \\\hline
select TCP	&	(256 fds)	&	11.050	&	.005	&	272	&	8	&	24	\x{} &	288	&	2	&	25	\x{} &	1,518	&	10	&	136	\x{}	 \\\hline
%select TCP	&	(512 fds)	&	23.463	&	.005	&	794	&	42	&	33	\x{} &	850	&	43	&	35	\x{} &	3,413	&	140	&	144	\x{}	 \\\hline
\hline																										
\multicolumn{2}{|l|}{pthread\_cond/mutex}			&	4.244	&	.644	&	6.384	&	.354	&	50	\% &	6.711	&	.000	&	58	\% &	11.981	&	.062	&	182	\%	 \\\hline
pthread\_mutex	&	(1 thread)	&	0.019	&	.000	&	0.019	&	.000	&	0	\% &	0.019	&	.000	&	0	\% &	0.019	&	.000	&	0	\%	 \\\hline
pthread\_mutex	&	(2 threads)	&	0.167	&	.008	&	0.154	&	.004	&	-8	\% &	0.156	&	.005	&	-7	\% &	0.159	&	.003	&	-5	\%	 \\\hline
pthread\_mutex	&	(4 threads)	&	0.386	&	.011	&	0.316	&	.009	&	-18	\% &	0.316	&	.004	&	-18	\% &	0.353	&	.005	&	-9	\%	 \\\hline
pthread\_mutex	&	(8 threads)	&	0.793	&	.016	&	0.758	&	.031	&	-4	\% &	0.802	&	.042	&	1	\% &	2.203	&	.501	&	178	\%	 \\\hline

\end{tabular}
\egroup

\caption{Performance of threading and scheduling operations, including cloning a thread, polling file descriptors, and synchronization primitives (signaling a conditional variable, acquiring a mutex). The comparison is among (1) native Linux processes; (2) \graphene{} on Linux host, both without and with \seccomp{} filter ({\bf +SC}) and reference monitor ({\bf +RM}); (3) \graphenesgx{}.
System call latency is in microseconds, and lower is better.
%The file system is measured in thousands operations per second, and higher is better.
Overheads are relative to Linux; negative overheads indicate improvement.} 
\label{tab:eval:libos:lmbench-threading}
\end{table}


Thread creation in \thelibos{}
is subject to additional overheads including allocating thread control blocks (TCB)
and synchronization between the caller of \syscall{clone} and new thread.
To emulate a simple \syscall{clone} system call, the overhead on the Linux host is \roughly{}84\%.
Within the overhead, 50\% is contributed
by the \hostapi{}, \palcall{ThreadCreate}, and the rest counts toward
TCB allocation and synchronization.
If \thelibos{} runs inside an enclave,
the overhead on \syscall{clone} is much more significant, reaching \roughly{}10\x{}.
Most of the cost on the \sgx{} host
is caused by synchronization since the overhead on \palcall{ThreadCreate} is only 2\x{}.
If comparing the latency of \syscall{pthread\_create}+\syscall{pthread\_join}, the overheads mostly double on both the Linux host and the \sgx{} host,
primarily because of extra synchronization cost
through emulated futexes
used by the pthread library for exit notifications.



Polling file descriptors using \syscall{select} is one of the most expensive system calls
emulated by \thelibos{}.
For instance, to poll 256 file descriptors,
Table~\ref{tab:eval:libos:lmbench-threading} shows up to 672\x{} overheads on the Linux host, or 1,301\x{} inside an enclave.
The reason for the high latency
is due to the definition of the \hostapi{}, \palcall{ObjectsWaitAny},
which only returns one handle at a time.
Therefore, to poll 256 file descriptors,
\thelibos{} has to call \palcall{ObjectsWaitAny} up to 256 times, to confirm
the readiness of all polled file descriptors.
A possible optimization
is to redefining the \hostapi{} to allow returning multiple handles,
which remains as future work.


The synchronization primitives offered by the pthread library,
including conditional variables and mutexes, 
are largely affected by the performance of futexes emulated by \thelibos{},
especially when one thread has to block for the others.
As one example,
the latency of signaling a blocking thread
through a condition variable
is 50\% slower on the Linux host, or 182\% slower in an enclave,
compared to the native performance.





