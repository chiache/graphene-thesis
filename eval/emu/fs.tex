\papersubsection{File systems}
\label{sec:eval:libos:fs}


\begin{table}[t!b!]
\footnotesize
\centering
\bgroup
\def\arraystretch{1.1}
\setlength{\tabcolsep}{0.4em}
\begin{tabular}{|ll|>{\palign{r}}p{3.5em}r|>{\palign{r}}p{3.5em}rr|>{\palign{r}}p{3.5em}rr|>{\palign{r}}p{3.5em}rr|}
\hline
& & \multicolumn{11}{c|}{System call latency (\usec{}), +/- Confidence Interval, \% Overhead} \\
\hline
\multicolumn{2}{|c|}{{\bf Test}} &
\multicolumn{2}{c|}{{\bf Linux \linuxversion{}}} &
\multicolumn{3}{c|}{{\bf \graphene{}}} & \multicolumn{3}{c|}{{\bf \graphene{}+SC+RM}} & \multicolumn{3}{c|}{{\bf \graphenesgx{}}} \\
& &
\usec{} & +/- & 
\usec{} & +/- & \%O &
\usec{} & +/- & \%O &
\usec{} & +/- & \%O \\
\hline

{\tt open}	&	(d=2,len=\hspace{.5em}8)	&	0.947	&	.072	&	2.337	&	.013	&	147	&	2.719	&	.007	&	187	&	16.600	&	.007	&	1,653		 \\\hline
{\tt open}	&	(d=4,len=16)	&	1.011	&	.074	&	2.627	&	.009	&	160	&	2.922	&	.009	&	189	&	17.168	&	.016	&	1,598		 \\\hline
%{\tt open}	&	(d=6,len=24)	&	1.074	&	.068	&	2.719	&	.008	&	153	&	3.131	&	.007	&	192	&	17.527	&	.016	&	1,532		 \\\hline
{\tt open}	&	(d=8,len=32)	&	1.131	&	.074	&	3.360	&	.000	&	197	&	3.812	&	.007	&	237	&	18.415	&	.016	&	1,528		 \\\hline
\hline																										
{\tt stat}	&	(d=2,len=\hspace{.5em}8)	&	0.361	&	.000	&	0.502	&	.000	&	39	&	0.499	&	.000	&	38	&	0.487	&	.000	&	35		 \\\hline
{\tt stat}	&	(d=4,len=16)	&	0.420	&	.000	&	0.585	&	.000	&	39	&	0.584	&	.000	&	39	&	0.571	&	.001	&	36		 \\\hline
%{\tt stat}	&	(d=6,len=24)	&	0.486	&	.000	&	0.685	&	.000	&	41	&	0.685	&	.000	&	41	&	0.671	&	.000	&	38		 \\\hline
{\tt stat}	&	(d=8,len=32)	&	0.553	&	.000	&	0.780	&	.000	&	41	&	0.780	&	.000	&	41	&	0.767	&	.000	&	39		 \\\hline
\hline																										
{\tt fstat} 	&	(any length)	&	0.120	&	.000	&	0.193	&	.000	&	61	&	0.193	&	.000	&	61	&	0.187	&	.000	&	56		 \\\hline
\hline																										
\hline																										
%{\tt read} 	&	(0.25KB)	&	0.207	&	.072	&	0.252	&	.000	&	22	&	0.255	&	.000	&	23	&	0.342	&	.000	&	65		 \\\hline
{\tt read} 	&	(\hspace{.5em}1KB)	&	0.227	&	.072	&	0.435	&	.000	&	92	&	0.434	&	.000	&	91	&	0.805	&	.001	&	255		 \\\hline
{\tt read} 	&	(\hspace{.5em}4KB)	&	0.315	&	.072	&	0.545	&	.001	&	73	&	0.607	&	.000	&	93	&	9.545	&	.006	&	2,930		 \\\hline
{\tt read} 	&	(16KB)	&	1.022	&	.072	&	1.308	&	.000	&	28	&	1.356	&	.000	&	33	&	11.437	&	.022	&	1,019		 \\\hline
%{\tt read} 	&	(64KB)	&	3.931	&	.072	&	4.190	&	.001	&	7	&	4.189	&	.001	&	7	&	17.071	&	.004	&	334		 \\\hline
\hline																										
%{\tt write} 	&	(0.25KB)	&	0.515	&	.002	&	0.285	&	.000	&	-45	&	0.287	&	.000	&	-44	&	0.490	&	.000	&	-5		 \\\hline
{\tt write} 	&	(\hspace{.5em}1KB)	&	0.535	&	.001	&	0.575	&	.000	&	7	&	0.580	&	.000	&	8	&	1.420	&	.002	&	165		 \\\hline
{\tt write} 	&	(\hspace{.5em}4KB)	&	0.618	&	.000	&	0.856	&	.002	&	39	&	0.909	&	.002	&	47	&	9.784	&	.006	&	1,483		 \\\hline
{\tt write} 	&	(16KB)	&	2.034	&	.000	&	2.303	&	.013	&	13	&	2.356	&	.001	&	16	&	19.730	&	.021	&	870		 \\\hline
%{\tt write} 	&	(64KB)	&	7.614	&	.001	&	7.929	&	.020	&	4	&	7.971	&	.002	&	5	&	59.899	&	.017	&	687		 \\\hline

\hline
\hline
& & \multicolumn{11}{c|}{System call throughput (operations/s), +/- Confidence Interval, \% Overhead} \\
\hline
\multicolumn{2}{|c|}{{\bf Test}} &
\multicolumn{2}{c|}{{\bf Linux \linuxversion{}}} &
\multicolumn{3}{c|}{{\bf \graphene{}}} & \multicolumn{3}{c|}{{\bf \graphene{}+SC+RM}} & \multicolumn{3}{c|}{{\bf \graphenesgx{}}} \\
& &
ops/s & +/- & 
ops/s & +/- & \%O &
ops/s & +/- & \%O &
ops/s & +/- & \%O \\
\hline
create	&	(\hspace{.5em}0KB)	&	151,819	&	734	&	122,526	&	343	&	24	&	116,195	&	205	&	31	&	40,471	&	248	&	275		 \\\hline
delete	&	(\hspace{.5em}0KB)	&	247,750	&	1,048	&	133,397	&	424	&	86	&	120,683	&	138	&	105	&	37,706	&	127	&	557		 \\\hline
create	&	(\hspace{.5em}4KB)	&	154,318	&	21	&	83,880	&	201	&	84	&	73,797	&	993	&	109	&	21,989	&	37	&	602		 \\\hline
delete	&	(\hspace{.5em}4KB)	&	250,097	&	461	&	109,782	&	504	&	128	&	101,480	&	480	&	146	&	35,355	&	14	&	607		 \\\hline
create	&	(10KB)	&	102,749	&	90	&	64,693	&	134	&	59	&	62,891	&	72	&	63	&	18,194	&	6	&	465		 \\\hline
delete	&	(10KB)	&	186,029	&	458	&	93,833	&	232	&	98	&	89,493	&	129	&	108	&	33,368	&	94	&	458		 \\\hline
\end{tabular}
\egroup
\caption{File-related system call performance based on \lmbench{}. 
Comparison is among (1) native Linux processes; (2) \graphene{} on Linux host, both without and with \seccomp{} filter ({\bf +SC}) and reference monitor ({\bf +RM}); (3) \graphenesgx{}.
System call latency is in microseconds, and lower is better.
System call throughput is in operations per second, and higher is better. 
Overheads are relative to Linux; negative overheads indicate improvement.} 
\label{tab:eval:libos:lmbench-fs}
\end{table}



Figure~\ref{tab:eval:libos:lmbench-fs}
lists the latency or throughput of system calls
for accessing an isolated host file system mounted in a \thelibos{} instance,
or a {\bf chroot} file system.
Each system call in a chroot file system
accesses a file or a directory in the host file system,
and therefore requires
translation to one or multiple
host system calls.
As a result, the system call latency
is determined by the underlying \hostapi{} latency and the translation cost inside \thelibos{}.
%Besides, as previously stated, \thelibos{}
%can optimize system calls such as \syscall{read} and \syscall{write}
%by buffering read or written data.



System calls like \syscall{open} and \syscall{stat}
%access a specific path
%in the file system.
%The performance of this type of system calls
are sensitive to lengths and depths (i.e., numbers of components) in the requested paths.
For optimization,
\thelibos{} implements a file system directory cache
to store path information and file attributes retrieved from the host OS.
Because the \lmbench{} tests %for \syscall{stat} and \syscall{open}
access the same path repeatedly,
the directory cache
is guaranteed to optimize every system calls measured.
As a result,
\syscall{stat} in both \graphene{} and \graphenesgx{} is only 35--41\% slower than native
and mostly irrelevant from the host system call latency. 
\syscall{fstat} also benefits from directory caching
(35--41\% overheads).
%Different from \syscall{stat},
For \syscall{open}, %despite the optimization of directory caching,
\graphene{} imposes
extra overheads for opening PAL handles and allocating file descriptors in \thelibos{}.
To access a path with 2--8 components,
the overheads on \syscall{open} are 147--197\% for \graphene{} on Linux host, and 187--237\% with \seccomp{} filter and reference monitor.
For \graphenesgx{}, the overheads are 15.2--16.5$\times$
without considering the checksum calculation costs.


For \syscall{read} and \syscall{write},
the latency in \graphene{} depends on the buffering strategy in \thelibos{}.
The experiments
are based on a strategy which
buffers reads and writes smaller than 4KB (not including 4KB)
using a 16KB buffer directly mapped from the file.
In Figure~\ref{tab:eval:libos:lmbench-fs}, buffered reads and writes (256 bytes and 1KB) on Linux host
are 22--92\% and -45--8\% slower than native, respectively.
For unbuffered reads and writes (4KB and 16KB),
the latency in \graphene{} and \graphenesgx{}
is closer to native with larger reading and writing sizes.
Unbuffered reads and writes
on \graphenesgx{}
have significant overheads around 10--29$\times$, primarily for copying file contents
between enclave and untrusted memory.

\lmbench{} also tests the throughput of creating and deleting a large amount of files,
measured in operations per second.
For both \graphene{} and \graphenesgx{}, deletion has higher overheads than creation; \graphenesgx{} also imposes much more significant overheads than \graphene{}, at 2.7--6$\times$.


