\papersubsection{File systems}
\label{sec:eval:libos:fs}

File system performance in \thelibos{}
is subject to several optimizations including directory caching and buffering.
To reduce the impact of \hostapi{} latency,
a key to the chroot file system implementation
is to lower the average number of expensive \hostapis{}
needed for emulating each system call.
For instance, the directory cache
in \thelibos{} stores path existence and metadata
in spare \picoproc{} memory,
to skip the redundant cost
of querying the host file system
when accessing the same path in the future.
Table~\ref{tab:eval:libos:lmbench-fs} lists the latency of \syscall{open} and \syscall{stat}
for repeatedly opening or querying the same path.
Directory caching
reduces the overheads on \syscall{stat} to 35--41\%
regardless of the hosts.
The latency of \syscall{open}
also benefits from directory caching, but the cost of opening the file in the host OSes
overshadows the optimization,
causing 187--237\% overheads on the Linux host with the \seccomp{} filter and reference monitor,
or 15.2--16.5\x{} in an enclave.

%can be categorized as two types.
%One type of overheads is the costs of directory caching,
%for storing file system hierarchy and metadata
%in \picoproc{} memory to avoid redundant storage lookup.
%Directory caching also
%helps identifying resources among multiple
%chroot'ed file systems
%mounted from host directories
%and pseudo file systems such as \code{/proc} and \code{/dev}.
%\thelibos{}
%adopt a similar design as the directory cache
%in a Linux kernel,
%with an optimization for searching long paths
%in a warm cache~\cite{tsai15dcache}. 


%Figure~\ref{tab:eval:libos:lmbench-fs}
%lists the latency or throughput of system calls
%for accessing an isolated host file system mounted in a \thelibos{} instance,
%or a {\bf chroot} file system.
%Each system call in a chroot file system
%accesses a file or a directory in the host file system,
%and therefore requires
%translation to one or multiple
%host system calls.
%As a result, the system call latency
%is determined by the underlying \hostapi{} latency and the translation cost inside \thelibos{}.
%%Besides, as previously stated, \thelibos{}
%%can optimize system calls such as \syscall{read} and \syscall{write}
%%by buffering read or written data.



%System calls like \syscall{open} and \syscall{stat}
%%access a specific path
%%in the file system.
%%The performance of this type of system calls
%are sensitive to path lengths and depths (i.e., numbers of components).
%As an optimization,
%\thelibos{} implements a file system directory cache
%to store path information and file attributes retrieved from the host OS.
%Because the \lmbench{} tests %for \syscall{stat} and \syscall{open}
%access the same path repeatedly,
%the directory cache
%is guaranteed to optimize every system calls measured.
%As a result,
%\syscall{stat} in both \graphene{} and \graphenesgx{} is only 35--41\% slower than native
%and mostly irrelevant from the host system call latency. 
%\syscall{fstat} also benefits from directory caching
%(35--41\% overheads).
%%Different from \syscall{stat},
%For \syscall{open}, %despite the optimization of directory caching,
%\graphene{} imposes
%extra overheads for opening PAL handles and allocating file descriptors in \thelibos{}.
%To access a path with 2--8 components,
%the overheads on \syscall{open} are 147--197\% for \graphene{} on Linux host, and 187--237\% with \seccomp{} filter and reference monitor.
%For \graphenesgx{}, the overheads are 15.2--16.5\x{}
%without considering the checksum calculation costs.


For \syscall{read} and \syscall{write},
the latency in \graphene{} depends on the buffering strategy in \thelibos{}.
The experiments
are based on a strategy which
buffers reads and writes smaller than 4KB (not including 4KB)
using a 16KB buffer directly mapped from the file.
In Table~\ref{tab:eval:libos:lmbench-fs}, buffered reads and writes (256 bytes and 1KB) on Linux host
are 22--92\% and -45--8\% slower than native, respectively.
The latency of unbuffered reads and writes (4KB and 16KB),
is closer to native
when running on a Linux host,
but suffers significant overheads (10--29\x{}) in an enclave due to copying file contents.



Table~\ref{tab:eval:libos:lmbench-fs} also lists the throughputs of creating and deleting a large number of files,
measured in operations per second.
Among all the benchmark results, deletion throughputs tend to have much higher overheads than creation throughputs.
Compared to running on the Linux host, 
both file creation and deletion from an enclave suffer significantly higher overheads
(2.7--6\x{}).

\clearpage
\begin{table}[p]
\footnotesize
\centering
\bgroup
\def\arraystretch{1.1}
\setlength{\tabcolsep}{0.4em}
\begin{tabular}{|l|>{\palign{r}}p{4em}r|>{\palign{r}}p{4em}rr|>{\palign{r}}p{4em}rr|>{\palign{r}}p{4em}rr|}
\hline
&\multicolumn{11}{c|}{System call latency (\usec{}), +/- Confidence Interval, \% Overhead} \\
\hline
\multicolumn{1}{|c|}{{\bf Test}} &
\multicolumn{2}{c|}{{\bf Linux \linuxversion{}}} &
\multicolumn{3}{c|}{{\bf \graphene{}}} & \multicolumn{3}{c|}{{\bf \graphene{}+SC+RM}} & \multicolumn{3}{c|}{{\bf \graphenesgx{}}} \\
&
\usec{} & +/- & 
\usec{} & +/- & \%O &
\usec{} & +/- & \%O &
\usec{} & +/- & \%O \\
\hline

\parbox{\widthof{open}}{open} (d=2,len=08)	&	0.947	&	.072	&	2.474	&	.007	&	161	&	2.892	&	.007	&	205	&	16.676	&	.018	&	1,661		 \\\hline
\parbox{\widthof{open}}{open} (d=4,len=16)	&	1.011	&	.074	&	2.895	&	.034	&	186	&	3.241	&	.007	&	221	&	17.456	&	.017	&	1,627		 \\\hline
\parbox{\widthof{open}}{open} (d=6,len=24)	&	1.074	&	.068	&	3.363	&	.070	&	213	&	3.708	&	.007	&	245	&	18.084	&	.018	&	1,584		 \\\hline
\parbox{\widthof{open}}{open} (d=8,len=32)	&	1.131	&	.074	&	4.259	&	.007	&	277	&	4.601	&	.007	&	307	&	19.196	&	.020	&	1,597		 \\\hline
\parbox{\widthof{open}}{stat} (d=2,len=08)	&	0.361	&	.000	&	0.904	&	.000	&	150	&	0.857	&	.000	&	137	&	0.846	&	.000	&	134		 \\\hline
\parbox{\widthof{open}}{stat} (d=4,len=16)	&	0.420	&	.000	&	1.277	&	.001	&	204	&	1.278	&	.006	&	204	&	1.264	&	.001	&	201		 \\\hline
\parbox{\widthof{open}}{stat} (d=6,len=24)	&	0.486	&	.000	&	1.730	&	.000	&	256	&	1.730	&	.001	&	256	&	1.718	&	.000	&	253		 \\\hline
\parbox{\widthof{open}}{stat} (d=8,len=32)	&	0.553	&	.000	&	2.247	&	.000	&	306	&	2.249	&	.001	&	307	&	2.238	&	.001	&	305		 \\\hline
\parbox{\widthof{open}}{fstat} (any length)	&	0.120	&	.000	&	0.193	&	.000	&	61	&	0.193	&	.000	&	61	&	0.187	&	.000	&	56		 \\\hline
																								
\parbox{\widthof{write}}{read} (256bytes)	&	0.207	&	.072	&	0.252	&	.000	&	22	&	0.255	&	.000	&	23	&	0.342	&	.000	&	65		 \\\hline
\parbox{\widthof{write}}{read} (1KB)	&	0.227	&	.072	&	0.435	&	.000	&	92	&	0.434	&	.000	&	91	&	0.805	&	.001	&	255		 \\\hline
\parbox{\widthof{write}}{read} (4KB)	&	0.315	&	.072	&	1.163	&	.001	&	269	&	1.190	&	.000	&	278	&	2.744	&	.006	&	771		 \\\hline
\parbox{\widthof{write}}{read} (16KB)	&	1.022	&	.072	&	4.174	&	.009	&	308	&	4.291	&	.010	&	320	&	10.518	&	.022	&	929		 \\\hline
\parbox{\widthof{write}}{read} (64KB)	&	3.931	&	.072	&	9.504	&	.002	&	142	&	9.790	&	.004	&	149	&	14.619	&	.024	&	272		 \\\hline
\parbox{\widthof{write}}{write} (256bytes)	&	0.515	&	.002	&	0.285	&	.000	&	-45	&	0.287	&	.000	&	-44	&	0.490	&	.000	&	-5		 \\\hline
\parbox{\widthof{write}}{write} (1KB)	&	0.535	&	.001	&	0.575	&	.000	&	7	&	0.580	&	.000	&	8	&	1.420	&	.002	&	165		 \\\hline
\parbox{\widthof{write}}{write} (4KB)	&	0.618	&	.000	&	1.726	&	.003	&	179	&	1.767	&	.000	&	186	&	5.172	&	.006	&	737		 \\\hline
\parbox{\widthof{write}}{write} (16KB)	&	2.034	&	.000	&	6.490	&	.004	&	219	&	6.451	&	.002	&	217	&	19.128	&	.021	&	840		 \\\hline
\parbox{\widthof{write}}{write} (64KB)	&	7.614	&	.001	&	19.250	&	.040	&	153	&	19.545	&	.044	&	157	&	57.574	&	.017	&	656		 \\\hline

\hline
\hline
&\multicolumn{11}{c|}{System call throughput (operations/s), +/- Confidence Interval, \% Overhead} \\
\hline
\multicolumn{1}{|c|}{{\bf Test}} &
\multicolumn{2}{c|}{{\bf Linux \linuxversion{}}} &
\multicolumn{3}{c|}{{\bf \graphene{}}} & \multicolumn{3}{c|}{{\bf \graphene{}+SC+RM}} & \multicolumn{3}{c|}{{\bf \graphenesgx{}}} \\
&
kops/s & +/- & 
kops/s & +/- & \%O &
kops/s & +/- & \%O &
kops/s & +/- & \%O \\
\hline
create (0KB)	&	151,819	&	734	&	122,526	&	343	&	24	&	116,195	&	205	&	31	&	40,471	&	248	&	275		 \\\hline
delete (0KB)	&	247,750	&	1,048	&	133,397	&	424	&	86	&	120,683	&	138	&	105	&	37,706	&	127	&	557		 \\\hline
create (4KB)	&	154,318	&	21	&	83,880	&	201	&	84	&	73,797	&	993	&	109	&	21,989	&	37	&	602		 \\\hline
delete (4KB)	&	250,097	&	461	&	109,782	&	504	&	128	&	101,480	&	480	&	146	&	35,355	&	14	&	607		 \\\hline
create (10KB)	&	102,749	&	90	&	64,693	&	134	&	59	&	62,891	&	72	&	63	&	18,194	&	6	&	465		 \\\hline
delete (10KB)	&	186,029	&	458	&	93,833	&	232	&	98	&	89,493	&	129	&	108	&	33,368	&	94	&	458		 \\\hline
\end{tabular}
\egroup
\caption{File-related system call performance based on \lmbench{}. 
The comparison is among (1) native Linux processes; (2) \graphene{} on Linux host, both without and with \seccomp{} filter ({\bf +SC}) and reference monitor ({\bf +RM}); (3) \graphenesgx{}.
System call latency is in microseconds, and lower is better.
System call throughput is in operations per second, and higher is better. 
Overheads are relative to Linux; negative overheads indicate improvement.} 
\label{tab:eval:libos:lmbench-fs}
\end{table}
\clearpage