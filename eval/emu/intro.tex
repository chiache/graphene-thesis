This section evaluates the performance
of Linux system calls
serviced by the \graphene{} \libos{}, or \thelibos{}.
%The evaluation is based on benchmark results
%of Linux system call inside a \picoproc{},
%since the Linux system call table is the main target that \thelibos{} emulates
%as a guest.
%Linux system call performance
%on \thelibos{}
%is affected by the emulation strategies
%or the underlying host abstractions (i.e., \hostapis{})
%that \thelibos{}
%selects to implement system call.
%This section
%shows the emulation overhead of \thelibos{}
%by comparing the benchmark results
%of Linux system call in a \picoproc{} with the native system call performance in a Linux process
%and the related \hostapis{}, as evaluated in Section~\ref{sec:eval:pal}.
%Finally,
%the evaluation shows that with the right emulation strategies,
%\thelibos{} can achieve acceptable performance on system calls.
The evaluation is based on
an extended version of \lmbenchwithver{}~\cite{McVoy:lmbench}.
%including
%extra system call tests.
Each benchmark result
reports a mean and 95\% confidence interval,
based on the assumption
that the results of each test are normally distributed.
%to improve the precision,
%this evaluation increases the number of iterations in each to at least 111 times, which effectively
%lowers the variance in most tests to acceptable range.
%Although assuming a normal distribution may not be realistic for most benchmark results~\cite{staelin05lmbench},
%the error is likely to be marginal with the low variance
%observed in the tests.
%we use the default number of iterations for each test case.
%We have added code to \lmbench{} to also calculate 95\% confidence intervals 
%within a run~\footnote{The lmbench authors deliberately exclude variation statistics
%because most methods assume a known distribution, generally a normal distribution---an 
%assumption which is often not the case for a computer microbenchmark~\cite{staelin05lmbench}.
%Though confidence intervals should be taken with a grain of salt, 
%we include them because they clearly indicate that these experiments have very low variance. In 
%a few cases of minor performance improvement, one can assess the impact of noise.}.
To measure the marginal cost of host security checks,
the experiments test both with
and without the \seccomp{} filter and reference monitor
on the Linux host.
The experiments on the SGX host are based on a default security policy,
which enforces security checks on most inputs from the untrusted host OS,
but does not automatically encrypt or authenticate
data files or network traffics. 

% security checks on most enclave call results,
%except for encryption and authentication
%on file contents and
%network traffics.
%RPC streams are protected with TLS connections by default.


The evaluation results categorize
system calls as three types.
The first type of system calls can be completely serviced inside \thelibos{};
in this case, the latency of system calls
in \picoprocs{}
is likely to be close to native
or even be more optimized.
%depending on how optimized the emulation is.
For these system calls,
\thelibos{} emulates the semantics
in the guest space,
gaining performance benefits
by avoiding host system calls or expensive enclave exits.
%to use system calls in the host OS or even exit enclaves.
%the evaluation results show that these system calls are even faster than native, because \thelibos{} does not switch context into the kernal space
%to service the system calls.
For instance,
\syscall{getppid}, which only returns an OS state
(parent process ID) from \thelibos{},
is 67\% faster in \graphene{} than in a native Linux process,
regardless of which host target
the \libos{} is running on.


The second type of system calls
requires
always interacting with the host OS via \thehostabi{},
or enclave calls on \sgx{}.
%translation to a native host system calls,
%with enclave exits if running on SGX.
For example, to implement \syscall{open}, \thelibos{} needs to call \palcall{DkStreamOpen} to open a file handle in the host OS.
The overheads on emulating these system calls
tends to be more significant
than others, as including translation costs and overheads of host security mechanisms.
%There system calls are slower in \graphene{} or \graphenesgx{} than native,
%because emulating each of the system calls
%requires calling the native host system call at least once,
%or more if the \thelibos{} or PAL implementation
%requires so.
Especially if emulating a system call
requires multiple \hostapis{}
or enclave calls,
the overheads multiply as the time spent on accessing host system calls increases.

%for accessing the host file,
%and ends up calling \syscall{open} in the host OS.


The third type of system calls are also implemented
with the involvement
of \hostapis{}, but can benefit from optimizations inside \libos{}, using techniques
such as \hostapi{} batching, I/O buffering,
and directory caching. 
%\thelibos{} can batch
%the results of several operations by buffering
%or caching data inside of the \libos{} instance.
For these system calls,
the overheads generally depend on
the number of \hostapis{}
that \thelibos{} can prevent for each instance of system calls,
since \thelibos{}
cannot reduce the slowdown on each \hostapi{}.
For instance,
\thelibos{} buffers file contents for system calls like \syscall{read} and \syscall{write}
by mapping parts of the file inside \picoprocs{}.
Since \thelibos{}
can batch the effect of multiple small reads or writes
at nearby locations in a file,
%without necessarily flushing the written data to the disks.
the operations
can be up to \roughly{}45\% faster than in a native Linux process.  


%The system call overheads
%on the \graphene{} \libos{} are subject to frequency.
%If \thelibos{} requires no host system calls
%or can defer host system calls
%until a later moment, the emulation cost contains only in-memory operation latency.
%Otherwise,
%using \hostapis{}, which eventually translate to host system calls, tends to cause more significant
%overheads than servicing inside the \libos{}.
%%\thelibos{} tends to impose larger overheads
%%if it has to directly translate guest system calls to \hostapis{}, and eventually to host system calls.





