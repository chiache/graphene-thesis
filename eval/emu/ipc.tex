
\papersubsection{Inter-process communication}


Table~\ref{tab:eval:sysvipc} lists the micro-benchmarks
which exercise each System V message queue function,
within one \picoproc{} (in process), across two concurrent \picoprocs{} (inter process),
and across two non-concurrent \picoprocs{} (persistent).
Linux comparisons for persistent are missing, since message queues 
survive processes in kernel memory.


\begin{table}[t!b!]
\footnotesize
\centering
\bgroup
\def\arraystretch{1.1}
\setlength{\tabcolsep}{0.25em}
\begin{tabular}{|>{\palign{l}}p{3.5em}>{\palign{r}}p{6em}|>{\palign{r}}p{3.5em}>{\palign{r}}p{2em}|>{\palign{r}}p{3.5em}>{\palign{r}}p{2em}>{\palign[\em]{r}}p{2.5em}|>{\palign{r}}p{3.5em}>{\palign{r}}p{2em}>{\palign[\em]{r}}p{2.5em}|>{\palign{r}}p{3.5em}>{\palign{r}}p{2em}>{\palign[\em]{r}}p{2.5em}|}
\hline
& &
\multicolumn{11}{c|}{System call latency (\usec{}), +/- Confidence Interval, \% Overhead} \\
\hline
\multicolumn{2}{|c|}{{\bf Test}} &
\multicolumn{2}{c|}{{\bf Linux \linuxversion{}}} &
\multicolumn{3}{c|}{{\bf \graphene{}}} & \multicolumn{3}{c|}{{\bf \graphene{}+SC+RM}} & \multicolumn{3}{c|}{{\bf \graphenesgx{}}} \\
& &
\usec{} & +/- & 
\usec{} & +/- & \%O &
\usec{} & +/- & \%O &
\usec{} & +/- & \%O  \\
\hline
\code{SIGUSR1}	&	in-process	&		&		&		&		&		&		&		&		&		&		&		 \\
	&	inter-process	&		&		&		&		&		&		&		&		&		&		&		 \\
\hline
\hline
msgget	&	in-process	&		&		&		&		&		&		&		&		&		&		&		 \\
(create)	&	inter-process	&		&		&		&		&		&		&		&		&		&		&		 \\
\hline
msgget	&	in-process	&		&		&		&		&		&		&		&		&		&		&		 \\
	&	inter-process	&		&		&		&		&		&		&		&		&		&		&		 \\
\hline
msgsnd	&	in-process	&		&		&		&		&		&		&		&		&		&		&		 \\
	&	inter-process	&		&		&		&		&		&		&		&		&		&		&		 \\
\hline
msgrcv	&	in-process	&		&		&		&		&		&		&		&		&		&		&		 \\
	&	inter-process	&		&		&		&		&		&		&		&		&		&		&		 \\
\hline
\hline
semget	&	in-process	&		&		&		&		&		&		&		&		&		&		&		 \\
(create)	&	inter-process	&		&		&		&		&		&		&		&		&		&		&		 \\
\hline
semget	&	in-process	&		&		&		&		&		&		&		&		&		&		&		 \\
	&	inter-process	&		&		&		&		&		&		&		&		&		&		&		 \\
\hline
semop	&	in-process	&		&		&		&		&		&		&		&		&		&		&		 \\
(wait)	&	inter-process	&		&		&		&		&		&		&		&		&		&		&		 \\
\hline
semop	&	in-process	&		&		&		&		&		&		&		&		&		&		&		 \\
(signal)	&	inter-process	&		&		&		&		&		&		&		&		&		&		&		 \\
\hline
%msgget   & in-process    & 3320	& 0.7 &  2823 & 0.3 &	-15	\%		\\
%(create) & inter-process & 3336	& 0.5 &  2879 & 3.6 &	-14	\%		\\
%& persistent    &  N/A	&     & 10015 & 0.7 &	202	\%      \\
%\hline								
%msgget   & in-process    & 3245	& 0.5 &   137 & 0.0 &	-96	\%		\\
%& inter-process & 3272	& 3.4 &  8362 & 2.4 &	156	\%		\\
%& persistent    &  N/A	&     &  9386 & 0.4 &	189	\%      \\
%\hline								
%msgsnd   & in-process    &  149	& 0.2 &   443 & 0.2 &	191	\%		\\
%& inter-process &  153	& 0.3 &   761 & 1.1 &	397	\%		\\
%& persistent    &  N/A	&     &   471 & 0.8 &	216	\%	    \\
%\hline								
%msgrcv   & in-process    &  149	& 0.1 &   237 & 0.2 &	60	\%		\\
%& inter-process &  153	& 0.1 &   779 & 2.2 &	409	\%  	\\
%& persistent    &  N/A	&     &   979 & 0.6 &	561	\%	    \\
%\hline
\end{tabular}
\egroup
\caption{Micro-benchmark comparison for System V message queues
between a native Linux process and \graphene{} \picoprocs{}.
Execution time is in microseconds, and lower is better.
overheads are relative to Linux, and negative overheads indicate improved performance.}
\label{tab:eval:sysvipc}
\end{table}

In-process queue creation and lookup are faster than Linux.
In-process send and receive overheads are higher
because of locking on the internal data structures; the current implementation acquires and releases four
fine-grained locks, two of which could be elided by using RCU to eliminate locking for the readers~\cite{mckenney04rcu}.
Most of the costs of persisting message queue contents are also attributable to locking.

Although inter-process send and receive still induce substantial overhead, the optimizations
discussed in \S\ref{sec:libos:namespaces:lessons} reduced overheads compared to a naive implementation
by a factor of ten.  The optimizations of asynchronous sending and migrating ownership of queues
when a producer/consumer pattern were detected were particularly helpful.

%%\begin{comment}
%\paragraph{Scalability.}
%We compare the scalability of \graphene{}'s RPC substrate with the scalability
%of Linux pipes, using a \libos{} that compares the cost of ping-ponging a no-op RPC within a sandbox.
%%In order to evaluate the scalability of \graphene{}'s m
%%we created a microbenchmark where processes within one sandbox ping-pong a no-op RPC,
%%in figure~\ref{fig:rpc4core}.For comparison we executed a similar test where processes send equal sized
%For this experiment, we used a 48-core SuperMicro SuperServer, with four 12-core AMD Opteron 6172 chips running at 2.1 GHz and 64 GB of RAM.
%The performance of \graphene{} closely matches Linux (Figure~\ref{fig:graphene:rpc48core}),
%indicating that the \graphene{} RPC mechanism doesn't introduce any scalability bottlenecks above
%the scalability of IPC on the host OS (Linux).
%The relative performance differences are more variable above 24 cores, which we believe are the 
%result of host-level scheduling.
%We hasten to note that these are worst-case stress tests; based on our application behaviors as well as 
%tuning experience, we expect RPC messages to be infrequent and to scale further in practice.


%% messages over pipes on native Linux.The results indicated that the RPC substrate itself 
%% scales in a pattern closely following native Linux pipes. This was on a 4 core machine.
%% %%The RPC overheads indicate that the RPC substrate itself easily scales to 32 \picoprocs{}
%% %%within one sandbox.  Beyond 32 \picoprocs{}, the IPC helper threads are not always available 
%% %%to service requests and latency starts to increase more rapidly.
%% We re-ran the tests on a 48 core machine and found that the RPC substrate scales more gracefully
%% in this case. The scheduling on a 48 core machine improved the scalability.Figure
%% %%We remind the reader that this degree of scaling was achieved on only a 4 core machine.
%% %%For comparison, we also include a comparable {\tt libLinux}-level 
%% %%signal ping-pong benchmark.  This is less scalable than the RPC substrate due to coarse
%% %%locking on the sigaction structures in {\tt libLinux}.
%% These tests indicate that \graphene{}'s RPC mechanism is sufficiently scalable within a sandbox for 
%% any reasonable multi-processing application.
%%\end{comment}


%\begin{figure}[t!]
%\centering
%\includeplot[0.5]{rpc-scalability}
%\centerline{\includegraphics[width=\linewidth]{figures/48core.png}}
%\vspace{-10pt}
%\caption{Scalability of Linux pipes and \graphene{} RPC on a 48 core machine. Pairs of processes concurrently exchange 10,000 1-byte messages.
%\label{fig:rpc48core}}
%\end{figure}


%% \fixmedp{Qualitative tests we should do:
%% %
%% Daemonize apache, disconnect PAL channel.
%% %
%% Externally disconnect a stream; handle PID transparently (e.g., looks like other PID died and goes away, deliver sigchld, etc.).
%% %
%% Dynamic reattachment (merge)
%% %
%% Group migration of a multi-process workload?
%% %
%% }



%% \paragraph{exec-after-fork.}
%% \emph{exec-after-fork} is a special case that show the
%% generalization of out namespace coordination design. 
%% Because of assumption of maintain PID domain, a thread or process
%% has to inhabit in the process who allocates the PID or direct child of it.
%% However, \emph{exec-after-fork} causes the thread migrated
%% the grandchild process, and lose direct control. We use an extra
%% Namespace operation {\tt regroup} to force all the connected process
%% to replace PID in leases or channel records.
%% Our experiment shows that an exec-after-fork program can
%% successfully wait on its child. 







%\papersubsection{Multi-Processing and Security}

%This subsection details the microbenchmarks
%and stress tests we use to exercise \graphene{}'s multi-processing
%support and security isolation.

%\paragraph{Security isolation.}

%We validate that the \graphene{} host can dynamically move a \picoproc{} into a new sandbox, and that the \graphene{} handles this transparently to the application. In this test case, the host forcibly disconnects all streams between two guests after execution begins. Both library OSes behave as if the other \picoproc{} terminated, delivering exit notifications, closing application-visible pipes. The \picoproc{} which was not already leader for IPC coordination  assumes this role in the new sandbox. Thus, we show that \graphene{} gracefully handles dynamic sandboxing or disconnection of collaborating \picoprocs{}.




%\paragraph{Seccomp filter optimization.~}
%Our benchmark shows that applying a seccomp filter can cause \fixme{xxx}\% overhead on {\tt getppid} syscall latency. We use a seccomp optimization patch, which just-in-time compiles filter code into machine code~\cite{seccomp-jit}.
%The result shows that the patch can improve xxx\% of {\tt getppid} latency.
%This patch is not officially adopted by Linux kernel, but it can be confirmed secure at least for x86-64 architecture.


