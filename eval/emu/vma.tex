\papersubsection{Virtual Memory}
\label{sec:eval:libos:vma}

The latency of virtual memory allocation
and deallocation in \thelibos{}
contains both the cost of updating the internal list of VMAs (virtual memory areas) in \thelibos{}
and the latency of corresponding \hostapis{}, (\palcall{VirtMemAlloc} and \palcall{VirtMemFree}.
Section~\ref{sec:eval:pal:memory}
reports the latency of \palcall{VirtMemAlloc} and \palcall{VirtMemFree}
on the Linux PAL
to be close to the native \syscall{mmap}
and \syscall{munmap} latency.
Table~\ref{tab:eval:libos:lmbench-vma} shows a significant 116--126\% overheads
on the Linux host,
primarily as the cost of VMA list updating.
For \graphenesgx{},
the latency of virtual memory allocation
and deallocation will be dominated by the \hostapis{}, especially since the \sgx{} PAL has to zero allocated pages before returning to \thelibos{}.



\begin{table}[t!b!]
\footnotesize
\centering
\bgroup
\def\arraystretch{1.1}
\setlength{\tabcolsep}{.5em}
\begin{tabular}{|ll|>{\palign{r}}p{4em}r|>{\palign{r}}p{4em}rr|>{\palign{r}}p{4em}rr|}
\hline
& & \multicolumn{8}{c|}{System call latency (\usec{}), +/- Confidence Interval, \%/$\times$ Overhead} \\
\hline
\multicolumn{2}{|c|}{{\bf Test}} &
\multicolumn{2}{c|}{{\bf Linux \linuxversion{}}} &
%\multicolumn{3}{c|}{{\bf \graphene{}}} &
\multicolumn{3}{c|}{{\bf \graphene{}+SC+RM}} &
\multicolumn{3}{c|}{{\bf \graphenesgx{}}} \\
& &
\usec{} & +/- & 
%\usec{} & +/- & O &
\usec{} & +/- & O &
\usec{} & +/- & O \\
\hline

mmap	&	(\hspace{.5em}1MB)	&	0.854	&	.011	& \iffalse	11.397	&	.089	&	12	$\times$ & \fi	11.382	&	.077	&	12	$\times$ &	53	&	0	&	61	$\times$	 \\\hline
mmap	&	(\hspace{.5em}4MB)	&	0.872	&	.055	& \iffalse	11.631	&	.103	&	12	$\times$ & \fi	11.524	&	.129	&	12	$\times$ &	279	&	13	&	319	$\times$	 \\\hline
mmap	&	(16MB)	&	0.872	&	.056	& \iffalse	11.575	&	.086	&	12	$\times$ & \fi	11.371	&	.061	&	12	$\times$ &	7,762	&	30	&	8,901	$\times$	 \\\hline
%mmap	&	(64MB)	&	0.889	&	.050	& \iffalse	11.574	&	.145	&	12	$\times$ & \fi	11.419	&	.063	&	12	$\times$ &	36,832	&	70	&	41,430	$\times$	 \\\hline
\hline																										
mmap/access	&	(\hspace{.5em}1MB)	&	7.930	&	.052	& \iffalse	14.6	&	.054	&	84	\% & \fi	14.664	&	.049	&	85	\% &	53	&	0	&	6	$\times$	 \\\hline
mmap/access	&	(\hspace{.5em}4MB)	&	33.440	&	11.567	& \iffalse	48.5	&	10.873	&	45	\% & \fi	45.189	&	10.214	&	35	\% &	264	&	10	&	7	$\times$	 \\\hline
mmap/access	&	(16MB)	&	101.416	&	11.547	& \iffalse	93.7	&	8.347	&	-8	\% & \fi	93.381	&	12.239	&	-8	\% &	7,738	&	33	&	75	$\times$	 \\\hline
%mmap/access	&	(64MB)	&	313.162	&	30.100	& \iffalse	302.3	&	23.752	&	-3	\% & \fi	315.187	&	23.690	&	1	\% &	36,832	&	44	&	117	$\times$	 \\\hline
\hline																										
brk	&	(\hspace{.5em}1KB)	&	0.445	&	.002	& \iffalse	0.159	&	.000	&	-64	\% & \fi	0.146	&	.000	&	-67	\% &	0.136	&	.000	&	-69	\%	 \\\hline
brk	&	(\hspace{.5em}4KB)	&	0.446	&	.003	& \iffalse	0.159	&	.000	&	-64	\% & \fi	0.146	&	.000	&	-67	\% &	0.136	&	.000	&	-70	\%	 \\\hline
brk	&	(16KB)	&	0.444	&	.004	& \iffalse	0.159	&	.000	&	-64	\% & \fi	0.146	&	.000	&	-67	\% &	0.136	&	.000	&	-69	\%	 \\\hline
%brk	&	(64KB)	&	0.444	&	.001	& \iffalse	0.159	&	.000	&	-64	\% & \fi	0.146	&	.000	&	-67	\% &	0.136	&	.000	&	-69	\%	 \\\hline
\hline																										
brk/access	&	(\hspace{.5em}1KB)	&	1.783	&	.059	& \iffalse	0.169	&	.000	&	-91	\% & \fi	0.167	&	.000	&	-91	\% &	0.160	&	.000	&	-91	\%	 \\\hline
brk/access	&	(\hspace{.5em}4KB)	&	1.924	&	.266	& \iffalse	0.199	&	.000	&	-90	\% & \fi	0.197	&	.000	&	-90	\% &	0.190	&	.000	&	-90	\%	 \\\hline
brk/access	&	(16KB)	&	4.195	&	.033	& \iffalse	0.320	&	.000	&	-92	\% & \fi	0.317	&	.000	&	-92	\% &	0.311	&	.000	&	-93	\%	 \\\hline
%brk/access	&	(64KB)	&	14.067	&	.030	& \iffalse	1.309	&	.009	&	-91	\% & \fi	1.323	&	.024	&	-91	\% &	1.292	&	.001	&	-91	\%	 \\\hline

\end{tabular}
\egroup

\caption{\syscall{mmap} and \syscall{brk} latency. The comparison is among (1) native Linux processes; (2) \graphene{} on Linux host, with \seccomp{} filter ({\bf +SC}) and reference monitor ({\bf +RM}); (3) \graphenesgx{}.
System call latency is in microseconds, and lower is better.
Overheads are relative to Linux; negative overheads indicate improvement.} 
\label{tab:eval:libos:lmbench-vma}
\end{table}


As an opposite example, allocation and deallocation
with \syscall{brk} are
significantly faster inside \thelibos{}
than native,
because \thelibos{} encapsulates most of the operations.
The emulation of \syscall{brk} in \thelibos{}
preallocates a large enough
heap space for consequential allocations,
so \syscall{brk} can be mostly serviced in \thelibos{}
without making frequent \hostapis{}.
As a result, either on the Linux host or the SGX host,
the latency on \syscall{brk}
is up to 70\% faster than native.

