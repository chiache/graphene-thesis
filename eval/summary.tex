\papersection{Summary}
\label{sec:eval:summary}

This section evaluates
various performance factors of using the \graphene{} \libos{} for running unmodified Linux applications.
As one of the dominating factor,
\thehostabi{} performance is primarily determined by the latency of underlying host system calls,
but burdened with security checks such as system call restriction and reference monitoring.
As one of the worst examples,
the \sgx{} PAL tends to suffers significant overheads on several \hostapis{}.
The potential factors that are responsible for the overheads
include enclave exit and re-entrance,
copying \hostapi{} arguments across the enclave boundary,
and security checks based on cryptographic techniques.


Despite the potential significant overheads
on \thehostabi{},
\thelibos{} introduces several optimizations
based on reducing the average number of \hostapis{} per emulated system call.
Several system calls, such as \syscall{getppid} and similar system calls,
accessing pseudo files (e.g., \code{/dev/zero}),
\syscall{brk},
have shown close-to-native or even better performance
which is irrelevant from
\thehostabi{} implementation.
For file system operations,
caching and buffering can effectively reduce the impact of \hostapi{} overheads,
by deploying spared \picoproc{} process memory
to store temporarily abstraction states.
As a result, both server and command-line applications show comparable end-to-end performance
on either \graphene{} or a native Linux kernel.
For \graphenesgx{},
most of the tested workloads show less than 2\x{} overheads
in general.

