\section{\graphene{}}
\label{sec:eval:graphene}





%\fixmedp{Missing back-of-the envelope for fork overheads, fraction of system calls using ipc}


This section evaluates \graphene{}'s multi-process coordination, security, cross-host migration, memory footprint, and performance.
We drive this evaluation with a selection of real-world applications that leverage multiple processes in \graphene{},
as well as with microbenchmarks and stress tests.
We organize the evaluation around the following questions:
\begin{compactenum}
\item How do \graphene{}'s startup and migration costs compare to running an application in a dedicated VM?
\item Given that RAM is often the limiting factor in VMs per system, how does \graphene{}'s memory footprint compare to other virtualization techniques?
\item What are the performance overheads of \graphene{} relative to a native Linux process or VM?
\item What additional overheads are added by the reference monitor?
\item How do \graphene{}'s overheads scale with the number of processes in a sandbox?
\item Does the \graphene{} reference monitor enforce security isolation comparable to running the application in a VM?  
\item What fraction of recent Linux vulnerabilities would \graphene{} prevent?
\end{compactenum}

%We note that no recent single-process library OSes are both publicly available
%and able to execute unmodified Linux binaries.

Except for scalability measurements, 
all measurements were collected on a 
Dell Optiplex 790 with 
a 4-core 3.40 GHz Intel Core i7 CPU,
4 GB RAM, and a 250GB, 7200 RPM ATA disk.
Our host system runs Ubuntu 12.04 server with host Linux kernel version 3.5, 
which includes KVM on 
QEMU version 1.0.
%Adding more detail of KVM environment
Each KVM Guest is deployed with 4 virtual CPU with EPT, 2GB RAM, a 30GB virtual disk image, Virtio enabled for network and disk, bridged connection with TAP, and runs the same Ubuntu and Linux kernel image.
We note that recent library OSes are either not openly available
or cannot execute unmodified Linux binaries.
Unless otherwise noted, \graphene{} measurements include the reference monitor.
%In order to assess the relative overhead of the monitor (\S\ref{sec:eval:micro}), 
%we include some microbenchmark measurements with and without the monitor.

%\paragraph{Seccomp filter optimization.~}
%Our benchmark shows that applying a seccomp filter can cause \fixme{xxx}\% overhead on {\tt getppid} syscall latency. We use a seccomp optimization patch, which just-in-time compiles filter code into machine code~\cite{seccomp-jit}.
%The result shows that the patch can improve xxx\% of {\tt getppid} latency.
%This patch is not officially adopted by Linux kernel, but it can be confirmed secure at least for x86-64 architecture.


\begin{table}[t!b!]
\footnotesize
\centering
\begin{tabular}{|l|r|rr|rr|rr|}
\hline
{\bf Test } & {\bf Linux } & \multicolumn{2}{c|}{{\bf KVM}}
& \multicolumn{2}{c|}{{\bf \graphene{}}} \\\hline

%Start-up time& 88.40& 3293296.77 & 3725349\% & 125350.73 & 1417\%& 622.00 & 604\% \\
Start-up   & 208 \us{} & 3.3 s & 15K\x{} & 641 \us{} & 3.1\x{}  \\
\hline
Checkpoint & N/A   & 0.987 s  &  &   416 \us{} &  \\
\hline
Resume     & N/A   & 1.146 s  &  &  1387 \us{} & \\
\hline\hline
Checkpoint size & N/A & \multicolumn{2}{l|}{105 MB} & \multicolumn{2}{l|}{376 KB}   \\
\hline
\end{tabular}
\caption[Startup, checkpoint, and resume times in Linux, KVM, and \graphene{}]
{Startup, checkpoint, and resume times for (1) a native Linux process,
(2) a KVM virtual machine,
(3) a \graphene{} \picoproc{},
(4) a \graphene{} \picoproc{} in a SGX enclave,
 where appropriate. Lower is better.  
Overheads are relative to Linux.} 
\label{tab:graphene:startup}
\end{table}

\subsection{Process migration and startup}
\label{sec:eval:graphene:startup}

%A key feature of VMs is the ability to move a running application to a new system.
\graphene{} supports migration of an application from a \picoproc{} on one machine
to a \picoproc{} on another machine by checkpointing the application,
copying the checkpoint over the network, and then resuming the checkpoint.
%We note that resuming a checkpoint executes entirely in userspace.
Table~\ref{tab:graphene:startup} shows the time to start
up a process, VM, or \picoproc{}; as well as the checkpoint and resume time for KVM and \graphene{}.
Migration across machines is a function of network bandwidth,
%% and checkpoint size, 
so we report checkpoint size instead. % of migration time.
%Only KVM and \graphene{} have the ability to checkpoint and resume, and are the only data points listed.


\graphene{} shows dramatically faster initialization times than a VM. This is not surprising,
since \graphene{} is substantially smaller than an OS kernel. 
Similarly, checkpointing and restoring a 4 MB application on \graphene{} is 1--4 times faster than checkpointing or restoring a KVM guest.



%\fixmedp{Error bars on mem usage graphs missing}

\subsection{Memory footprint}
\label{sec:eval:graphene:memory}

We begin by measuring the minimal memory footprint of a simple ``hello world'' program 
on Linux (352 KB) and \graphene{} (1.4 MB).
Thus, one would expect roughly 1 MB of extra memory usage for any single-\picoproc{} application.
Because of copy-on-write sharing, however, the incremental cost of adding additional ``hello world'' children
is only about 790 KB per process.
%, although this will vary based on usage.

%Figure~\ref{fig:graphene:memusage} lists memory overheads of 
%a diverse set of {\em unmodified}
%applications, including 
%a {\tt make -j 4} of Gra\-phene's {\tt libLinux}
%using the {\tt \gcc{}} compiler (v4.4.5), 
%%\busy, a software which combines tiny versions of most common UNIX utilities into a single
%%small executable;
%the \light{} web server (v1.4.30) with 4 threads,
%the Apache web server (v2.4.3) with 4 processes,
%and the \busy{} shell (v4.1) executing the shell script test ({\tt multi.sh}) from the Unixbench suite (v5.1.3)~\cite{unixbench}.
%We measure memory utilization based on the maximum kernel-reported resident set size
%of each process or VM. For most applications, memory usage was fairly constant across inputs,
%so we only display representative examples.
%
%\begin{figure}[t!]
%\centering
%\includeplot[0.5]{apps-memusage}
%\centerline{\includegraphics[width=\linewidth]{figures/memusage.pdf}}
%\caption{Memory usage of applications on Linux, \graphene{}, and KVM, in MB.  Lower s better.
%\label{fig:graphene:memusage}}
%\end{figure}


We found that the memory footprints of compilation were a function of the 
size of the source base, even on Linux; we select compile of {\tt libLinux} as a representative example.
\graphene{} adds less than 15\% overhead in all cases.
%every compilation we measured on \graphene{} had less than 10\% memory overhead.

Unixbench on \graphene{} uses substantially more memory at a given time than native Linux---more than double.
In these samples, however, \graphene{} also had 3--4\x{} as many processes
running; this is because Unixbench simply spawns all of the tasks in the background, rather than
executing them sequentially.   Because \graphene{} processes execute more slowly (attributable to a slower {\tt fork}---\S\ref{sec:eval:graphene}),
a given sample will include more \picoprocs{}, pushing total memory usage higher.
Thus, we expect that further tuning {\tt fork} performance will lower sampled memory usage.


Across all workloads, \graphene{}'s memory footprint is 3--20\x{}  
smaller than KVM.  
For all tests, we used a minimal KVM disk image, 
%formatted as a 30GB\fixmewkj{this can be changed to be smaller, but I don't know if disk size would impact memory} raw disk image with ext4. The root image was 
generated using debootstrap 1.0.39ubuntu0.3 and supplemented only by packages required to obtain, compile, and run the experiments.
In order to make memory footprint measurements as fair as possible to KVM, 
we used both the virtio balloon driver and kernel same page merging (KSM)~\cite{ksm}.
We also reduced the RAM allocated to the VM to the smallest size without harming performance---128MB.  We note that memory measured includes memory used by QEMU for VM device emulation, 
adding a few dozen MB.
%hence the total is 
%a few dozen MB higher than the RAM allocated to the VM.

If the smallest usable Linux VM consumes about 150 MB of RAM, our measurements indicate that 
one could run anywhere from 12--188 libOSes within the same footprint.


\begin{table}[t!b!]
\footnotesize
%\tiny
\centering

\begin{tabular}{|l|rr|rrr|rrr|}
\hline
&\multicolumn{8}{c|}{Execution time (s), +/- Conf. Interval, \% Overhead} \\
\hline
{\bf \gcc/make} & \multicolumn{2}{c|}{\bf Linux} & \multicolumn{3}{c|}{{\bf in KVM}} & \multicolumn{3}{c|}{{\bf \graphene{} + SC + RM}}  \\
\hline
% gcc
%\multicolumn{2}{|l|}{helloworld} & .02 & .02 & 0 \% & .03 & 100 \% \\ 
%\hline
%oggenc & 1.42 & .01 & 1.51 & .00 & 6 \% & 1.43 & .00 & 1 \% \\\hline
gcc (helloworld)  & 0.020 & .000 &  0.022 & .000 & 7 \% &  0.023 & .000 &  12 \%  \\\hline
gcc (.7M\loc{})   & 21.64 &  .00 &  23.37 &  .02 & 8 \% &  21.88 &  .00 &   1 \%  \\\hline
make bzip2        & 2.332 & .000 &  2.448 & .000 & 5 \% &  2.407 & .000 &   3 \%  \\\hline
make -j4 bzip2    & 0.888 & .000 &  0.967 & .000 & 9 \% &  0.923 & .004 &   4 \%  \\\hline
make libLinux     & 4.832 & .000 &  5.037 & .000 & 4 \% &  5.112 & .001 &   6 \%  \\\hline
make -j4 libLinux & 1.361 & .000 &  1.413 & .000 & 4 \% &  1.459 & .000 &   7 \%  \\\hline



\hline\hline
Ap. Bnch     & \multicolumn{8}{c|}{Avg Throughput (MB/s), +/- Conf. Interval, \% Overhead} \\
\hline
{\bf Apache} & \multicolumn{2}{c|}{\bf Linux} & \multicolumn{3}{c|}{{\bf in in KVM}} & \multicolumn{3}{c|}{{\bf \graphene{} + SC + RM}}  \\
\hline
25 conc     & 6.282 & .005 & 5.327 & .031 & 18 \% & 4.586 & .002 & 36 \%  \\\hline
50 conc     & 6.305 & .002 & 5.420 & .060 & 16 \% & 4.555 & .012 & 38 \%  \\\hline
100 conc    & 6.347 & .010 & 5.152 & .036 & 22 \% & 4.572 & .009 & 39 \%  \\\hline

{\bf \lighttpd{}} & \multicolumn{2}{c|}{\bf Linux} & \multicolumn{3}{c|}{{\bf in KVM}} & \multicolumn{3}{c|}{{\bf \graphene{} + SC + RM}} \\
\hline

25 conc    & 6.66 &.01 &6.46 &.03 &3 \% &5.65 &.00 &18 \%  \\\hline
50 conc    & 6.65 &.13 &6.41 &.02 &4 \% &4.79 &.00 &39 \%  \\\hline
100 conc   & 6.69 &.01 &6.39 &.03 &5 \% &4.56 &.01 &47 \%  \\\hline

\hline\hline
%%%%% UnixBench

  & \multicolumn{8}{c|}{Execution Time (s), +/- Conf. Interval, \% Overhead} \\
\hline
{\bf bash } & \multicolumn{2}{c|}{\bf Linux} & \multicolumn{3}{c|}{{\bf in KVM}} & \multicolumn{3}{c|}{{\bf \graphene{} + RM}} \\
\hline
%2.1 s linux
%graphene 10.06s

%bash
Unix utils & 0.87 & .00 & 1.10 & .01 & 26 \% & 2.01 & .00 & 134 \%  \\\hline
Unixbench  & 0.55 & .00 & 0.55 & .00 &  0 \% & 1.49 & .00 & 192 \%  \\\hline
\end{tabular}
\caption[Application benchmark results in Linux, KVM and \graphene{}]
{Application benchmark execution times in a (1) native Linux process, (2) a process inside a KVM virtual machine, (3) a \graphene{} \picoproc{} with the SECCOMP filter ({\bf +SC}) and reference monitor ({\bf +RM}). }
\label{tab:graphene:apps}
\end{table}


\subsection{Application performance}
\label{sec:eval:graphene:apps}

Table~\ref{tab:graphene:apps} lists 
execution time of our {\em unmodified} 
application benchmarks (detailed in \S\ref{sec:eval:graphene}).
All applications create multiple processes,
except for \lighttpd{}, which only creates multiple threads.
Each data point is the average of at least six runs, 
and 95\% confidence intervals are listed in the table.

We exercise  {\tt \gcc{}}/{\tt make}
with inputs of varying sizes:
%the pbzip2 compression utility (v1.1.6, 4 KLoC),
%the ogg sound encoder (v1.0.1, 5 KLoC),
%GNU {\tt make} (v3.82, \fixmedp{XX} KLoC)
bzip2 (v1.0.6, 5KLoC, 13 files),
\graphene{}'s {\tt libLinux} (31 KLoC, 78 files)
and \gcc{} (v3.5.0, 551 KLoC, collected as a single source file). 
We benchmark Apache (4 preforked workers) and \lighttpd{} (4 threads) with 
\ab{},
which issues 25, 50, and 100 concurrent requests
to download a 100 byte file 50,000 times.

We exercised \busy{} with 
300 iterations of the Unixbench benchmark~\cite{unixbench}, as well as 
300 iterations of a simple shell script benchmark that runs 6 common shell script commands
({\tt cp}, {\tt rm}, {\tt ls}, {\tt cat}, {\tt date}, and {\tt echo}).

Compilation workloads incur overheads ranging from 5--30\%.
%All compilation workloads have minimal overheads compared to native when executed sequentially;
Parallel compilation on both \graphene{} and Linux yields comparable  speedups  over sequential,
but the percent overhead increases for parallel \graphene{}.
For instance, {\tt make -j4 libLinux} speeds up 3.7\x{} on Linux and 3.4$\times$ on \graphene{}.
The compilation overheads are primarily from the reference monitor---nearly all for bzip2 and gcc, and half for {\tt libLinux}.
%Nearly all of the overhead from the bzip2 and gcc compilations are from the reference monitor; about half of the overheads of {\tt libLinux} compile 
%are the reference monitor and half are \graphene{} itself.
In the case of both \busy{} workloads, the key bottleneck is the {\tt fork} system call.
Profiling indicates that half of the time in {\tt libLinux} is spent on {\tt fork} in both benchmarks.
The trend is exacerbated in Unixbench, which creates all of the processes at the beginning and
waits for them all to complete; because \graphene{} cannot create children as quickly as native, this leads to 
load imbalance throughout the rest of the benchmark.  

With the reference monitor disabled,  \lighttpd{} has equivalent throughput to a native Linux process;
as discussed in the next subsection, these overheads come from checking paths in the monitor.
The Apache web server loses about half of its throughput relative to \lighttpd{} on \graphene{}.
The primary bottleneck in Apache relative to \lighttpd{} is System V semaphores,
%, which account for about 
%one third of the additional time spent in {\tt libLinux}
%relative to \light{}, 
and the remaining overheads are attributable to more time spent waiting for input.
The overheads for both \lighttpd{} and Apache on KVM are primarily attributable
to bridged networking.





\subsection{Micro-benchmarks}
\label{sec:eval:graphene:mbench}

In order to understand the overheads of individual system calls,
Table~\ref{tab:graphene:lmbench} lists 
a representative sample of 
tests from the
\lmbench{} suite, version 2.5~\cite{McVoy:lmbench}.
Each row reports a mean and 95\% confidence interval;
we use the default number of iterations for each test case.
%We have added code to \lmbench{} to also calculate 95\% confidence intervals 
%within a run~\footnote{The lmbench authors deliberately exclude variation statistics
%because most methods assume a known distribution, generally a normal distribution---an 
%assumption which is often not the case for a computer microbenchmark~\cite{staelin05lmbench}.
%Though confidence intervals should be taken with a grain of salt, 
%we include them because they clearly indicate that these experiments have very low variance. In 
%a few cases of minor performance improvement, one can assess the impact of noise.}.
To measure the marginal cost of the reference monitor, we report numbers with and without 
the reference monitor.

%The performance of \graphene{} relative to Linux varies
%based on the system call.  
In general, calls that can be serviced inside the library are faster than native,
whereas calls that require translation to a native call incur overheads typically under 100\%.
For instance, 
the self-signaling test (sig overhead)
just calls the signal handler as a function,
which is almost twice as fast
as the Linux kernel implementation.  

\begin{table}[t!b!]
\footnotesize
\centering
\begin{tabular}{|l|rr|rrr|rrr|}
\hline
{\bf Test } & \multicolumn{2}{c|}{{\bf Linux}} & \multicolumn{3}{c|}{{\bf \graphene{}
}} & \multicolumn{3}{c|}{{\bf \graphene{} + SC + RM}} \\
&
\usec{} & +/- & 
\usec{} & +/- & \%O &
\usec{} & +/- & \%O \\

\hline

syscall        &  0.045 & .000 &  0.014 & .000 & -68.9 &    0.014 & .000 & -68.9   \\\hline
read           &  0.115 & .000 &  0.118 & .000 &   2.6 &    0.118 & .000 &   2.6   \\\hline
write          &  0.115 & .000 &  0.115 & .000 &   0.0 &    0.115 & .000 &   0.0   \\\hline
stat           &  0.332 & .000 &  1.154 & .000 & 247.6 &    1.161 & .000 & 249.7   \\\hline
fstat          &  0.107 & .000 &  0.189 & .000 &  76.6 &    0.189 & .003 &  76.6   \\\hline
open/close     &  0.880 & .003 &  2.552 & .005 & 190.0 &    2.816 & .002 & 330.0   \\\hline
select file    &  3.360 & .001 & 11.536 & .003 & 243.3 &   11.647 & .004 & 246.6   \\\hline
select tcp     & 10.590 & .002 & 11.679 & .002 &  10.3 &   12.050 & .000 &  13.8   \\\hline
sig install    &  0.126 & .000 &  0.126 & .000 &   0.0 &    0.126 & .000 &   0.0   \\\hline
sigusr1        &  0.951 & .000 &  0.184 & .000 & -80.7 &    0.188 & .000 & -80.2   \\\hline
sigsegv        &  0.263 & .000 &  1.467 & .000 & 457.8 &    1.681 & .000 & 539.2   \\\hline
\hline
TCP            &  8.354 & .001 &  9.121 & .002 &   1.8 &    9.699 & .002 &  16.1   \\\hline
UDP            &  7.222 & .002 &  9.450 & .002 &  30.8 &    9.995 & .001 &  38.4   \\\hline
AF\_UNIX       &  4.937 & .011 &  5.024 & .001 &   1.8 &    6.155 & .010 &  24.7   \\\hline
\hline
fork+exit      &  66.86 & 0.09 &   380.47 & 3.55 & 469 &   442.83 & 1.39 &   562   \\\hline
fork+fork+exit & 148.32 & 0.55 &   778.00 & 6.87 & 424 &   915.67 & 3.35 &   517   \\\hline
vfork+exec     & 141.53 & 0.18 &   487.67 & 5.05 & 245 &   547.30 & 5.22 &   286   \\\hline
fork+exec      & 194.93 & 0.20 &   810.00 & 3.27 & 315 &   878.50 & 1.89 &   350   \\\hline
fork+fork+exec & 266.89 & 0.36 & 1,200.60 & 5.20 & 349 & 1,387.75 & 6.21 &   420   \\\hline
fork+sh        & 499.64 & 0.67 & 1,726.75 & 6.14 & 245 & 1,912.00 & 3.83 &   283   \\\hline
%\hline
%UDP lat.&7.26&0.0&10.35&0.1&43&11.63&0.1&60 \\\hline
%TCP lat.&8.70&0.0&9.84&0.1&13&11.48&0.1&32 \\\hline
%\hline
%%%  \bf{Test} & K/s & +/- & K/s & +/- & \%O & K/s & +/- & \%O \\\hline
%%% %% \bf{Test} & \bf{K/s} & \bf{+/-} & \bf{K/s} & \bf{+/-} & \bf{\%O} & \bf{K/s} & \bf{+/-} & \bf{\%O} \\\hline
%%% 0K creat & 174 & 0.9 & 72 & 0.1 & 143 & 77 & 0.2 & 125 \\\hline
%%% 0K del & 232 & 1.2 & 130 & 0.5 & 79 & 117 & 0.2 & 98 \\\hline
%%% 4K creat & 120 & 0.1  & 33 & 0.0 & 265 & 31 & 0.0 & 284 \\\hline
%%% 4K del & 184 & 1.3  & 112 & 0.1 & 64 & 104 & 0.2 & 78 \\\hline
%%% 10K creat & 83 & 0.1  & 28 & 0.0 & 195 & 27 & 0.0 & 204 \\\hline
%%% 10K del & 149 & 0.3  & 94 & 0.1 & 58 & 88 & 0.2 & 69 \\\hline
\end{tabular}
\caption[\lmbench{} benchmarking results in Linux, KVM and \graphene{}]
{\lmbench{} comparison among (1) native Linux processes, (2) \graphene{} \picoprocs{} on Linux host, both without and with the SECCOMP filter ({\bf +SC}) and reference monitor ({\bf +RM}), and (3) \graphene{} in SGX enclaves.
Execution time is in microseconds, and lower is better. 
%The file system is measured in thousands operations per second, and higher is better.
Overheads are relative to Linux; negative overheads indicate improved performance.} 
\label{tab:graphene:lmbench}
\end{table}


The most expensive system calls occur when {\tt libLinux} inadvertently duplicates work
with the host kernel.  
For instance, many of the file path and handle management calls duplicate some of the effort of the host file system,
leading to a 1--3\x{} slower implementation than native.
As the worst example,
{\tt fork+exit} is 5.9\x{} slower than Linux.
Profiling indicates that about one sixth of this overhead is in process creation, which 
takes additional work to create a clean \picoproc{} on Linux; we expect this overhead could be reduced
with a kernel-level implementation of the process creation ABI, rather than emulating this behavior on {\tt clone}.
Another half of the overhead comes from the
{\tt libLinux} checkpointing code (commensurate with the data in Table~\ref{tab:graphene:lmbench}), which 
includes a substantial amount of serialization effort which might be reduced by checkpointing the data structures in place.
A more competitive {\tt fork} will require host support and additional {\tt libLinux} tuning.
%Thus, we think a competitive {\tt fork} implementation will require both a more suitable host kernel
%and more tuning in the {\tt libLinux} code. 
%\fixmewkj{explain why TCP faster than UDP?}

We also measure the overhead of isolating a \graphene{} \picoproc{} inside the reference monitor.
Because most filtering rules can be statically loaded into the kernel,
the cost of filtering is negligible with few exceptions.
%TCP and UDP latency is increased slightly because the monitor checks
%the arguments of the {\tt connect} system call.
Only calls that
involve path traversals, such as {\tt open} and {\tt exec}, result in substantial overheads relative to \graphene{}.
%%, adding an additional 100--200\% overhead. This is 
%because the AppArmor extensions must verify that
%the requested paths are in the permitted file system view.
%Alternative implementations, such as a {\tt chroot-ed} environment using the aufs unioning file system version 3.0~\cite{aufs},
%resulted in opens that were {\em twice} as slow as our LSM extensions.
An efficient implementation of an environment similar to FreeBSD  jails~\cite{jails}
would make all reference monitoring overheads negligible.

\begin{table}[t!b!]
\footnotesize
\centering
\begin{tabular}{|ll|rr|rrr|}
\hline
\multicolumn{2}{|c|}{{\bf Test}} &
\multicolumn{2}{c|}{{\bf Linux}} &
\multicolumn{3}{c|}{{\bf \graphene{}}} \\
 & & \us{} & +/- & \us{} & +/- & \\
\hline
msgget   & in-process    & 3320	& 0.7 &  2823 & 0.3 &	-15	\%		\\
(create) & inter-process & 3336	& 0.5 &  2879 & 3.6 &	-14	\%		\\
         & persistent    &  N/A	&     & 10015 & 0.7 &	202	\%      \\
\hline								
msgget   & in-process    & 3245	& 0.5 &   137 & 0.0 &	-96	\%		\\
         & inter-process & 3272	& 3.4 &  8362 & 2.4 &	156	\%		\\
         & persistent    &  N/A	&     &  9386 & 0.4 &	189	\%      \\
\hline								
msgsnd   & in-process    &  149	& 0.2 &   443 & 0.2 &	191	\%		\\
         & inter-process &  153	& 0.3 &   761 & 1.1 &	397	\%		\\
         & persistent    &  N/A	&     &   471 & 0.8 &	216	\%	    \\
\hline								
msgrcv   & in-process    &  149	& 0.1 &   237 & 0.2 &	60	\%		\\
         & inter-process &  153	& 0.1 &   779 & 2.2 &	409	\%  	\\
         & persistent    &  N/A	&     &   979 & 0.6 &	561	\%	    \\
         \hline
\end{tabular}
\caption[The micro-benchmark results for System V message queues in Linux, KVM, and \graphene{}]
{Micro-benchmark comparison for System V message queues
between a native Linux process and \graphene{} \picoprocs{}.
Execution time is in microseconds, and lower is better.
overheads are relative to Linux, and negative overheads indicate improved performance.}
%We measure the cost of sending and receiving from the same process (in process),
%across two processes (inter process), and non-concurrently (persistent).
\label{tab:graphene:msgq}
\end{table}

\paragraph{System V IPC.}
Table~\ref{tab:graphene:msgq} lists the micro-benchmarks
which exercise each System V message queue function,
within one \picoproc{} (in process), across two concurrent \picoprocs{} (inter process),
and across two non-concurrent \picoprocs{} (persistent).
Linux comparisons for persistent are missing, since message queues 
survive processes in kernel memory.

In-process queue creation and lookup are faster than Linux.
In-process send and receive overheads are higher
because of locking on the internal data structures; the current implementation acquires and releases four
fine-grained locks, two of which could be elided by using RCU to eliminate locking for the readers~\cite{mckenney04rcu}.
Most of the costs of persisting message queue contents are also attributable to locking.

Although inter-process send and receive still induce substantial overhead, the optimizations
discussed in \S\ref{sec:libos:namespaces:lessons} reduced overheads compared to a naive implementation
by a factor of ten.  The optimizations of asynchronous sending and migrating ownership of queues
when a producer/consumer pattern were detected were particularly helpful.

%%\begin{comment}
%\paragraph{Scalability.}
%We compare the scalability of \graphene{}'s RPC substrate with the scalability
%of Linux pipes, using a \libos{} that compares the cost of ping-ponging a no-op RPC within a sandbox.
%%In order to evaluate the scalability of \graphene{}'s m
%%we created a microbenchmark where processes within one sandbox ping-pong a no-op RPC,
%%in figure~\ref{fig:rpc4core}.For comparison we executed a similar test where processes send equal sized
%For this experiment, we used a 48-core SuperMicro SuperServer, with four 12-core AMD Opteron 6172 chips running at 2.1 GHz and 64 GB of RAM.
%The performance of \graphene{} closely matches Linux (Figure~\ref{fig:graphene:rpc48core}),
%indicating that the \graphene{} RPC mechanism doesn't introduce any scalability bottlenecks above
%the scalability of IPC on the host OS (Linux).
%The relative performance differences are more variable above 24 cores, which we believe are the 
%result of host-level scheduling.
%We hasten to note that these are worst-case stress tests; based on our application behaviors as well as 
%tuning experience, we expect RPC messages to be infrequent and to scale further in practice.


%% messages over pipes on native Linux.The results indicated that the RPC substrate itself 
%% scales in a pattern closely following native Linux pipes. This was on a 4 core machine.
%% %%The RPC overheads indicate that the RPC substrate itself easily scales to 32 \picoprocs{}
%% %%within one sandbox.  Beyond 32 \picoprocs{}, the IPC helper threads are not always available 
%% %%to service requests and latency starts to increase more rapidly.
%% We re-ran the tests on a 48 core machine and found that the RPC substrate scales more gracefully
%% in this case. The scheduling on a 48 core machine improved the scalability.Figure
%% %%We remind the reader that this degree of scaling was achieved on only a 4 core machine.
%% %%For comparison, we also include a comparable {\tt libLinux}-level 
%% %%signal ping-pong benchmark.  This is less scalable than the RPC substrate due to coarse
%% %%locking on the sigaction structures in {\tt libLinux}.
%% These tests indicate that \graphene{}'s RPC mechanism is sufficiently scalable within a sandbox for 
%% any reasonable multi-processing application.
%%\end{comment}


%\begin{figure}[t!]
%\centering
%\includeplot[0.5]{rpc-scalability}
%\centerline{\includegraphics[width=\linewidth]{figures/48core.png}}
%\vspace{-10pt}
%\caption{Scalability of Linux pipes and \graphene{} RPC on a 48 core machine. Pairs of processes concurrently exchange 10,000 1-byte messages.
%\label{fig:rpc48core}}
%\end{figure}


%% \fixmedp{Qualitative tests we should do:
%% %
%% Daemonize apache, disconnect PAL channel.
%% %
%% Externally disconnect a stream; handle PID transparently (e.g., looks like other PID died and goes away, deliver sigchld, etc.).
%% %
%% Dynamic reattachment (merge)
%% %
%% Group migration of a multi-process workload?
%% %
%% }



%% \paragraph{exec-after-fork.}
%% \emph{exec-after-fork} is a special case that show the
%% generalization of out namespace coordination design. 
%% Because of assumption of maintain PID domain, a thread or process
%% has to inhabit in the process who allocates the PID or direct child of it.
%% However, \emph{exec-after-fork} causes the thread migrated
%% the grandchild process, and lose direct control. We use an extra
%% Namespace operation {\tt regroup} to force all the connected process
%% to replace PID in leases or channel records.
%% Our experiment shows that an exec-after-fork program can
%% successfully wait on its child. 







%\subsection{Multi-Processing and Security}

%This subsection details the microbenchmarks
%and stress tests we use to exercise \graphene{}'s multi-processing
%support and security isolation.

%\paragraph{Security isolation.}

%We validate that the \graphene{} host can dynamically move a \picoproc{} into a new sandbox, and that the \graphene{} handles this transparently to the application. In this test case, the host forcibly disconnects all streams between two guests after execution begins. Both library OSes behave as if the other \picoproc{} terminated, delivering exit notifications, closing application-visible pipes. The \picoproc{} which was not already leader for IPC coordination  assumes this role in the new sandbox. Thus, we show that \graphene{} gracefully handles dynamic sandboxing or disconnection of collaborating \picoprocs{}.

\subsection{Security study}


\issue{1.1.d}{Add a kernel coverage study}
Demonstrating security is always challenging, as it requires exploring all possible attacks.
We instead offer statistics that indicate an overall reduction in attack surface,
and qualitative validation where appropriate.
\graphene{} runs substantial Linux applications using less than 15\% of the
Linux system call table
--- reducing this attack surface.
We wrote several tests that attempt to issue illegal system calls with inline assembly;
we validate that all system calls are redirected to {\tt libLinux},
and that signals and other IPC cannot cross sandbox boundaries.
%This protection adds only \~{}2,500 lines of code to the trusted computing base
%of the system.  
%Admittedly, Linux itself adds millions of lines to the TCB,
%but independent results indicate the \graphene{} ABI
%could be implemented on a substantially smaller kernel~\cite{porter11drawbridge, baumann13bascule}.




%\graphene{}’s goal is to provide a lightweight alternative to isolate applications. Instead of leveraging a heavy-weighed VM to isolate an application or a set of collaborating processes, one can leverage \graphene{} to achieve isolation with much less memory overhead. Application isolation has many advantages from a security viewpoint: any vulnerability, malicious computations and bugs are isolated in the application sandbox. It is very hard for a malicious application to affect another, for example, obtaining/leaking sensitive information,  crashing or affecting performance. 

\begin{comment}
In our design, only trusted applications such as system utilities can run as a non-\graphene{} \picoproc{}, as they need many more functionality that the \graphene{} Linux ABI support. From a security standpoint, nothing prevents a \graphene{} \picoprocs{} from acting maliciously, as application code runs unmodified in \graphene{}.  However, it is important to note that \graphene{}’s goal is not to secure a system but to isolate applications in a competitive level compared to the VM alternative. 
\end{comment}

%Thus, the security issues discussed in this section are related to \emph{isolation}, especially in comparison to the VM environment. In fact, VM environment can suffer from malicious processes in the same guest. 

%For example, a malicious process in a VM could hog system resources on purpose (e.g, continuously computing the factorial of a number to waste CPU cycles) to affect performance/availability of another VM system. 


\paragraph{Isolation experiments.}
This subsection tests whether \graphene{} meets its goal of providing equivalent isolation
to a VM.
We conducted an evaluation of the security of the isolation mechanisms in \graphene{} and analyzed whether a malicious \graphene{} \picoproc{} could 
(i) fork a non-\graphene{} process, 
(ii)  kill processes from another sandbox or a non \graphene{} process, 
(iii) access files not prescribed in its Manifest, and 
(iv) discover secrets from \picoprocs{} in other sandboxes or from non-\graphene{} process through side-channels via the {\tt /proc} file system. 
The first three attacks use inline assembly to directly issue a system call, and are blocked by the reference monitor; the fourth creates
an attack similar to Memento~\cite{memento} and is frustrated by the fact that {\tt /proc} is implemented within {\tt libLinux} and
the system {\tt /proc} is inaccessible from \graphene{}.

\begin{comment}
Consider the configuration illustrated in Figure \ref{fig:coordination} and assume that \picoproc{} 1 is malicious . In our first experiment, \picoproc{} 1 unsuccessfully attempts to fork a non-\graphene{} process by invoking the {\tt fork()} function and also invoking the {\tt fork} system call directly. Since \graphene{} hooks system calls at the kernel level, it does not matter how the fork occurs since the kernel checks whether it is a \graphene{} \picoproc{} that is calling {\tt fork} and creates a new \graphene{} process if it is. 

In our second experiment \picoproc{} 1 attempts to kill \picoproc{} 3 in a separate sandbox. Picoprocess 1 is blocked from calling the kill command on any process that is not explicitly contained in its sandbox since the PID is checked against the sandbox's PID list. 

%Each \graphene{} application includes a manifest, which specifies restrictions, including such as network firewall rules and subsets of the file system it is allowed to access. 

In a third experiment \picoproc{} 1 tries to access files that do not belong in its Manifest. Picoprocess 1 attempts to circumvent the Manifest rules through the function {\tt open}, {\tt fopen}, and also by calling the open system call directly. When the open system call is called, it’s filename parameter is checked against a list of acceptable directories and rejected if it does not match anything in this list. This blocks the \picoproc{} 1 from accessing illegal files.

In our last experiment, \picoproc{} 1 attempts to obtain secrets from \picoproc{} 3, in an attack similar to Memento \cite{memento}. \graphene{} processes are restricted from using the proc file system in order gain information about system processes.  In our experiment, \picoproc{} 1's manifest file was setup to deny access to this directory and, as a result, it was unable access system or other processes' information outside its sandbox.
\end{comment}

%This attack involves correlating a web browser’s data resident size (inspected via the {\tt proc} file system) to the web sites visited by a user. 
%Since the proc file system is accessed as though it was part of the normal file system, any access to it has to go through the system  open command.
\paragraph{Analysis of Linux vulnerabilities.}
\graphene{} restricts \picoprocs{} to 15\% of the system call table.
To evaluate the impact on system security,
we \emph{manually} analyzed
all reported Linux vulnerabilities from 2011--2013 (a total
of 291 vulnerabilities)~\cite{linuxvuln}.
We categorized these exploits by kernel component, listed in Table~\ref{table:vulnerabilities}.
Roughly half of these vulnerabilities required a system
call  which \graphene{} blocks in order to exploit the system.
\graphene{} would only allow 5 of the relevant vulnerabilities
through its system call filtering and reference monitor.
The remaining half of the vulnerabilities were entirely within the kernel or modules,
    such as bugs in the virtual memory subsystem.

\begin{table}[t!b!]
\footnotesize
\centering
\begin{tabular}{|l|r|rr|}
\hline
{\bf Category } & {\bf Total} & \multicolumn{2}{|c|}{{\bf Prevented by \graphene{}}}\\

%&& \multicolumn{2}{|c|}{{\bf by \graphene{}}}\\
\hline
System call      & 118        &   \hspace{0.2in} 113 &96\% \\\hline 
Network          & 73         &   \hspace{0.2in}30 & 41\% \\\hline 
File system      & 33         &  \hspace{0.2in} 2  & 6\% \\\hline 
Drivers          & 37         &   \hspace{0.2in} 0 &\\\hline 
VM subsystem     & 15         &   \hspace{0.2in} 0 &\\\hline 
Application vulnerabilities & 2   & \hspace{0.2in} 2 & 100\% \\\hline 
Kernel other     & 13      & \hspace{0.2in} 0 &\\\hline 
Total            & 291     & \hspace{0.2in} 147 & 51\% \\\hline 
%\hline
\end{tabular}
\caption[Analysis of Linux vulnerabilities prevented by \graphene{}]
{Manual analysis of Linux vulnerabilities from 2011-2013 and \graphene{}'s prevention.}
\label{table:vulnerabilities}
\end{table}


%One particularly interesting observation was that nearly 10\% of vulnerabilities
%were in uncommonly-used network protocols, and 

%%% For each
%%% vulnerability we analyzed whether or not \graphene{} restrictions,
%%% especially related to system calls, would have prevented the
%%% vulnerability. Table \ref{table:vulnerabilities} shows the results of
%%% our analysis.

%System
%call interface, Network subsystem , File System, Drivers, VM support
%subsystem, vulnerable applications and general kernel issues.
%%% Even though \graphene{}'s main goal is to provide a light-weight
%%% alternative for isolating a process or a group of cooperating
%%% processes, we wanted to discover how this paradigm affect the overall
%%% security of a system. To accomplish that we 


 
\begin{comment}
As Table \ref{table:vulnerabilities} illustrates almost all Linux
system call-related vulnerabilities are prevented through the \graphene{}
paradigm. These vulnerabilities are related to system calls or flags
that are blocked or restricted in \graphene{}, such as I/O control, event
polling and notification, huge pages, socket options, task stats,
\emph{etc.}  \graphene{} also restricts rarely used network protocols
such as VSOCK, TIPC and ROSE, and this prevents 41\% of network
vulnerabilities in \graphene{} systems. The Manifest file used to
restrict portions of the file system also prevented 2 FS-related
vulnerabilities. We also considered vulnerabilities in Linux
applications as preventable by \graphene{}, as all the effects are
confined in the correspondent \picoproc{} or sandbox. 
\end{comment}



Despite the fact that our primary security goal is isolation, 
these results indicate that moving the system call table into 
the application has the potential to substantially reduce exploitable 
system vulnerabilities.

\paragraph{New opportunities.}
To explore new use cases of the \graphene{} sandboxing model, we modified the 
Apache {\tt mod\_\-auth\_\-basic.so} module to call the new library OS function
{\tt sandbox\_create} after user authentication.
The work\-er process that services the user request executes in a separate sandbox 
with file system access restricted to only data required for that user.
Similarly, this worker's libosannot coordinate shared OS abstractions with other worker processes,
limiting the risk to other users if this process is exploited.
We see interesting opportunities to expand this model in future work.
\fixme{Describe in details.}

%The module detaches a child process that serves the request, put it in a more
%restrictive sandbox with limited access to the filesystem, without downgrading functionality.
%Although a sandboxed
%process cannot be reused for further requests, it quarantines the authenticated
%website users in a jail. The quarantined users are still abled to download the contents which they are allowed to.


%%% Our results show
%%% that \graphene{} paradigm prevents 50\% of all Linux security
%%% vulnerabilities reported for the last 3 years, even though the
%%% immediate goal of the paradigm is isolation.



%As an example of our analysis process, consider the description of Linux vulnerability with CVE-2013-1828: \emph{The {\tt sctp\_getsockopt\_assoc\_stats} function in net/sctp/socket.c in the Linux kernel before 3.8.4 does not validate a size value before proceeding to a {\tt copy\_from\_user} operation, which allows local users to gain privileges via a crafted application that contains an SCTP\_GET\_ASSOC\_STATS {\tt getsockopt} system call}. As the {\tt getsockopt} system call is not provided in the PAL and gets blocked even if the application invoked it directly through inline assembly, this vulnerability will not occur in \graphene{} applications.

%processes sharing I/O context, perf events, key control and related functions, ptrace, capabilities, module loading/unloading, tun tap, partitions, TPM encruption, sigqueueand and Hyper V.
%, NFC, NETROM, LLC, L2TP, IUCV, IRDA, CAIF, BLUETOOTH


%\fixmedp{Please list the version of each, and write a sentence for each one
%describing how it was .}
% released under GPL.

%% don issues:
%% Why fork+sh fast in graphene?
%% Why stat slow in graphene
%% why unix sockets fast in kvm
%% why file system so fast?





%% This experiment show how \graphene{} isolate child processes by
%% shutdown namespace coordination channel. We enforce an easy
%% isolation rule (defined by manifest) that isolate process
%% after their parent explicitly exits. The child process loses the view
%% of namespace of its parent, and  becomes a namespace master.
