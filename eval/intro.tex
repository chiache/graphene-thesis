This chapter evaluates the performance and security of the \graphene{} \libos{} on both the Linux and SGX host.
For both hosts,
the evaluation includes the overhead on start-up times, memory footprints, server throughputs and command-line program latencies, and micro-benchmark results regarding \linuxapi{} overheads.
The evaluation on the Linux host also includes a study of Linux kernel vulnerabilities contained by \graphene{}.
The evaluation on the SGX host includes a summary of trusted computing base (TCB) that \graphenesgx{} adds to an enclave, in lines of code (LoC).

All experiment results are collected on a Dell Optiplex 790 Small-Form Desktop,
with a 4-core 3.20 GHz Intel Core i5-6500 CPU (no hyper-threading, with 6MB cache),
8 GB RAM, and a 512GB, 7200 RPM SATA disk.
%We disable SpeedStep and TurboBoost to prevent interference.
The host OS is Ubuntu 16.04.4 LTS, with Linux kernel 4.10.0.
Each machine uses a 1 Gbps Ethernet card connected to a dedicated local network.





We use version 1.9 of 
the Intel \sgx{} Linux SDK~\cite{intel-sgx-linux-sdk} and driver~\cite{intel-sgx-linux-driver}.



%Except for scalability measurements, 
%all measurements were collected on a 
%Dell Optiplex 790 with 
%a 4-core 3.40 GHz Intel Core i7 CPU,
%4 GB RAM, and a 250GB, 7200 RPM ATA disk.
%Our host system runs Ubuntu 12.04 server with host Linux kernel version 3.5, 
%which includes KVM on 
%QEMU version 1.0.
%%Adding more detail of KVM environment
%Each KVM Guest is deployed with 4 virtual CPU with EPT, 2GB RAM, a 30GB virtual disk image, Virtio enabled for network and disk, bridged connection with TAP, and runs the same Ubuntu and Linux kernel image.
%We note that recent library OSes are either not openly available
%or cannot execute unmodified Linux binaries.
%Unless otherwise noted, \graphene{} measurements include the reference monitor.
%%In order to assess the relative overhead of the monitor (\S\ref{sec:eval:micro}), 
%%we include some microbenchmark measurements with and without the monitor.





\section{Performance Evaluation}
\label{sec:eval:perf}



%\fixmedp{Missing back-of-the envelope for fork overheads, fraction of system calls using ipc}


This section evaluates \graphene{}'s multi-process coordination, security, cross-host migration, memory footprint, and performance.
We drive this evaluation with a selection of real-world applications that leverage multiple processes in \graphene{},
as well as with microbenchmarks and stress tests.
We organize the evaluation around the following questions:
\begin{compactenum}
	\item How do \graphene{}'s startup and migration costs compare to running an application in a dedicated VM?
	\item Given that RAM is often the limiting factor in VMs per system, how does \graphene{}'s memory footprint compare to other virtualization techniques?
	\item What are the performance overheads of \graphene{} relative to a native Linux process or VM?
	\item What additional overheads are added by the reference monitor?
	\item How do \graphene{}'s overheads scale with the number of processes in a sandbox?
	\item Does the \graphene{} reference monitor enforce security isolation comparable to running the application in a VM?  
	\item What fraction of recent Linux vulnerabilities would \graphene{} prevent?
\end{compactenum}

%We note that no recent single-process library OSes are both publicly available
%and able to execute unmodified Linux binaries.



%We evaluate the performance overheads on running unmodified Linux applications
%on SGX.
%Unlike the existing \sgx{} shielding systems which focuses on cloud-based systems or microservices,
\graphenesgx{} is designed to be general-purpose, supporting a broad range of
server and command-line applications. 
%, including both server and desktop workloads.
%\fixmedp{I wouldn't really call gcc or R ``desktop''}
\fixme{get rid of the part that we need to compile source code}
We thus evaluate performance overheads of unmodified Linux applications, using binaries 
from an Ubuntu installation.
%, or,
%in the case of Apache, Lighttpd, and NGINX,
%compiled from original source code using the default configurations.}
%(Apache, Lighttpd, and NGINX are compiled from source).} %\fixmedp{which apps are compiled?}
Depending on the workload, we measure application throughput or latency.
%According to the types of workloads, we design experiments to evaluate whether \graphenesgx{} can support applications with acceptable throughput or latency.

%All applications in our experiments are real, unmodified Linux applications,
%either directly taken from a disk with Ubuntu installation,
%or compiled using the default configuration given by the developers.
%No source code modification or change of compilation environment is required in the experiments.
%In our experiments, we either test on application binaries taken off-the-shelf from the Ubuntu {\tt APT} repositories,
%or applications that are built from the default configurations
%provided by the developers.
%Either source code modification or adjustment of the compilation environment is avoided in the setup of experiments.
%Our evaluation simulates the realistic results of running unmodified Linux applications in \graphenesgx{}.
%running Linux COTS applications on commodity hardware.

In order to differentiate SGX-specific overheads 
from Graphene overheads,
%, which also
%runs on a Linux host without SGX, 
we use both
Linux processes and \graphene{} on a Linux host without SGX as baselines
for comparison.
%Since a large portion of the \libos{} design is inherited from the \graphene{} \libos{},
%the performance impact of adopting \graphenesgx{} is largely affected by the design choices made in \graphene{}.
%In order to show the performance impact of \graphenesgx{} over \graphene{}, we use both native Linux and \graphene{} on a Linux host as the baselines for comparison.
Note that \graphene{} includes two optional kernel extensions:
one that creates a reference monitor to protect the host kernel from 
the library OS, and one that optimizes fork by 
with copy-on-write for large (page-sized) RPC messages.
%implementing a multi-page RPC
%abstraction.
Neither of these extensions are currently supported in \graphenesgx{}.
%so we measure baseline Graphene with these disabled.
% can be optionally run with a reference monitor for security isolation, and a bulk IPC abstraction for fork optimization.
%Since neither of the features is ported in \graphenesgx{},
%we mostly compare \graphenesgx{} to \graphene{} with these two features disabled,
%to show a more meaningful comparison.