This chapter evaluates the performance overheads of \graphene{}
on two different host ABI implementations;
one is an unmodified Linux kernel, and the other is inside of an \sgx{} enclave running on an untrusted Linux host.
The evaluation
targets the following four aspects:
(1) translation, isolation, and shielding costs of the host ABI and startup time;
(2) emulation overheads of the \libos{} on system calls latency and throughput;
(3) end-to-end performance of sample applications;
(4) resource costs, including both memory footprint and CPU occupancy.
%The evaluation compares experiment results
%%(either execution time, processing bandwidth, or resource usage)
%%of benchmarks and applications,
%between \graphene{}, \graphenesgx{}, and native processes running on an unmodified, generic Linux kernel.
The results
show the cost of leveraging \graphene{} for compatibility and security isolation.
%of a micro-benchmark or sample application on \graphene{} or \graphenesgx{}, with the results collected on a native Linux kernel.





% the overhead on start-up times, memory footprints, server throughputs and command-line program latencies, and micro-benchmark results regarding system call overheads.
%The evaluation on the Linux host also includes a study of Linux kernel vulnerabilities contained by \graphene{}.
%The evaluation on the \sgx{} host includes a summary of trusted computing base (TCB) that \graphenesgx{} adds to an enclave, in lines of code (LoC), and attack surfaces.


%\fixmedp{Missing back-of-the envelope for fork overheads, fraction of system calls using ipc}


%This section evaluates \graphene{}'s multi-process coordination, security, cross-host migration, memory footprint, and performance.
%We drive this evaluation with a selection of real-world applications that leverage multiple processes in \graphene{},
%as well as with microbenchmarks and stress tests.
%We organize the evaluation around the following questions:
%\begin{compactenum}
%\item How do \graphene{}'s startup and migration costs compare to running an application in a dedicated VM?
%\item Given that RAM is often the limiting factor in VMs per system, how does \graphene{}'s memory footprint compare to other virtualization techniques?
%\item What are the performance overheads of \graphene{} relative to a native Linux process or VM?
%\item What additional overheads are added by the reference monitor?
%\item How do \graphene{}'s overheads scale with the number of processes in a sandbox?
%\item Does the \graphene{} reference monitor enforce security isolation comparable to running the application in a VM?  
%\item What fraction of recent Linux vulnerabilities would \graphene{} prevent?
%\end{compactenum}

%We note that no recent single-process library OSes are both publicly available
%and able to execute unmodified Linux binaries.



%%We evaluate the performance overheads on running unmodified Linux applications
%%on \sgx{}.
%%Unlike the existing \sgx{} shielding systems which focuses on cloud-based systems or microservices,
%\graphenesgx{} is designed to be general-purpose, supporting a broad range of
%server and command-line applications. 
%%, including both server and desktop workloads.
%%\fixmedp{I wouldn't really call gcc or R ``desktop''}
%\fixme{get rid of the part that we need to compile source code}
%We thus evaluate performance overheads of unmodified Linux applications, using binaries 
%from an Ubuntu installation.
%%, or,
%%in the case of Apache, Lighttpd, and NGINX,
%%compiled from original source code using the default configurations.}
%%(Apache, Lighttpd, and NGINX are compiled from source).} %\fixmedp{which apps are compiled?}
%Depending on the workload, we measure application throughput or latency.
%%According to the types of workloads, we design experiments to evaluate whether \graphenesgx{} can support applications with acceptable throughput or latency.

%All applications in our experiments are real, unmodified Linux applications,
%either directly taken from a disk with Ubuntu installation,
%or compiled using the default configuration given by the developers.
%No source code modification or change of compilation environment is required in the experiments.
%In our experiments, we either test on application binaries taken off-the-shelf from the Ubuntu {\tt APT} repositories,
%or applications that are built from the default configurations
%provided by the developers.
%Either source code modification or adjustment of the compilation environment is avoided in the setup of experiments.
%Our evaluation simulates the realistic results of running unmodified Linux applications in \graphenesgx{}.
%running Linux COTS applications on commodity hardware.

%In order to differentiate \sgx{}-specific overheads 
%from Graphene overheads,
%%, which also
%%runs on a Linux host without \sgx{}, 
%we use both
%Linux processes and \graphene{} on a Linux host without \sgx{} as baselines
%for comparison.
%%Since a large portion of the \libos{} design is inherited from the \graphene{} \libos{},
%%the performance impact of adopting \graphenesgx{} is largely affected by the design choices made in \graphene{}.
%%In order to show the performance impact of \graphenesgx{} over \graphene{}, we use both native Linux and \graphene{} on a Linux host as the baselines for comparison.
%Note that \graphene{} includes two optional kernel extensions:
%one that creates a reference monitor to protect the host kernel from 
%the library OS, and one that optimizes fork by 
%with copy-on-write for large (page-sized) RPC messages.
%%implementing a multi-page RPC
%%abstraction.
%Neither of these extensions are currently supported in \graphenesgx{}.
%%so we measure baseline Graphene with these disabled.
%% can be optionally run with a reference monitor for security isolation, and a bulk IPC abstraction for fork optimization.
%%Since neither of the features is ported in \graphenesgx{},
%%we mostly compare \graphenesgx{} to \graphene{} with these two features disabled,
%%to show a more meaningful comparison.



\paragraph{Experimental setup.}

The evaluation is based on \graphene{} \grapheneversion{}\footnote{\graphene{} is released at \url{https://github.com/oscarlab/graphene}}.
All experiment results are collected on a Dell Optiplex 790 Small-Form Desktop,
with a 4-core 3.2GHz Intel Core i5-6500 CPU
without hyper-threading
%(no hyper-threading, with 256KB I-Cache and D-Cache, 1MB L2 Cache, and 6MB L3 Cache),
two 4GB DIMM 1600MHz DDR3 RAMs (8GB in total), and a Seagate 512GB, 7200 RPM SATA disk formatted as EXT4.
\sgx{} is enabled on the CPU
%The Intel CPUs are configured with \sgx{} (Software Guard Extensions, requires BIOS update) and EPT (Extended Page table) enabled, and
with 128MB EPC
(enclave page cache).
To prevent fluctuation in experiment results,
Turbo Boost, SpeedStep, and CPU idle states are disabled for
all experiments.
%We disable SpeedStep and TurboBoost to prevent interference.
All networked servers are evaluated over an 1Gbps Ethernet card connected to a dedicated local network,
except that some micro-benchmarks are evaluated over a local loopback device (i.e., localhost).




The host OS for evaluating the Linux and \sgx{} ports is Ubuntu \ubuntuversion{} with Linux kernel \linuxversion{},
which is also the baseline for comparison.
All test programs and applications are dynamically linked
with a modified \glibc{} \glibcversion{}.
%The experiments for 
\graphenesgx{}
uses the Intel \sgx{} Linux SDK~\cite{intel-sgx-linux-sdk} and driver~\cite{intel-sgx-linux-driver} \sgxdriverversion{}.
%The host OS also
%includes KVM on QEMU version 2.5 and Xen 4.6.5 for comparison.
%%%Adding more detail of KVM environment
%Each KVM or Xen Guest is deployed with 4 virtual CPU with VT and EPT (Extended Page Table), 1GB RAM, a 30GB virtual disk image, VirtIO enabled for network and disk, bridged connection with TAP, and runs the same Ubuntu base system and Linux kernel as the host OS.










